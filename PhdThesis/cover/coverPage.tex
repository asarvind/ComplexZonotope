% % % la ligne ci-dessous est à insérer obligatoirement dans le préambule du document avant \begin{document}
% % 
% % \usepackage[a4paper]{meta-donnees}


% les lignes en bas sont à insérer obligatoirement après \begin{document}

%%%%%%%%%%%%%%%%%%%%%%%%%%%%%%%%%%%%%%%%%%%%%%%%%%%%%%
%%             Commandes Meta-données               %%
%%   à renseigner par les auteurs pour générer      %%
%%     la couverture modèle Univ. Grenoble          %%
%%%%%%%%%%%%%%%%%%%%%%%%%%%%%%%%%%%%%%%%%%%%%%%%%%%%%%
%%      Fichier encodé au format ISO-8859-16        %%

%\Sethpageshift{???mm}   %%optionnel : à décommenter si besoin pour ajout d'espace afin de center la couvérture horizontalement (valeur par défaut est -5.5mm)
%\Setvpageshift{???mm}   %%optionnel : à décommenter si besoin pour ajout d'espace afin de center la couvérture verticalement (valeur par défaut est -15.5mm)


%\Universite{}    %%optionnel : à décommenter et à renseigenr si vous voulez changer le non d'université
%\Grade{}         %%optionnel : à décommenter et à renseigenr si vous voulez changer le grade
\Specialite{Math\'ematiques et Informatique}
\Arrete{}
\Auteur{Santosh Arvind Adimoolam}
\Directeur{Thao Dang}
%\CoDirecteur{}    %%optionnel : à décommenter et à renseigenr si présence d'un Co-directeur de thèse
\Laboratoire{du laboratoire {Verimag}}
\EcoleDoctorale{Ecole Doctorale Math\'ematiques, Sciences et Technologies de l'Information, Informatique}         
\Titre{Calculus of Complex Zonotopes}
%\Soustitre{}      %%optionnel : à décommenter et à renseigenr si présence d'un sous-titre de thèse
% \Depot{DD Month 2017}       


% Commande pour création de nouvelles catégories dans le jury:

%\UGTNewJuryCategory{...NomDeLaCategorie...}{...Definition...}

% Exemple \UGTNewJuryCategory{UGTFamille}{Membre de la famille} que nous ajoutons dans la commande \Jury ci-dessous sous la forme \UGTFamille{Jean Rousseau}{(...titre_et_affiliation...s'il_y_en_a...)}


%% \Jury{
%% \UGTPresident{...Civilit, Prnom\_et\_Nom...}{...titre\_et\_affiliation...}
%% \UGTPresidente{...Civilit, Prnom\_et\_Nom...}{...titre\_et\_affiliation...}

%% \UGTRapporteur{...Civilit, Prnom\_et\_Nom...}{...titre\_et\_affiliation...}      %% 1er rapporteur
%% \UGTRapporteur{...Civilit, Prnom\_et\_Nom...}{...titre\_et\_affiliation...}      %% second rapporteur
 
%% \UGTExaminateur{...Civilit, Prnom\_et\_Nom...}{...titre\_et\_affiliation...}     %% 1er examinateur
%% \UGTExaminateur{...Civilit, Prnom\_et\_Nom...}{...titre\_et\_affiliation...}     %% second examinateur
%% \UGTExaminatrice{...Civilit, Prnom\_et\_Nom...}{...titre\_et\_affiliation...}    %% 3ème examinateur

%% \UGTDirecteur{...Civilit, Prnom\_et\_Nom...}{...titre\_et\_affiliation...}       %% Directeur de thèse
%% \UGTCoDirecteur{...Civilit, Prnom\_et\_Nom...}{...titre\_et\_affiliation...}     %% Co-Directeur de thèse s'il y en a

%% \UGTInvite{...Civilit, Prnom\_et\_Nom...}{...titre\_et\_affiliation...}
%% \UGTInvitee{...Civilit, Prnom\_et\_Nom...}{...titre\_et\_affiliation...}
%% }

 \MakeUGthesePDG    %% très important pour générer la couvérture de thèse


