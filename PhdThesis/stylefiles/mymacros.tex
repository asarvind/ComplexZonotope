%% ========================================================
%%% Basic symbols
%% ========================================================

% \def    \s      {\sigma}
\def    \S      {\Sigma}
\def    \d      {\delta}
% \def    \D      {\Delta}
\def    \ee     {\epsilon}
\def    \e      {\varepsilon}
% \def    \g      {\gamma}
\def    \se     {\subseteq}
% 
\def    \N      {\mathbb{N}}
% \def    \Z      {\mathbb{Z}}
\def    \Q      {\mathbb{Q}}
\def    \R      {\mathbb{R}}
\def    \B      {\mathbb{B}}
\def    \A      {\mathcal{A}}
 
% 
\def    \qed    {\hfill\ensuremath{\square}\vspace{.7em}}%                
\def    \QED    {\hfill\ensuremath{\blacksquare}}%


%% ========================================================
%%% New commands
%% ========================================================

\newcommand{\sem}[1]{[\![#1]\!]}

%%% membership queries
\def \mq          {{\textsc{mq}}}
\def \eq          {{\textsc{eq}}}

%%% shortcuts
\newcommand{\lt}{\left}
\newcommand{\rt}{\right}
\newcommand{\pr}{\prime}
\newcommand{\dpr}{{\pr\pr}}
\newcommand{\tcz}{\tcztope}
\newcommand{\imp}{\implies}
\newcommand{\realset}{\reals}
\newcommand{\diag}{\diagonal}
\newcommand{\tbf}{\textbf}
\newcommand{\pair}[2]{\lt(#1,#2\rt)}
\newcommand{\mc}{\mathcal}
\newcommand{\maxaffine}{\maxapprox}
\newcommand{\minaffine}{\minapprox}

%% environments
% double vertically spaced row in table
\newcommand{\doublerow}[1]{\multirow{2}{*}{#1}}


%%% operations
% absolute value
\newcommand{\absolute}[1]{\lt|#1\rt|}
% infimum norm
\newcommand{\infnorm}[1]{\left\|#1\right\|_{\infty}}
% square norm
\newcommand{\sqnorm}[1]{\left\|#1\right\|_{2}}
% euclidean norm
\newcommand{\norm}[1]{\left\|#1\right\|_2}
% determinant
\newcommand{\determinant}[1]{\operatorname{det}\lt(#1\rt)}
% Minkowski sum
\newcommand{\minsum}[2]{#1\oplus #2}
% Intersection
\newcommand{\intersection}[2]{#1\bigcap #2}
% Meet
\newcommand{\meet}[2]{#1\bigwedge #2}
% Join
\newcommand{\join}[2]{#1\bigvee #2}
% vectormin
\newcommand{\vectormin}[1]{\underset{#1}{\bigwedge}}
% vectormax
\newcommand{\vectormax}[1]{\underset{#1}{\bigvee}}



%%% functions
% function specification
\newcommand{\func}[3]{#1:#2\rightarrow #3}
% support of a vector
\newcommand{\support}[3]{\rho\left(#1,#2,#3\right)}
% \support function
\newcommand{\supp}[2]{\rho\lt(#1,#2\rt)}
% over approximation error
\newcommand{\approxerror}[2]{\delta\lt(#1,#2\rt)}
% min-approximation 
\newcommand{\minapprox}[2]{\lt\llbracket\bigwedge\rt\rrbracket\lt(#1,#2\rt)}
% min-approximation function symbol
\newcommand{\minapproxsymbol}[2]{\llbracket\bigwedge\rrbracket}
% max-approximation
\newcommand{\maxapprox}[2]{\lt\llbracket\bigvee\rt\rrbracket\lt(#1,#2\rt)}
% max-approximation function symbol
\newcommand{\maxapproxsymbol}[2]{\llbracket\bigvee\rrbracket}
% Hausdorff distance
\newcommand{\hausdorff}[2]{\d_H\lt(#1,#2\rt)}
% over-approximation error
\newcommand{\err}[4]{\Delta\lt(#1,#2,#3,#4\rt)}
% shifted lower bound
\newcommand{\slb}[4]{\overline{l}_{#4}\lt(#1,#2,#3\rt)}
% shifted upper bound
\newcommand{\sub}[4]{\overline{u}_{#4}\lt(#1,#2,#3\rt)}
% contraction
\newcommand{\contraction}[2]{\chi\lt(#2,#1\rt)}
% concretization
\newcommand{\concrete}[1]{\lt\llbracket#1\rt\rrbracket}


%%% numbers
\newcommand{\reals}{\mathbb{R}}
\newcommand{\compnums}{\mathbb{C}}
\newcommand{\integers}{\mathbb{Z}}
% imaginary part
\newcommand{\img}{\operatorname{Im}}
% real part
\newcommand{\real}{\operatorname{Re}}




%%% arrays and vectors
% set of matrices
\newcommand{\mat}[3]{\mathbb{M}_{#1\times#2}\left(#3\right)}
% diagonal matrix
\newcommand{\diagonal}[1]{\operatorname{diag}\left(#1\right)}
% repeat elements
\newcommand{\repmat}[3]{\lt[#1\rt]_{#2\times #3}}
% inverse matrix
\newcommand{\inv}[1]{{#1}^{-1}}
% pseudo-inverse
\newcommand{\pinv}[1]{{#1}^\dagger}
% transpose
\newcommand{\transpose}[1]{{#1}^T}
% matrix
\newcommand{\mymatrix}[1]{\lt[\begin{matrix}#1\end{matrix}\rt]}
% orthonormal basis
\newcommand{\orthnorm}[1]{#1^*}
% column vector of identity matrix
\newcommand{\idvec}[1]{\bm{e}_{#1}}
% identity matrix
\newcommand{\identity}[2]{\bm{Id}_{#1\times #2}}
% vector with one component zero and rest equals one
\newcommand{\compidvec}[1]{\bm{\alpha}_{#1}}
% diagonal matrix with above vector
\newcommand{\compid}[1]{\diagonal{\bm{\alpha}_{#1}}}



%%% relations
% convex order relation between complex zonotopes
\newcommand{\order}{\sqsubseteq}
\newcommand{\equivalent}{\Leftrightarrow}



%%% set representations
% set 
\newcommand{\set}[1]{\left\{#1\right\}}
% polynomial ring
\newcommand{\polyring}[2]{#2\lt[#1\rt]}
% nullspace
\newcommand{\nullspace}[1]{\operatorname{null}\lt(#1\rt)}
% polytope
\newcommand{\polytope}[2]{\mathit{Poly}\lt(#1,#2\rt)}
% real zonotope
\newcommand{\rztope}[2]{\mathit{\mathcal{Z}}\lt(#1,#2\rt)}
% complex zonotope
\newcommand{\cztope}[2]{\mathit{\mathcal{C}}\lt(#1,#2\rt)}
% template complex zonotope
\newcommand{\tcztope}[3]{\mathit{\operatorname{\mathcal{T}}}\lt(#1,#2,#3\rt)}
% interval zonotope
\newcommand{\iztope}[3]{\mathit{\operatorname{\mathcal{I}}}\lt(#1,#2,#3\rt)}
% sub-parallelotope
\newcommand{\ptope}[3]{\mathcal{P}\left(#1,#2,#3\right)}
% augmented complex zonotope
\newcommand{\acztope}[6]{\mathcal{G}\left(#1,#2,#3,#4,#5,#6\right)}
% error interval zonotope
\newcommand{\errtope}[5]{\Delta_{#5}\lt(#1,#2,#3,#4\rt)}
% boundary of a set
\newcommand{\boundary}[1]{\operatorname{\bm{Bd}}\lt(#1\rt)}
% convexhull
\newcommand{\convexhull}[1]{\bm{Conv}\lt(#1\rt)}



%%% variables in set representation 
% primary template
\newcommand{\ptemp}{\mathcal{V}}
% secondary template
\newcommand{\sectemp}{\mathcal{W}}
% parallelotope template
\newcommand{\partemp}{\mathcal{K}}
% center
\newcommand{\cen}{c}
% scaling factors
\newcommand{\sfact}{s}
% secondary template
\newcommand{\stemp}{\mathcal{W}}
% lower interval bound
\newcommand{\lb}{l}
% upper interval bound
\newcommand{\ub}{u}
% parallelotope template
\newcommand{\qtemp}{\mathcal{K}}
% lower parallelotope bound
\newcommand{\plb}{\widehat{\lb}}
% upper parallelotope bound
\newcommand{\pub}{\widehat{\ub}}



%%% Variables in equations
% transfer matrix inclusion-checking
\newcommand{\tmat}{X}
% transfer vector in inclusion-checking
\newcommand{\tvect}{y}
% optimal variable substitution
\newcommand{\optimal}[1]{#1^*}




%%% System specifications
% system notation
\newcommand{\system}{\mathbb{H}}
% locations
\newcommand{\locations}{Q}
% location dynamics map
\newcommand{\locmap}{\Gamma}
% staying bounds
\newcommand{\stay}{\gamma}
% linear map
\newcommand{\linmap}{\mathcal{A}}
% Set of edges
\newcommand{\edgeset}{E}
% trajectory
\newcommand{\trj}[2]{{\bf #1}\lt(#2\rt)}
% continuous trajectory
\newcommand{\ctrj}[1]{\trj{x}{#1}}
% discrete trajectory
\newcommand{\dtrj}[1]{\trj{\loc}{#1}}
% hybrid trajectory
\newcommand{\htrj}[1]{\lt(\ctrj{#1},\dtrj{#1}\rt)}
% hybrid trajectory map
\newcommand{\maphtrj}{\lt(\tbf{x},\tbf{\loc}\rt)}
% input trajectory
\newcommand{\inptrj}[1]{{\bf u}\lt(#1\rt)}
% input set
\newcommand{\inputset}{U}
% initial set
\newcommand{\init}{\Psi_0}
% location
\newcommand{\loc}{q}
% edge
\newcommand{\edge}{\sigma}
% upper bound
\newcommand{\usys}[1]{#1^+}
% lower bound
\newcommand{\lsys}[1]{#1^-}
% initial template 
\newcommand{\inittemp}{\ptemp^{\operatorname{init}}}
% initial center
\newcommand{\initcen}{\cen^{\operatorname{init}}}
% initial scaling factors
\newcommand{\initsfact}{\sfact^{\operatorname{init}}}
% input template
\newcommand{\inputtemp}{\ptemp^{\operatorname{inp}}}
% input center
\newcommand{\inputcen}{\cen^{\operatorname{inp}}}
% input secondary template
\newcommand{\inputstemp}{\mathcal{W}^{\operatorname{inp}}}
% input lower interval bound
\newcommand{\inputlb}{\lb^{\operatorname{inp}}}
% input upper interval bound
\newcommand{\inputub}{\ub^{\operatorname{inp}}}
% input scaling factors
\newcommand{\inputsfact}{\sfact^{\operatorname{inp}}}
% Next reachable set from a given set of states
\newcommand{\nextset}[1]{\mathcal{R}\lt(#1\rt)}
% Next continuous reachable set 
\newcommand{\contreach}[2]{\mathcal{R}_{#1}\lt(#2\rt)}
% Reachable set at a time point from an initial set of states
\newcommand{\reachset}[2]{\mathcal{R}^{#1}\lt(#2\rt)}
% Set of trajectories
\newcommand{\trajectoryset}{\Gamma}
% Linear Invariance property
\newcommand{\invariance}[2]{\lt(#1,#2\rt)}
% continuous projection
\newcommand{\proj}[2]{#1_#2}
% positive invariance with respect to a transition
\newcommand{\posinv}[2]{#1\order\pair{\system}{#2}}
% lower sampling bound
\newcommand{\lsb}{\bm{\underline{\tau}}}
% upper sampling bound
\newcommand{\usb}{\bm{\overline{\tau}}}




%% ========================================================
%%% Referencing miscellaneous
%% ========================================================

% number equation inside align*
\newcommand\numberthis{\addtocounter{equation}{1}\tag{\theequation}}

%% ========================================================
%%% New operators
%% ========================================================

\DeclareMathOperator*{\argmax}{\rm arg\max}
\DeclareMathOperator*{\argmin}{\rm arg\min}

%% ========================================================
%%% Editing Commands
%% ========================================================
\def    \toupdate#1 {\textcolor{Periwinkle}{#1 } }
\def    \comment#1 { }
\def    \note#1    {\textbf{[ \marginpar{\texttt{note}}}{ \small #1 }\textbf{] }}
\def    \todo#1    {\marginpar{ \small \textcolor{red}{TODO: #1}}}


%% ========================================================
%%% Document appearance
%% ========================================================
 
\def    \ni     {\noindent}

%% centering vertically and horizontally in tabular
\newcolumntype{C}[1]{>{\centering\arraybackslash}m{#1} } 

%%% Changing the line spacing of the algorithms
\def \algsp {1.1} % line spacing in algorithms

%% Uniform style for the paragraph titles
\def \partitle#1 {\vspace{.7em}\ni\textbf{#1. } }

% styling lists (itemize)
\setlist[itemize]{itemsep=0pt,topsep=6pt}
\setlist[itemize,1]{label=--}

%% styling tabulars - set line spread
\newcommand{\ra}[1]{\renewcommand{\arraystretch}{#1}}

%% ========================================================
%%% New environments
%% ========================================================
\theoremstyle{plain}
\newtheorem{theorem}{Theorem}[section]
\newtheorem{proposition}[theorem]{Proposition}
\newtheorem{lemma}[theorem]{Lemma}
\newtheorem{corollary}[theorem]{Corollary}
\theoremstyle{definition}
\newtheorem{definition}[theorem]{Definition}
\newtheorem{example}[theorem]{Example}
\newtheorem{remark}[theorem]{Remark}
\newtheorem{problem}[theorem]{Problem}
\newtheorem{thm}{Theorem}[section]

\newtheorem{lem}[theorem]{Lemma} \newtheorem{prop}[thm]{Proposition}
\newtheorem{cor}[theorem]{Corollary} \newtheorem{defn}[thm]{Definition}
\newtheorem{rem}[theorem]{Remark} \newtheorem{prp}[thm]{Property}
\newtheorem{prem}[theorem]{Premise} \newtheorem{egp}[thm]{Example}
\newcommand{\initstate}{\tbf{x}_0} \newcommand{\camount}{\mathcal{\xi}}

%% ========================================================
%%% Define colors 
%% ========================================================
\definecolor{myred}{RGB}{204,0,0}
\definecolor{mygreen}{RGB}{0,204,0}
\definecolor{myblue}{RGB}{0,0,204}
\definecolor{mypurple}{HTML}{880088}

\colorlet{colorhighlight}{black!10}
