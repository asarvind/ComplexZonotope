%% ========================================================
%%% Basic symbols
%% ========================================================

% \def    \s      {\sigma}
\def    \S      {\Sigma}
\def    \d      {\delta}
% \def    \D      {\Delta}
\def    \ee     {\epsilon}
\def    \e      {\varepsilon}
% \def    \g      {\gamma}
\def    \se     {\subseteq}
% 
\def    \N      {\mathbb{N}}
% \def    \Z      {\mathbb{Z}}
\def    \Q      {\mathbb{Q}}
\def    \R      {\mathbb{R}}
\def    \B      {\mathbb{B}}
\def    \A      {\mathcal{A}}
 
% 
\def    \qed    {\hfill\ensuremath{\square}\vspace{.7em}}%                
\def    \QED    {\hfill\ensuremath{\blacksquare}}%


%% ========================================================
%%% New commands
%% ========================================================

\newcommand{\sem}[1]{[\![#1]\!]}

%%% membership queries
\def \mq          {{\textsc{mq}}}
\def \eq          {{\textsc{eq}}}

%%% shortcuts
\newcommand{\lt}{\left}
\newcommand{\rt}{\right}
\newcommand{\pr}{\prime}


%%% operations
% absolute value
\newcommand{\absolute}[1]{\lt|#1\rt|}
%infimum norm
\newcommand{\infnorm}[1]{\left\|#1\right\|_{\infty}}
% Minkowski sum
\newcommand{\minsum}[2]{#1\oplus #2}

%%% functions
% function specification
\newcommand{\func}[3]{#1:#2\rightarrow #3}
% coefficients for generator based representations of a point
\newcommand{\coeff}[2]{\Theta\lt(#1,#2\rt)}

%%% numbers
\newcommand{\reals}{\mathbb{R}}
\newcommand{\compnums}{\mathbb{C}}

%%% arrays
% set of matrices
\newcommand{\mat}[3]{\mathbb{M}_{#1\times#2}\left(#3\right)}
% diagonal matrix
\newcommand{\diagonal}[1]{\mathcal{D}\left(#1\right)}



%%% set representations
% set 
\newcommand{\set}[1]{\left\{#1\right\}}
% polytope
\newcommand{\polytope}[2]{\mathit{Poly}\lt(#1,#2\rt)}
% real zonotope
\newcommand{\rztope}[2]{\mathit{\mathcal{Z}}\lt(#1,#2\rt)}
% complex zonotope
\newcommand{\cztope}[2]{\mathit{\mathcal{C}}\lt(#1,#2\rt)}
% template complex zonotope
\newcommand{\tcztope}[3]{\mathit{\mathcal{C}}\lt(#1,#2,#3\rt)}

%%% variables in set representation 
% primary template
\newcommand{\ptemp}{\mathcal{V}}
% secondary template
\newcommand{\sectemp}{\mathcal{W}}
% parallelotope template
\newcommand{\partemp}{\mathcal{K}}
% center
\newcommand{\cen}{c}
% scaling factors
\newcommand{\sfact}{s}

%%% Variables in equations
% transfer matrix inclusion-checking
\newcommand{\tmat}{X}


%% ========================================================
%%% New operators
%% ========================================================

\DeclareMathOperator*{\argmax}{\rm arg\max}
\DeclareMathOperator*{\argmin}{\rm arg\min}

%% ========================================================
%%% Editing Commands
%% ========================================================
\def    \toupdate#1 {\textcolor{Periwinkle}{#1 } }
\def    \comment#1 { }
\def    \note#1    {\textbf{[ \marginpar{\texttt{note}}}{ \small #1 }\textbf{] }}
\def    \todo#1    {\marginpar{ \small \textcolor{red}{TODO: #1}}}


%% ========================================================
%%% Document appearance
%% ========================================================
 
\def    \ni     {\noindent}

%% centering vertically and horizontally in tabular
\newcolumntype{C}[1]{>{\centering\arraybackslash}m{#1} } 

%%% Changing the line spacing of the algorithms
\def \algsp {1.1} % line spacing in algorithms

%% Uniform style for the paragraph titles
\def \partitle#1 {\vspace{.7em}\ni\textbf{#1. } }

% styling lists (itemize)
\setlist[itemize]{itemsep=0pt,topsep=6pt}
\setlist[itemize,1]{label=--}

%% styling tabulars - set line spread
\newcommand{\ra}[1]{\renewcommand{\arraystretch}{#1}}

%% ========================================================
%%% New environments
%% ========================================================
\theoremstyle{plain}
\newtheorem{theorem}{Theorem}[chapter]
\newtheorem{proposition}[theorem]{Proposition}
\newtheorem{lemma}[theorem]{Lemma}
\theoremstyle{definition}
\newtheorem{definition}[theorem]{Definition}
\newtheorem{example}[theorem]{Example}

%% ========================================================
%%% Define colors 
%% ========================================================
\definecolor{myred}{RGB}{204,0,0}
\definecolor{mygreen}{RGB}{0,204,0}
\definecolor{myblue}{RGB}{0,0,204}
\definecolor{mypurple}{HTML}{880088}

\colorlet{colorhighlight}{black!10}
