\documentclass[11pt,a4paper,twoside,openright]{article}  
%alt: [11/12pt,a4paper/letterpaper,oneside/twoside,openany/openright]{book/memoir}
\usepackage{graphicx}

\title{Calculus of Complex Zonotopes}
\author{Santosh Arvind Adimoolam}
\date{}

\begin{document}

\maketitle Many approaches in formal verification of cyber-physical
systems require computing the set of reachable states in a bounded or
unbounded span of time. Exactly computing the reachable sets of hybrid
models of cyber-physical systems, that describe the dynamics by an
interplay between continuous and discrete variables, is either
undecidable or computationally expensive.  Therefore, reachability
analysis approaches have generally concentrated on computing an
over-approximation of reachable set by using specific data
structures~\cite{cousot1978automatic,Sankaranarayanan+Dang+Ivancic-08-Symbolic,kvasnica2004multi,DBLP:journals/tecs/AllamigeonGSGP16,kurzhanskiy2006ellipsoidal,DBLP:conf/esop/AdjeGG10,prajna2004safety},
which we call as {\it set representations}.  In numerical abstract
interpretation, a set representation is usually an {\it abstract
  domain}, which is a lattice with an abstraction and a concretization
map~\cite{jeannet2009apron}.  Our research is in general concerned
with discrete time affine hybrid systems.  We consider the problem of
efficiently computing a small enough over-approximation of the
unbounded time reachable set that is useful for verifying the safety
the system.  The unbounded time reachable set of the system is usually
approximated by a {\it positive invariant}, which is a set of states
whose next reachable set is contained within itself.


For a set representation, efficient closure under reachability
computation operations and existence of an efficiently representable
positive invariant are desirable characteristics.  However, well known
set representations posess some of the above characteristics, but not
all.  For example, polytopes are efficiently closed under linear
transoformation, mutual intersection and intersection with linear
constraints.  But the Minkowski sum of polytopes can lead to either a
quadratic increase in the number of faces or an exponential increase
in the number of faces.  Moreover, even for the simpler case of a
stable linear transformation, a positively invariant polytope can
require arbitrarily large representation size.  Ellipsoids are
efficiently closed under linear transformation and can also
efficiently represent the positive invariant of a stable linear
transformation.  But ellipsoids are not closed under Minkowski sum and
approximating the Minkowski sum of ellipsoids by another ellipsoid can
incur a large over-approximation error.  The
zonotope~\cite{DBLP:conf/hybrid/Girard05} and quadratic
zonotope~\cite{DBLP:conf/aplas/AdjeGW15} set representations are
specified as linear and quadratic functions of real valued intervals,
respectively.  Geometrically, a simple zonotope is a Minkowski sum of
line segments and hence a type of polytope.  Zonotopes have the
advantage that they are efficiently closed under linear transformation
and Minkowski sum operations.  Therefore, they have been successfully
applied to the computation of bounded time reachable sets of uncertain
linear systems and some examples of affine hybrid systems.  But
concerning the over-approximation of the unbounded time reachable set
by a positively invariant real zonotope, we have the following
drawback.  It is not known whether a positively invariant zonotope
even exists for the simpler case of a stable linear transformation
having complex eigenvalues.  When the eigenvectors of a stable linear
transformation are real, then collecting the eigenvectors of the
matrix among the generators of the zonotope gives a positive
invariant.  However, this result does not hold when the eigenvectors
have complex values, i.e., have non-zero imaginary and real parts.



\section {Contribution}
\paragraph{Complex zonotope.}  To address the above drawback of real
zonotopes, we extend them to the complex valued domain by a set
representation called {\it complex zonotope}.  The most basic
representation of a complex zonotope is a linear combination of
complex valued vectors with complex combining coefficients, whose
absolute values are bounded within unity.  We show that a complex
zonotope captures contraction along the complex valued eigenvectors of
a linear transformation, which relates to existence of a positively
invariant complex zonotope.  The real projection of a complex zonotope
can represent not only Minkowski sums of line segments, but also some
ellipsoids.  Hence, they can represent non-polyhedral sets as well as
polytopic zonotopes.  They are different from quadratic zonotopes,
because while the latter is a quadratic function of real valued
intervals, a complex zonotope is a complex linear combination of
circles in the complex plane.
%
\begin{figure}
  \begin{minipage}{0.5\textwidth}
    \includegraphics[scale=0.6]{figures/complex-zonotope.png}
  \end{minipage}
  %
  \hspace{5em}
  %
  \begin{minipage}{0.4\textwidth}
    \includegraphics[scale=0.3]{figures/CZhull.png}
  \end{minipage}
  %
  \caption{Figure of a complex zonotope}
\end{figure}
%
\paragraph{Template complex zonotope.}  In order to easily refine a
complex zonotope by adding arbitrary generators while approximating a
given set, we introduce the {\it template complex zonotope}
representation.  In a template complex zonotope, the absolute values
of combining coefficients have variable bounds called scaling factors.
The scaling factors can be adjusted to find better over-approximations
while adding arbitrary generators in a {\it template}.  Just like
simple zonotopes, template complex zonotopes are also efficiently
closed under linear transformation and Minkowski sum operations.  But
in addition, template complex zonotope can easily represent a positive
invariant for a linear transformation based on the eigenstructure of
the matrix, as described previously.  We propose a sufficient
condition for checking inclusion between template complex zonotopes,
expressed as a set of second-order conic (convex) constraints on the
scaling factors and other auxiliary variables.  This condition is
useful in practice, which we demonstrate through our experiments in
safety and stability verification on some benchmark examples.

\paragraph{Augmented complex zonotope.}  Template complex
zonotopes are not closed under intersection with sub-level sets of
linear inequalities.  This is also a drawback of real zonotopes.
Previous work addressed this drawback for real zonotopes by allowing
linear constraints on the combining coefficients, in a more general
representation called {\it constrained
  zonotopes}~\cite{scott2016constrained}.  But if we emulate this
approach for complex zonotopes, we get linear constraints on the
complex valued combining co-coefficients in addition to the quadratic
absolute value bounds.  However, for the resultant set representation
having quadratic absolute value constraints as well as linear
constraints, the problem of finding a reasonable convex-relaxation for
inclusion-checking, that works well in practice, seems intractable.
Instead, we define a different representation called {\it augmented
  complex zonotope}, which is geometrically equivalent to a complex
zonotope, but whose representation allows efficient computation of a
non-trivial over-approximation of the intersection with a class of
linear constraints.  The over-approximation is computed as a simple
algebraic expression of the parameters of the augmented complex
zonotope.  In the course of the derivation of this result, we also
find a general mathematical result about intersection of Minkowski sum
of convex sets with another convex set.

\paragraph{Unbounded time safety verification.}
We express operations on augmented complex zonotopes like linear
transformation, Minkowski sum, over-approximation of intersection with
constraints and support of a vector as affine expressions of some of
the specification parameters.   We also derive a second order conic
program for checking inclusion between two augmented complex
zonotopes.  Henceforth, we derive a second order conic
program for checking the safety of an affine hybrid system.  The fact
that a complex zonotope can both capture the contraction along complex
eigenvectors and also efficiently represent linear transformation and
Minkowski sum makes it an efficient set representation for unbounded
time safety verification.  We demonstrate the efficiency of the safety
verification approach by implementing on some benchmark examples.

\paragraph{Stability verification of nearly periodic linear impulsive
  systems.}  %% The specification of an augmented complex
%% zonotope has a complex matrix, called {\it primary template}, a real
%% matrix called {\it secondary template}, the center of the constituting
%% template complex zonotope, called {\it primary offset}, bounds on the
%% absolute values of the complex combining co-efficients, called {\it
%%   scaling factors} and the {\it lower and upper interval bounds} on
%% the real combining co-efficients.
 %% These
%% expressions are affine functions of the primary offset, scaling
%% factors and lower and upper interval bounds. 
  Our analysis of template complex zonotopes for the case of discrete
  time switched linear systems is extended to verify stability of {\it nearly
    periodic linear impulsive systems}.  We first derive bounds on the
  amount of contraction of a complex zonotope after a linear
  transformation.  Then we derive a second order conic program to
  synthesize a contractive complex zonotope for a collection of linear
  transformations.  We apply these results to verify the exponential
  stability of nearly periodic linear impulsive systems.  We implement
  the procedure on some benchmark examples to demonstrate its
  efficiency.

\section {Organization}
The dissertation contains five main chapters and a conclusive chapter.
In Chapter 1, we briefly discuss the use of set representation for
computing reachable sets of hybrid systems and the desirable
characteristics of a good set representation.  In light of these
characteristics, we provide a short review of some set representations
that discusses their advantages and drawbacks.  Then we review
simple/real zonotopes and the commonly used operations on them in
reach set computation.  We also review the definitions polynomial and
constrained zonotopes, which are previously known extensions of the
simple zonotope.  Then we shall describe the shortcoming of the real
zonotope in computing a positive invariant.

In Chapter 2, we introduce the complex zonotope set representation
that addresses the drawback of real zonotope described in the previous
chapter.  We define the basic representation of a complex zonotope and
state a result about how a complex zonotope can efficiently represent
a positive invariant for a stable linear transformation based on the
eigenstructure.  Next we define the more general template complex
zonotope representation, which permits a way of refining a complex
zonotope while approximating a given set.  We discuss operations on
template complex zonotopes that are used in reach set computation and
derive a second order conic program for checking inclusion between
template complex zonotopes.  Using the results, we derive a second
order conic program for unbounded time safety verification of linear
systems and illustrate it with an example.

In Chapter 3,
we introduce augmented complex zonotopes for computing non-trivial
intersection with a class of sub-level sets of linear inequalities
called sub-parallelotopes.  Apart from the intersection, we discuss
other operations on augmented complex zonotope used in the computation
of reachable sets.  Also, we state a second order conic program for
checking inclusion between augmented complex zonotopes.

In Chapter 4, we apply augmented complex zonotopes for unbounded time
safety verification of affine hybrid systems.  We derive a second
order conic program for checking safety, which is based on computing a
suitable positively invariant augmented complex zonotope.  We
implement the approach on some benchmark examples to demonstrate its
efficiency.

In Chapter 5, we apply template complex zonotopes for verifying
exponential stability of linear impulsive systems with sampling time
uncertainty.  We discuss a procedure for stability verification, which
consists of synthesizing a possibly contractive template complex
zonotopes and verifying their contraction.  We implement this
procedure on some benchmark examples to demonstrate its efficiency.

The concluding remarks are stated in Chapter 6.








\bibliographystyle{plain}
\bibliography{PhdThesis/mylib}





\end{document}
