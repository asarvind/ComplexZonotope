We shall discuss the computation of some basic operations on template complex zonotopes
like linear transformation, Minkowski sum and intersection in special
cases.  In the next section, we shall discuss how to check inclusion
between two template complex zonotopes.  For the rest of this chapter,
we shall consider
%
\begin{align*}
\ptemp\in\mat{n}{m}{\compnums},~~\sfact\in\reals^m_{\geq 0},~~\cen\in\reals^n,
\end{align*}
%
unless otherwise specified.

The set of template complex zonotopes is closed under linear transformation and
{Minkowski sum} operations, which are straightforward algebraic
computations just like in the case of simple (real) zonotopes.
%
\begin{lemma}[Linear transformation]~\label{lem:lin-transform}
Let $A\in\mat{n}{n}{\reals}$ and $\ptemp\in\mat{n}{m}{\compnums}$.

%
\begin{equation*}~\label{eqn:lin-transform}
\text{Then}~~~~A\tcztope{\ptemp}{\cen}{\sfact}=\tcztope{A\ptemp}{A\cen}{\sfact}.
\end{equation*}
%
\end{lemma}
%
\begin{proof}
  We derive
  %
  \begin{align*}
&
    A\tcztope{\ptemp}{\cen}{\sfact}=A\set{\cen+\ptemp\zeta:~\zeta\in\compnums^m,~\absolute{\zeta}\leq
    \sfact}\\
    & =\set{A\cen+A\ptemp\zeta:~\zeta\in\compnums^m,~\absolute{\zeta}\leq
    \sfact}=\tcztope{A\ptemp}{A\cen}{\sfact}.~\hspace{3em}\qedhere
  \end{align*}
  %
\end{proof}
%
We see that the template of a complex zonotope can change after a
linear transformation.  But when the column vectors of the
template are the eigenvectors of the linear transformation and the
center is the origin, we can represent the transformed complex
zonotope using the same template by changing the scaling factors.
This result can be seen as an extension of Proposition~\ref{lem:eig-invariance} to the
case of template complex zonotopes.
%
\begin{proposition}[Eigenstructure based scaling]
Let us consider $\ptemp\in\mat{n}{n}{\compnums}$ consists of the complex
eigenvectors of a matrix $A\in\mat{n}{n}{\reals}$ as column vectors
and $\mu\in\compnums^n$ be the vector of eigenvalues such that
$A\ptemp=\ptemp\diagonal{\mu}$.  Let
$\tcztope{\ptemp}{0}{\sfact}\subset\compnums^n$ be a template complex
zonotope.  Then,
%
\begin{equation}~\label{eqn:eigen-scaling}
A\tcztope{\ptemp}{0}{\sfact}=\tcztope{\ptemp}{0}{\diagonal{\absolute{\mu}}\sfact}.
\end{equation}
%
\end{proposition}
%
\begin{proof}
Based on Equation~\ref{eqn:lin-transform} and the fact that
$A\ptemp=\ptemp\diagonal{\mu}$, we get
%
\[
A\tcztope{\ptemp}{0}{\sfact}=\tcztope{A\ptemp}{0}{\sfact}=\tcztope{\ptemp\diagonal{\mu}}{0}{\sfact}.
\]
%
Then using Equation~\ref{eqn:normalization}, we get
%
\[
\tcztope{\ptemp\diagonal{\mu}}{0}{\sfact}=\tcztope{\ptemp}{0}{\diagonal{\absolute{\mu}}\sfact}
\]
\end{proof}

The resultant template complex zonotope from the Minkowski sum of two
template complex zonotopes can be computed as follows.
%
\begin{lemma}[Minkowski sum]~\label{lem:min-sum}
Let us consider $\ptemp\in\mat{n}{m}{\compnums^m}$ and
$\ptemp^\pr\in\mat{n}{m}{\compnums^r}$.  Then
%
\begin{equation}
\minsum{\tcztope{\ptemp}{\cen}{\sfact}}{\tcztope{\ptemp^\pr}{\cen^\pr}{\sfact^\pr}}
= \tcztope{\begin{bmatrix}\ptemp & \ptemp^\pr\end{bmatrix}}{\cen+\cen^\pr}{\begin{bmatrix}\sfact\\\sfact^\pr\end{bmatrix}}
\end{equation}
\end{lemma}
%
\begin{proof}
We derive the following.
%
\begin{align*}
&
  \tcztope{\ptemp}{\cen}{\sfact}\oplus\tcztope{\ptemp^\pr}{\cen^\pr}{\sfact^\pr}\\
& =
  \set{\cen+\ptemp\zeta:~\zeta\in\compnums^n,~\absolute{\zeta}\leq\sfact}\oplus\set{\cen+\ptemp\zeta:~\zeta\in\compnums^n,~\absolute{\zeta}\leq\sfact}\\
& =
  \set{\lt(\cen+\cen^\pr\rt)+\ptemp\zeta+\ptemp^\pr\zeta^\pr:~\zeta\in\compnums^m,~\zeta^\pr\in\compnums^r,~\absolute{\zeta}\leq\sfact,~\absolute{\zeta^\pr}\leq\sfact^\pr}\\
& =
  \set{\lt(\cen+\cen^\pr\rt)+\mymatrix{\ptemp&\ptemp^\pr}\mymatrix{\zeta\\\zeta^\pr}:~\mymatrix{\zeta\\\zeta^\pr}\in\compnums^{m+r},~\absolute{\mymatrix{\zeta\\\zeta^\pr}\leq\mymatrix{\sfact\\\sfact^\pr}}}\\
& =  \tcztope{\begin{bmatrix}\ptemp & \ptemp^\pr\end{bmatrix}}{\cen+\cen^\pr}{\begin{bmatrix}\sfact\\\sfact^\pr\end{bmatrix}}.~\hspace{3em}\qedhere
\end{align*}
%
\end{proof}
%
We see that that the representation size of a template complex
zonotope can increase after a Minkowski sum with another template
complex zonotope.  But when two template complex
zonotopes have the same template, their Minkowski sum results in a
complex zonotope with the same template, i.e., the representation size
does not increase.  This is described in the following Proposition.
%
\begin{proposition}[Minkowski sum with common template]
The following is true.
%
\begin{equation}~\label{eqn:common-minkowski}
\minsum{\tcztope{\ptemp}{\cen}{\sfact}}{\tcztope{\ptemp}{\cen^\pr}{\sfact^\pr}}=\tcztope{\ptemp}{\cen+\cen^\pr}{\sfact+\sfact^\pr}
\end{equation}
%
\end{proposition}
%
\begin{proof}
We have $\minsum{\tcztope{\ptemp}{c}{s}}{\tcztope{\ptemp}{c^\pr}{s^\pr}}=$
%
\begin{align}~\label{eqn:proof-commin2}
&
\set{\lt(\cen+\ptemp\zeta\rt)+\lt(\cen^\pr+\ptemp\zeta^\pr\rt):\zeta,\zeta^\pr\in\compnums^m,~\absolute{\zeta}\leq
\sfact,~\absolute{\zeta^\pr}\leq \sfact^\pr}\nonumber\\
& = \set{\lt(\cen+\cen^\pr\rt)+\ptemp\lt(\zeta+\zeta^\pr\rt):\zeta,\zeta^\pr\in\compnums^m,~\absolute{\zeta}\leq
\sfact,~\absolute{\zeta^\pr}\leq \sfact^\pr}
\end{align}
%
First, we shall show that
\begin{equation}~\label{eqn:proof-commin1}
\set{\zeta+\zeta^\pr:\zeta,\zeta^\pr\in\compnums^m,~\absolute{\zeta}\leq
\sfact,~\absolute{\zeta^\pr}\leq\sfact^\pr}=\set{\zeta^\dpr:\absolute{\zeta^\dpr}\leq\sfact+\sfact^\pr }.
\end{equation}
%
To prove that the L.H.S of Equation~\ref{eqn:proof-commin1} is
contained within the R.H.S of Equation~\ref{eqn:proof-commin1},
consider $\zeta,\zeta^\pr\in\compnums^m$ where $\absolute{\zeta}\leq
\sfact,~\absolute{\zeta^\pr}\leq\sfact^\pr$.  For any
$i\in\set{1,...,m}$, we have
$\absolute{\zeta_i+\zeta^\pr_i}\leq \absolute{\zeta_i}+\absolute{\zeta^\pr_i}=\sfact+\sfact^\pr$
by the triangular inequality.  This proves that the L.H.S of
Equation~\ref{eqn:proof-commin1} is contained within the R.H.S of
Equation~\ref{eqn:proof-commin1}.

To prove the vice-versa, let us
consider
$\zeta^\dpr\in\compnums:\absolute{\zeta^\dpr}\leq \sfact+\sfact^\pr$.
Let
$\zeta=\sfact\frac{\zeta^\dpr}{\sfact+\sfact^\pr},~\zeta^\pr=\sfact^\pr\frac{\zeta^\dpr}{\sfact+\sfact^\pr}$.
Then $\absolute{\zeta}\leq \sfact$ and
$\absolute{\zeta^\pr}\leq\sfact^\pr$ because
$\absolute{\frac{\zeta^\dpr}{\sfact+\sfact^\pr}}\leq 1$.  So,
$\zeta^\dpr=\zeta+\zeta^\pr$ where $\absolute{\zeta}\leq \sfact$ and
$\absolute{\zeta^\pr}\leq \sfact^\pr$.  This proves that the R.H.S of
Equation~\ref{eqn:proof-commin1}  is contained inside the L.H.S of
Equation~\ref{eqn:proof-commin1}.

Hence, we have proved Equation~\ref{eqn:proof-commin1}.  Then combining 
Equation~\ref{eqn:proof-commin1}  with
Equation~\ref{eqn:proof-commin2} gives us
%
\begin{align*}
&\minsum{\tcztope{\ptemp}{c}{s}}{\tcztope{\ptemp}{c^\pr}{s^\pr}}=\set{\cen+\cen^\pr+\ptemp\zeta^\dpr:\absolute{\zeta^\dpr}\leq\sfact+\sfact^\pr}\\
&=\tcztope{\ptemp}{\cen+\cen^\pr}{\sfact+\sfact^\pr}.
\end{align*}
%
\end{proof}
%
The support of a template complex zonotope along any real vector is computed by a simple affine expression, as follows
%
\begin{lemma}[Support of a real vector]~\label{lem:support-tcz}
{\color{red} TODO}
\end{lemma}
%
\begin{proof}
{\color{red}TODO}.
\end{proof}
%
Like in the case of simple
zonotopes, template complex zonotopes are also not closed under mutual
intersection.  Moreover, the intersection between template complex
zonotopes with a common template and center need not also be a
tempalte complex zonotope in general.  Therefore, even for a fixed
template and center, there may not exist the smallest template complex
zonotope over-approximation of a given set. {\color{red} Explain with
figure}.

However, we observe that template complex zonotopes with a common
non-singular (invertible) template and center are closed under
intersection and the intersection can be expressed algebraically.
This is explained in the following lemma.
%
\begin{lemma}[Mutual intersection with common invertible template and center]~\label{lem:intersection}
Let us consider two template complex zonotopes
$\tcztope{\ptemp}{\cen}{\sfact},\tcztope{\ptemp}{\cen}{\sfact^\pr}\subset\compnums^n$
where their common template $\ptemp$ is a non-singular (invertible)
square matrix.  Then
%
\begin{equation}~\label{eqn:intersection}
\tcztope{\ptemp}{\cen}{\sfact}\bigcap\tcztope{\ptemp}{\cen}{\sfact^\pr}=\tcztope{\ptemp}{\cen}{\meet{\sfact}{\sfact^\pr}}
\end{equation}
%
\end{lemma}
%
\begin{proof}
Let us consider a point
$z\in\tcztope{\ptemp}{\cen}{\meet{\sfact}{\sfact^\pr}}$ which is
written in as a linear combination of the template vectors plus center
as $z=c+\ptemp\zeta:~\absolute{\zeta}\leq \meet{\sfact}{\sfact^\pr}$.
This means that $\absolute{\zeta}\leq\sfact$ and also
$\absolute{\zeta}\leq \sfact^\pr$.  So,
$z\in\tcztope{\ptemp}{\cen}{\sfact}$ and also
$z\in\tcztope{\ptemp}{\cen}{\sfact^\pr}$.  Therefore,
$\tcztope{\ptemp}{\cen}{\meet{\sfact}{\sfact^\pr}}\subseteq \tcztope{\ptemp}{\cen}{\sfact}\bigcap\tcztope{\ptemp}{\cen}{\sfact^\pr}$.

Next consider another point
$x\in\tcztope{\ptemp}{\cen}{\sfact}\bigcap\tcztope{\ptemp}{\cen}{\sfact^\pr}$.
Since $\ptemp$ is an invertible matrix, $x$ is uniquely written as a
linear combination of tempalte vectors plus center as
$\cen+\ptemp\lt(\inv{\ptemp}\lt(x-\cen\rt)\rt)$ where
$\absolute{\inv{\ptemp}\lt(x-\cen\rt)}\leq
\sfact$ and $\absolute{\inv{\ptemp}\lt(x-\cen\rt)}\leq
\sfact^\pr$, which means
$\absolute{\inv{\ptemp}\lt(x-\cen\rt)}\leq\meet{\sfact}{\sfact^\pr}$.  Therefore,
$\tcztope{\ptemp}{\cen}{\sfact}\bigcap\tcztope{\ptemp}{\cen}{\sfact^\pr}\subseteq\tcztope{\ptemp}{\cen}{\meet{\sfact}{\sfact^\pr}}$.

Combining the above two conclusions, we get that
$\tcztope{\ptemp}{\cen}{\sfact}\bigcap\tcztope{\ptemp}{\cen}{\sfact^\pr}=\tcztope{\ptemp}{\cen}{\meet{\sfact}{\sfact^\pr}}$.

\end{proof}
%
As explained before, in general, even for a fixed template and center,
there may not exist the smallest template complex zonotope
over-approximation of a given set.  However, since template complex
zonotopes with a fixed center and and an invertible square matrix
template are closed under intersection based on
Lemma~\ref{lem:intersection}, in this case there exists a smallest
template complex zonotope over-approximation of a given set.  This
smallest over-approximation is expressed as follows.
%
\begin{theorem}~\label{thm:min-abstraction}
Let $\Psi\subseteq\compnums^n$ be a bounded set.  Let
$\ptemp\in\mat{n}{n}{\compnums}$ be an invertible square matrix and
$\cen\in\compnums^n$.  Consider the set $S=\set{\sfact\in\reals_{\geq
0}^n:\Psi\subseteq\tcztope{\ptemp}{\cen}{\sfact}}$ and
$\sfact^*=\vectormin{s\in S} s$.  Then
%
\begin{equation}
\Psi\subseteq\bigcap_{\sfact\in S}\tcztope{\ptemp}{\cen}{\sfact}=\tcztope{\ptemp}{\cen}{\sfact^*}.
\end{equation}
%
and
%
\begin{equation}
\forall i\in\set{1,...,n},~\sfact^*_i=\max_{x\in\Psi}\lt(\inv{\ptemp}\lt(x-\cen\rt)\rt)_i.
\end{equation}
%
\end{theorem}
%
\begin{proof}
Based on Lemma~\ref{lem:intersection}, it follows that
%
\[
\bigcap_{\sfact\in
S}\tcztope{\ptemp}{\cen}{\sfact}=\tcztope{\ptemp}{\cen}{\vectormin{s\in
S}s}=\tcztope{\ptemp}{\cen}{\sfact^*}.
\]
%
Furthermore, as $S$ is defined as the set of all scaling factors
whose corresponding template complex zonotope is an over-approximation
of $\Psi$, we also get
%
\[
\Psi\subseteq\bigcap_{\sfact\in S}\tcztope{\ptemp}{\cen}{\sfact}=\tcztope{\ptemp}{\cen}{\sfact^*}.
\]
%
Next, consider $x\in\Psi$ and $\sfact^\pr\in\reals_{\geq 0}^n$ where
%
\[
\forall
i\in\set{1,...,n}:~\sfact^\pr=\max_{x\in\Psi}\lt(\inv{\ptemp}\lt(x-\cen\rt)\rt)_i.
\]
%
Since $\ptemp$ is invertible, $x$ is uniquely written as a linear
combination of template comlumn vectors of $\ptemp$ plus the center
$\cen$ as $x=\cen+\ptemp\lt(\inv{\ptemp}\lt(x-\cen\rt)\rt)$, where the
vector of combining coefficients is
$\lt(\inv{\ptemp}\lt(x-\cen\rt)\rt)$.  So, if $s\in S$, then 
$x\in\Psi\subseteq\tcztope{\ptemp}{\cen}{\sfact}$ and hence \[\forall
i\in\set{1,...,n}:~\lt(\inv{\ptemp}\lt(x-\cen\rt)\rt)_i\leq\sfact_i.\]
Therefore,
%
\begin{equation}~\label{proof-min-abstraction-1}
\forall
i\in\set{1,...,n}:~\sfact^\pr_i=\max_{x\in\Psi}\lt(\inv{\ptemp}\lt(x-\cen\rt)\rt)_i\leq\min_{s\in
S}s_i=\sfact^*_i.
\end{equation}
%
Furthermore, if $x\in\Psi$, then by the definition of $\sfact^\pr$, we
get $\lt(\inv{\ptemp}\lt(x-\cen\rt)\rt)\leq \sfact^\pr$.  Therefore,
$\Psi\subseteq\tcztope{\ptemp}{\cen}{\sfact^\pr}$.  This implies
$\sfact^\pr\in S$.  So, $\sfact^\pr\geq\sfact^*$.  Combining the
former conclusion with the conclusion in
Equation~\ref{proof-min-abstraction-1} gives $\sfact^\pr=\sfact^*$,
which proves the second part of the theorem.
\end{proof}

