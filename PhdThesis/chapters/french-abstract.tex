Le calcul de l'ensemble atteignable est une approche 
utilis\'ee dans de nombreuses m\'ethodes de v\'erification formelles pour
des syst\`emes hybrides. Mais le calcul exact de l'ensemble atteignable
est un probl\`eme d\^ur pour plusieurs classes de syst\`emes
hybrides, soit en raison de l'ind\'ecidabilit\'e ou de la complexit\'e de
calcul \'elev\'ee. Alternativement, beaucoup de recherches ont \'et\'e centr\'ees
sur le d\'ev\'eloppement des repr\'esentations d'ensembles qui peuvent \^etre
manipul\'ees efficacement pour calculer une surapproximation suffisamment
pr\'ecise de l'ensemble atteignable. Les zonotopes sont une
repr\'esentation d'ensemble utile pour l'analyse d'accessibilit\'e
en raison de leur fermeture et de leur faible complexit\'e pour le
calcul de la transformation lin\'eaire et de la somme de
Minkowski. Mais pour approximer l'ensemble atteignable pour une dur\'ee de temps non born\'ee
par des invariants positifs, les zonotopes pr\'esentent
l'inconv\'enient suivant. L'efficacit\'e d'une repr\'esentation d'ensemble
pour calculer un invariant positif d\'epend de l'encodage efficace des
directions de convergence des \'etats vers un point d'\'equilibre. Dans un syst\'eme
hybride affine, certaines des directions de convergence peuvent \^etre
d\'eriv\'ees \`a partir des vecteurs propres des matrices de
dynamiques continues. Mais la repr\'esentation zonotopique usuelle ne peut pas
exploiter la structure des vecteurs propres complexe de ces matrices de transformation
car la repr\'esentation zonotopique usuelle ne se d\'efinit avec des g\'en\'erateurs \'a valeurs r\'eelles.

Par cons\'equent, nous \'etendons les zonotopes r\'eels au domaine 
complexe afin de capturer la contraction le long des
vecteurs propres complexes, ce qui am\`ene \`a une nouvelle repr\'esentation
d'ensemble appel\'ee {\em zonotope complexe}. G\'eom\'etriquement parlant, les zonotopes
complexes repr\'esentent une classe d'ensembles plus large qui
comprennent des ensembles non-polytopiques ainsi que des zonotopes
polytopiques. Ils conservent les avantages des zonotopes r\'eels permettant
d'effectuer efficacement la transformation lin\'eaire et les
op\'erations de somme de Minkowski et calculer la fonction de
support. De plus, nous d\'eveloppons des
algorithmes approximatifs pour le test d'inclusion
et le calcul d'intersection avec des demi-espaces. En utilisant ces op\'erations
sur des zonotopes complexes, nous d\'eveloppons ensuite des programmes convexes
pour v\'erifier les propri\'et\'es d'invariance lin\'eaire des syst\'emes
hybrides affines \'a temps discret et la stabilit\'e exponentielle des
syst\'emes impulsifs lin\'eaires. Nos r\'esultats exp\'erimentaux sur certains exemples de
benchmarks d\'emontrent l'efficacit\'e nos techniques de v\'erification
bas\'ees sur des zonotopes complexes.
