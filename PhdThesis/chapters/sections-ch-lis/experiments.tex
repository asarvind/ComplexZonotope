We evaluated our algorithm on two benchmark examples of linear
impulsive systems below and compared it with other state-of-the-art
approaches.  For convex optimization, we use CVX version 2.1 with
Matlab 8.5.0.197613 (R2015a).  The reported experimental results were
obtained on Intel(R) Core(TM) i5-3470 CPU @ 3.20GHz.

\paragraph{ Example 1.  } We consider a networked control system with
uncertain but bounded transmission period.  A networked control system
is composed of a plant and a controller that interact with each other
by transmission of feedback input from controller to the plant.  If
the system dynamics is linear with linear feedback, then for uncertain
but bounded transmission period, we can equivalently represent it
as a linear impulsive system where
%
\[A_c=\lt(\begin{array}{ccc} A_p & 0 & B_p\\ 0 & 0 &
  0\\ 0 & 0 & 0
\end{array}\rt),~A_r=\lt(\begin{array}{ccc}
\mb{I} & 0 & 0\\
B_oC_p & A_o & 0\\
D_oC_p & C_o & 0
\end{array}\rt)\]
%
for some parameter matrices $A_p$, $B_p$, $B_o$,
$C_p$, $A_o$, $C_o$, and $D_o$.  The sampling interval $\Delta$ of the
linear impulsive system specifies bounds on the transmission interval.
%
Our example of a networked control system is taken from Bj\"{o}rn et
al.~\cite{wittenmark2002computer}.  The system is originally described
by discrete time transfer functions, which has an equivalent state
space representation with parameter matrices
%
\begin{align*}
& A_p=\lt(\begin{array}{cc}-1 & 0\\ 1 & 0\end{array}\rt),
 ~ B_p=\lt(\begin{array}{c}1\\ 0\end{array}\rt),
~ C_p=\lt(\begin{array}{cc}0 & 1\end{array}\rt),\\& A_o=0.4286,~
      B_o=-0.8163,~ C_o=-1 D_o=-3.4286.
\end{align*}
%
Given the lower bound on the transmission period as $t_{min}=0.8$, we
want to find as high a value of $t_{max}$ as possible for which
the system is GES.

\begin{table}
\begin{minipage}{1\textwidth}
  \center
\begin{tabular}{|l|c|r|}
  \hline
  Reference & $t_{min}$ & $t_{max}$ \\
\hline
  %& &\\
Value recommended in~\cite{wittenmark2002computer} & 0.08 & 0.22\\
  \hline
  %& &\\
  NCS toolbox~\cite{BauLoo_NECSYS12a} & 0.08 & 0.4 \\
\hline
  %& &\\
  Template complex zonotope & 0.08 & 0.58 \\
  \hline
\end{tabular}
\caption{Experimental results: Example 1}
\label{tab:com1}
\end{minipage}
\end{table}

\begin{table}
\begin{minipage}{1\textwidth}
\label{tab:com2}
\center
\begin{tabular}{|l|c|r|}
\hline
    Reference & $t_{min}$ & $t_{max}$ \\
  \hline
  %& &\\
  Lyapunov, parametric LMI~\cite{2013hetel} & 0.1 & 0.3 \\
  \hline
  %& &\\
  Polytopic set contractiveness~\cite{2014-fiacchini-set} & 0.1 & 0.475 \\
  \hline
  %& &\\
Khatib et al.~\cite{AlKhatib2015} & 0.1 & 0.514\\
\hline
 % & &\\
  Template complex zonotope & 0.1 & 0.496 \\
  \hline
\end{tabular}
\caption{Experimental results: Example 2}
\label{tab:com2}
\end{minipage}
\end{table}

\paragraph{ Example 2.  } We consider the following linear impulsive system
from Hetel et. al.~\cite{2013hetel}, that describes an LMI based
approach to verify stability. The specification is given by
\[A_c=\left(\begin{array}{ccc} 0 & -3 & 1\\ 1.4 & -2.6 & 0.6\\ 8.4 &
  -18.6 & 4.6
\end{array}\rt),~
A_r=\lt(\begin{array}{ccc} 1 & 0 & 0\\ 0 & 1 & 0\\ 0 & 0 & 0
\end{array}\rt).
\]

\paragraph{Setting and Results.}  While implementing the algorithm for
stability verification, we used first order Taylor expansion, a
tolerance of $tol=0.01$ for Example 1 and $tol=0.006$ for Example 2.
We required $k=3$ number of reachability operators for both examples,
for synthesizing a suitable template complex zonotope used in checking
contraction.  We could verify exponential stability in a sampling
interval $[0.08, 0.58]$ for Example 1 and $[0.1,0.496]$ for Example 2.
For the first example, our method outperforms other approaches as
given in Table~\ref{tab:com1}.  For the second example, our method
larger bounds than the Lyapunov function approach~\cite{2013hetel} and
polytopic set contractiveness based
approach~\cite{2014-fiacchini-set}, as reported in
Table~\ref{tab:com2}.  Although our bound is smaller than the one
found by the approach of~\cite{AlKhatib2015}, still the difference is
only 0.018.
