A nearly periodic linear impulsive system is specified by a tuple
%
\[\system=\left(\reals^n,A_r,A_c,\Delta\right)\] where $\reals^n$ is the state
space with dimension $n\in\mathbb{Z}_+$, $A_r$ and $A_c$ are $n\times
n$ real matrices called the impulse matrix and the linear vector field
matrix, respectively.  The time interval $\Delta=[\tau_m,\tau_M]$ such
that $0<\tau_m<\tau_M$ is called the sampling period interval.  A
\emph{trajectory} of the nearly periodic linear impulsive system is
described by a function $\tbf{x}:\mathbb{R}_{\geq 0}\rightarrow
\reals^n$, such that there exists a sequence of sampling times
$(t_k)_{k=0}^\infty$ satisfying all the following.
%
\begin{equation}\label{eqn:system} \begin{split}
&{\trj{\dot{x}}{t}}=A_c\tbf{x}(t)~~\forall t\in [t_k,t_{k+1})~~\forall k\in\mb{Z}_{\geq 0}\\
&\trj{x}{t_k^+}=A_r\tbf{x}(t_k^-)~~\forall k\in\mb{Z}_{>0}\\ 
&t_{k+1}-t_k\in\Delta~~~\forall k\in\mb{Z}_{\geq 0}\\
&\tbf{x}(0)=\initstate
\end{split} \end{equation}
%
Here, $\tbf{x}(t)\in\mathbb{R}^n$ is the state of the system at a time
instant $t$.  For any $k\in\mathbb{N}$, we shall denote
$\tbf{x}_k=\tbf{x}(t_k^+)$.  When there is continuous evolution of a
state $x$ until time $t$, then the state reached at time $t$ is
$\trj{x}{t}=e^{A_ct}x$.  If the system experiences a reset at time
$t=0$, then the state reached immediately afterwards is
$A_r\initstate$.  Then the state reached at time $t$ after a reset at
time $t=0$ is
%
\[
\trj{x}{t}=e^{A_ct}A_r\initstate.
\]
%
We call the change in the state of the system from just before one
switch to another switch as \emph{a step}.  Without loss of
generality, let us assume that the first switch occurs at time
$t_0=0$.  Since the impulse period can range in the interval $\Delta$,
for any set $\Psi\subset \mathbb{R}^n$, the set of all reachable
points after the first step is
        %
        \[ R(\Psi)=\{e^{A_ct}A_r \initstate:
        \initstate \in \Psi, t\in\Delta\}.
        \]
        %
Then, the set of all points
        reachable after $k$ steps is
%
\begin{equation*}%\label{eqn:reach}
R^k(\Psi)=\left\{\left(\Pi_{i=1}^ke^{A_ct_i}A_r\right)\initstate: \initstate\in
\Psi, \forall i\in\{1,...,k\} t_i\in\Delta\right\}. 
\end{equation*}
%
Therefore, we define the {\em
$k$-step reachability operator} corresponding to a sequence of $k$ switching times
%        
\begin{align*}
& t_1, \ldots ,t_k:~\forall
  i\in\{1,\ldots,k-1\}:(t_{i+1}-t_{i})\in\Delta\\
& \text{ as } H_{t_1,...,t_k}=\Pi_{i=1}^k\lt(e^{A_ct_i}A_r\rt).
\end{align*}
%
A $k$-step reachability operator when acting on a state gives another
state that is reachable after $k$ steps.  Mathematically, we get
%
\[
R^k(\Psi)=\set{H_\tau\Psi:~\tau\in\Delta^k}.
\]
%
Since the the sampling period interval $\Delta$ is an uncountable set,
have a uncountable number of $k$-step reachability operators.  For a
sub-interval $[\tau_1,\tau_2]\subseteq\Delta$, we shall denote the
$k$-step reachability operators as
%
\begin{equation}\label{eqn:operators}
O_k([\tau_1,\tau_2])=\left\{H_\tau:\tau\in[\tau_1,\tau_2]^k\right\}.
\end{equation}



