A nearly periodic linear impulsive system is specified by a tuple
%
\[\system=\left(\reals^n,A_r,A_c,\Delta\right)\] where $\reals^n$ is the state
space with dimension $n\in\mathbb{Z}_{>0}$, $A_r$ and $A_c$ are
$n\times n$ real matrices called the impulse matrix and the linear
vector field matrix, respectively.  The interval
$\Delta=[\lsb,\usb]$, such
that $0<\tau_m<\tau_M$, is called the \emph{sampling interval}.  A
\emph{trajectory} of the system is
a function $\tbf{x}:\mathbb{R}_{\geq 0}\rightarrow
\reals^n$, such that there exists a sequence of sampling times
$(t_k)_{k=0}^\infty$ satisfying all the following for any $k\in\integers_{\geq 0}$.
%
\begin{align*}
&{\trj{\dot{x}}{t}}=A_c\tbf{x}(t)~~\forall t\in [t_k,t_{k+1})\\
&\trj{x}{t_k^+}=A_r\tbf{x}(t_k^-)~\numberthis\label{eqn:reset1}\\ 
&t_{k+1}-t_k\in\Delta\\
&\tbf{x}(0)=\initstate
\end{align*}
%
Here, $\tbf{x}(t)\in\mathbb{R}^n$ is the state of the system at a time
instant $t$.  For any $k\in\mathbb{N}$, we shall denote the state
reached just after the impulse at $t_k$ as $\tbf{x}_k=\tbf{x}(t_k^+)$.
When there is continuous evolution of a state $x$ until time $t$, then
the state reached at time $t$ is $\trj{x}{t}=e^{A_ct}x$.  If there is
an impulse at time $t=0$, then the state reached immediately
afterwards is $A_r\initstate$.  Therefore, using
Equation~\ref{eqn:reset1}, we get
%
\[
\tbf{x}_{k+1}{t}\in \set{e^{A_ct}A_r\tbf{x}_k:~t\in\Delta}.
\]
%
We call the change in the state of the system from the time just
before one impulse to the next impulse as \emph{a step}.  Therefore,
for any set $\Psi\subset \mathbb{R}^n$, the set of all reachable
points from a state $x$ in one step is 
        %
        \[ R(\Psi)=\{e^{A_ct}A_r \initstate:
        \initstate \in \Psi, t\in\Delta\}.
        \]
        %
Then, the set of all points
        reachable after $k$ steps is
%
\begin{equation*}~x\label{eqn:reach}
R^k(\Psi)=\left\{\left(\Pi_{i=1}^ke^{A_ct_i}A_r\right)\initstate: \initstate\in
\Psi, \forall i\in\{1,...,k\} t_i\in\Delta\right\}. 
\end{equation*}
%
Therefore, we define a {\em
reachability operator} as follows.
%        
\begin{align*}
& \forall t\in\Delta,
 \text{  } H_t=e^{A_ct}A_r.
\end{align*}
%
As the the sampling period interval $\Delta$ is an uncountable set,
there are uncountable number of reachability operators.  For a
sub-interval $[\tau_1,\tau_2]\subseteq\Delta$, we shall denote the set of
reachability operators as
%
\begin{equation}\label{eqn:operators}
O([\tau_1,\tau_2])=\left\{H_\tau:\tau\in[\tau_1,\tau_2]\right\}.
\end{equation}



