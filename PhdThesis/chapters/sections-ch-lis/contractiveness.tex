%
For any $\rho:0\leq \rho\leq \epsilon$, we want to derive a bound on the
contraction of the operator $H_{t+\rho}$ as a function of $H_t$ and
$\epsilon$.  Using Taylor expansion of an order $r$, we
can write
\begin{align*}
&
  H_{t+\rho}=e^{(A_c\rho)}H_t=P_r(\rho)H_t+E_r(\delta)H_t~~\text{where}\\
&  P_r(\rho)=\sum_{i=0}^r\frac{A_c^i\rho^i}{i!},~~~~~~
 E_r(\delta)=\frac{A_c^{r+1}\delta^{r+1}}{(r+1)!}:~~\delta\in[0,\epsilon].~\numberthis\label{eqn:taylor}
\end{align*}
%
To use the above expansion for deriving the bound on contraction, we
shall describe some of its properties.  The following
lemma states that the contraction upon transformation
by the product of any two matrices is bounded by the product of the
contractions by individual matrices.  Also, the contraction
upon transformation by the sum of two matrices is bounded by the sum
of the contractions by the individual matrices.
%
\begin{lem}\label{lem:composition}
Let us consider $J_1,J_2\in\mat{n}{n}{\reals^n}$ and
  $\Psi\subset\reals^n$.  Then the all of the following is true.
\begin{enumerate}
\item $\chi(\Psi,J_1+J_2)\leq\chi(\Psi,J_1)+\chi(\Psi,J_2)$.
\item $\chi(\Psi,J_1J_2)\leq\chi(\Psi,J_1)\chi(\Psi,J_2)$.
\end{enumerate}
\end{lem}
%
\begin{proof}
  For proving the first part, we derive the following.
  %
  \begin{align*}
&  \text{Since}\hspace{1em}   J_1\Psi\subseteq
    \chi(\Psi,J_1)\Psi~\text{ and }
    J_2\Psi\subseteq \chi(\Psi,J_2),~\text{we get}\\
&  (J_1+J_2)\Psi\subseteq
  \chi(\Psi,J_1)\Psi \oplus \chi(\Psi,J_2)\Psi=
  (\chi(\Psi,J_1) + \chi(\Psi,J_2))\Psi.
  \end{align*}
  %
For proving the second part, we derive the following.
\begin{align*}
  &  J_1J_2\Psi=J_1(J_2\Psi)\subseteq J_1(\chi(\Psi,J_2)\Psi)\\
 &=\chi(\Psi,J_2)\lt(J_1\Psi\rt)\subseteq
  \chi(\Psi,J_1)\chi(\Psi,J_2)\Psi.\hspace{3em}\qedhere
\end{align*}
%
\end{proof}
%
If a matrix is embedded inside the convex hull of a set of matrices,
then we get the following bound on contraction by the matrix.
%
\begin{lemma}~\label{lem:conv-bound}
Let us consider that $J\in\convexhull{\set{A_1,...,A_r}}$ and
$\Psi\subset\reals^n$.  Then
%
\begin{align*}
& \contraction{J}{\Psi}\leq\max_{i=1}^r\contraction{A_i}{\Psi}.
\end{align*}
%
\end{lemma}
%
\begin{proof}
As $J\in\convexhull{\set{A_1,...,A_r}}$, there exists
$\alpha_1,...,\alpha_r\in\reals_{\geq 0}$ such that
$\sum_{i=1}^r\alpha_i=1$ and $J=\sum_{i=1}^r\alpha_iA_i$.  Then by
using Lemma~\ref{lem:composition}, we get
%
\begin{align*}
  & \contraction{J}{\Psi}\leq\sum_{i=1}^r\contraction{A_i}{\Psi}\\
  & \leq \lt(\sum_{i=1}^r\alpha_i\rt)\max_{i=1}^r\contraction{A_i}{\Psi}=\max_{i=1}^r\contraction{A_i}{\Psi}.\hspace{3em}\qedhere
\end{align*}
%
\end{proof}
%
If we want to bound the contraction of a polynomial with matrix
co-coefficients, where the variable has a bound, then the following
lemma is useful.  The set of all possible values of
the polynomial can be embedded inside a convex hull of a finite set of
matrices, as described below.
%
\begin{lem}~\label{lem:convex}~\cite{2013hetel}
Let us consider $\set{A_0,A_1,...,A_r}\subseteq\mat{n}{n}\reals$ where
%
\[
\forall j\in\{0,...,r\},~ U_j(\rho)=\sum_{i=0}^jA_i\rho^i.
\]
%
If
$0\leq\rho\leq \epsilon$, then $U_r(\rho)\in
\convexhull{U_0(\epsilon),U_1(\epsilon),...,U_r(\epsilon)}$.
\end{lem}
\begin{proof}
This has been proved in~\cite{2013hetel}.
\end{proof}

Using the above results, we derive the following bound on contraction
by the operator $H_{t+\rho}$, when $\rho\in[0,\epsilon]$.
%
\begin{lemma}\label{lem:conv}
Let us consider $\Psi\subset\reals^n$ and $\rho\in[0,\epsilon]$.  If
$0\leq \rho\leq \epsilon$, then
%
\begin{align*}
& \contraction{H_{t+\rho}}{\Psi}\leq\max_{i=1}^r\contraction{P_r(\epsilon)H_t}{\Psi}+\contraction{\frac{A_c^{r+1}}{(r+1)!}H_t}{\Psi}\epsilon^{r+1}.
\end{align*}
%
\end{lemma}
%
\begin{proof}
Using Equation~\ref{eqn:taylor}, there exists
$\delta\in[0,\epsilon]$ such that,
%
\begin{align*}
&
  \contraction{H_{t+\rho}}{\Psi}=\contraction{P_r(\rho)+E_r(\delta)}{\Psi}\\
& \%\%~~\text{by Lemma~\ref{lem:composition}}\\
&
  \leq\contraction{P_r(\rho)H_t}{\Psi}+\contraction{\frac{A_c^{r+1}}{(r+1)!}H_t}{\Psi}\delta^{r+1}\\
  & \%\%~~\text{by Lemmas~\ref{lem:conv-bound} and~\ref{lem:convex}}\\
&
  \leq\max_{i=1}^r\contraction{P_r(\epsilon)H_t}{\Psi}+\contraction{\frac{A_c^{r+1}}{(r+1)!}H_t}{\Psi}\delta^{r+1}\\
& \leq\max_{i=1}^r\contraction{P_r(\epsilon)H_t}{\Psi}+\contraction{\frac{A_c^{r+1}}{(r+1)!}H_t}{\Psi}\epsilon^{r+1}.\hspace{3em}\qedhere
\end{align*}
%
\end{proof}
%

