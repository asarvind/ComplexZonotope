A nearly periodic linear impulsive system is globally exponentially
stable (GES) if any point in the state space reaches arbitrarily close
to the origin at an exponential rate.  This property is mathematically
stated as follows.
%
\begin{defn}\label{defn:exp-stable}[Global exponential stability (GES)] The
system $\system$ is globally exponentially stable
(GES) if there exists $\lambda\in[0,1)$ and $c>0$ such that for all $\initstate
\in\mb{R}^n$ and $k\in\mb{Z}_+$,
$||\tbf{x}_k||\leq c\lambda^k||\tbf{x}_0||$.
\end{defn}
%
The following is the stability verification problem.
%
\begin{problem}
Given a sampling period interval $\Delta=[\lsb,\usb]$, verify
that the system $H$ is globally exponentially stable.
\end{problem}
%
The GES of a system is related to the reachable sets of the
system. Indeed if all bounded sets containing the origin eventually
contract to arbitrarily small sets around the origin, then every point
eventually reaches close to the origin and the system is thus GES.
Instead of verifying the contraction of all bounded sets containing
the origin, we can verify the contraction of any compact and convex
set containing the origin, because it can be scaled to include any
bounded set. We call such sets as $C$-sets, defined as follows.
%
\begin{definition}
A set $\Psi\subset\reals^n$ is called a $C$-set if it is compact,
convex and contains the origin in its interior.
\end{definition}
%

The contraction of a $C$-set is defined as follows.
%
\begin{definition}[\cite{2014-fiacchini-set}]
Given $\lambda\in[0,1]$, a set $C$-set $\Psi\subset\reals^n$ is
$\lambda$-contractive for the system $\system$ iff
\[\forall x\in\Psi,~\forall t\in\Delta,~H_tx\in\lambda\Psi.\]
\end{definition}
%
\begin{remark} It has been shown previously
in~(\cite{2014-fiacchini-set,athanasopoulos2014alternative,AlKhatib2015})
that globally exponentially stability of a nearly-periodic linear
impulsive system is equivalent to the existence of a
$\lambda$-contractive $C$-set for a $\lambda\in[0,1)$.  So, we can
  find a $\lambda$-contractive $C$-set for $\lambda<1$ to prove  global
  exponential stability.
\end{remark}
%
  A stability verification algorithm was proposed
  in~\cite{2014-fiacchini-set} that definitely computes a contractive
  $C$-set for a globally exponentially stable system.  But the
  algorithm involves iterative intersections.  But during iterative
  intersections, the complexity of representing the set can grow
  uncontrollably.  So, the
  algorithm~\cite{2014-fiacchini-set} can be costly, especially in
  higher dimensions.  Alternatively, we propose a stability
  verification algorithm using complex zonotope and convex
  optimization, where the size of the template is fixed a priori.
  Although the existence of a contractive complex zonotope is only a
  sufficient condition for global exponential stability, we demonstrate
  the efficiency of our procedure by experiments on some benchmark
  examples.  Our algorithm, described in the next section, uses some
  properties of contraction of sets, which we shall discuss now.

Since the state change of the system can be identified by the
transformation by a reachability operator, we define contraction by
a matrix as follows.  From here on, we denote $J$ as an
$n\times n$ real matrix.
%
\begin{defn} The amount of contraction of a set
  $\Psi\subset\mathbb{R}^n$ upon transformation by the matrix $J$,
  denoted as $\chi(\Psi,J)$, is $$\chi(\Psi,J) =
  \inf\{a\in\mathbb{R}_{\geq 0} : J(\Psi)\subseteq a
  \Psi\}.$$ \end{defn}
% \begin{defn} The amount of contraction of a set $\Psi\subset\mathbb{R}^n$
% after $k$ steps for some $k\in\mathbb{Z}^+$, denoted by  $\chi_k(\Psi)$, is
% $$\chi_k(\Psi)=\inf\{a\in\mathbb{R}_{\geq 0} : R^k(\Psi)\subseteq a\Psi\}.$$  
% %We shall denote the contraction of $\Psi$ after $k$ steps as $\chi_k(\Psi)$.
% \end{defn}
The amount of contraction being greater than one would indicate that
the set is actually \emph{expanding}, which will also be referred
mathematically as ``contraction'' parameter, in a general sense.
%
%
For any $\rho:0\leq \rho\leq \epsilon$, we want to derive a bound on the
contraction of the operator $H_{t+\rho}$ as a function of $H_t$ and
$\epsilon$.  Using Taylor expansion of an order $r$, we
can write
\begin{align*}
&
  H_{t+\rho}=e^{(A_c\rho)}H_t=P_r(\rho)H_t+E_r(\delta)H_t~~\text{where}\\
&  P_r(\rho)=\sum_{i=0}^r\frac{A_c^i\rho^i}{i!},~~~~~~
 E_r(\delta)=\frac{A_c^{r+1}\delta^{r+1}}{(r+1)!}:~~\delta\in[0,\epsilon].~\numberthis\label{eqn:taylor}
\end{align*}
%
To use the above expansion for deriving the bound on contraction, we
shall describe some of its properties.  The following
lemma states that the contraction upon transformation
by the product of any two matrices is bounded by the product of the
contractions by individual matrices.  Also, the contraction
upon transformation by the sum of two matrices is bounded by the sum
of the contractions by the individual matrices.
%
\begin{lem}\label{lem:composition}
Let us consider $J_1,J_2\in\mat{n}{n}{\reals^n}$ and
  $\Psi\subset\reals^n$.  Then the all of the following is true.
\begin{enumerate}
\item $\chi(\Psi,J_1+J_2)\leq\chi(\Psi,J_1)+\chi(\Psi,J_2)$.
\item $\chi(\Psi,J_1J_2)\leq\chi(\Psi,J_1)\chi(\Psi,J_2)$.
\end{enumerate}
\end{lem}
%
\begin{proof}
  For proving the first part, we derive the following.
  %
  \begin{align*}
&  \text{Since}\hspace{1em}   J_1\Psi\subseteq
    \chi(\Psi,J_1)\Psi~\text{ and }
    J_2\Psi\subseteq \chi(\Psi,J_2),~\text{we get}\\
&  (J_1+J_2)\Psi\subseteq
  \chi(\Psi,J_1)\Psi \oplus \chi(\Psi,J_2)\Psi=
  (\chi(\Psi,J_1) + \chi(\Psi,J_2))\Psi.
  \end{align*}
  %
For proving the second part, we derive the following.
\begin{align*}
  &  J_1J_2\Psi=J_1(J_2\Psi)\subseteq J_1(\chi(\Psi,J_2)\Psi)\\
 &=\chi(\Psi,J_2)\lt(J_1\Psi\rt)\subseteq
  \chi(\Psi,J_1)\chi(\Psi,J_2)\Psi.\hspace{3em}\qedhere
\end{align*}
%
\end{proof}
%
If a matrix is embedded inside the convex hull of a set of matrices,
then we get the following bound on contraction by the matrix.
%
\begin{lemma}~\label{lem:conv-bound}
Let us consider that $J\in\convexhull{\set{A_1,...,A_r}}$ and
$\Psi\subset\reals^n$.  Then
%
\begin{align*}
& \contraction{J}{\Psi}\leq\max_{i=1}^r\contraction{A_i}{\Psi}.
\end{align*}
%
\end{lemma}
%
\begin{proof}
As $J\in\convexhull{\set{A_1,...,A_r}}$, there exists
$\alpha_1,...,\alpha_r\in\reals_{\geq 0}$ such that
$\sum_{i=1}^r\alpha_i=1$ and $J=\sum_{i=1}^r\alpha_iA_i$.  Then by
using Lemma~\ref{lem:composition}, we get
%
\begin{align*}
  & \contraction{J}{\Psi}\leq\sum_{i=1}^r\alpha_i\contraction{A_i}{\Psi}\\
  & \leq \lt(\sum_{i=1}^r\alpha_i\rt)\max_{i=1}^r\contraction{A_i}{\Psi}=\max_{i=1}^r\contraction{A_i}{\Psi}.\hspace{3em}\qedhere
\end{align*}
%
\end{proof}
%
If we want to bound the contraction of a polynomial with matrix
co-coefficients, where the variable has a bound, then the following
lemma is useful.  The set of all possible values of
the polynomial can be embedded inside a convex hull of a finite set of
matrices, as described below.
%
\begin{lem}~\label{lem:convex}~\cite{2013hetel}
Let us consider $\set{A_0,A_1,...,A_r}\subseteq\mat{n}{n}\reals$ where
%
\[
\forall j\in\{0,...,r\},~ U_j(\rho)=\sum_{i=0}^jA_i\rho^i.
\]
%
If
$0\leq\rho\leq \epsilon$, then $U_r(\rho)\in
\convexhull{U_0(\epsilon),U_1(\epsilon),...,U_r(\epsilon)}$.
\end{lem}
\begin{proof}
This has been proved in~\cite{2013hetel}.
\end{proof}

Using the above results, we derive the following bound on contraction
by the operator $H_{t+\rho}$, when $\rho\in[0,\epsilon]$.
%
\begin{lemma}\label{lem:conv}
Let us consider $\Psi\subset\reals^n$ and $\rho\in[0,\epsilon]$.  If
$0\leq \rho\leq \epsilon$, then
%
\begin{align*}
& \contraction{H_{t+\rho}}{\Psi}\leq\max_{i=1}^r\contraction{P_r(\epsilon)H_t}{\Psi}+\contraction{\frac{A_c^{r+1}}{(r+1)!}H_t}{\Psi}\epsilon^{r+1}.
\end{align*}
%
\end{lemma}
%
\begin{proof}
Using Equation~\ref{eqn:taylor}, there exists
$\delta\in[0,\epsilon]$ such that,
%
\begin{align*}
&
  \contraction{H_{t+\rho}}{\Psi}=\contraction{P_r(\rho)+E_r(\delta)}{\Psi}\\
& \%\%~~\text{by Lemma~\ref{lem:composition}}\\
&
  \leq\contraction{P_r(\rho)H_t}{\Psi}+\contraction{\frac{A_c^{r+1}}{(r+1)!}H_t}{\Psi}\delta^{r+1}\\
  & \%\%~~\text{by Lemmas~\ref{lem:conv-bound} and~\ref{lem:convex}}\\
&
  \leq\max_{i=1}^r\contraction{P_r(\epsilon)H_t}{\Psi}+\contraction{\frac{A_c^{r+1}}{(r+1)!}H_t}{\Psi}\delta^{r+1}\\
& \leq\max_{i=1}^r\contraction{P_r(\epsilon)H_t}{\Psi}+\contraction{\frac{A_c^{r+1}}{(r+1)!}H_t}{\Psi}\epsilon^{r+1}.\hspace{3em}\qedhere
\end{align*}
%
\end{proof}
%



