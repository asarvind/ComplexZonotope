We verify global exponential stability by finding a contractive
complex zonotope that contains the origin in its interior.  A
motivation for considering complex zonotope for stability verification
is that they can capture the contraction along the eigenvectors, which
was earlier explained in Lemma~\ref{lem:eig-scaling}.  Accordingly, we
find a complex zonotope by sampling the eigenvectors of some of the
reachability operators, synthesizing suitable scaling factors and
latter verifying that the complex zonotope contracts.  Our algorithm
for stability verification has two stages.  In the first stage, we
synthesize a complex zonotope that contracts with respect to a few
sampled operators.  In the next stage, we verify that the synthesized
complex zonotope contracts with respect to all the sampled operators.
We fix the template a priori by a set of eigenvectors of selected
reachability operators, because of which we have control over the
representation size of the contractive set.

{\bf Synthesizing a candidate template complex zonotope.}  
Let us sample uniform $k$ time points in the interval $\Delta$ as
%
\begin{align*}
  & \omega^k_i=\lsb+i\frac{(\usb-\lsb)}{k}:~i\in\set{0,...,k-1}\\
  & \Lambda_k=\set{\omega^k_i:i\in\set{1,\ldots,k}}.
\end{align*}
%
Let us denote $\Theta_i^k$ as a matrix with $n$ columns consisting of
all possible unit eigenvectors of $H_{\omega_i^k}$ with possible
repetition.  Let us consider $\eigenvectors_k$ as the matrix
containing all the eigenvectors of $k$ uniformly sampled operators,
i.e.,
%
\[
\eigenvectors_k=\mymatrix{\Theta^k_1 &\ldots&\Theta^k_k}.
\]
%
We fix $\eigenvectors_k$ as the template of the complex zonotope and
synthesize suitable scaling factors based on the following theorem.
The theorem uses the inclusion checking condition from
Theorem~\ref{thm:suff-inclusion}.
%
\begin{thm}
For a vector of scaling factors $\sfact\in\mb{R}^{m}_{\geq 0}$, the
template complex zonotope $\tcz{\eigenvectors_k}{0}{s}$ is
$\lambda$-contractive with respect to all $H_t:~ t\in\Lambda_k$ and
represent a $C$-set containing the origin in its interior,
if all of following is true.
%
\begin{align*}
  & \tcztope{\identity{n}{n}}{0}{\repmat{1}{n}{1}}\order\tcztope{\eigenvectors_k}{0}{\sfact}~\numberthis\label{eqn:cset}\\
  &\forall t\in\Lambda_k,~ \tcztope{H_t\eigenvectors_k}{0}{\sfact}\order\tcztope{\eigenvectors}{0}{\lambda\sfact}.~\numberthis\label{eqn:contractive}
\end{align*}
\end{thm}
%
\begin{proof}
A complex zonotope is a compact and convex set.  Furthermore, the real
projection of $\tcztope{\identity{n}{n}}{0}{\repmat{1}{n}{1}}$ is the
hypercube containing $0$.  So, by Theorem~\ref{thm:suff-inclusion} and
Equation~\ref{eqn:cset}, $\tcztope{\eigenvectors_k}{0}{\sfact}$
contains the origin in its interior. Therefore,
$\tcztope{\eigenvectors_k}{0}{\sfact}$ is a $C$-set.  By
Lemma~\ref{lem:lin-transform}, Theorem~\ref{thm:suff-inclusion} and
Equations~\ref{eqn:normalization} and~\ref{eqn:contractive}, we get
$H_t\tcztope{\eigenvectors_k}{0}{\sfact}\subseteq \lambda\tcztope{\eigenvectors_k}{0}{\sfact}$.
\end{proof}
%

{\bf Verifying contraction:} To verify that the template complex
zonotope synthesized in the first stage contracts with respect to all
the reachability operators $H_t:~t\in\Delta$, we divide the sampling
interval into small enough sub-intervals and verify contraction in
each interval.  For this, we need to find a bound on contraction of a
complex zonotope with respect to any reachability operator.  When the
synthesized complex zonotope contains the eigenvectors of a
reachability operator, then the contraction by the operator is bounded
by the maximum magnitude of the eigenvalues of the operator.  This is
described in the following lemma.
%
\begin{lemma}
Let us consider that
$H_{\omega_i^k}\Theta_i^k=\Theta^k_i\diagonal{\mu}:~\mu\in\compnums^n$.
Let $\sfact\in\reals_{\geq 0}^n$.  Then,
%
\[
\contraction{H_{\omega^k_i}}{\tcztope{\eigenvectors}{0}{\sfact}}=\max_{i=1}^n\absolute{\mu_i}.
\]
%
\end{lemma}
%
\begin{proof}
  By using Lemma~\ref{lem:eig-scaling}, we get
%
\begin{align*}
&
 H_{\omega^k_i}\tcztope{\Theta^k_i}{0}{\sfact} =
 \tcztope{\Theta^k_i}{0}{\diagonal{\absolute{\mu}}{\sfact}}.\\
& \therefore  \contraction{H_{\omega^k_i}}{\tcztope{\Theta^k_i}{0}{\sfact}}\leq\max_{i=1}^n\absolute{\mu_i}.~\hspace{3em}\qedhere.
\end{align*}
\end{proof}
%
However, when the eigenvectors of multiple reachability operators are
sampled to form the template, the above lemma can not be used to
compute the contraction bound.  In the latter case, we compute the
contraction bound using convex optimization, as follows.  Let us
define
\[\beta_k(\sfact,J)=\min\{\|\tmat\|_\infty:\tmat\in\mat{m}{m}{\compnums}~\wedge~
\eigenvectors_k\diagonal{\sfact}X=J\eigenvectors_k\diagonal{\sfact}\}\] 
%
\begin{lemma}\label{lem:contLin}
We get $\chi\lt(J,\tcztope{\eigenvectors_k}{0}{\sfact}\rt)\leq \beta_k(\sfact,J)$.
%
\end{lemma}
%
\begin{proof}
Let us consider
$J\eigenvectors_k\diagonal{\sfact}=\eigenvectors_k\diagonal{\sfact}X$.
We have to prove that
\[
J\tcz{\eigenvectors_k}{0}{s}=\tcztope{J\eigenvectors_k}{0}{\sfact}\subseteq
\|\tmat\|_\infty\tcztope{\eigenvectors_k}{0}{\sfact}.
\]
We derive the following.  By using Lemma~\ref{lem:normalization}, we get
%
\begin{align*}
&  \tcztope{J\eigenvectors_k}{0}{s}=\cztope{J\eigenvectors_k\diagonal{\sfact}}{0}\\
&=
  \{J\eigenvectors_k\diagonal{\sfact}\zeta^\pr:\|\zeta^\pr\|_{\infty} \leq 1\}\\
 & =
  \{J\diagonal{\sfact}\tmat\zeta^\pr:\|\zeta^\pr\|_\infty\leq 1\}\\
 & \subseteq
  \|\tmat\|_\infty\{\eigenvectors_k\diagonal{\sfact}\zeta^\pr:\|\zeta^\pr\|_{\infty}\leq 1\}\\
 & =\|\tmat\|_\infty\cztope{\eigenvectors_k\diagonal{\sfact}}{0}
=
\|X\|_{\infty}\tcztope{\eigenvectors_k}{0}{\sfact}.\hspace{3em}\qedhere
\end{align*}
\end{proof}
%
In our verification procedure, we verify contraction in the
neighborhoods of a finite set of time points in the sampling interval,
such that the union of neighborhoods contains the time interval.
Therefore, we derive a bound on the contraction in a
neighborhood of a time point, as follows.  For any $t\in\Delta$,
$r\in\integers_{\geq 0}$ and
$\epsilon\geq 0$, let us denote
%
\[
\cont{k}{r}{\sfact,t,\epsilon}=\max_{i=1}^k\beta_k\lt(\sfact,P_i\lt(\epsilon\rt)H_t\rt)+\beta_k\lt(\frac{A_c^{r+1}}{(r+1)!}H_t\rt)\epsilon^{r+1}.
\]
%
\begin{theorem}~\label{thm:main-contraction}
  The following is true.
  %
  \begin{align*}
& \max_{\rho\in[t,t+\epsilon]}\contraction{H_\rho}{\tcztope{\eigenvectors_k}{0}{\sfact}}\leq\cont{k}{r}{\sfact,t,\epsilon}.
  \end{align*}
  %
\end{theorem}
%
\begin{proof}
Let us denote $\Psi=\tcztope{\eigenvectors_k}{0}{\sfact}$.  Using
Lemma~\ref{lem:conv}, for any
$\rho\in[t,t+\epsilon]$, we derive the following.
  %
  \begin{align*}
&
    \contraction{H_\rho}{\Psi}\leq\max_{i=1}^r\contraction{P_r(\epsilon)H_t}{\Psi}+\contraction{\frac{A_c^{r+1}}{(r+1)!}H_t}{\Psi}\epsilon^{r+1}\\
    & \%\%~~\text{by Lemma~\ref{lem:contLin}}\\
& \leq \max_{i=1}^k\beta_k\lt(\sfact,P_i\lt(\epsilon\rt)H_t\rt)+\beta_k\lt(\frac{A_c^{r+1}}{(r+1)!}H_t\rt)\epsilon^{r+1}.\hspace{3em}\qedhere  
  \end{align*}
  %
\end{proof}
%
 \begin{algorithm}
\caption{Exponential stability verification of $\system$}
\label{alg:1step}
\begin{algorithmic}[1]
\State Initialize $k=3$.
\State Choose an $M\in\integers_{>3}$ as the bound of $k$ and a $tol>0$ as
the discretization parameter.
\State Find a vector of scaling factors $\sfact$ solving Equations~\ref{eqn:cset} and~\ref{eqn:contractive}.
\State Initialize $t=\lsb$, $h=tol$, $r:=$ order of Taylor
expansion (typically $\leq 2$).
\While{$k \leq M$ and $t<\usb$ and $h\geq \; tol$}
\If{$\cont{k}{r}{\sfact,t,h}<1$}
\State $t\gets t+h; h\gets h+tol$
\Else
\State $h\gets h-tol$
\EndIf
\If{$h<tol$}
\State $k\gets k+1$.  
\EndIf
\EndWhile
\If {$t \geq \usb$}
\State \emph{System is exponentially stable}
\Else
\State \emph{Inconclusive}
\EndIf
\end{algorithmic}
 \end{algorithm}
%
{\bf Verification algorithm.} We begin with $k=3$ reference
 operators that correspond to the two end points of the sampling
 interval and the middle point.  The algorithm first finds suitable
 scaling factors $\sfact$ such that the template complex zonotope with
 $\eigenvectors_k$ as the template operators contracts with respect to
 $\omega_i^k\forall i\in\set{1,\ldots,k}$.  Then we check whether the
 contraction of the template complex zonotope by all the reachability
 operators is less than one. For checking contraction, we first
 discretize the sampling interval by a grid of size $tol>0$ and then
 compute the bound on contraction in the forward $tol$ neighborhood of
 each gird point using Theorem~\ref{thm:main-contraction}.  We choose
 an order of Taylor expansion, typically not greater than two.
 Starting with $t=\lsb$ and $h=tol$, for any $t$ and $h$, we check
 that $\cont{k}{r}{\sfact,t,h}<1$.  If successfully, we increment the
 value of $t$ to $t+h$ and $h$ to $h+tol$.  If not successful, we
 reduce the value of $h$ to $h-tol$.  If we have verified contraction
 until a point $t>\usb$, then we have successfully verified global
 exponential stability of the system.  Otherwise, if $h$ becomes less
 than zero at some point, we start another loop by increasing $k$ to
 $k+1$ and repeat the procedure.  We hope to verify contraction within
 a reasonable value of $k$.  Otherwise, we terminate without any
 conclusion.  The steps of the procedure are systematically described
 in Algorithm~\ref{alg:1step}.  We shall discuss the experimental results
 based on this algorithm in the next section.
%

