

    Since computers work with digital signals and the physical system
    they control operates in the analog world, sampling is required.
    Various parameters related to sampling of the system can be
    subject to uncertainty like the sampling period, delay, digital
    output of the controller, output feedback from the system, etc.
    These uncertainties can lead to instability of the system.
    Henceforth, we are faced with the challenge of verifying system
    stability in the presence of these uncertainties.  We address this
    issue by considering the problem of verifying stability of
    nearly-periodic impulsive systems, which can be used to model
    sampled-data systems \cite{naghshtabrizi2007delay} and networked
    control systems \cite{2008-naghshtabrizi-exponential}.  This
    problem has been tackled using control approaches, which mainly
    involve deriving stability conditions in terms of Lyapunov
    functions and checking these conditions using optimization (Linear
    Matrix Inequalities (LMI) or Sum of Squares (SOS)). In this work,
    we use a stability condition based on set contractiveness proposed
    in
    \cite{athanasopoulos2014alternative,2014-fiacchini-set,AlKhatib2015}
    and propose a new method for checking it using computational
    techniques, inspired by hybrid systems verification
    techniques. Globally exponential stability (GES) of
    nearly-periodic linear impulsive systems can be proved by showing
    the contractiveness of a compact and convex set containing the
    origin in its interior, also called as a $C$-set
    \cite{Lazar2013,athanasopoulos2014alternative,2014-fiacchini-set,AlKhatib2015}.
    The uncertainty in impulse times for which stability can be proved
    depends on the choice of the contractive $C$-set.  A
    nearly-periodic linear impulsive system is stable only if all of
    its reachability operators, which compute the state reached after
    an impulse, are stable.  This motivates us to use complex
    zonotopes since we can find good candidate complex zonotope for
    contractive sets using the eigenvectors of the reachability
    operators.  For proving stability condition, we consider complex
    zonotopes whose generator sets are chosen among the eigenvectors
    of reachability operators of the nearly-periodic linear impulsive
    system.  We then derive a condition for the contractiveness of a
    complex zonotope that can be verified by convex optimization.
    Concerning experimental results, our approach is either
    competitive or better compared to the existing approaches, in
    terms of the largeness of uncertainty of sampling periods for
    which stability could be proved.

    The chapter is organized into five main sections.  We discuss the
    related work in Section~\ref{lis-rel}.  In
    Section~\ref{sec:dynamics}, we describe the dynamics of a
    nearly-periodic linear impulsive system.  In
    Section~\ref{sec:ges}, we explain global exponential stability,
    the verification problem and its relation to finding a contractive
    $C$-set of the system.  In Section~\ref{sec:alg}, we explain our
    algorithm for stability verification based on complex zonotopes.
    The experiments on two benchmark examples are discussed in Section~\ref{sec:experiments}



\section{Related work}~\label{lis-rel}
A major approach to stability analysis of aperiodic sampling control uses time-delay 
systems, and stability can be proved using Lyapunov Krakovskii functional 
\cite{Mikheev1988,Teel1998,2010-liu-stability,Mazenc2013}, 
or time-dependent Lyapunov functional \cite{2010-fridman-refined}. Using a continuous-time model,  
discrete-time Lyapunov functions is proposed for stability condition \cite{2012-seuret-novel}, 
which can be checked using Sum of Squares (SOS) \cite{2013-seuret-stability}. Robust stability with respect to 
time-varying input delay can also be handled by input/output approach 
\cite{Mirkin2007,DBLP:journals/automatica/Fujioka09,DBLP:conf/amcc/KaoW14,Omran2013,Omran2014}.
Another important approach is based on the hybrid systems modeling framework, in particular time-varying impulsive systems 
\cite{Hu2003,nevsic2004framework,Goebel2009,Cai2008,BauLoo_NECSYS12a} and employs Lyapunov-based 
methods in various forms including discontinuous time-independent \cite{2008-naghshtabrizi-exponential} 
or time-dependent Lyapunov functions \cite{2010-fridman-refined}. Another popular 
approach involves using convex embedding \cite{HetelDaafouz2006,Fujioka2009,HetelKruszewski2011,2013hetel,Omran2014}. 
In this approach, stability tests can be formulated as parametric Linear Matrix Inequalities (LMIs) \cite{HetelDaafouz2006}, 
or as set contractiveness (such as, polytopic set contractiveness is equivalent to polyhedral Lyapunov 
functions) \cite{2014-fiacchini-set,2013-briat-convex,Lazar2013,athanasopoulos2014alternative,AlKhatib2015}. 
In this work, we focus on exponential stability and are inspired by
set theory conditions
\cite{2014-fiacchini-set,athanasopoulos2014alternative,AlKhatib2015} to derive a stability condition which is more conservative 
but can be efficiently verified. The novelty of our work lies in the use of complex zonotopes to efficiently 
find contractive sets. Computationally speaking, our approach is close in spirit to abstract interpretation
and hybrid systems analysis. Indeed the way we find contractive sets using such zonotopes is similar 
to the way invariant sets are computed using some zonotope
\cite{Girard05reachabilityof,Althoff2011,DBLP:conf/sas/GoubaultPV12} and
template-polyhedral abstract domains \cite{Sriram2008,jeannet2009apron}.



\section{Dynamics}~\label{sec:dynamics}
In a discrete time affine hybrid system, the state of the system is
specified by a discrete valued variable, called location, and a
continuous variable whose valuation is in the real Euclidean space of
a finite dimenstion.  The state of the system in each location has to
stay within a polyhedral set, called the staying condition.  The state
of the system can change by two kinds of transitions, {\it continuous
  transition} and {\it discrete transiton}.  In a continuous
transition, the discrete state of the system remains constant while
the continuous state changes by an affine transformation.  The affine
transformation has possible additive disturbance input, which is
bounded.  The parameters of the affine transformation of a continuous
transition depend on the location in which the transition takes place.
In a discrete transition, there is a change in the discrete variable
accompanied by an affine transformation of the continuous variable.
The transition is has precondition specified by a linear constraint,
called a guard, while the post-condition is the staying condition in
the location reached after transition.  A set of edges specifies
the possible discrete transitions, vis a vis, the locations between
which a discrete transition takes place, the parameters of the affine
transformation and the guard.   

{\it Sub-parallelotopic guards and staying conditions}: In this paper,
we consider hybrid systems where the guards and staying conditions can
be specified by a sub-parallelotope with a common template.  We note
that the class of sub-parallelotopic constraints are quite general and
can be used in the specification of many practical affine hybrid
systems.  

{\bf Model.}  We specify the discrete time affine hybrid system
by a tuple 
 %
\[
\system = \lt(\locations,\qtemp,\stay,\linmap,\inputset,\edgeset,\Psi\rt).
\]
%
The finite set of locations is $\locations$.  The sub-parallelotopic
template for specifying the guards and staying conditions is
$\qtemp\in\mat{k}{n}{\reals}$.  The staying set in a location
$\loc\in\locations$ is a sub-parallelotope
$\ptope{\qtemp}{\lsys{\stay_\loc}}{\usys{\stay_\loc}}$, whose pair of
lower and upper interval bounds is
$\stay_\loc=\lt(\lsys{\stay_\loc},\usys{\stay_\loc}\rt)$ .  The
parameters affine transformation in a location $\loc\in\locations$
consist of a linear transformation, specified by a matrix
$\linmap_\loc\in\mat{n}{n}{\reals}$, and a bounded additive
disturbance input set $\inputset_\loc\subset\reals^n$.  The set of
edges is $E$.  An edge $\edge\in\edgeset$ is specified by a tuple
%
\[
\edge=\lt(\edge_1,\edge_2,\usys{\edge},\lsys{\edge},\linmap_\edge,\inputset_\edge\rt).
\]
%
The before and after locations of a discrete transition along an edge
$\edge$ are $\edge_1,\edge_2$.  The guard on the transition along the
edge $\edge$ is the sub-parallelotope
$\ptope{\qtemp}{\lsys{\edge}}{\usys{\edge}}$, whose pair of
lower and upper interval bounds is
$\lt({\lsys{\edge}},{\usys{\edge}}\rt)$.
The parameters of the affine transformation for the discrete
transition along the edge $\edge$ consists of a linear map specified
by the matrix $\linmap_\edge\in\mat{n}{n}{\reals}$ and a bounded
additive disturbance input set $\inputset_\edge\subset\reals^n$.  The
set of initial states of the system is $\Psi\subseteq\locations\times\reals^n$.

{\bf Dynamics.}  A state of the hybrid system is a pair $(x,\loc)$,
where $x\in\reals^n$, called the {\it continuous state}, and
$\loc\in\locations$, called the {\it discrete state}.  A {\it
  trajectory} specifies the evolution of the state of the system as a
function of discrete time instants.  A trajectory is a function
$\maphtrj:\integers_{\geq 0}\ra\reals^n\times\locations$, such that
$\forall t\in\integers_{\geq 0}$, one of the following conditions is
true.
%
\begin{enumerate}
\item todo
\item todo
\end{enumerate}
%































{\color{red} TODO.}

\section{Globally exponential stability and set contraction}~\label{sec:ges}
A nearly periodic linear impulsive system is globally exponentially
stable (GES) if any point in the state space reaches arbitrarily close
to the origin at an exponential rate.  This property is mathematically
stated as follows.
%
\begin{defn}\label{defn:exp-stable}[Global exponential stability] The
system~(\ref{eqn:system}) is said to be globally exponentially stable
(GES) if there exist $\lambda\in[0,1)$ and $c>0$ such that for all $\initstate
\in\mb{R}^n$ and $k\in\mb{Z}_+$,
$||\tbf{x}_k||\leq c\lambda^k||\tbf{x}_0||$.
\end{defn}
%
The GES of a system is related to the reachable sets of the
system. Indeed if all bounded sets containing the origin eventually
contract to arbitrarily small sets around the origin, then every point
eventually reaches close to the origin and the system is thus
GES. Instead of verifying the contraction of all bounded sets
containing the origin, we can verify the contraction of any compact
and convex set containing the origin, because it can be scaled to
include any bounded set. We
call such sets as $C$-sets.

The contractiveness of a set is defined as follows.
%
\begin{definition}[\cite{todo:all}]
Given $\lambda\in[0,1]$, a set $C$-set $\Psi\subset\reals^n$ is
$\lambda$-contractive for the system $\system$ iff
\[\forall x\in\Psi,~\forall t\in\Delta,~H_tx\in\lambda\Psi.\]
\end{definition}
%
The following is the stability verification problem.
%
\begin{problem}
Given a sampling period interval $\Delta=[\lsb,\usb]$, verify
that the system $H$ is globally exponentially stable.
\end{problem}
%
\begin{remark} It has been shown previously
in~(\cite{2014-fiacchini-set,athanasopoulos2014alternative,AlKhatib2015})
that the globally exponentially stablility of the system is equivalent
to the existence of a $\lambda$-contractive $C$-set for a
$\lambda\in[0,1)$, as stated in the following theorem.
\end{remark}
%
Therefore, the global exponential stability of the system can be
verified by finding a contractive $C$-set.  A stability verification
algorithm was proposed in~\cite{todo} that definitely computes a
contractive $C$-set for a globally exponentially stable system.
However, the algorithm in~\cite{todo} involves iterative
intersections, where the complexity of the set can grow in size during
recursion.  Therefore, the algorithm can be costly, especially in
higher dimensions.  Alternatively, in this paper we propose an
algorithm that synthesizes a possibly contractive complex zonotope by
convex optimization and later verifies its contraction.  Although the
existence of a contractive complex zonotope is only a sufficient
condition for global exponential stability, we demostrate that our
procedure by experiments on some benchmar examples.  Our procedure for
stability verification, which we describe in the next section, shall
use some properties of contraction of sets that we shall discuss now.

Since the state change of the system can be identified by the
transformation by a reachability operator, we define contraction by
a matrix as follows.  From here on, $J$ denotes an
$n\times n$ real matrix.
%
\begin{defn} The amount of contraction of a set
  $\Psi\subset\mathbb{R}^n$ upon transformation by the matrix $J$,
  denoted as $\chi(\Psi,J)$, is $$\chi(\Psi,J) =
  \inf\{a\in\mathbb{R}_{\geq 0} : J(\Psi)\subseteq a
  \Psi\}.$$ \end{defn}
% \begin{defn} The amount of contraction of a set $\Psi\subset\mathbb{R}^n$
% after $k$ steps for some $k\in\mathbb{Z}^+$, denoted by  $\chi_k(\Psi)$, is
% $$\chi_k(\Psi)=\inf\{a\in\mathbb{R}_{\geq 0} : R^k(\Psi)\subseteq a\Psi\}.$$  
% %We shall denote the contraction of $\Psi$ after $k$ steps as $\chi_k(\Psi)$.
% \end{defn}
The amount of contraction being greater than one would indicate that
the set is actually \emph{expanding}, which also will be referred in
general
as ``contraction'' sense in our paper.
%
%
For any $\rho:0\leq \rho\leq \epsilon$, we want to derive a bound on the
contraction of the operator $H_{t+\rho}$ as a function of $H_t$ and
$\epsilon$.  Using Taylor expansion of an order $r$, we
can write
\begin{align*}
&
  H_{t+\rho}=e^{(A_c\rho)}H_t=P_r(\rho)H_t+E_r(\delta)H_t~~\text{where}\\
&  P_r(\rho)=\sum_{i=0}^r\frac{A_c^i\rho^i}{i!},~~~~~~
 E_r(\delta)=\frac{A_c^{r+1}\delta^{r+1}}{(r+1)!}:~~\delta\in[0,\epsilon].~\numberthis\label{eqn:taylor}
\end{align*}
%
To use the above expansion for deriving the bound on contraction, we
shall describe some properties of contraction.  The following
lemma states that the amount of contraction under transformation
by the product of two matrices is bounded by the product of amount of
contraction of individual matrices.  Also, the amount of contraction
under transformation by the sum of two matrices is bounded by the sum
of the amount of contraction of individual matrices.
%
\begin{lem}\label{lem:composition}
Let $J_1$ and $J_2$ be two linear operators and $\Psi$ be a $C$-set.
  Then the all of the following is true.
\begin{enumerate}
\item $\chi(\Psi,J_1+J_2)\leq\chi(\Psi,J_1)+\chi(\Psi,J_2)$.
\item $\chi(\Psi,J_1J_2)\leq\chi(\Psi,J_1)\chi(\Psi,J_2)$.
\end{enumerate}
\end{lem}
%
\begin{proof}
  For proving the first part, we derive the following.
  %
  \begin{align*}
&  \text{Since}\hspace{1em}   J_1\Psi\subseteq
    \chi(\Psi,J_1)\Psi~\text{ and }
    J_2\Psi\subseteq \chi(\Psi,J_2),~\text{we get}\\
&  (J_1+J_2)\Psi\subseteq
  \chi(\Psi,J_1)\Psi \oplus \chi(\Psi,J_2)\Psi=
  (\chi(\Psi,J_1) + \chi(\Psi,J_2))\Psi.
  \end{align*}
  %
For proving the second part, we derive the following.
\begin{align*}
  &  J_1J_2\Psi=J_1(J_2\Psi)\subseteq J_1(\chi(\Psi,J_2)\Psi)\\
 &=\chi(\Psi,J_2)\lt(J_1\Psi\rt)\subseteq
  \chi(\Psi,J_1)\chi(\Psi,J_2)\Psi.\hspace{3em}\qedhere
\end{align*}
%
\end{proof}
%
If a matrix is embedded inside the convex hull of a set of matrices,
then we get the following bound on contraction by the matrix.
%
\begin{lemma}
Let us consider that $J\in\convexhull{\set{A_1,...,A_r}}$ and
$\Psi\subset\reals^n$.  Then
%
\begin{align*}
& \contraction{J}{\Psi}\leq\max_{i=1}^r\contraction{A_i}{\Psi}.
\end{align*}
%
\end{lemma}
%
\begin{proof}
As $J\in\convexhull{\set{A_1,...,A_r}}$, there exists
$\alpha_1,...,\alpha_r\in\reals_{\geq 0}$ such that
$\sum_{i=1}^r\alpha_i=1$ and $J=\sum_{i=1}^r\alpha_iA_i$.  Then by
using Lemma~\ref{lem:composition}, we get
%
\begin{align*}
  & \contraction{J}{\Psi}\leq\sum_{i=1}^r\contraction{A_i}{\Psi}\\
  & \leq \lt(\sum_{i=1}^r\alpha_i\rt)\max_{i=1}^r\contraction{A_i}{\Psi}=\max_{i=1}^r\contraction{A_i}{\Psi}.\hspace{3em}\qedhere
\end{align*}
%
\end{proof}
%
\begin{lem}~\label{lem:convex}~\cite{todo}
Let us consider $\set{A_0,A_1,...,A_r}\subseteq\mat{n}{n}\reals$ such
that 
%
\[
\forall j\in\{0,...,r\},~ U_j(\rho)=\sum_{i=0}^jA_i\rho^i.
\]
%
If
$0\leq\rho\leq \epsilon$, then $U_r(\rho)\in
\convexhull{U_0(\epsilon),U_1(\epsilon),...,U_r(\epsilon)}$.
\end{lem}
\begin{proof}
This has been proved in~\cite{todo}.
\end{proof}

Using the above results, we derive the following bound on contraction
by the operator $H_{t+\rho}$, when $\rho\in[0,\epsilon]$.
%
\begin{lemma}\label{lem:conv}
Let us consider $\Psi\subset\reals^n$ and $\rho\in[0,\epsilon]$.  If
$0\leq \rho\leq \epsilon$, then
%
\begin{align*}
& \contraction{H_{t+\rho}}{\Psi}\leq\max_{i=1}^r\contraction{P_r(\epsilon)H_t}{\Psi}+\contraction{\frac{A_c^{r+1}}{(r+1)!}H_t}{\Psi}\epsilon^{r+1}.
\end{align*}
%
\end{lemma}
%
\begin{proof}
Using Equation~\ref{eqn:taylor}, there exists
$\delta\in[0,\epsilon]$ such that,
%
\begin{align*}
&
  \contraction{H_{t+\rho}}{\Psi}=\contraction{P_r(\rho)+E_r(\delta)}{\Psi}\\
& \%\%~~\text{by Lemma~\ref{lem:composition}}\\
&
  \leq\contraction{P_r(\rho)H_t,\Psi}+\contraction{\frac{A_c^{r+1}}{(r+1)!}H_t}{\Psi}\delta^{r+1}\\
  & \%\%~~\text{by Lemma~\ref{lem:convex}}\\
&
  \leq\max_{i=1}^r\contraction{P_r(\epsilon)H_t}{\Psi}+\contraction{\frac{A_c^{r+1}}{(r+1)!}H_t}{\Psi}\delta^{r+1}\\
& \leq\max_{i=1}^r\contraction{P_r(\epsilon)H_t}{\Psi}+\contraction{\frac{A_c^{r+1}}{(r+1)!}H_t}{\Psi}\epsilon^{r+1}.\hspace{3em}\qedhere
\end{align*}
%
\end{proof}
%
As $P_r(\epsilon)$ is a function of $H_t$ and 




\section{Stability verification using complex zonotope}~\label{sec:alg}
By Lemma~\ref{lem:pi-ver}, to prove a linear invariance property, we
can compute a positive invariant containing the initial set and which
satisfies the property.  We shall derive a convex program to compute a
positively invariant satisfying a linear invariance property and
containing an initial set, whose continuous projection in any location
is represented as an augmented complex zonotope.  Our procedure
apriori fixes the templates of the augmented complex zonotope and
synthesizes the set of scaling factors, and lower and upper interval
bounds for verifying the property.  We shall discuss in a latter
section how to select a suitable primary template.  But the secondary
template has to be the pseudo-inverse of the sub-parallelotopic
template of the system, so that we can over-approximate the
intersection with the guards and staying conditions based on
Theorem~\ref{thm:main-intersection}.

Therefore, we consider a set of states $\Omega$ whose projection onto
continuous states in a location $\loc$ is an augmented complex
zonotope specified as
%
\[
\Omega_\loc=\acztope{\ptemp}{\cen_\loc}{\sfact_\loc}{\pinv{\qtemp}}{\lb_\loc}{\ub_\loc}.
\]
%
where $\ptemp\in\mat{n}{m}{\compnums}$.  Furthermore, we require a
condition the following condition on the templates for computing a
sound intersection with the guards and staying conditions, based on
Theorem~\ref{thm:main-intersection}.
%
\begin{equation}~\label{eqn:boxcenter}
\begin{split}
 & \forall
 i\in\set{1,....,k},~\forall \loc\in\locations\\
 & \tcztope{\compid{i}\qtemp\ptemp}{\compid{i}\qtemp\cen_\loc}{\sfact_\loc}\order\tcztope{\qtemp\ptemp}{\qtemp\cen_\loc}{\sfact_\loc}
 \end{split}
\end{equation}
%
We consider an
over-approximation of the input
disturbance set in any transition function, by an
augmented complex zonotope, as follows.

%
\begin{align}
& \forall
\loc\in\locations,~\inputset_\loc\subseteq\acztope{\inputtemp_\loc}{\inputcen_\loc}{\inputsfact_\loc}{\inputstemp_\loc}{\inputlb_\loc}{\inputub_\loc}\\
& \forall \edge\in\edgeset,~\inputset_\edge\subseteq\acztope{\inputtemp_\edge}{\inputcen_\edge}{\inputsfact_\edge}{\inputstemp_\edge}{\inputlb_\edge}{\inputub_\edge}.
\end{align}
%
{\bf Min and Max-approximation functions:} The over-approximation of
an intersection between an augmented complex zonotope and a
sub-parallelotope, given in Theorem~\ref{thm:main-intersection}
requires computing componentwise minimum (meet) and maximum (join) of
two real vectors.  The meet and join operations on real vectors are in
general not affine functions of the arguments.  Since we are
interested in deriving a convex program, we want to find an affine
expression for the meet and join operations.  In this regard, we
observe that under a certain affine constraint on the variables, the
meet and join operations can be expressed as affine expressions of the
variables.  We consider two affine functions
called \emph{min-approximation} and \emph{max-approximation}
functions, defined as follows.  A function
$\minapproxsymbol:\reals^k\times\lt(\reals\cup\infty\rt)^k$ is a
min-approximation function if for all $i\in\set{1,...,k}$
%
\[
\lt(\minapprox{\ub}{\pub}\rt)_i=
\lt\{
\begin{array}{l}
\ub_i~\text{if}~\pub_i=\infty\\
\pub_i~\text{if}~\pub_i<\infty
\end{array}
\rt.
\]
%
By the above definition, $\minapprox{\ub}{\pub}$ is an affine function of its
first argument $\ub$, and is a finite valued real vector.
%
\begin{equation}~\label{eqn:minapprox}
\minapprox{\ub}{\pub}\geq\meet{\ub}{\pub}.
\end{equation}
%
Similarly, a function
$\maxapproxsymbol:\reals^k\times\lt(\reals\cup\lt(-\infty\rt)^k\rt)$
is a max-approximation function if for all $i\in\set{1,...,k}$
%
\[
\lt(\maxapprox{\lb}{\plb}\rt)_i=
\lt\{
\begin{array}{l}
\lb_i~\text{if}~\plb_i=-\infty\\
\plb_i~\text{if}~\plb_i>-\infty
\end{array}
\rt.
\]
%
By the above definition, $\maxapprox{\lb}{\plb}$ is an affine function
of its first argument $\lb$, and is a finite valued real vector.
%
\begin{equation}~\label{eqn:minapprox}
\minapprox{\lb}{\plb}\geq\join{\lb}{\plb}.
\end{equation}
%
The following lemma states an affine condition when the meet and join
operations can be equivalently computed by min and max-approximation
functions, respectively.
%
\begin{lemma}~\label{lem:min-max-approximation}
Let us consider vectors $\lb,\ub\in\reals^k$ and
$\plb,\pub\in\set{\reals,-\infty,\infty}^k$.  If
$\lb\leq\maxapprox{\lb}{\plb}\leq\minapprox{\ub}{\pub}\leq\ub$, then
%
\begin{align*}
& \join{\lb}{\plb}=\maxapprox{\lb}{\plb}~\numberthis\label{eqn:max1}\\
& \meet{\ub}{\pub}=\minapprox{\ub}{\pub}.~\numberthis\label{eqn:max2}
\end{align*}
%
\end{lemma}
%
\begin{proof}
For any $i\in\set{1,...,k}$, we derive results for the following four cases

{\it Case 1: }  Let us consider that $\plb_i>-\infty$. So,
$\lt(\maxaffine{\lb}{\plb}\rt)_i=\plb_i$.  Then by
Equation~\ref{eqn:max1}, we get
%
\begin{align*}
& \lt(\join{\lb}{\plb}\rt)_i=\plb_i=\lt(\maxaffine{\lb}{\plb}\rt)_i.
\end{align*}
%

{Case 2: }  Let us consider that $\plb_i=-\infty$.  Then
$\lt(\maxaffine{\lb}{\plb}\rt)_i=\lb_i$.   Then by
Equation~\ref{eqn:max1}, we get
%
%
\begin{align*}
& \lt(\join{\lb}{\plb}\rt)_i=\lb_i=\lt(\maxaffine{\lb}{\plb}\rt)_i.
\end{align*}
%

{\it Case 3: }  Let us consider that $\pub_i<\infty$. So,
$\lt(\minaffine{\ub}{\pub}\rt)_i=\pub_i$.  Then by
Equation~\ref{eqn:max2}, we get
%
\begin{align*}
& \lt(\meet{\ub}{\pub}\rt)_i=\plb_i=\lt(\minaffine{\ub}{\pub}\rt)_i.
\end{align*}
%

{Case 4: }  Let us consider that $\pub_i=\infty$.  Then
$\lt(\minaffine{\ub}{\pub}\rt)_i=\ub_i$.  Then by
Equation~\ref{eqn:max2}, we get
%
\begin{align*}
& \lt(\meet{\ub}{\pub}\rt)_i=\ub_i=\lt(\minaffine{\ub}{\pub}\rt)_i.
\end{align*}
%
\end{proof}
%
{\bf Deriving sufficient conditions for positive invariance.}  We
introduce the following condition, which along with
Equation~\ref{eqn:boxcenter} is sufficient for the inclusion of the
set of continuous reachable states $\contreach{\loc}{\Omega}$ inside
$\Omega_\loc$.  We shall prove this inclusion in
Lemma~\ref{lem:pi-cont}.
%
\begin{definition}
  For $\loc\in\locations$, we say that
  $\posinv{\Omega}{q}$ iff  all of the following is
  collectively true.
%
\begin{align}
& \exists l^\pr,u^\pr,l^\dpr,u^\dpr\in\reals^k:\nonumber\\
& \lb_\loc\leq\maxapprox{\lb_\loc}{\lsys{\stay_\loc}}\leq\minapprox{\ub_\loc}{\usys{\stay_\loc}}\leq\ub_{\loc},~\label{eqn:continv1}\\
& l^\pr= \maxapprox{\lb_\loc}{\lsys{\stay_\loc}},~~u^\pr= \minapprox{\ub_\loc}{\usys{\stay_\loc}},~\label{eqn:continv2}\\
& \acztope{
\mymatrix{\linmap_\loc\ptemp_\loc & \inputtemp_\loc}
}
{
\cen_\loc+\inputcen_\loc
}
{
\mymatrix{\sfact_\loc\\ \inputsfact_\loc}
}
{
\mymatrix{\linmap_\loc\qtemp & \inputstemp_\loc}
}
{
\mymatrix{l^\pr\\\inputlb_\loc}
}
{
\mymatrix{u^\pr\\ \inputub_\loc}
},\nonumber\\
& \order 
\acztope{\ptemp_\loc}{\cen_\loc}{\sfact_\loc}{\qtemp}{l^\dpr}{u^\dpr},~\label{eqn:continv3}\\
&
l^\dpr\leq\maxapprox{l^\dpr}{\lsys{\stay_\loc}}\leq\minapprox{u^\dpr}{\usys{\stay_\loc}}\leq\ub^\dpr,~\label{eqn:continv4}\\
& \lb_\loc\leq\maxapprox{l^\dpr}{\lsys{\stay_\loc}},~~\minapprox{u^\dpr}{\usys{\stay_\loc}}\leq\ub_\loc~\label{eqn:continv5}.
\end{align}
%
\end{definition}
%
\begin{lemma}~\label{lem:pi-cont}
For a location $\loc\in\locations$, if $\posinv{\Omega}{\loc}$, and
Equation~\ref{eqn:boxcenter} holds, then
$\contreach{\loc}{\Omega}\subseteq\Omega$.
\end{lemma}
%
\begin{proof}
By Theorem~\ref{thm:main-intersection},
Equations~\ref{eqn:continv1},~\ref{eqn:continv2},~\ref{eqn:boxcenter}
and Lemma~\ref{lem:min-max-approximation}, we get
%
\begin{align*}
& \Omega_q\bigcap{\ptope{\qtemp}{\lsys{\stay_\loc}}{\usys{\stay_\loc}}}
  \subseteq \acztope{\ptemp_\loc}{\cen_\loc}{\sfact_\loc}{\qtemp}{l^\pr}{u^\pr}.
\end{align*}
%
Using the above over-approximation and the expressions for the linear
transformation and Minkowski sum of augmented complex zonotopes, we
get
%
\begin{align}
& \minsum{A\lt(\Omega_q\bigcap{\ptope{\qtemp}{\lsys{\stay_\loc}}{\usys{\stay_\loc}}}\rt)}{\inputset_\loc}\nonumber\\
& \subseteq \acztope{
\mymatrix{\linmap_\loc\ptemp_\loc & \inputtemp_\loc}
}
{
\cen_\loc+\inputcen_\loc
}
{
\mymatrix{\sfact_\loc\\ \inputsfact_\loc}
}
{
\mymatrix{\linmap_\loc\qtemp & \inputstemp_\loc}
}
{
\mymatrix{l^\pr\\\inputlb_\loc}
}
{
\mymatrix{u^\pr\\ \inputub_\loc}
}\nonumber\\
& \%\%~~\text{by Theorem~\ref{thm:acz-inclusion} and Equation~\ref{eqn:continv3}}\nonumber\\
& \subseteq \acztope{\ptemp_\loc}{\cen_\loc}{\sfact_\loc}{\qtemp}{l^\dpr}{u^\dpr}~\label{eqn:proof-continv1}.
\end{align}
%
Again by Theorem~\ref{thm:main-intersection},
Equations~\ref{eqn:continv4},~\ref{eqn:boxcenter}
and Lemma~\ref{lem:min-max-approximation}, we
get
%
\begin{align}
& \acztope{\ptemp_\loc}{\cen_\loc}{\sfact_\loc}{\qtemp}{l^\dpr}{u^\dpr}\bigcap\ptope{\qtemp}{\lsys{\stay_{\edge_2}}}{\usys{\stay_{\edge_2}}}\nonumber\\
& \subseteq \acztope{\ptemp_\loc}{\cen_\loc}{\sfact_\loc}{\qtemp}{\maxapprox{l^\dpr}{\lsys{\stay_\loc}}}{\minapprox{u^\dpr}{\usys{\stay_\loc}}}\nonumber\\
& \%\%~~\text{by Equation~\ref{eqn:continv5}}\nonumber\\
& \subseteq \acztope{\ptemp_\loc}{\cen_\loc}{\sfact_\loc}{\qtemp}{\lb_\loc}{\ub_\loc}=\Omega_\loc~\label{eqn:proof-continv2}.
\end{align}
%
Using Equations~\ref{eqn:contreach},~\ref{eqn:proof-continv1},
and~\ref{eqn:proof-continv2}, we get
$\contreach{\loc}{\Omega}\subseteq \Omega_\loc$.
\end{proof}
%
For the set of continuous states reached from $\Omega$ after a
discrete transition, $\contreach{\edge}{\Omega}$, to be contained
within 
$\Omega_{\edge_2}$, we introduce the following condition.  We shall
show in Lemma~\ref{lem:pi-dis} that this condition is sufficient for
$\contreach{\edge}{\Omega}\subseteq\Omega_{\edge_2}$.
%
\begin{definition}
For an edge $\edge\in\edgeset$, we say that $\posinv{\Omega}{\edge}$
iff all of the following is collectively true.
%
\begin{align}
& \exists l^\pr,u^\pr,l^\dpr,u^\dpr\in\reals^k:\nonumber\\
& \lb_{\edge_1}\leq \maxapprox{\lb_{\edge_1}}{\join{\lsys{\edge}}{\lsys{\stay_{\edge_1}}}}\leq
\minapprox{\ub_{\edge_1}}{\meet{\usys{\edge}}{\usys{\stay_{\edge_1}}}}\leq\ub_{\edge_1}~\label{eqn:disinv1}\\
&
l^\pr=\maxapprox{\lb_{\edge_1}}{\join{\lsys{\edge}}{\lsys{\stay_{\edge_1}}}},~~
u^\pr=\minapprox{\ub_{\edge_1}}{\meet{\usys{\edge}}{\usys{\stay_{\edge_1}}}}~\label{eqn:disinv2}\\
& \acztope{
\mymatrix{\linmap_{\edge}\ptemp_{\edge_1} & \inputtemp_{\edge}}
}
{
\cen_{\edge_1}+\inputcen_{\edge}
}
{
\mymatrix{\sfact_{\edge_1}\\ \inputsfact_{\edge}}
}
{
\mymatrix{\linmap_{\edge}\qtemp & \inputstemp_{\edge}}
}
{
\mymatrix{l^\pr\\\inputlb_{\edge}}
}
{
\mymatrix{u^\pr\\ \inputub_{\edge}}
},\nonumber\\
& \order 
\acztope{\ptemp_{\edge_2}}{\cen_{\edge_2}}{\sfact_{\edge_2}}{\qtemp}{l^\dpr}{u^\dpr},~\label{eqn:disinv3}\\
&
l^\dpr\leq\maxapprox{l^\dpr}{\lsys{\stay_{\edge_2}}}\leq\minapprox{u^\dpr}{\usys{\stay_{\edge_2}}}\leq
u^\dpr,~\label{eqn:disinv4}\\
& \lb_{\edge_2}\leq\maxapprox{l^\dpr}{\lsys{\stay_{\edge_2}}},~~\minapprox{u^\dpr}{\usys{\stay_{\edge_2}}}\leq\ub_{\edge_2}~\label{eqn:disinv5}.
\end{align}
%
\end{definition}
%
\begin{lemma}~\label{lem:pi-dis}
For an edge $\edge\in\edgeset$, if $\posinv{\Omega}{\edge}$, then $\contreach{\edge}{\Omega}\subseteq\Omega_{\edge_2}$.
\end{lemma}
%
\begin{proof}
By Theorem~\ref{thm:main-intersection},
Equations~\ref{eqn:disinv1},~\ref{eqn:disinv2},~\ref{eqn:boxcenter},
and Lemma~\ref{lem:min-max-approximation} we get
%
\begin{align*}
& \Omega_{\edge_1}\bigcap\ptope{\qtemp}{\join{\lsys{\edge}}{\lsys{\stay_{\edge_1}}}}{\meet{\usys{\edge}}{\usys{\stay_{\edge_1}}}}
\subseteq \acztope{\ptemp_{\edge_1}}{\cen_{\edge_1}}{\sfact_{\edge_1}}{\qtemp}{l^\pr}{u^\pr}.
\end{align*}
%
Using the above over-approximation and the expressions for the linear
transformation and Minkowski sum of augmented complex zonotopes, we
get
%
\begin{align}
& \minsum{A\lt(\Omega_{\edge_1}\bigcap\ptope{\qtemp}{\join{\lsys{\edge}}{\lsys{\stay_{\edge_1}}}}{\meet{\usys{\edge}}{\usys{\stay_{\edge_1}}}}\rt)}{\inputset_{\edge}}\nonumber\\
& \subseteq \acztope{
\mymatrix{\linmap_{\edge_1}\ptemp_{\edge_1} & \inputtemp_\edge}
}
{
\cen_{\edge_1}+\inputcen_\edge
}
{
\mymatrix{\sfact_{\edge_1}\\ \inputsfact_\edge}
}
{
\mymatrix{\linmap_{\edge_1}\qtemp & \inputstemp_\edge}
}
{
\mymatrix{l^\pr\\\inputlb_\edge}
}
{
\mymatrix{u^\pr\\ \inputub_\edge}
}\nonumber\\
& \%\%~~\text{by Theorem~\ref{thm:acz-inclusion} and Equation~\ref{eqn:disinv3}}\nonumber\\
& \subseteq \acztope{\ptemp_{\edge_2}}{\cen_{\edge_2}}{\sfact_{\edge_2}}{\qtemp}{l^\dpr}{u^\dpr}~\label{eqn:proof-dis1}.
\end{align}
%
Again by Theorem~\ref{thm:main-intersection},
Equations~\ref{eqn:disinv4},~\ref{eqn:boxcenter} and Lemma~\ref{lem:min-max-approximation}, we get
%
\begin{align}
& \acztope{\ptemp_{\edge_2}}{\cen_{\edge_2}}{\sfact_{\edge_2}}{\qtemp}{l^\dpr}{u^\dpr}\bigcap\ptope{\qtemp}{\lsys{\stay_{\edge_2}}}{\usys{\stay_{\edge_2}}}\nonumber\\
& \subseteq \acztope{\ptemp_{\edge_2}}{\cen_{\edge_2}}{\sfact_{\edge_2}}{\qtemp}{\maxapprox{l^\dpr}{\lsys{\stay_{\edge_2}}}}{\minapprox{u^\dpr}{\usys{\stay_{\edge_2}}}}\nonumber\\
& \%\%~~\text{by Equation~\ref{eqn:disinv5}}\nonumber\\
& \subseteq \acztope{\ptemp_{\edge_2}}{\cen_{\edge_2}}{\sfact_{\edge_2}}{\qtemp}{\lb_{\edge_2}}{\ub_{\edge_2}}=\Omega_{\edge_2}~\label{eqn:proof-dis2}.
\end{align}
%
Using Equations~\ref{eqn:disreach},~\ref{eqn:proof-dis1}
and~\ref{eqn:proof-dis2}, we get $\contreach{\edge}{\Psi}
\subseteq \Omega_{\edge_2}$.
\end{proof}
%
The following condition is sufficient to check a linear
invariance property.
%
\begin{theorem}~\label{thm:maintheorem}
We get $\lt(\system,\Psi\rt)\models\invariance{T}{d}$ if all of the
following is true.
%
\begin{align*}
& \forall
 i\in\set{1,....,k},~\forall \loc\in\locations,~\forall\edge\in\edgeset\\
& \tcztope{\compid{i}\qtemp\ptemp}{\compid{i}\qtemp\cen_\loc}{\sfact_\loc}\order\tcztope{\qtemp\ptemp}{\qtemp\cen_\loc}{\sfact_\loc},\\
& \posinv{\Omega}{\loc}~\wedge~\posinv{\Omega}{\edge},\\
& T\lt(\cen_\loc+\pinv{\qtemp}\frac{\lb_\loc+\ub_\loc}{2}\rt)+\absolute{T\mymatrix{\ptemp
& \pinv{\qtemp}}}\mymatrix{\sfact_\loc\\\frac{\ub_\loc-\lb_\loc}{2}}\leq d~\numberthis\label{eqn:mainsup}.
\end{align*}
%
\end{theorem}
%
\begin{proof}
By Lemmas~\ref{lem:pi-cont} and~\ref{lem:pi-dis}, we get that $\Omega$ is
a positive invariant.  According to the expression for support
function in Lemma~\ref{lem:support-acz} which matches the last part of
Equation~\ref{eqn:mainsup}, we get that all continuous states
$x\in\Omega_\loc$ for all the locations satisfy $Tx\leq d$.  By
Lemma~\ref{lem:pi-ver}, the linear invariance property is satisfied.
\end{proof}
%
{\bf Algorithm for verification: } For fixed primary template and the
secondary template chosen as explained previously, the verification
procedure is given in Algorithm~\ref{alg:ver}.  The algorithm if
succesful guarantees the verification of a linear invariance property.
But it can not invalidate a linear invariance property because it is
based on a sufficient but not necessary condition.  It can be
implemented by a single step of
\emph{second order conic programming}.  This follows from the fact
that min and max-approximation functions are affine and the relation
``$\order$'' between complex zonotopes is equivalent to a set of
second order conic constraints on the center, scaling factors and
lower and upper interval bounds.
%
 \begin{algorithm}
\caption{Verification of linear invariance property}
\label{alg:ver}
For all $\loc\in\locations$, solve for $\cen_\loc$,
$\sfact_\loc$, $\lb_\loc$ and $\ub_\loc$ satisfying Equation~\ref{eqn:mainsup}. 
 \end{algorithm}
%

{\bf Selecting the primary template:  }  We note that adding any
arbitrary vector to a primary template increases the accuracy of the
verification procedure because the scaling factors are adjusted by the
optimizer.  However, there the computational cost also increases by
adding more vectors to a template.  Therefore, we have to select the
primary template wisely.  In this regard, we provide some suggestions
for chosing the primary template, which are based on the properties of
complex zonotopes that we derived earlier.
%
\begin{enumerate}
\item \emph{Eigenvectors:  }  We can add eigenvectors of the linear
transformation matrices and their products.  This choice is based on
Lemma~\ref{lem:eig-scaling} which states that a complex zonotope can
capture the contraction by a linear transformation along the
eigenvectors of the transformation.
\item \emph{Orthonormal vectors and their projections:  }  We can add
the orthonormal vectors in the null space of the secondary template
and the projection of template vectors in the space orthogonal to the
null space.  This is based on Theorem~\ref{thm:main-intersection}
where the upper bound on the error in over-approximation of the intersection
between an augmented complex zonotope and sub-parallelotope is
proportional to the orientation between the primary template and the
sub-parallelotopic template.
\item \emph{Template of the input set and its transformations:  }  We can add the templates
used to over-approximate the disturbance input set and its
transformations by the system matrices.  This is based on
Proposition~\ref{prop:commin} which states that the template size of
the resultant complex zonotope from the Minkowski sum of two complex
zonotopes does not increase when their templates are the same.  Since
we take Minkowski sum with the disturbance input in our verification
procedure, we expect to increase accuracy by incorporating the input
template and its transformations in the primary template.
\item Adding any vector to the primary template will increase the
accuracy because the scaling factors can be adjusted by the
optimizer.
\end{enumerate}


\section{Experiments}~\label{sec:experiments}
We evaluated our algorithm on two benchmark examples of linear
impulsive systems below and compared it with other state-of-the-art
approaches.  For convex optimization, we use CVX version 2.1 with
Matlab 8.5.0.197613 (R2015a).  The reported experimental results were
obtained on Intel(R) Core(TM) i5-3470 CPU @ 3.20GHz.

\paragraph{ Example 1.  } We consider a networked control system with
uncertain but bounded transmission period.  A networked control system
is composed of a plant and a controller that interact with each other
by transmission of feedback input from controller to the plant.  If
the system dynamics is linear with linear feedback, then for uncertain
but bounded transmission period, we can equivalently represent it
as a linear impulsive system where
%
\[A_c=\lt(\begin{array}{ccc} A_p & 0 & B_p\\ 0 & 0 &
  0\\ 0 & 0 & 0
\end{array}\rt),~A_r=\lt(\begin{array}{ccc}
\mb{I} & 0 & 0\\
B_oC_p & A_o & 0\\
D_oC_p & C_o & 0
\end{array}\rt)\]
%
for some parameter matrices $A_p$, $B_p$, $B_o$,
$C_p$, $A_o$, $C_o$, and $D_o$.  The sampling interval $\Delta$ of the
linear impulsive system specifies bounds on the transmission interval.
%
Our example of a networked control system is taken from Bj\"{o}rn et
al.~\cite{wittenmark2002computer}.  The system is originally described
by discrete time transfer functions, which has an equivalent state
space representation with parameter matrices
%
\begin{align*}
& A_p=\lt(\begin{array}{cc}-1 & 0\\ 1 & 0\end{array}\rt),
 ~ B_p=\lt(\begin{array}{c}1\\ 0\end{array}\rt),
~ C_p=\lt(\begin{array}{cc}0 & 1\end{array}\rt),\\& A_o=0.4286,~
      B_o=-0.8163,~ C_o=-1 D_o=-3.4286.
\end{align*}
%
Given the lower bound on the transmission period as $t_{min}=0.8$, we
want to find as high a value of $t_{max}$ as possible for which
the system is GES.

\begin{table}
\begin{minipage}{1\textwidth}
  \center
\begin{tabular}{|l|c|r|}
  \hline
  Reference & $t_{min}$ & $t_{max}$ \\
\hline
  %& &\\
Value recommended in~\cite{wittenmark2002computer} & 0.08 & 0.22\\
  \hline
  %& &\\
  NCS toolbox~\cite{BauLoo_NECSYS12a} & 0.08 & 0.4 \\
\hline
  %& &\\
  Template complex zonotope & 0.08 & 0.58 \\
  \hline
\end{tabular}
\caption{Experimental results: Example 1}
\label{tab:com1}
\end{minipage}
\end{table}

\begin{table}
\begin{minipage}{1\textwidth}
\label{tab:com2}
\center
\begin{tabular}{|l|c|r|}
\hline
    Reference & $t_{min}$ & $t_{max}$ \\
  \hline
  %& &\\
  Lyapunov, parametric LMI~\cite{2013hetel} & 0.1 & 0.3 \\
  \hline
  %& &\\
  Polytopic set contractiveness~\cite{2014-fiacchini-set} & 0.1 & 0.475 \\
  \hline
  %& &\\
Khatib et al.~\cite{AlKhatib2015} & 0.1 & 0.514\\
\hline
 % & &\\
  Template complex zonotope & 0.1 & 0.496 \\
  \hline
\end{tabular}
\caption{Experimental results: Example 2}
\label{tab:com2}
\end{minipage}
\end{table}

\paragraph{ Example 2.  } We consider the following linear impulsive system
from Hetel et. al.~\cite{2013hetel}, that describes an LMI based
approach to verify stability. The specification is given by
\[A_c=\left(\begin{array}{ccc} 0 & -3 & 1\\ 1.4 & -2.6 & 0.6\\ 8.4 &
  -18.6 & 4.6
\end{array}\rt),~
A_r=\lt(\begin{array}{ccc} 1 & 0 & 0\\ 0 & 1 & 0\\ 0 & 0 & 0
\end{array}\rt).
\]

\paragraph{Setting and Results.}  While implementing the algorithm for
stability verification, we used first order Taylor expansion, a
tolerance of $tol=0.01$ for Example 1 and $tol=0.006$ for Example 2.
We required $k=3$ number of reachability operators for both examples,
for synthesizing a suitable template complex zonotope used in checking
contraction.  We could verify exponential stability in a sampling
interval $[0.08, 0.58]$ for Example 1 and $[0.1,0.496]$ for Example 2.
For the first example, our method outperforms other approaches as
given in Table~\ref{tab:com1}.  For the second example, our method
larger bounds than the Lyapunov function approach~\cite{2013hetel} and
polytopic set contractiveness based
approach~\cite{2014-fiacchini-set}, as reported in
Table~\ref{tab:com2}.  Although our bound is smaller than the one
found by the approach of~\cite{AlKhatib2015}, still the difference is
only 0.018.

