
An invariant property of a system is a property that is true about the
state of the sytem at all time instants.  In many cases, a
safety condition can be expressed as an invariance property.  For
example, if a set of states of a system are considered safe, then the
safety condition is the invariance property that the state of the
system at all time instants should belong to the safe set.  We
consider the problem of verifying {\it linear invariance} property of
discrete time affine hybrid systems.  A linear invariance property is
a linear constraint (inequality) that will be true for the
state of the system at every time instant.   To prove a linear
invariance property, it suffices to compute a positive invariant
containing the initial state and contained within the intersection of
sub-level set of linear inequalities.  Our approach is to compute an
augmented complex zonotope, which is a positive invariant and
satisfies the given linear constraints.  We derive a second order
conic program to compute the required augmented complex zonotope
positive invariant.

This chapter is organized as follows.  In the following section, we
discuss some of the related work on the problem of computing positive
invariants for affine hybrid systems.  We discuss the advantage of the
complex zonotope set representation in comparison to the other
approaches, which was discussed in more detail in the introductory
chapter.  In Section~\ref{sec:hybrid-system}, we define a discrete time affine
hybrid system.  In Section~\ref{sec:linear-invariance}, we define a linear invariance
property, positive invariant and show how we can verify linear
invariance by finding a positive invariant.  In Section~\ref{sec:verification-invariance}, we
discuss our procedure for verification of linear invariance based on
computing an augmented complex zonotope positive invariant.  We derive
a set of convex constraints, which are collectively equivalently to
second order conic constraint, compute the desired augmented complex
zonotope.  We also explain how to choose the template of the augmented
complex template, which is fixed apriori, while other parameters are
found by convex optimization.  In Sections~\ref{todo},~\ref{todo},
and~\ref{todo}, we discuss our experiments on three benchmark examples
that demonstrate the efficiency of our approach.

\section{Related Work}~\label{sec:invariance-related-work}
{\color{red} TODO}.  

\section{Discrete time affine hybrid system}~\label{sec:hybrid-system}
In a discrete time affine hybrid system, the state of the system is
specified by a discrete valued variable, called location, and a
continuous variable whose valuation is in the real Euclidean space of
a finite dimenstion.  The state of the system in each location has to
stay within a polyhedral set, called the staying condition.  The state
of the system can change by two kinds of transitions, {\it continuous
  transition} and {\it discrete transiton}.  In a continuous
transition, the discrete state of the system remains constant while
the continuous state changes by an affine transformation.  The affine
transformation has possible additive disturbance input, which is
bounded.  The parameters of the affine transformation of a continuous
transition depend on the location in which the transition takes place.
In a discrete transition, there is a change in the discrete variable
accompanied by an affine transformation of the continuous variable.
The transition is has precondition specified by a linear constraint,
called a guard, while the post-condition is the staying condition in
the location reached after transition.  A set of edges specifies
the possible discrete transitions, vis a vis, the locations between
which a discrete transition takes place, the parameters of the affine
transformation and the guard.   

{\it Sub-parallelotopic guards and staying conditions}: In this paper,
we consider hybrid systems where the guards and staying conditions can
be specified by a sub-parallelotope with a common template.  We note
that the class of sub-parallelotopic constraints are quite general and
can be used in the specification of many practical affine hybrid
systems.  

{\bf Model.}  We specify the discrete time affine hybrid system
by a tuple 
 %
\[
\system = \lt(\locations,\qtemp,\stay,\linmap,\inputset,\edgeset,\Psi\rt).
\]
%
The finite set of locations is $\locations$.  The sub-parallelotopic
template for specifying the guards and staying conditions is
$\qtemp\in\mat{k}{n}{\reals}$.  The staying set in a location
$\loc\in\locations$ is a sub-parallelotope
$\ptope{\qtemp}{\lsys{\stay_\loc}}{\usys{\stay_\loc}}$, whose pair of
lower and upper interval bounds is
$\stay_\loc=\lt(\lsys{\stay_\loc},\usys{\stay_\loc}\rt)$ .  The
parameters affine transformation in a location $\loc\in\locations$
consist of a linear transformation, specified by a matrix
$\linmap_\loc\in\mat{n}{n}{\reals}$, and a bounded additive
disturbance input set $\inputset_\loc\subset\reals^n$.  The set of
edges is $E$.  An edge $\edge\in\edgeset$ is specified by a tuple
%
\[
\edge=\lt(\edge_1,\edge_2,\usys{\edge},\lsys{\edge},\linmap_\edge,\inputset_\edge\rt).
\]
%
The before and after locations of a discrete transition along an edge
$\edge$ are $\edge_1,\edge_2$.  The guard on the transition along the
edge $\edge$ is the sub-parallelotope
$\ptope{\qtemp}{\lsys{\edge}}{\usys{\edge}}$, whose pair of
lower and upper interval bounds is
$\lt({\lsys{\edge}},{\usys{\edge}}\rt)$.
The parameters of the affine transformation for the discrete
transition along the edge $\edge$ consists of a linear map specified
by the matrix $\linmap_\edge\in\mat{n}{n}{\reals}$ and a bounded
additive disturbance input set $\inputset_\edge\subset\reals^n$.  The
set of initial states of the system is $\Psi\subseteq\locations\times\reals^n$.

{\bf Dynamics.}  A state of the hybrid system is a pair $(x,\loc)$,
where $x\in\reals^n$, called the {\it continuous state}, and
$\loc\in\locations$, called the {\it discrete state}.  A {\it
  trajectory} specifies the evolution of the state of the system as a
function of discrete time instants.  A trajectory is a function
$\maphtrj:\integers_{\geq 0}\ra\reals^n\times\locations$, such that
$\forall t\in\integers_{\geq 0}$, one of the following conditions is
true.
%
\begin{enumerate}
\item todo
\item todo
\end{enumerate}
%































{\color{red} TODO.}x

\section{Linear invariance property}~\label{sec:linear-invariance}
A linear invariance property is a set of linear inequalities that are
satisfied by the state of the system at all time instants for every
trajectory starting in a given initial set.  Mathematically, it is
defined as follows.
%
\begin{definition}[Linear Invariance]
Let us consider a set of states
$\Psi\subseteq\reals^n\times\locations$, a real matrix 
${T\in\mat{r}{n}{\reals}}$ and a real vector $d\in\reals^r$.  We
say that \[\lt(\system,\Psi\rt)\models\invariance{T}{d}~~\text{(Linear invariance property)}~~\text{iff}\] 
%
\begin{align*}\
& \forall t\in\integers_{\geq 0},\forall
 x\in\bigcup_{q\in\locations}\lt(\reachset{t}{\Psi}\rt)_\loc:~~
 Tx\leq d.
\end{align*}
%
\end{definition}
%
To prove that a set of initial states satisfies a linear invariance
property, we can equivalently show the existence of a positive
invariant containing the initial states and satisfying the linear
constraints given in the property specification.  This is described below.
%
\begin{lemma}~\label{lem:pi-ver}
We have
$\lt(\system,\Psi\rt)\models\invariance{T}{d}$ iff there
exists a positive invariant $\Omega$ such that $\Psi\subseteq\Omega$
and $\forall \loc\in\locations~\forall x\in\Omega_{\loc}:~Tx\leq d$.
\end{lemma}
%
\begin{proof}
{\it Case 1:} Let us consider that there exists a positive invariant
$\Omega$ such that $\Psi\subseteq\Omega$ and $\forall
\loc\in\locations~\forall x\in\Omega_{\loc}:~Tx\leq d$.  Since
$\Omega$ is a positive invariant, we have
%
\[
\bigcup_{t\in\integers_{\geq 0}}\reachset{t}{\Omega}\subseteq\Omega
\]
%
\begin{align*}
  & \therefore\forall t\in\integers_{\geq 0}~\forall q\in\locations~\forall x\in\lt(\reachset{t}{\Omega}\rt)_\loc:~Tx\leq d\\
  & \%\%~~\text{since}~\Psi\subseteq\Omega\\
  & \forall t\in\integers_{\geq 0}~\forall q\in\locations~\forall x\in\lt(\reachset{t}{\Psi}\rt)_{\loc}:~Tx\leq d\\
  & \lt(\Psi,\system\rt)\models\invariance{T}{d}.
\end{align*}
%
{\it Case 2:}  Let us consider that $\lt(\system,\Psi\rt)\models\invariance{T}{d}$.  Let us denote 
%
\[
\Omega=\bigcup_{t=0}^\infty\reachset{t}{\Psi}.
\]
% 
Then by Lemma~\ref{lem:exact-pi}, $\Omega$ is a positive invariant.  By
the definition of linear invariant property, we get 
%
\begin{align*}
& \forall t\in\integers_{\geq 0},\forall
 x\in\bigcup_{q\in\locations}\lt(\reachset{t}{\Psi}\rt)_\loc:~~
 Tx\leq d\\
 & \equivalent \forall q\in\locations~\forall
x\in\lt(\bigcup_{t=0}^\infty\reachset{t}{\Psi}\rt)_\loc:~~Tx\leq d\\
 & \equivalent \forall q\in\locations~\forall
x\in\Omega_\loc:~~Tx\leq d.
\end{align*}
%
The lemma follows from the results in both the above cases.
\end{proof}
%


\section{Verification using complex zonotope}~\label{sec:verification-invariance}
{\color{red} TODO}.

\section{TODO}