
In the previous chapter, we have explained the advantage of simple
zonotopes that they are closed under linear transformation and
Minkowski sum operations and these can be computed efficiently.
Regarding computation of a positive invariant real zonotope for a
linear system, we showed in Proposition~\ref{prop:eig-rztope} that it
can be done efficiently when the eigenvectors are real valued.
However, when the eigenvectors of a stable linear system are complex
valued, we can not guarantee the existence of a non-zero real
zonotopic positive invariant.  Overcoming this drawback of real
zonotopes, we extend them to a new class of sets called \emph{complex
zonotopes} by which we can easily specify positive invariants of a
stable linear system using the eigenstructure.  Henceforth, we believe
that even for affine hybrid systems, a complex zonotope can capture
contraction along the complex eigenvectors of some of the
transformation matrices.  Complex zonotopes are also geometrically
more expressive, since their real projections can describe Minkowski
sums of some ellipsoids along with line segments.  Apart from
computing simple operations on complex zonotopes like linear
transformation and Minkowski sum, we shall derive a convex program for
checking inclusion between two complex zonotopes.  The inclusion
relation is a key ingredient for efficient invariant computation,
which we shall discuss in latter chapters.

This chapter is organized into three main sections.  In
Section~\ref{sec:representation}, we shall introduce the basic
representation of a complex zonotope that naturally extends the
Definition~\ref{defn:rztope} of real zonotopes.  Further on, we shall
introduce a more general but geometrically equivalent representation,
called \emph{template complex zonotope}, which allows efficient
modification of complex zonotopic sets for increasing accuracy of
abstraction of sets.  In Section~\ref{sec:operations-tcz}, we shall
discuss basic operations on a template complex zonotope like linear
transformation, Minkowski sum and computation of support function.  In
Section~\ref{sec:inclusion-tcz}, we shall derive a convex program for
checking the inclusion between two template complex zonotopes.  

\section{Representation of a complex zonotope}~\label{sec:representation}
The basic representation of a complex zonotope is a linear combination
of complex valued vectors with complex combining coefficients whose
absolute value is bounded by unity.  This is a generalization of the
representation of a simple zonotope given in Definition~\ref{todo} of
previous chapter to the space of complex numbers.  However, the real
projection of a complex zonotope is expressive because it can
represent some non-polyhedral sets in addition to the polyhedral
zonotopes, which we shall discuss later.
%
\begin{definition}[Complex zonotope]
Let $\ptemp\in\mat{m}{n}{\compnums}$ be a complex valued matrix
whose columns are called {\it generators} and $\cen\in\compnums^n$ be a
complex valued vector called the {\it center}.  The following is the
representation of a
complex zonotope.
%
\begin{equation}
\cztope{\ptemp}{\cen} := \set{\ptemp\zeta+\cen:~\zeta\in\compnums^m,~\infnorm{\zeta}\leq 1}.
\end{equation}
%
\end{definition}
%
A motivation for extending simple zonotopes to complex zonotopes is
that a complex zonotope with its generators as the complex
eigenvectors of a discrete time linear system will be a positively
invariant if the complex eigenvalues corresponding to the generators
are bounded within unity in their absolute values.  This property is
explained mathematically in the following proposition.
%
\begin{proposition}[Eigenstructure based invariance]~\label{lem:eig-invariance}
Let us consider $\ptemp\in\mat{n}{n}{\compnums}$ consists
of the complex eigenvectors of a matrix $A\in\mat{n}{n}{\reals}$ as
its column vectors and $\mu\in\compnums^m$ be the vector of complex
eigenvalues such that $A\ptemp
= \ptemp\diagonal{\mu}$.  Then \[A\lt(\cztope{\ptemp}{0}\rt)
= \cztope{\ptemp\diagonal{\mu}}{0}.\] Furthermore, if
$\infnorm{\mu}\leq 1$, then
$A\lt(\rztope{\ptemp}{0}\rt)\subseteq \rztope{\ptemp}{0}$.
\end{proposition}
% 
\begin{proof}
We derive
  %
\begin{align*}
& A\lt(\cztope{\ptemp}{0}\rt) =
A\set{\ptemp\zeta:~\zeta\in\compnums^m,\infnorm{\zeta}\leq 1}\\
& =\set{A\ptemp\zeta:~\zeta\in\compnums^m,\infnorm{\zeta}\leq 1}
= \cztope{A\ptemp}{0}=\cztope{\ptemp\diagonal{\mu}}{0}.
\end{align*}
%
which proves the first part of the
Proposition.

For the second part, we are given that $\infnorm{\mu}\leq 1$.
Consider a point
%
\begin{align*}
  & y\in \cztope{\ptemp\diagonal{\mu}}{0}~~\text{where}\\
  &y = \ptemp\diagonal{\mu}\delta:\infnorm{\delta}\leq
1.
\end{align*}
%
Let $\zeta = \diagonal{\mu}\delta$. Then $\infnorm{\zeta} \leq
\infnorm{\mu}\infnorm{\delta} \leq 1$.  So,
%
\begin{align*}
  & y=\ptemp\zeta~~\text{ where }~
  \infnorm{\zeta}\leq 1.
\end{align*}
%
This implies $y\in \cztope{\ptemp}{0}$.
Therefore, $A\lt(\cztope{\ptemp}{0}\rt)\subseteq
\cztope{\ptemp}{0}$ when $\infnorm{\mu}\leq 1$.
\end{proof}
%
{\it Shape of complex zonotope:} Geometrically speaking, the real
projection of a complex zonotope can be a Minkowski sum of line
segments as well as some ellipsoids.  Therefore, complex zonotopes are
more expressive than real zonotopes.  An example of such a
non-polyhedral complex zonotope projection in real space is
illustrated with Figure~\ref{fig:cz}, where the generators are
$\ptemp=\lt(\begin{array}{lll}(1+2i) & 1 & (2+i)\\(1-2i) & 1 &
  (2-i)\end{array}\rt)$ and the center is the origin.  Furthermore,
complex zonotopes are different from polynomial zonotopes.  While a
polynomial zonotope is a polynomial function of real valued intervals,
a complex zonotope is a Minkowski sum of \emph{linearly transformed
  transformed circles} in the the complex plane.  A complex zonotope
is symmetric around the center.  To see this, consider a point in a
complex zonotope centered at the origin, written as $y=\ptemp\zeta$
where $\ptemp$ defines the generator set and $\zeta$ is the vector of
combining coefficients.  Since, $\infnorm{-\zeta}=\infnorm{\zeta}\leq
1$, so $-y=\ptemp(-\zeta)$ also belongs to the complex zonotope.


If we add more generators to the above representation of a complex
zonotope, it would increase the size of the complex zonotope.
Threfore, we can not find better approximations of a given set by
only adding more generators to the complex zonotope.  Moreover, adding a
generator can violate positive invariance, as shown in
Figure~\ref{todo}.  Alternatively, to refine a complex zonotope, we
can adjust the magnitude of contribution of each generator to the size
of the set while also preserving the positive
invariance.  This way can also add more generators to refine the
complex zonotope, by adjusting the
magnitude of contribution of each generator.

In order to conveniently perform algebraic manipulations on the
magnitude of each generator, we can explicity specify values
proportional to their magnitudes as part of the set representation.
In this regard, we introduce a {\it template complex zonotope}
representation, where the magnitude of each combining coefficient is
bounded in its absolute value by a positive real, called a {\it
  scaling factor}.  We call the matrix whose column vectors generate a
template complex zonotope as a {\it template}.  This representation is
similar in spirit to the known template based set
representations~\cite{todo} in abstract interpretation, where for some
fixed template, subsets
of metric spaces are mapped to points in a lattice.  In the case of a
template complex zonotope, for a fixed template, subsets of the
complex vector space can be mapped to the {\it scaling factors}.
%
\begin{definition}[Template complex zonotope]
Let us consider $\ptemp\in\mat{n}{m}{\compnums}$ called the template,
$\sfact\in\reals^m_{\geq 0}$ called scaling factors and
$\cen\in\compnums^n$ called the center.  Then the following is a template
complex zonotope.
%
\begin{equation}
\tcztope{\ptemp}{\cen}{\sfact}
= \set{\ptemp\zeta+\cen:~\absolute{\zeta_i}\leq \sfact_i~\forall
i\in\set{1,...,m}}.
\end{equation}
\end{definition}
%
%% The scaling factors of a template complex zonotope can be treated as
%% variables that can be adjusted to find better over-approximations of a
%% reachable set.  For example, Figure~\ref{toref} illustrates the
%% different over-approximations of a convex hull of five points by a
%% complex zonotopes with a fixed template but different scaling factors.
A template complex zonotope can be converted to the basic
representation of the complex zonotope by multiplying the diagonal
matrix of scaling factors to the template.  This is described in the
following lemma.
%
\begin{lemma}[Normalization]~\label{lem:normalization}
Let us consider $\tcztope{\ptemp}{\cen}{\sfact}\subset\compnums^n$ where $\ptemp\in\mat{n}{m}{\compnums}$ and
${\mu\in\compnums^m}$.
%
\begin{align*}
\text{Then}\hspace{3em}&\tcztope{\ptemp\diagonal{\mu}}{\cen}{\sfact}=\tcztope{\ptemp}{\cen}{\diagonal{\absolute{\mu}}\sfact}.~\numberthis\label{eqn:normalization}\\
\text{Therefore},\hspace{3em} & \tcztope{\ptemp}{\cen}{\sfact}=\cztope{\ptemp\diagonal{\sfact}}{\cen}.
\end{align*}
%
\end{lemma}
%
\begin{proof}
Consider a point $x\in\tcztope{\ptemp\diagonal{\mu}}{\cen}{\sfact}$,
where
%
\[
x=\cen+\ptemp\diagonal{\mu}\zeta:\absolute{\zeta}\leq\sfact.
\]
%
Let $\zeta^\pr=\diagonal{\mu}\zeta$.  Then, $x=c+\ptemp\zeta^\pr$.
We get
%
\[
\absolute{\zeta^\pr}=\diagonal{\absolute{\mu}}\absolute{\zeta}\leq\diagonal{\absolute{\mu}}\sfact.
\]

Therefore, ${x\in\tcztope{\ptemp}{\cen}{\diagonal{\mu}\sfact}}$.  This
means,
%
\[
\tcztope{\ptemp\diagonal{\mu}}{\cen}{\sfact}\subseteq\tcztope{\ptemp}{\cen}{\diagonal{\absolute{\mu}}\sfact}
\]
%
Next consider a point
$y\in\tcztope{\ptemp}{\cen}{\diagonal{\absolute{\mu}\sfact}}$ where
%
\[
y=\cen+\ptemp\epsilon:~\absolute{\epsilon}\leq
\diagonal{\absolute{\mu}}\sfact.
\]
%
Let us consider $\epsilon^\pr\in\compnums^m$, such that
%
\[\forall i\in\set{1,...,m},~~
\epsilon_i=\left\{
\begin{array}{l}
\frac{\epsilon_i}{\mu_i}~\text{if}~\mu_i\neq 0\\
0~\text{if}~\mu_i=0.
\end{array}
\right.
\]
%
We shall show that $\epsilon=\epsilon^\pr\diagonal{\mu}$, i.e., for
any $i\in\set{1,...,m}$, $\epsilon_i=\epsilon^\pr_i\mu_i$.  We prove
it in the following two cases.
\begin{enumerate}
\item Let us consider $\epsilon_i\neq 0$.  As
$\absolute{\epsilon}\leq\absolute{\diagonal{\mu_i}}\sfact$, so
  $\mu_i\neq 0$.  Therefore,
  \[
  \epsilon_i=\frac{\epsilon_i}{\mu_i}\mu_i=\epsilon^\pr_i\mu_i.
  \]
\item Let us consider $\epsilon_i=0$.  As
$\absolute{\epsilon}\leq\absolute{\diagonal{\mu_i}}\sfact$, so $\mu_i=
  0$.  This implies
  \[
  0=\epsilon=\epsilon^\pr_i\times
  0=\epsilon^\pr_i\mu_i.
  \]
  %
\end{enumerate}
%
So, we get $y=\cen+\ptemp\diagonal{\mu_i}\epsilon^\pr$.  By the definition of
$\epsilon^\pr$, we get
%
\[\forall i\in\set{1,...,m}~~
\absolute{\epsilon^\pr_i}\leq
\left\{
\begin{array}{l}
\absolute{\frac{\epsilon_i}{\mu_i}}\leq\frac{\absolute{\mu_i}\sfact_i}{\absolute{\mu_i}}=\sfact_i~\text{if}~\mu_i\neq
0\\
0~\text{if}~\mu_i=0
\end{array}
\right.
\]
%
Therefore, $\absolute{\epsilon^\pr}\leq\sfact$.  So,
$y\in\tcztope{\ptemp\diagonal{\mu}}{\cen}{\sfact}$.  Therefore,
%
\[
\tcztope{\ptemp}{\cen}{\diagonal{\absolute{\mu}}\sfact}\subseteq\tcztope{\ptemp\diagonal{\mu}}{\cen}{\sfact}.
\]
%
Combining the previous two conclusions, we get
Equation~\ref{eqn:normalization}.

By definition,
%
\begin{align*}
& \cztope{\ptemp\diagonal{\sfact}}{\cen}=\tcztope{\ptemp\diagonal{\sfact}}{\cen}{\repmat{1}{m}{1}}\\
& \%\%~~\text{by Equation~\ref{eqn:normalization}}\\
& =\tcztope{\ptemp}{\cen}{\diagonal{{\sfact}}\repmat{1}{m}{1}}=\tcz{\ptemp}{\cen}{\sfact}.~\hspace{3em}\qedhere
\end{align*}
%
\end{proof}
%
In further discussion, by the {\it representation size} of a template complex
zonotope, we refer to the size of the template matrix.
%


\section{Basic operations}~\label{sec:operations-tcz}
We shall discuss about other operations on augmented complex zonotopes
like are linear transformation, Minkowski sum, support function and
checking inclusion.

Augmented complex zonotopes are closed under linear transformation and
Minkowski sum, as described in the following two lemmas.  This
is because an augmented complex is specified as a Minkowski sum of a template
complex zonotope and an interval zonotope, both of which are closed
under these operations.
%
\begin{lemma}[Linear transformation]
Let us consider $A\in\mat{n}{n}{\reals}$.  Then
%
\[
A\acztope{\ptemp}{\cen}{\sfact}{\stemp}{\lb}{\ub}=\acztope{A\ptemp}{A\cen}{\sfact}{A\stemp}{\lb}{\ub}.
\]
%
\end{lemma}
%
\begin{proof}
  %
  \begin{align*}
    & A\acztope{\ptemp}{\cen}{\sfact}{\stemp}{\lb}{\ub}=
    A\lt(\minsum{\tcztope{\ptemp}{\cen}{\sfact}}{\iztope{\stemp}{\lb}{\ub}}\rt)\\
    & \%\%~\text{by Lemmas~\ref{lem:lin-transform} and~\ref{lem:iz-lin-transform}}\\
    &
    = \minsum{\tcztope{A\ptemp}{A\cen}{\sfact}}{\iztope{A\stemp}{\lb}{\ub}}\\
    & = \acztope{A\ptemp}{A\cen}{\sfact}{A\stemp}{\lb}{\ub}.~~~~~~~~~~~~~\qedhere
  \end{align*}
  %
\end{proof}
%
\begin{lemma}[Minkowski sum]
The following is true.
%
\begin{align*}
& \minsum{\acztope{\ptemp}{\cen}{\sfact}{\stemp}{\lb}{\ub}}
  {\acztope{\ptemp^\pr}{\cen^\pr}{\sfact^\pr}{\stemp^\pr}{\lb^\pr}{\ub^\pr}}\\
& = \acztope
{\lt[\begin{matrix}
    \ptemp &
    \ptemp^\pr
  \end{matrix}\rt]
}
{\begin{matrix}
    \cen+\cen^\pr
  \end{matrix}
}
{\lt[\begin{matrix}
    \sfact\\
    \sfact^\pr
  \end{matrix}\rt]
}
{\lt[\begin{matrix}
    \stemp &
    \stemp^\pr
  \end{matrix}\rt]
}
{\lt[\begin{matrix}
    \lb\\
    \lb^\pr
  \end{matrix}\rt]
}
{\lt[\begin{matrix}
    \ub\\
    \ub^\pr
  \end{matrix}\rt]
}
\end{align*}
%
\end{lemma}
%
\begin{proof}
  %
  \begin{align*}
& \minsum{\acztope{\ptemp}{\cen}{\sfact}{\stemp}{\lb}{\ub}}
    {\acztope{\ptemp^\pr}{\cen^\pr}{\sfact^\pr}{\stemp^\pr}{\lb^\pr}{\ub^\pr}}\\
& =
    \minsum{\tcztope{\ptemp}{\cen}{\sfact}}{\iztope{\stemp}{\lb}{\ub}}
    \oplus\minsum{\tcztope{\ptemp^\pr}{\cen^\pr}{\sfact^\pr}}{\iztope{\stemp^\pr}{\lb^\pr}{\ub^\pr}}\\
&
    =\lt(\minsum{\tcztope{\ptemp}{\cen}{\sfact}}{\tcztope{\ptemp^\pr}{\cen^\pr}{\sfact^\pr}}\rt)
    \oplus\lt(\minsum{\iztope{\stemp}{\lb}{\ub}}{\iztope{\stemp^\pr}{\lb^\pr}{\ub^\pr}}\rt)\\
    & \%\%~\text{by Lemmas~\ref{lem:min-sum} and~\ref{lem:iz-min-sum}}\\
& = \minsum{\tcztope{\begin{bmatrix}\ptemp &
          \ptemp^\pr\end{bmatrix}}{\cen+\cen^\pr}{\begin{bmatrix}\sfact\\\sfact^\pr\end{bmatrix}}}
             {\iztope{\mymatrix{\stemp &
                   \stemp^\pr}}{\mymatrix{\lb\\\lb^\pr}}{\mymatrix{\ub\\\ub^\pr}}}\\
& =  \acztope
{\lt[\begin{matrix}
    \ptemp &
    \ptemp^\pr
  \end{matrix}\rt]
}
{\begin{matrix}
    \cen+\cen^\pr
  \end{matrix}
}
{\lt[\begin{matrix}
    \sfact\\
    \sfact^\pr
  \end{matrix}\rt]
}
{\lt[\begin{matrix}
    \stemp &
    \stemp^\pr
  \end{matrix}\rt]
}
{\lt[\begin{matrix}
    \lb\\
    \lb^\pr
  \end{matrix}\rt]
}
{\lt[\begin{matrix}
    \ub\\
    \ub^\pr
  \end{matrix}\rt]
}.~~~~~~~~~~~~~~~~~~~\qedhere            
    \end{align*}
%
\end{proof}
%
To problem of checking inclusion between two augmented complex
zonotopes can be expressed as the problem of checking
inclusion between two geometrically equivalent template complex
zonotopes.  For this, we propose the following conversion between the
real projection of an augmented complex zonotope and a template
complex zonotope.
%
\begin{lemma}~\label{lem:acz-tcz-conversion}
Let us consider an augmented complex zonotope
$\acztope{\ptemp}{\cen}{\sfact}{\stemp}{\lb}{\ub}$.  Then we have the
following equivalence to the real projection of a template complex zonotope.
%
\begin{align*}
  \real\lt(\acztope{\ptemp}{\cen}{\sfact}{\stemp}{\lb}{\ub}\rt)
  = \real\lt(\tcztope
  {\lt[\begin{matrix}
      \ptemp &
      \stemp
    \end{matrix}\rt]
  }
  {\begin{matrix}
      \cen+\frac{\ub+\lb}{2}
    \end{matrix}
  }
  {\lt[\begin{matrix}
      \sfact\\
      \frac{\ub-\lb}{2}
    \end{matrix}\rt]
  }
  \rt)
\end{align*}
%
\end{lemma}
%
\begin{proof}
  %
  \begin{align*}
    & \real\lt(\acztope{\ptemp}{\cen}{\sfact}{\stemp}{\lb}{\ub}\rt)
    =
    \real\lt(\tcztope{\ptemp}{\cen}{\sfact}\rt)\oplus\iztope{\stemp}{\lb}{\ub}\\
    & \%\%~\text{By Lemma~\ref{lem:iz-tcz-conversion}}\\
    & =
    \real\lt(\tcztope{\ptemp}{\cen}{\sfact}\rt)\oplus\real\lt(\tcztope{\stemp}{\stemp\frac{\ub+\lb}{2}}{\frac{\ub-\lb}{2}}\rt)\\
    &
  = \real\lt(\tcztope
  {\lt[\begin{matrix}
      \ptemp &
      \stemp
    \end{matrix}\rt]
  }
  {\begin{matrix}
      \cen+\frac{\ub+\lb}{2}
    \end{matrix}
  }
  {\lt[\begin{matrix}
      \sfact\\
      \frac{\ub-\lb}{2}
    \end{matrix}\rt]
  }
  \rt).\hspace{4em}\qedhere
   \end{align*}
  %
\end{proof}
%
 In Definition~\ref{defn:inclusion-tcz} of the previous chapter, we
 introduced a partial order $''\order''$, as a sufficient condition
 for checking inclusion between two template complex zonotopes.  The
 relation could be checking by second order conic programming.  We
 extend the inclusion-checking relation to augmented complex zonotopes.
%
\begin{definition}[Ordering between augmented complex zonotopes]~\label{defn:inclusion-acz}
We say that
%
\begin{align*}
& \acztope{\ptemp^\pr}{\cen^\pr}{\sfact^\pr}{\stemp^\pr}{\lb^\pr}{\ub^\pr} \order
 \acztope{\ptemp}{\cen}{\sfact}{\stemp}{\lb}{\ub}~~\text{iff}\\
& \tcztope
  {\lt[\begin{matrix}
      \ptemp^\pr &
      \stemp^\pr
    \end{matrix}\rt]
  }
  {\begin{matrix}
      \cen^\pr+\frac{\ub^\pr+\lb^\pr}{2}
    \end{matrix}
  }
  {\lt[\begin{matrix}
      \sfact^\pr\\
      \frac{\ub^\pr-\lb^\pr}{2}
    \end{matrix}\rt]
  }  
  \order
  \tcztope
  {\lt[\begin{matrix}
      \ptemp &
      \stemp
    \end{matrix}\rt]
  }
  {\begin{matrix}
      \cen+\frac{\ub+\lb}{2}
    \end{matrix}
  }
  {\lt[\begin{matrix}
      \sfact\\
      \frac{\ub-\lb}{2}
    \end{matrix}\rt]
  }.
\end{align*}
%
\end{definition}
%
\begin{proof}
{\color{red} TODO}.
\end{proof}
%
\begin{theorem}[Inclusion-checking and partial order]~\label{thm:acz-inclusion}
All of the following is true.
\begin{enumerate}
  \item Sufficient condition for inclusion:
%
    \begin{align*}
& \acztope{\ptemp^\pr}{\cen^\pr}{\sfact^\pr}{\stemp^\pr}{\lb^\pr}{\ub^\pr}\order\acztope{\ptemp}{\cen}{\sfact}{\stemp}{\lb}{\ub}\\    
&\implies \real\lt(\acztope{\ptemp^\pr}{\cen^\pr}{\sfact^\pr}{\stemp^\pr}{\lb^\pr}{\ub^\pr}\rt)\subseteq\real\lt(\acztope{\ptemp}{\cen}{\sfact}{\stemp}{\lb}{\ub}\rt).
\end{align*}
%
\item The relation ``$\order$'' is a partial order on the set of
  augmented complex zonotopes.
\end{enumerate}
%
\end{theorem}
%
\begin{proof}
  {\color{red} TODO}
\end{proof}
%
The following propostion states that the above sufficient condition
for checking the inclusion between the real projections of augmented
complex zonotopes is equivalently a second order conic constraint on a
variable of size proportional to the representation size of the
augmented complex zonnotopes.
%
\begin{proposition}
Let us consider two augmented complex zonotopes\\
$\acztope{\ptemp}{\cen}{\sfact}{\stemp}{\lb}{\ub},
\acztope{\ptemp^\pr}{\cen^\pr}{\sfact^\pr}{\stemp^\pr}{\lb^\pr}{\ub^\pr}\subset\compnums^n$.
If $\ptemp$ and $\ptemp^\pr$ are considered constants, then the
relation
%
\[
\acztope{\ptemp^\pr}{\cen^\pr}{\sfact^\pr}{\stemp^\pr}{\lb^\pr}{\ub^\pr} \order
 \acztope{\ptemp}{\cen}{\sfact}{\stemp}{\lb}{\ub}
\]
%
is equivalent to a second order conic constraint on a variable of
length {\color{red} todo}, which comprises the primary offset $\cen$, scaling
factors $\sfact$ and lower and upper interval bounds $\lb$ and $\ub$,
respectively.
\end{proposition}
%
The support function of an augmented complex zonotope for a fixed
vector is as an affine expression of the primary offset, scaling
factors and the upper and lower interval bounds.  This is described in
the following lemma.
%
\begin{lemma}[Support function]~\label{lem:support-acz}
Let us consider an augmented complex zonotope
$\acztope{\ptemp}{\cen}{\sfact}{\stemp}{\lb}{\ub}$ and a vector
$v\in\compnums^n$.  Then
%
\begin{align*}
  & \support{w}{v}{\acztope{\ptemp}{\cen}{\sfact}{\stemp}{\lb}{\ub}}\\
  & = \mymatrix{\ptemp &
    \stemp}\lt(\cen+\frac{\ub+\lb}{2}-w\rt)+v^T\mymatrix{\ptemp & \stemp}\mymatrix{\sfact\\\frac{\ub-\lb}{2}}.
\end{align*}
%
\end{lemma}
%
\begin{proof}
{\color{red} }.
\end{proof}
%




\section{Checking inclusion}~\label{sec:inclusion-tcz}
Checking inclusion between two complex zonotopes amounts to
solving a non-convex optimization problem which can be computationally
intractable.  However, we shall provide a sufficient condition for
the inclusion that can be checked by convex optimization.  In
this regard we define the following relation between two complex
zonotopes, which we later show is sufficient for the inclusion.
%
\begin{definition}[Relation for checking inclusion]~\label{defn:inclusion}
Let us consider $\ptemp^\pr\in\mat{n}{r}{\compnums}$ and
$\stemp^\pr\in\mat{n}{h}{\reals}$.  We say that
$\acztope{\ptemp^\pr}{\cen^\pr}{\sfact^\pr}{\stemp^\pr}{\lb^\pr}{\ub^\pr}
\order \acztope{\ptemp}{\cen}{\sfact}{\stemp}{\lb}{\ub}$
if all of the following is collectively true.
%
\begin{align*}
  & \exists X\in\mat{(m+k)}{(r+h)}{\compnums},~y\in\compnums^{m+k}:\\
   & \mymatrix{\ptemp & \stemp}{y} = \cen^\pr+\stemp^\pr\frac{\ub^\pr+\lb^\pr}{2}-\cen-\stemp\frac{\ub+\lb}{2}, \\
  & \mymatrix{\ptemp & \stemp} X = \mymatrix{\ptemp^\pr &
    \stemp^\pr}\diagonal{\mymatrix{\sfact
      \\ \frac{\ub^\pr-\lb^\pr}{2}}},\\
  & \forall i\in\set{1,\ldots,m}~\absolute{y_i}+\sum_{j = 1}^{r+h}\absolute{X_{i_j}}
  \leq \sfact_i,\\
  & \forall i\in\set{1,\ldots,k}~\absolute{y_i}+\sum_{j =
    1}^{r+h}\absolute{X_{(m+i)_j}}\leq \lt(\frac{\ub-\lb}{2}\rt)_i.
\end{align*}
%
\end{definition}
%
The following lemma states that the above relation between complex
zonotopes is sufficient to guarantee the inclusion between them.
%
\begin{lemma}[Sufficient condition for inclusion]~\label{lem:inclusion}
Let us consider that
$\acztope{\ptemp^\pr}{\cen^\pr}{\sfact^\pr}{\stemp^\pr}{\lb^\pr}{\ub^\pr}
\order \acztope{\ptemp}{\cen}{\sfact}{\stemp}{\lb}{\ub}$.  Then
$\acztope{\ptemp^\pr}{\cen^\pr}{\sfact^\pr}{\stemp^\pr}{\lb^\pr}{\ub^\pr}
\subseteq \acztope{\ptemp}{\cen}{\sfact}{\stemp}{\lb}{\ub}$.
\end{lemma}
%
\begin{proof}
Let us consider a point
$x\in\acztope{\ptemp^\pr}{\cen^\pr}{\sfact^\pr}{\stemp^\pr}{\lb^\pr}{\ub^\pr}$.
Then there exists $\zeta^\pr\in\compnums^r$ and
$\epsilon^\pr\in\reals^k$ such that
%
\begin{align*}
x=\ptemp^\pr\zeta^\pr+\stemp^\pr\epsilon^\pr,~\absolute{\zeta^\pr}\leq\sfact^\pr,~\text{and}~\lb\leq\epsilon^\pr\leq\ub.%~\numberthis\label{eqn:spec}
\end{align*}
%
Let us consider $\alpha\in\compnums^r$ and $\beta\in\reals^h$ defined as follows.
%
\begin{align*}
&  \forall i\in\set{1,\ldots,r}~\alpha_i=\lt\{\begin{array}{l}
  \frac{\zeta_i}{\sfact^\pr_i}~\text{if}~\zeta_i\neq 0\\
  0~\text{otherwise}.
  \end{array}
  \rt.\\
  & \forall i\in\set{1,\ldots,h}~\beta_i=\lt\{
  \begin{array}{l}
  \frac{2\zeta^\pr-\ub^\pr+\lb^\pr}{\ub^\pr-\lb^\pr}~\text{if}~\zeta^\pr\neq\frac{\ub^\pr-\lb^\pr}{2}\\
  0~\text{otherwise}.
    \end{array}
    \rt.\\ \text{Then}~
%% \end{align*}
%% %
%% Then we get
%% %
%% \begin{align*}
&  \zeta^\pr=\diagonal{\sfact^\pr}\alpha,~\epsilon^\pr=\diagonal{\frac{\ub^\pr-\lb^\pr}{2}}\beta+\frac{\ub^\pr+\lb^\pr}{2},\\
&  \infnorm{\alpha}\leq 1,~\infnorm{\beta}\leq 1.%~\numberthis\label{eqn:conversion}
\end{align*}
%
Let us consider $\mymatrix{\zeta\\\epsilon}=y+X\mymatrix{\alpha\\\beta}+\mymatrix{0\\\frac{\ub-\lb}{2}}$.
We derive
%
\begin{align*}
  &  x=\cen^\pr+\ptemp^\pr\zeta^\pr+\stemp^\pr\epsilon^\pr\\
&   = \cen^\pr + \mymatrix{\ptemp^\pr & \stemp^\pr}\mymatrix{\diagonal{\sfact^\pr}\alpha\\
  \diagonal{\frac{\ub^\pr-\lb^\pr}{2}}\beta +
  \frac{\ub^\pr+\lb^\pr}{2}}\\
  & = \cen^\pr+\stemp^\pr\frac{\ub^\pr+\lb^\pr}{2}+\mymatrix{\ptemp &
    \stemp}X\mymatrix{\alpha\\\beta}\\
  & = \cen +\stemp\frac{\ub-\lb}{2}+\mymatrix{\ptemp &
    \stemp}{\lt(y+X\mymatrix{\alpha\\\beta}\rt)}\\
  & = \cen+\mymatrix{\ptemp & \stemp}\mymatrix{\zeta\\ \epsilon}.~\numberthis\label{eqn:conversion1}
\end{align*}
%
We derive the following bounds on $\zeta$ and $\epsilon$.
%
\begin{align*}
& \forall
i\in\set{1,\ldots,m}~\absolute{\zeta_i}=\absolute{y_i+X_i\alpha}\\
& \leq \absolute{y_i}+\sum_{i=1}^{r+h}\absolute{X_{i_j}}\leq \sfact_i~\lt(\because
\infnorm{\alpha}\leq 1\rt)~\numberthis\label{eqn:bound1}\\
& \forall i\in\set{1,\ldots,k}~\absolute{y_i+X_i\beta}\\
& \leq \absolute{y_i}+\sum_{i=1}^{r+h}\absolute{X_{i_j}}\leq\frac{\ub-\lb}{2}~\lt(\because
\infnorm{\beta}\leq 1\rt)~\numberthis\label{eqn:bound2}\\
& \%\%~\because \epsilon_i =
y_i+X_i\beta_i+\frac{\ub+\lb}{2},~\text{by
  Equation~\ref{eqn:bound2}}:\\
& \lb\leq\epsilon\leq \ub.~\numberthis\label{eqn:bound3}
\end{align*}
%
By Equations~\ref{eqn:conversion1},~\ref{eqn:bound1}
and~\ref{eqn:bound3}, we get that
$x\in\acztope{\ptemp}{\cen}{\sfact}{\stemp}{\lb}{\ub}$.  As this is
true for all
$x\in\acztope{\ptemp^\pr}{\cen^\pr}{\sfact^\pr}{\stemp^\pr}{\lb^\pr}{\ub^\pr}$,
we get
$\acztope{\ptemp^\pr}{\cen^\pr}{\sfact^\pr}{\stemp^\pr}{\lb^\pr}{\ub^\pr}\subseteq\acztope{\ptemp}{\cen}{\sfact}{\stemp}{\lb}{\ub}$.
\end{proof}
%
We note that for fixed templates, the relation in
Definition~\ref{defn:inclusion} is a set of convex constraints on the
scaling factors, primary offset and interval bounds of the two complex
zonotopes.  In fact, these constraints can be equivalently
expressed as a second order conic program of polynomial size in the
specification of the two complex zonnotopes.

