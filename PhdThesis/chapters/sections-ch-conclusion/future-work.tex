Some of the directions for extending this research on complex
zonotopes are discussed below.
%
\begin{enumerate}
\item \emph{Non-linear systems: } Hybrid systems with non-linear
  differential and difference equations are inherently more difficult
  to verify than affine hybrid systems having linear differential or
  difference equations.  For reachability analysis of affine hybrid
  systems, complex zonotopes have the advantage that they are closed
  and have low computational complexity under affine transformations.
  However, complex zonotopes are not closed under non-affine
  transformations, like polynomial functions, due to which extending
  their usage to non-affine hybrid systems is computationally
  challenging.  So, we have to develop convex relaxations to
  approximate the non-affine images of complex zonotopes.
  Alternatively, just like polynomial real zonotopes, we can extend
  complex zonotopes to polynomial complex zonotopes.  Then, we face
  the problem that the degree of polynomial complex zonotope grows
  exponentially with the iterative application of non-affine
  functions.  So, we have to approximate a polynomial complex zonotope
  of higher degree with a polynomial complex zonotope of lower degree.
\item \emph{Continuous time dynamics: } The set operations we have
  developed are mainly useful for computing positive invariants of
  discrete time dynamics or simpler cases of continuous dynamics that
  can be easily discretized in time.  But for computing positively
  invariant complex zonotopes for more general continuous time
  dynamics, we have to develop approximations for the lie derivatives
  of the complex zonotope.  
\item \emph{Program verification:  }  Complex zonotopes can be used as
  an abstract domain for program verification.  Indeed, our template
  based representation of complex zonotope is closer in spirit to the
  template based approaches in abstract interpretation.  We have to
  experimentally find out the performance of complex zonotopes when
  applied to numerical program verification.
\end{enumerate}
%
Indeed, more experimentation has to be done to understand the
practical issues with using complex zonotopes, even for affine hybrid
systems.  Nevertheless, complex zonotopes provide an exemplary
approach for future set representations to efficiently encode the
eigenstructure of a system for increasing the accuracy of reachability
analysis techniques.
