\section{Interval zonotope}
Geometrically, an interval zonotope is the same as a simple (real)
zonotope, but the difference is that in their respective
representations.  In an interval zonotope, we specify the bounds on
the combining coefficients as real valued intervals without specifying
the center.  We shall explain latter how this would allow us to
represent the intersection with a certain class of linear inequalities
as another interval zonotope whose interval bounds are a simple
algebraic function of the original interval bounds.
%
\begin{definition}
Let us consider $\stemp\in\mat{n}{k}{\reals}$ called the {\it
  template} and $\lb,\ub\in\reals^k$ called the upper and lower
interval bounds, respectively, such that $\lb\leq \ub$.  The following
is the representation of an interval zonotope.
%
\[
\iztope{\stemp}{\lb}{\ub}=\set{\stemp\zeta:~\zeta\in\reals^k,~\lb\leq\zeta\leq\ub}.
\]
%
\end{definition}
%
An interval zonotope can be equivalently represented as a simple
zonotope as follows.
%
\begin{proposition}
We have
%
\[
\iztope{\stemp}{\lb}{\ub}=\rztope{\stemp\diagonal{\frac{\ub-\lb}{2}}}{\frac{\ub+\lb}{2}}
\]
%
\end{proposition}
%
Although, in general, interval zonotopes are not closed under
intersection with the sub-level sets of linear inequalities, if the
linear inequalities are oriented with the interval zonotope, then the
intersection is closed and can be expressed algebraically.  This is
explained as follows.  We consider a class of sub-level sets of
linear inequalities, which we call as {\it sub-parallelotopes}, that
represent possibly unbounded parallelotopes, defined as follows.
%
\begin{definition}
Let us consider ${\qtemp}\in\mat{k}{n}{\set{-\infty,\infty,\reals}}$
such that $\lt(\qtemp\transpose{\qtemp}\rt)$ is an invertible square
matrix, called a {\it sub-parallelotopic template}, and
$\plb,\pub\in\reals^k$ such that $\plb\leq\pub$, called {\it lower and
  upper offsets}, respectively.  The following is the representation
of a sub-parallelotope.
%
\[
\ptope{\qtemp}{\plb}{\pub}=\set{x\in\reals^n:~\plb\leq\qtemp x\leq\pub}.
\]
%
\end{definition}
%
In other words, a sub-level set of linear inequalities is a
sub-parallelotope if and only if the linear functions on which the
inequalities are defined are linearly independent. For example, the
sub-level set of linear inequalities $-1\leq x+y-z\leq 1~\wedge~~
x-y+z\leq 3$ is equivalent to a sub-parallelotope
\[
\ptope{\lt[\begin{matrix}
    1~~~1~-1\\
    1~-1~1
  \end{matrix}
  \rt]}
{\lt[
    \begin{matrix}
      -1\\
      -\infty
    \end{matrix}
    \rt]
}
{\left[
    \begin{matrix}
      1\\
      3
    \end{matrix}
    \right]
},
\]
because the rows of the sub-parallelotopic template are linearly
independent.  On the other hand, the sub-level set of $-1\leq
x+y-z\leq 1~\wedge~~x+y+z\leq 2\wedge~~-1\leq x+y$ do not constitute a
sub-parallelotope, because the three row vectors $\lt[\begin{array}{c
      c c}1 & 1 & -1\end{array}\rt]$, $\lt[\begin{array}{c c c}1 & 1 &
    1\end{array}\rt]$, and $\lt[\begin{array}{c c c}1 & 1 &
    0\end{array}\rt]$ together are linearly dependent.
Sub-parallelotopes are related to interval zonotopes in the sense that
we can express a sub-parallelotope as a linear combination of real
vectors with corresponding interval bounds on the combining coefficients as
follows.
%
\begin{proposition}
  Consider a sub-parallelotope
  $\ptope{\qtemp}{\plb}{\pub}$ where $\qtemp\in\mat{k}{n}{\reals}$.
  Then,
  %
  \[
  \ptope{\qtemp}{\plb}{\pub}=\set{\cen+\pinv{\qtemp}\zeta:~c\in\reals^n,\zeta\in\reals^k,~\qtemp
  \cen=0,~\plb\leq
  \zeta\leq \pub
  }.
  \]
%
\end{proposition}
%
\begin{proof}
{\color{red} TODO}.
\end{proof}
%
We observe that when a sub-parallelotope has its template aligned with
that of an interval zonotope, their intersection can be exactly
represented by another interval zontope.  As a simple example, the
intersection of
%
\[
\iztope{\lt[\begin{array}{l l}1 & 0 \\ 0 &
      1\end{array}\rt]}{\lt[\begin{array}{c}-1\\ -1\end{array}\rt]}{\lt[\begin{array}{c}2\\ 2\end{array}\rt]}
\]
with a sub-parallelotope defined by the sub-level sets of
%
\begin{align*}
  x_1\leq 1,\\
  x_2\geq 0.5
\end{align*}
%
gives
%
\[
\iztope{\lt[\begin{array}{l
        l}1 & 0 \\ 0 &
      1\end{array}\rt]}{\lt[\begin{array}{c}-1\\ 0.5\end{array}\rt]}{\lt[\begin{array}{c}1\\ 2\end{array}\rt]}.
\]
%
The general case is described in the following lemma, where the
interval of the resultant interval zonotope after intersection is
a simple algebraic expression of the interval bounds of the original
zonotope and the offsets of th
%
\begin{lemma}~\label{lem:motivation}
Let $K\in\mat{k}{n}{R}$ such that $k\leq n$ and $\lt(\qtemp\transpose{\qtemp}\rt)$ is
non-singular.  Then
\[
\iztope{\pinv{\qtemp}}{\lb}{\ub} \bigcap \ptope{\qtemp}{\plb}{\pub}
= \iztope{\pinv{\qtemp}}{\lb\bigvee \plb}{\ub\bigwedge \pub}
\]
\end{lemma}
