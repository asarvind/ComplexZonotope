\section{Interval zonotope and Sub-parallelotope}
We introduce a set representation called interval zonotope, which is
geometrically the same as a simple (real) zonotope, but the difference
is in their respective specifications.  In an interval zonotope, we
specify the intervals in which the combining coefficients are
constrained, without explicitly specifying the center.
%
\begin{definition}[Interval Zonotope]
Let us consider $\stemp\in\mat{n}{k}{\reals}$, called the {\it
  template}, and $\lb,\ub\in\reals^k$, called the upper and lower
interval bounds, respectively, such that $\lb\leq \ub$.  The following
is the representation of an interval zonotope.
%
\[
\iztope{\stemp}{\lb}{\ub}=\set{\stemp\zeta:~\zeta\in\reals^k,~\lb\leq\zeta\leq\ub}.
\]
%
\end{definition}
%
An interval zonotope can be equivalently represented as a simple
zonotope, which is described in the following proposition.
%
\begin{proposition}
We have
%
\[
\iztope{\stemp}{\lb}{\ub}=\rztope{\stemp\diagonal{\frac{\ub-\lb}{2}}}{\frac{\ub+\lb}{2}}.
\]
%
\end{proposition}
%
Although, in general, interval zonotopes are not closed under
intersection with the sub-level sets of linear inequalities, if the
linear inequalities are oriented with the interval zonotope, then the
intersection gives another interval zonotope which is computed
algebraically.  We consider the intersection with a class of sub-level
sets of linear inequalities, called sub-parallelotopes, defined as
follows.  A sub-parallelotope can be seen as a generalization of
parallelotopes to possibly unbounded sets.
%
\begin{definition}[Sub-parallelotope]
Let us consider ${\qtemp}\in\mat{k}{n}{\reals}$ such that
$\lt(\qtemp\transpose{\qtemp}\rt)$ is an invertible square matrix.  We
call such a matrix $\qtemp$ as a sub-paralleotopic template.  Let us
consider a pair of vectors with possibly unbounded components,
$\plb,\pub\in\set{\reals,-\infty,\infty}^k$ such that $\plb\leq\pub$,
called {\it lower and upper offsets}, respectively.  The following is
the representation of a sub-parallelotope.
%
\[
\ptope{\qtemp}{\plb}{\pub}=\set{x\in\reals^n:~\plb\leq\qtemp x\leq\pub}.
\]
%
\end{definition}
%
In other words, a sub-level set of a set of linear inequalities is a
sub-parallelotope if and only if the linear functions on which the
inequalities are defined are linearly independent. For example, the
sub-level set of the linear inequalities
%
\begin{align*}
-1\leq x+y-z\leq 1\\
x-y+z\leq 3
\end{align*}
%
is a sub-parallelotope
%
\[
\ptope{\lt[\begin{matrix}
    1 & 1 & -1\\
    1 & 1 & 1
  \end{matrix}
  \rt]}
{\lt[
    \begin{matrix}
      -1\\
      -\infty
    \end{matrix}
    \rt]
}
{\left[
    \begin{matrix}
      1\\
      3
    \end{matrix}
    \right]
},
\]
because the row vectors $\lt[\begin{matrix}1 & 1 &
    -1 \end{matrix}\rt]$ and $\lt[\begin{matrix}1 & 1 &
    1 \end{matrix}\rt]$, which correspond to the linear functions in the
above inequalities, are linearly independent.  On the other hand, the
sub-level set of
%
\begin{align*}
  -1\leq x+y-z\leq 1\\
  x+y+z\leq 2\\
  -1\leq x+y
\end{align*}
%
is not a sub-parallelotope, because there is linear dependence among
the row vectors $\lt[\begin{matrix} 1 & 1 & -1\end{matrix}\rt]$,
$\lt[\begin{matrix} 1 & 1 & 1\end{matrix}\rt]$, and
$\lt[\begin{matrix} 1 & 1 & 0 \end{matrix}\rt]$.

Sub-parallelotopes have a similarity to interval zonotopes, in the
sense that we can express a sub-parallelotope as a linear combination
of real vectors with corresponding interval bounds on the combining
coefficients, as follows.  In the following proposition, recall that
$\pinv{\qtemp}=\transpose{\qtemp}\inv{\lt(\qtemp\transpose{\qtemp}\rt)}$,
which exists because $\lt(\qtemp\transpose{\qtemp}\rt)$ is defined to
be invertible for a sub-parallelotopic template $\qtemp$.
%
\begin{proposition}
  Consider a sub-parallelotope
  $\ptope{\qtemp}{\plb}{\pub}$ where $\qtemp\in\mat{k}{n}{\reals}$.
  Then,
  %
  \[
  \ptope{\qtemp}{\plb}{\pub}=\set{\cen+\pinv{\qtemp}\zeta:~c\in\reals^n,\zeta\in\reals^k,~\qtemp
  \cen=0,~\plb\leq
  \zeta\leq \pub
  }.
  \]
%
\end{proposition}
%
\begin{proof}
{\color{red} TODO}.
\end{proof}
%
We observe that when a sub-parallelotope has its template aligned with
that of an interval zonotope, their intersection can be exactly
represented by another interval zontope.  As an example, the
intersection of
%
\[
\iztope{\lt[\begin{array}{l l}1 & 0 \\ 0 &
      1\end{array}\rt]}{\lt[\begin{array}{c}-1\\ -1\end{array}\rt]}{\lt[\begin{array}{c}2\\ 2\end{array}\rt]}
\]
with a sub-parallelotope defined by the sub-level sets of
%
\begin{align*}
  x_1\leq 1,\\
  x_2\geq 0.5
\end{align*}
%
gives
%
\[
\iztope{\lt[\begin{array}{l
        l}1 & 0 \\ 0 &
      1\end{array}\rt]}{\lt[\begin{array}{c}-1\\ 0.5\end{array}\rt]}{\lt[\begin{array}{c}1\\ 2\end{array}\rt]}.
\]
%
The general case is described in the following lemma, where the
interval bounds of the resultant interval zonotope after intersection
is given by a simple algebraic expression.
%
\begin{lemma}~\label{lem:motivation}
Let $K\in\mat{k}{n}{R}$ be a sub-parallelotopic template.  Then
\[
\iztope{\pinv{\qtemp}}{\lb}{\ub} \bigcap \ptope{\qtemp}{\plb}{\pub}
= \iztope{\pinv{\qtemp}}{\lb\bigvee \plb}{\ub\bigwedge \pub}
\]
\end{lemma}
%
\begin{proof}
{\color{red} TODO}.
\end{proof}

