In the previous chapter, we derived a second order conic program for
unbounded time safety verification of linear systems, which was based
on approximating the reachable set by a template complex zonotope with
an apriori fixed template.  In that procedure, the scaling factors and
the center of the template complex zonotope were treated as variables
that were solved by convex optimization.  This was possible because of
the following reasons.
%
\begin{enumerate}
\item Template complex zonotopes are closed under Minkowski sum and
  linear transformation operations.
\item For apriori fixed templates, the scaling factors and the centers
  of resultant template complex zonotopes after linear transformation
  or Minkowski sum are expressed as affine functions.
\item For apriori fixed template, our sufficient condition for
  inclusion between template complex zonotopes is expressed as a
  set of second order conic constraints on the scaling factors, centers
  and some additional variables.
\end{enumerate}
%
But in the more general case of hybrid dynamical systems, a transition
may be controlled by constraints on the state variables which act as
preconditions for the transition.  For example, the transitions in an
affine hybrid system are controlled by linear constraints on the
variables, as we shall discuss in a latter chapter.  So,
over-approximating the reachable set of an affine hybrid system using
a set representation would require over-approximating, using the set
representation, the intersection with the sub-level sets of some
linear inequalities.  In this regard, complex zonotopes share the same
drawback as simple zonotopes that they are not closed under
intersection with sub-level sets of linear inequalities.  One way to
address this problem is to introduce a more general set representation
than template complex zonotope, which should serve the following
purposes.
%
\begin{enumerate}
\item The set representation should provide non-trivial approximation
  of the intersection with sub-level sets of linear constraints.  By a
  non-trivial over-approximation, we mean that if the actual
  intersected set is a smaller set than the original set, then the
  over-approximation should be a smaller set.
\item Similar to the case of template complex zonotopes, we should
  have variables in the set representation such that a sufficient
  condition for checking inclusion between any two sets is expresssed as
  convex constraints on the variables.  Furthermore, the sufficient
  condition should be strong enough to check inclusion in many
  practical cases.  Then we can derive an efficient convex program for
  unbounded time safety verification.
\end{enumerate}
%
Before we discuss our new set representation addressing the above problem, we review some
related work.

\paragraph{Related work.} {\color{red} TODO.}



Our new set representation, called {\it augmented complex zonotope} is
specified as a Minkowski sum of a template complex zonotope and
another representation called {\it interval zonotope}.  We shall first
discuss the interval zonotope representation and its intersection with
linear inequalities, which would provide a motivation for introducing the
augmented complex zonotope representation.
