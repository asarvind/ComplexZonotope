Computing reachable sets is a de facto approach used in many formal
verification methods for hybrid systems.  But exact computation of the
reachable set is an intractable problem for many kinds of hybrid
systems, either due to undecidability or high computational
complexity.  Alternatively, quite a lot of research has been focused
on using \emph{set representations} that can be efficiently
manipulated to compute sufficiently accurate over-approximation of the
reachable set.  Zonotopes are a useful set representation in
reachability analysis because of their closure and low complexity for
computing linear transformation and Minkowski sum operations.  But for
approximating the unbounded time reachable sets by \emph{positive
invariants}, zonotopes have the following drawback.  The effectiveness
of a set representation for computing a positive invariant depends on
efficiently encoding the directions for convergence of the states to
an equilibrium.  In an affine hybrid system, some of the directions
for convergence can be encoded by the complex valued eigenvectors of
the transformation matrices.  But the zonotope representation can not
exploit the complex eigenstructure of the transformation matrices
because it only has real valued generators.

Therefore, we extend real zonotopes to the complex valued domain in a
way that can capture contraction along complex valued vectors.  This
yields a new set representation called \emph{complex zonotope}.
Geometrically, complex zonotopes represent a wider class of sets that
include some non-polytopic sets as well as polytopic zonotopes.  They
retain the merit of real zonotopes that we can efficiently perform
linear transformation and Minkowski sum operations and compute the
support function.  Additionally, we show that they can capture
contraction along complex valued eigenvectors.  Furthermore, we
develop computationally tractable approximations for
inclusion-checking and intersection with half-spaces.  Using these set
operations on complex zonotopes, we develop convex programs to
verify \emph{linear invariance properties} of discrete time affine
hybrid systems and exponential stability of \emph{linear impulsive
systems}.  Our experiments on some benchmark examples demonstrate the
efficiency of the verification techniques based on     complex zonotopes.
