Our second example is a perturbed double integrator system given
in~\cite{rakovic2004computation}.  The closed loop system with a
feedback control is piecewise affine, having four different affine
dynamics in four different regions of space, as
$\trj{x}{t+1}=M_i\trj{x}{t}+w$, $i\in\{1,2,3,4\}$.  (the matrices and the regions are given in Appendix)
{\scriptsize
\begin{align*}~\label{eqn:pwa-regions}
i=\left\{\begin{array}{l}
1,~\text{if}~x_1\geq 0~\text{and}~x_2\geq 0\\
2,~\text{if}~x_1\leq 0~\text{and}~x_2\leq 0\\
3,~\text{if}~x_1\leq 0~\text{and}~x_2\geq 0\\
4,~\text{if}~x_1\geq 0~\text{and}~x_2\leq 0\\
\end{array} \rt.,
~M_1=M_2=\lt[\begin{matrix}
0.4103  &  0.0653\\
  -0.2949  &  0.5327
\end{matrix}\rt],~M_3=M_4=\lt[\begin{matrix}
0.4103  &  -0.0653\\
  0.2949  &  0.5327
\end{matrix}\rt]
\end{align*}}

The additive disturbance input $w$ is bounded as $\|w\|_{\infty}\leq
0.2$.  

We perform two different experiments on this system.  In the first
experiment, we try to verify the smallest possible magnitude of bounds
on the two coordinates, denoted $x_1$ and $x_2$. We compare these
bounds with that found by the SpaceEx tool.  In the second experiment,
we try to quickly compute a large invariant for the system under the
safety constraints given in~\cite{rakovic2004computation}.  The given
safety constraints are $\|x\|_{\infty}\leq 5$ and $\lt|K_i(x)\leq
1\rt|~\forall i\in\lt\{1,2\rt\}$.  In the latter case, we maximize the
sum of the scaling factors and differences of the upper and lower
interval bounds of the augmented complex zonotopic invaraint.
Furthermore, we decompose the given safety constraints as the
intersection of four different sets of safety constraints.  For each
set of safety constraints, we compute a large augmented complex
zonotopic invariant.  Then the desired invariant is the intersection
of four augmented complex zonotopic invariants.  Although we may not
find the largest possible (maximal) invariant by this approach, still
the optimizer would find a large invariant.  We draw comparison in
terms of the computation time with the reported result for the MPT
tool~\cite{rakovic2004computation}.

In our formalism, we model the system with $4$ locations and $12$
edges connecting all the locations.  Appropriate staying conditions
are specified in each location, reflecting the division of the state
space into different regions where the dynamics is affine. The initial
set is the origin. The same model is specified in SpaceEx.

\tbf{Size of model}: 2 dimensions, 4 locations and 12 edges.

\tbf{Experiment settings}.  For the primary template, we collected the
(complex) eigenvectors of all linear matrices of the affine maps and
their binary products. For the SpaceEx tool, we experimented with two
different templates, the octagon template and a template with 100
uniformly sampled support vectors.

\tbf{Results.}  In the first experiment, we verified smaller bounds
for $x_2$ than that of SpaceEx, while the bounds verified for $x_1$
were equal for both methods.  In our second experiment on this
example, the computation time for finding a large invariant by our
method is significantly smaller than that of the reported result for
the MPT tool.  The results are summarized in the
Tables~\ref{tab:smallinv-pdi} and~\ref{tab:largeinv-pdi}.

%% \tbf{Remark.}  The overapproximation quality of SpaceEx reduces when
%% there are large number of edges, because then in each step, a union of
%% many different polytopes has to be overapproximated using support
%% vectors.  In contrast, our approach uses optimization to learn an
%% appropriate invariant, avoiding the union operations.  Possibly
%% because of this reason, we verified smaller bounds than SpaceEx on
%% this example.
