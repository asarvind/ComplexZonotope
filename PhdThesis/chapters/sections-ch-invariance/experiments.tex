\subsection{Robot with a saturated controller}
Our first example is a veriication problem for the model of a
self-balancing two wheeled robot called
NXTway-GS1\footnote{\url{http://www.mathworks.com/matlabcentral/fileexchange/19147-nxtway-gs-self-balancing-two-wheeled-robot-controller-design}}
by Yorihisa Yamamoto, which was presented in the ARCH
workshop~\cite{heinz2014benchmark}. We consider the linearized sampled
data (discrete time) networked control system model from the paper.
The state of the plant is represented by a 6-dimensional vector
$x_p=(\dot{\theta},\theta,\dot{\psi},\psi,\dot{\phi},\phi)^T$, where
$\theta$ is the average angle of the left and right wheel, $\psi$ is
the body pitch angle, $\phi$ is the body yaw angle, and the rest
coordinates are their respective angular velocities.  The output of
the plant is represented by a 3-dimensional vector
$\lt(\dot{\psi}_{\operatorname*{out}},\theta_{m_1},\theta_{m_r}\rt)^T$
such that $y_p=C_px_p$.  The input to the plant is a two dimensional
vector $u_p$.  The dynamics of the plant is given by the differential
equation $\dot{x}_p=A_px_p+B_pu_p$.  In the sampled data system, the
state of the plant is sampled every 4s.

The controller state is represented by a 6-dimensional vector $x_c$,
and the input to the controller is denoted
$u_c=\lt({\hat{u}^T}_c,u^\dpr_c\rt)$.  The controller inputs
${\hat{u}^T}_c$ and $u^\dpr_c$ are both 2-dimensional inputs.  The
controller has a 2-dimensional output $y_c$ which is processed to
provide input to the plant as $u_p=D_py_c$.  The discrete time
controller dynamics is given by the equation
%
\begin{align*}
  & x_c(t+1)=\hat{A}_cx_c(t)+\hat{B}_cu_c(t),\\
  & y_c(t+1)=C_cx_c(t)+D_cu_c(t).
\end{align*}
%




In our experiment, we decoupled some unbounded directions of
the dynamics of the system from bounded directions by making an
appropriate linear transformation of the coordinates.  The
transformation is such that the coordinates corresponding to the
\emph{body pitch angle} and controller inputs are among the bounded
directions.  We do not explain the transformation here because it is
beyond the scope of this paper.


