A linear invariance property is a set of linear inequalities that are
satisfied by the state of the system at all time instants for every
trajectory starting in a given initial set.  Mathematically, it is
defined as follows.
%
\begin{definition}[Linear Invariance]
Let us consider a set of states
$\Psi\subseteq\reals^n\times\locations$, a real matrix 
${T\in\mat{r}{n}{\reals}}$ and a real vector $d\in\reals^r$.  We
say that \[\lt(\system,\Psi\rt)\models\invariance{T}{d}~~\text{(Linear invariance property)}~~\text{iff}\] 
%
\begin{align*}\
& \forall t\in\integers_{\geq 0},\forall
 x\in\bigcup_{q\in\locations}\lt(\reachset{t}{\Psi}\rt)_\loc:~~
 Tx\leq d.
\end{align*}
%
\end{definition}
%
To prove that a set of initial states satisfies a linear invariance
property, we can equivalently show the existence of a positive
invariant containing the initial states and satisfying the linear
constraints given in the property specification.  This is described below.
%
\begin{lemma}~\label{lem:pi-ver}
We have
$\lt(\system,\Psi\rt)\models\invariance{T}{d}$ iff there
exists a positive invariant $\Omega$ such that $\Psi\subseteq\Omega$
and $\forall \loc\in\locations~\forall x\in\Omega_{\loc}:~Tx\leq d$.
\end{lemma}
%
\begin{proof}
{\it Case 1:} Let us consider that there exists a positive invariant
$\Omega$ such that $\Psi\subseteq\Omega$ and $\forall
\loc\in\locations~\forall x\in\Omega_{\loc}:~Tx\leq d$.  Since
$\Omega$ is a positive invariant, we have
%
\[
\bigcup_{t\in\integers_{\geq 0}}\reachset{t}{\Omega}\subseteq\Omega
\]
%
\begin{align*}
  & \therefore\forall t\in\integers_{\geq 0}~\forall q\in\locations~\forall x\in\lt(\reachset{t}{\Omega}\rt)_\loc:~Tx\leq d\\
  & \%\%~~\text{since}~\Psi\subseteq\Omega\\
  & \forall t\in\integers_{\geq 0}~\forall q\in\locations~\forall x\in\lt(\reachset{t}{\Psi}\rt)_{\loc}:~Tx\leq d\\
  & \lt(\Psi,\system\rt)\models\invariance{T}{d}.
\end{align*}
%
{\it Case 2:}  Let us consider that $\lt(\system,\Psi\rt)\models\invariance{T}{d}$.  Let us denote 
%
\[
\Omega=\bigcup_{t=0}^\infty\reachset{t}{\Psi}.
\]
% 
Then by Lemma~\ref{lem:exact-pi}, $\Omega$ is a positive invariant.  By
the definition of linear invariant property, we get 
%
\begin{align*}
& \forall t\in\integers_{\geq 0},\forall
 x\in\bigcup_{q\in\locations}\lt(\reachset{t}{\Psi}\rt)_\loc:~~
 Tx\leq d\\
 & \equivalent \forall q\in\locations~\forall
x\in\lt(\bigcup_{t=0}^\infty\reachset{t}{\Psi}\rt)_\loc:~~Tx\leq d\\
 & \equivalent \forall q\in\locations~\forall
x\in\Omega_\loc:~~Tx\leq d.
\end{align*}
%
The lemma follows from the results in both the above cases.
\end{proof}
%
