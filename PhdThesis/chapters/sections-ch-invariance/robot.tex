Our first example is a verification problem for the model of a
self-balancing two wheeled robot called
NXTway-GS1\footnote{\url{http://www.mathworks.com/matlabcentral/fileexchange/19147-nxtway-gs-self-balancing-two-wheeled-robot-controller-design}}
by Yorihisa Yamamoto, which was presented in the ARCH
workshop~\cite{heinz2014benchmark}. We consider the linearized sampled
data (discrete time) networked control system model from the paper.
The state of the plant is represented by a 6-dimensional vector
$x_p=(\dot{\theta},\theta,\dot{\psi},\psi,\dot{\phi},\phi)^T$, where
$\theta$ is the average angle of the left and right wheel, $\psi$ is
the body pitch angle, $\phi$ is the body yaw angle, and the rest
coordinates are their respective angular velocities.  The output of
the plant is represented by a 3-dimensional vector
$\lt(\dot{\psi}_{\operatorname*{out}},\theta_{m_1},\theta_{m_r}\rt)^T$
such that $y_p=C_px_p$.  The input to the plant is a two dimensional
vector $u_p$.  The dynamics of the plant is given by the differential
equation $\dot{x}_p=A_px_p+B_pu_p$.  In the sampled data system, the
state of the plant is sampled every 4s.

The controller state is represented by a 6-dimensional vector $x_c$,
and the input to the controller is denoted
$u_c=\lt({u^\pr}_c,u^\dpr_c\rt)$.  The controller inputs
${{u^\pr}_c}$ and $u^\dpr_c$ are both 2-dimensional inputs.  The input
$u^\dpr$ is an uncertain input which is in the range $[-100,100]$.  The
controller dynamics is given by the equations
%
\begin{align*}
  & \dot{\tau}=1,~~\tau(4^+)=0\\
  & {u^\pr}_c(\tau)=\hat{u}_c(0)~\text{if}~\tau\in[0,4)\\
  & {u^\pr}_c(4)=y_p(\tau)\\
  & \dot{x_c}(\tau)={A}_cx_c(\tau)+{B}_cu_c(\tau)\\
  & y_c(\tau)=C_cx_c(\tau)+D_cu_c(\tau).
\end{align*}
%
The controller has a 2-dimensional output $y_c$ which is processed to
provide input to the plant.  The processor has a saturation limit on
the output received from the controller.  The saturated controller
output is given by the equation
%
\begin{align*}
u_p=D_p\lt(\join{\lt(\meet{y_c}{\mymatrix{v\\v}}\rt)}{\mymatrix{-v\\-v}}\rt).~\numberthis\label{eqn:sat1}
\end{align*}
%
where $v=100$ is a saturation limit.  The sampled data dynamics with
saturation can be modeled by an affine discrete time hybrid system,
where the switching is controlled by relevant guards on $u_p$.
However, if we consider the continuous state of the affine hybrid
system as $\lt(x_p,x_c,u_p\rt)^T$, we observed that some of the
directions are unbounded.  Therefore, we decoupled some unbounded
directions of the dynamics from the bounded directions by making
appropriate linear transformation of the coordinates.  The linear
transformation is composed by two transformations, one of which
involved Jordan decomposition in Matlab.

\begin{table}
{\scriptsize
\begin{align*}
& F_1  = \lt[\begin{matrix}
3.6929   &      0  &  0.7302  &  7.9715 &  14.5019 &   -0.0072 &
0.0720 &   -2.7354\\
    3.6929   &      0  &  0.7302  &  7.9715 &  14.5019 &  -0.0072  &  0.0720  & -2.7354\\
    0.9562    &     0  &  0.0019 &  -0.0021 &  -0.0022 &   -0.0000 &  -0.0001 &  -0.0002\\
         0 &   0.6910    &     0    &     0  &       0     &    0   &      0    &     0\\
    0.8833     &    0  & -0.1154 &  -1.2943 &  -2.3520  &  0.0012 &  -0.0118  &  0.4427\\
   -0.4712    &     0 &  -0.0812 &    0.1151  & -1.4845  &  0.0007 &  -0.0071  &  0.2819\\
   -0.1560     &    0 &  -0.0459 &  -0.3173  &  0.3650  &  0.0003  & -0.0023  &  0.1162\\
   -0.7719   &      0 &  -0.1248  & -1.4264 &  -2.5901  &  0.9973 &  -0.0131  &  0.4869\\
   -0.7544  &       0  & -0.1243 &  -1.4204 &  -2.5792  &  0.0013 &   0.9825 &   0.4796\\
   -0.1905   &      0  & -0.0148  & -0.2081 &  -0.3751  &  0.0002  &  0.0033  &  1.0651
\end{matrix}\rt]\\
& F_2 = \lt[\begin{matrix}
0.2543  &  0.2543\\
    0.2543  &  0.2543\\
   -0.0001 &  -0.0001\\
         0 &        0\\
   -0.0413 &  -0.0413\\
    0.0219  &  0.0219\\
    0.0102 &   0.0102\\
    0.0431 &   0.0431\\
    0.0428 &   0.0428\\
    0.0065 &   0.0065\\
\end{matrix}\rt],
~F_3 = 10^{-2}\times\lt[\begin{matrix}
 0.0000    &     0  & -0.0330 &   2.0218\\
    0   &      0  & -0.0330 &  -2.0218\\
    0  &       0 &   -0  &  0\\
   -0  &       0  &  0 &   0.0109\\
   -0.0118 &        0  &  0.0172  &  0 \\
    0.0436  &       0 &   0.0003 &  0 \\
   -0.0478   &      0  &  0.0034 &   0 \\
  -13.3924 &        0 &   0.0062 &   0 \\
    0.0909     &    0  &  0.0061 &  0\\
   -0.0798  &       0 &   0.0017  &  0\\
\end{matrix}\rt]
\end{align*}}
\caption{Matrices of the transformed system dynamics}~\label{tab:matrices-nxt}
\end{table}


After decomposition, the bounded dynamics with saturation could be
modeled in a 10-dimensional state space, which is described below.
%
\begin{align*}
\mymatrix{x(t+1)\\y(t+1)}=F_1\trj{x}{t}+F_2sat\lt(\trj{y}{t}\rt)+F_3\trj{u}{t},
\end{align*}
where $\trj{x}{t}\in\realset^8$ is the transformed state of the
composite system of plant and controller, $\trj{y}{t}\in\realset^2$ is
the input sent by the controller, $\trj{u}{t}\in\lt[-100,100\rt]^4$ is
the bounded additive disturbance input and $\operatorname*{sat}$ is
the saturation function which limits the controller input received by
the plant.  The body pitch angle is the first co-ordinate, i.e.,
$\psi=x_1$.  The matrices $F_1$, $F_2$ and $F_3$ are given in
Table~\ref{tab:matrices-nxt}.  The matrices for the original
(untransformed system) are given in the appendix.  The saturation
function is defined as follows.  The saturation function is given as
follows.
%
\begin{align*}
& \text{If
  saturated}~~\operatorname*{sat}\lt(y_i\rt) = max\lt(-\delta
 d_p,min\lt(y_i,\delta d_p\rt)\rt),~\forall i\in\{1,2\},\\
&\text{If unsaturated}~~ sat\lt(y_i\rt)=y_i~\forall i\in\{1,2\}.
\end{align*}
%
where $\delta=100$ and $d_p=0.0807$.  
%

{\bf Model complexity:} The 2-dimensional input $y$ on which the
positive and negative saturation is defined can thus be divided into 9
different regions, where the system exhibits different dynamics.  We
model each of the discrete time dynamics by a self edge on a common location.
Therefore, the saturated model consists of a single location with 9
self-edges having appropriate linear guards and transition matrices.
On the other hand, the unsaturated model is a linear system having
uncertain input, which is therefore modeled by only one location.

\emph{Size of saturated model}: 10 dimensional, 1 location and 9 edges.

\emph{Size of unsaturated model}: 10 dimensional, 1 location, 0 edges.

{\bf Linear invariance property to verify}: The safety requirement is
that the \emph{body pitch angle} of the robot, which in our model is
denoted by $x_1$, should be bounded within some value. In the
benchmark, it was suggested that for the saturated system
$x_1\in\lt[-\frac{\pi}{2}+\epsilon,\frac{\pi}{2}-\epsilon\rt]:~\epsilon>0$,
while $x_1\in\lt[\frac{-\pi}{2.26},~\frac{\pi}{2.26}\rt]$ for the
unsaturated system. The initial set is the origin.  Therefore, we want
to find a small bound $d$ on the valued of
$\absolute{\psi}=\absolute{x_1}$.  As the model is symmetric, it is
sufficient to find a bound along the positive direction.  Therefore,
we have to find a small enough $d$ such that $\lt(T,d\rt)$, where
$T=\mymatrix{1&\repmat{0}{1}{9}}$, is a linear invariance property.

{\bf Experiment settings.}

\emph{Augmented complex zonotope: }  The primary template for the hybrid system
is chosen as the collection of the (complex) eigenvectors of linear
matrices of all affine maps for the edge transitions, the orthonormal
vectors to the guarding hyperplane normals and the projections of the
eigenvectors on the subspace spanned by the orthonormal vectors.  For
the linear system, it consists of the eigenvectors of the linear map,
the input set template and its multiplication by the linear matrix
(related to affine map) and square of the linear matrix.

\emph{SpaceEx:      } Concerning
the experiment using SpaceEx, we tested with the octagon template and
a template with $400$ uniformly sampled support vectors distributed
uniformly in space.  
\begin{table}
\begin{minipage}{1\textwidth}
\centering
\begin{tabular}{|l|c|c|c|}
\hline
\multicolumn{2}{|c|}{\multirow{2}{*}{Method}} &
\multirow{2}{*}{Bound on pitch angle} & \multirow{2}{*}{Comp. time (s)}\\
\multicolumn{2}{|c|}{} & & \\
\hline
\multirow{4}{*}{SpaceEx} & octagon & \multirow{2}{*}{$>1000$} &
\multirow{2}{*}{Not terminate in $<180s$}\\
& template & & \\
\cline{2-4}
& 400 support & \multirow{2}{*}{UB} & \multirow{2}{*}{Not terminate in
  $<180s$}\\
& vectors & &\\
\hline
\multicolumn{2}{|c|}{\multirow{2}{*}{Suggested in~\cite{heinz2014benchmark}}} &
\multirow{2}{*}{$1.39$} & \multirow{2}{*}{n/a}\\
\multicolumn{2}{|c|}{} & &\\
\hline
\multicolumn{2}{|c|}{\multirow{2}{*}{Augmented complex zonotope}} & \multirow{2}{*}{$1.29$} &
\multirow{2}{*}{$4$}\\
\multicolumn{2}{|c|}{} & & \\
\hline
\end{tabular}
\caption{Unsaturated robot model: results}
~\label{tab:robot-unsaturated}
\end{minipage}
\hspace{0em}
\begin{minipage}{1\textwidth}
\centering
\begin{tabular}{|l|c|c|c|}
\hline
\multicolumn{2}{|c|}{\multirow{2}{*}{Method}} &
\multirow{2}{*}{Bound on pitch angle} & \multirow{2}{*}{Comp. time (s)}\\
\multicolumn{2}{|c|}{} & & \\
\hline
\multirow{4}{*}{SpaceEx} & octagon & \multirow{2}{*}{$>1000$} &
\multirow{2}{*}{Not terminate in $<180s$}\\
& template & & \\
\cline{2-4}
& 400 support & \multirow{2}{*}{UB} & \multirow{2}{*}{Not terminate $<180s$}\\
& vectors & & \\
\hline
\multicolumn{2}{|c|}{\multirow{2}{*}{Suggested in~\cite{heinz2014benchmark}}} &
\multirow{2}{*}{$1.571-\epsilon:~\epsilon>0$} & \multirow{2}{*}{n/a}\\
\multicolumn{2}{|c|}{} &  &\\
\hline
\multicolumn{2}{|c|}{\multirow{2}{*}{Augmented complex zonotope}} & \multirow{2}{*}{$1.16$} &
\multirow{2}{*}{45}\\
\multicolumn{2}{|c|}{} & &\\
\hline
\end{tabular}
\caption{Saturated robot model: results}
~\label{tab:robot-saturated}
\end{minipage}
\end{table}


\tbf{Results.}  For both the hybrid and the linear systems, we could
verify smaller magnitudes for the bounds on the pitch angle than what
is proposed in the benchmark~\cite{heinz2014benchmark}.  But the
SpaceEx tool could not find a finite bound for either of the above
systems.  The results are reported in the
Tables~\ref{tab:robot-unsaturated} and~\ref{tab:robot-saturated}.

\tbf{Remark.}  We have discussed in the review of polytopes that
although a linear system has a polytopic invariant, computing it can
be difficult.  The representation size of a polytopic invariant for a
fixed dimension can be arbitrarily large.  In our unsaturated model
which is linear, some of the eigenvalues are complex and their
magnitudes are close to one.  Possibly this is the reason SpaceEx
could not find an invariant even with 400 support vectors distributed 
uniformly in space.  But in our approach, since we use the complex
eigenstructure, we could find the desired invariant for the
unsaturated (linear) model.  Furthermore, we we also computed the
invariant for the saturated (hybrid) model.

