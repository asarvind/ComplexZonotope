Our third example is a model of a networked cooperative platoon of
vehicles, which is presented as a benchmark in the ARCH
workshop~\cite{makhlouf2014networked}.  The platoon consists of three
vehicles $M_1$, $M_2$ and $M_3$ along with a leader board ahead $M_4$.
The movement of the vehicles is dependent on the communication between
them their relative distances, velocities and accelerations.  The
distance between a vehicle $M_i$ and its next
vechicle $M_{i+1}$, relative to a reference distances $d_i^{ref}$ is
denoted $e_i$.  The acceleration of the leader vehicle is $a_L$ which
ranges between $[-9,1]m/s$.  The state of the system is denoted
by a vector
$x=\lt[e_1,\dot{e}_1,\ddot{e}_1,e_2,\dot{e}_2,\ddot{e}_2,e_3,\dot{e}_3,\ddot{e}_3\rt]$.
The dynamics of the platoon is different in the two cases when there
is full communication and when there is total failure of
communication.  These dynamics are described by differential equations,
%
\begin{align*}
& \dot{x}=A_cx+B_ca_L~\text{when there is communication}\\
& \dot{x}=A_nx+B_na_L~\text{when there is failure of communication}.
\end{align*}
%
where the pairs of matrices $(A_c,B_c)$ and $(A_n,B_n)$ are different.
The matrices are given in the appendix.  In any given mode, the
dynamics of the system is exponentially stable.  So, the system is
more stable when the minimum time for switching between the dynamics
is larger and less stable otherwise.  Therefore, in our case study, we
also experiment on a less stable system having integer switching time.
Therefore, in our evaluation of the benchmark, we also consider a less
stable system having integer switching times.

{\bf Time discretized models: } We could discretized the dynamics of
both models with large minimum switching time and integer switching
time.  The time discretized model having a large minimum switching
time has 2 locations and 4 edges, as given in Figure~\ref{todo}.  The
corresponding matrices Whereas, the model with integer switching times
is given has 2 locations and 2 edges, as given in Figure~\ref{todo}.
The continuous state is a 9-dimensional vector.  The matrices in the model
specifications described in the figures is given in Table~\ref{todo}.

{\bf Linear invariance property: } The verification challege proposed
in~\cite{makhlouf2014networked} is to find the minimum possible reference distances
$d_i^{ref}~\forall i\in\set{1,2,3}$, such that the vehicles do not
collide.  Any set of upper bounds on $-e_1$, $-e_2$, and $-e_3$ are
safe lower limits on the respective reference distances.  We express
the verification challege in terms of linear invariance properties, we
follows.  For each ${T_i:~1\leq
i\leq 3}$, where
%
$ T_1=\mymatrix{-1 & \repmat{0}{1}{8}},~
T_2=\mymatrix{\repmat{0}{1}{3} & -1 &\repmat{0}{5}{3}}$ and
 ${T_3=\mymatrix{\repmat{0}{1}{6}& -1 &\repmat{0}{2}{3}}},
$
%
find an upper bound on each $d_i:~i\in\set{1,2,3}$ such that
$\lt(T_i,d_i\rt)$ is a linear invariance property of the system.
