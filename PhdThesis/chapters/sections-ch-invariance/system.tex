In a discrete time affine hybrid system, the state of the system is
specified by a discrete valued variable, called location, and a
continuous variable whose valuation is in the real Euclidean space of
a finite dimension.  The state of the system in each location has to
stay within a polyhedral set, called the staying condition.  The state
of the system can change by two kinds of transitions, {\it continuous
  transition} and {\it discrete transit on}.  In a continuous
transition, the discrete state of the system remains constant while
the continuous state changes by an affine transformation.  The affine
transformation has possible additive disturbance input, which is
bounded.  The parameters of the affine transformation of a continuous
transition depend on the location in which the transition takes place.
In a discrete transition, there is a change in the discrete variable
accompanied by an affine transformation of the continuous variable.
The transition is has precondition specified by a linear constraint,
called a guard, while the post-condition is the staying condition in
the location reached after transition.  A set of edges specifies
the possible discrete transitions, vis a vis, the locations between
which a discrete transition takes place, the parameters of the affine
transformation and the guard.   

{\it Sub-parallelotopic guards and staying conditions}: In this paper,
we consider hybrid systems where the guards and staying conditions can
be specified by a sub-parallelotope with a common template.  We note
that the class of sub-parallelotopic constraints are quite general and
can be used in the specification of many practical affine hybrid
systems.  

{\bf Model.}  We specify the discrete time affine hybrid system
by a tuple 
 %
\[
\system = \lt(\locations,\qtemp,\stay,\linmap,\inputset,\edgeset\rt).
\]
%
The finite set of locations is $\locations$.  The sub-parallelotopic
template for specifying the guards and staying conditions is
$\qtemp\in\mat{k}{n}{\reals}$.  The staying set in a location
$\loc\in\locations$ is a sub-parallelotope
$\ptope{\qtemp}{\lsys{\stay_\loc}}{\usys{\stay_\loc}}$, whose pair of
lower and upper interval bounds is
$\stay_\loc=\lt(\lsys{\stay_\loc},\usys{\stay_\loc}\rt)$ .  The
parameters affine transformation in a location $\loc\in\locations$
consist of a linear transformation, specified by a matrix
$\linmap_\loc\in\mat{n}{n}{\reals}$, and a bounded additive
disturbance input set $\inputset_\loc\subset\reals^n$.  The set of
edges is $E$.  An edge $\edge\in\edgeset$ is specified by a tuple
%
\[
\edge=\lt(\edge_1,\edge_2,\usys{\edge},\lsys{\edge},\linmap_\edge,\inputset_\edge\rt).
\]
%
The before and after locations of a discrete transition along an edge
$\edge$ are ${\edge_1,\edge_2\in\locations}$.  The sub-parallelotope
$\ptope{\qtemp}{\lsys{\edge}}{\usys{\edge}}$ is the guard on the
transition along the edge $\edge$, where
$\lt({\lsys{\edge}},{\usys{\edge}}\rt)$ is the pair of lower and
upper interval bounds of the sub-parallelotope.  The
parameters of the affine transformation for the discrete transition
along the edge $\edge$ are specified by a matrix
${\linmap_\edge\in\mat{n}{n}{\reals}}$  and a bounded additive
disturbance input set $\inputset_\edge\subset\reals^n$.  The set of
initial states is
$\Psi\subseteq\locations\times\reals^n$.

{\bf Dynamics.}  A state of the hybrid system, called a \emph{hybrid
  state}, is a pair $(x,\loc)$, where $x\in\reals^n$, called the {\it
  continuous state}, and $\loc\in\locations$, called the {\it discrete
  state}.  A {\it trajectory} is the evolution of the state of the
  system as a function of discrete time instants.  A trajectory is a
  function $\maphtrj:\integers_{\geq 0}\ra\reals^n\times\locations$,
  such that $\forall t\in\integers_{\geq 0}$, one of the following
  conditions is true.
%
\begin{enumerate}
\item Continuous transition:
%
\begin{align}
& \exists u\in\inputset_{\dtrj{t}}~\text{such that all of the following are collectively true.}\nonumber\\
& \ctrj{t+1}=\linmap_{\dtrj{t}}+u,~~~\text{(affine map with uncertain input)}\nonumber\\
& \dtrj{t+1}=\dtrj{t}~~~\text{(location does not change)}\nonumber\\
& \ctrj{t},\ctrj{t+1}\in\ptope{\qtemp}{\lsys{\stay_{\dtrj{t}}}}{\usys{\stay_{\dtrj{t}}}}~~\text{(staying condition)}~~\label{eqn:cont-transition}
\end{align}
\item Discrete transition:
\begin{align}
& \exists \edge\in\edgeset~\text{and}~u\in\inputset_{\edge}~\text{all the following are collectively true.}\nonumber\\
& \dtrj{t}=\edge_1,~~\dtrj{t+1}=\edge_2~~~\text{(possible change in location)}\nonumber\\
& \%\%~~\text{satisfaction of guard and staying condition before transition:}\nonumber\\
& \ctrj{t}\in\ptope{\qtemp}{\join{\lsys{\edge}}{\lsys{\stay_{\dtrj{t}}}}}{\meet{\usys{\edge}}{\usys{\stay_{\dtrj{t}}}}}\nonumber\\
& \%\%~~\text{satisfaction of staying condition after transition:}\nonumber\\
& \ctrj{t+1}\in\ptope{\qtemp}{\lsys{\stay_{\dtrj{t+1}}}}{\usys{\stay_{\dtrj{t}}}}~~\label{eqn:dis-transition}
\end{align}
\end{enumerate}
%
In the definition, we use the following notation for the set of
continuous states corresponding to a set of hybrid states $\Psi$ for a
fixed location $\loc$.  
%
\[
\Psi_\loc=\set{x\in\reals^n:~(x,\loc)\in\Psi}.
\]
%
We denote the set of all possible trajectories of the system as
$\trajectoryset$.  We denote the set of next reachable states by every
possible transition from a set of states $\Psi$ as $\nextset{\Psi}$,
which is mathematically defined as follows.
%
\begin{definition}[Next set of reachable states]
  Let us consider a subset of states
  $\Psi\subset\reals^n\times\locations$.  The next set of reachable
  states of $\Psi$ is
%
\[
\nextset{\Psi}=\set{\htrj{t}:~\htrj{0}\in\Psi,~\maphtrj\in\trajectoryset}\subseteq.
\]
%
\end{definition}
%
So, we define the set of reachable states at a time $t$ starting from an initial
set of states $\Psi$, denoted $\reachset{t}{\Psi}$, as
follows.
%
\begin{definition}[Reachable states at a time point]
Let us consider a subset of states
$\Psi\subseteq\reals^n\times\locations$.  For all $t\in\integers_{\geq
0}$, we define $\reachset{t}{\Psi}$ inductively as follows.
%
\begin{enumerate}
\item $\reachset{0}{\Psi}=\Psi$.
\item If $t\geq 1$, $\reachset{t}{\Psi}=\nextset{\reachset{t-1}{\Psi}}$.
\end{enumerate}
%
\end{definition}
%
Based on Equation~\ref{eqn:cont-transition}, the continuous projection
of the set of reachable states of a system from a set of
states $\Psi$, after a continuous transition in a location $\loc$,
denoted $\contreach{\loc}{\Psi}$, can be represented as follows.
%
\begin{align}
&
\contreach{\loc}{\Psi}=\lt(\minsum{\linmap_\loc\lt(\Psi_q\bigcap{\ptope{\qtemp}{\lsys{\stay_\loc}}{\usys{\stay_\loc}}}\rt)}{\inputset_\loc}\rt)\bigcap
\ptope{\qtemp}{\lsys{\stay_\loc}}{\usys{\stay_\loc}}~\label{eqn:contreach}.
\end{align}
%
Similarly, based on Equation~\ref{eqn:dis-transition}, the continuous
projection of the set of reachable states of a system from a set of
states $\Psi$, after a discrete transition along an edge $\edge$,
denoted $\contreach{\edge}{\Psi}$, can be represented as follows.
%
\begin{align}
&
\contreach{\edge}{\Psi}=\lt(\minsum{\linmap_\edge\lt(\Psi_{\edge_1}\bigcap\ptope{\qtemp}{\join{\lsys{\edge}}{\lsys{\stay_{\edge_1}}}}{\meet{\usys{\edge}}{\usys{\stay_{\edge_1}}}}\rt)}{\inputset_{\edge}}\rt)\bigcap 
\ptope{\qtemp}{\lsys{\stay_{\edge_2}}}{\usys{\stay_{\edge_2}}}~\label{eqn:disreach}.
\end{align}
%
Therefore, we can represented the set of reachable states of the
system in one time step from a set of states $\Psi$ as follows.
%
\begin{align*}
& \nextset{\Psi}=\bigcup_{\loc\in\locations}\pair{\contreach{\loc}{\Psi}}{\loc}\bigcup_{\edge\in\edgeset}\pair{\contreach{\edge}{\Psi}}{\edge_2}.
\end{align*}
%
If a set of states of the system are such that its next set of states
is contained within itself, then it is called a {\it positive invariant}.
%
\begin{definition}[Positive invariant]
A subset of states $\Omega\subseteq\reals^n\times\locations$ is a
positive invariant iff $\nextset{\Omega}\subseteq\Omega$.
\end{definition}
%
The reachable set of states for all time instants can be
over-approximated by computing a positive invariant which contains the
initial set.  This is described in the following lemma.
%
\begin{lemma}[Positive invariant based over-approximation of reachable
  states]
Let us consider a positive invariant $\Omega$ such that
$\Psi\subseteq\Omega$.  Then 
%
\[
\forall t\in\integers_{\geq 0}~\reachset{t}{\Psi}\subseteq\Omega
\]
%
\end{lemma}
%
\begin{proof}
We prove the result by induction.  We have
$\reachset{0}{t}=\Psi\subseteq\Omega$ as given, which means that the
result holds for $t=0$.  Assume that for a $t\geq 0$,
$\reachset{t}{\Psi}\subseteq\Omega$.  Then we derive the following.
%
\begin{align*}
&
\reachset{t+1}{\Psi}=\nextset{\reachset{t}{\Psi}}\subseteq\nextset{\Omega}\\
& \%\%~~\text{by the definition of positive invariant}\\
& \subseteq \Omega.
\end{align*}
%
The lemma then follows by the principle of induction.
\end{proof}
%
The set of all reachable states of the system at all discrete time
instants is itself a positive invariant.  This is described in the
following lemma.
%
\begin{lemma}~\label{lem:exact-pi}
Let us consider a subset of states
$\Psi\subseteq\reals^n\times\locations$ and
%
\[
\Omega=\bigcup_{t=0}^\infty\reachset{t}{\Psi}.
\]
%
Then $\Omega$ is a positive invariant.
\end{lemma}
%
\begin{proof}
We derive the following.
%
\begin{align*}
& \nextset{\Omega}=\nextset{\bigcup_{t=0}^\infty\reachset{t}{\Psi}} = \bigcup_{t=0}^\infty\nextset{\reachset{t}{\Psi}}\\
& = \bigcup_{t=0}^\infty\reachset{t+1}{\Psi}= \bigcup_{t=1}^\infty\reachset{t}{\Psi}\\
& \subseteq  \bigcup_{t=0}^\infty\reachset{t}{\Psi}=\Omega.\hspace{3em}\qedhere
\end{align*}
%
\end{proof}
%






















