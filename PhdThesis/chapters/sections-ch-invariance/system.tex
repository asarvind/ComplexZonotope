In a discrete time affine hybrid system, the state of the system is
specified by a discrete valued variable, called location, and a
continuous variable whose valuation is in the real Euclidean space of
a finite dimenstion.  The state of the system in each location has to
stay within a polyhedral set, called the staying condition.  The state
of the system can change by two kinds of transitions, {\it continuous
  transition} and {\it discrete transiton}.  In a continuous
transition, the discrete state of the system remains constant while
the continuous state changes by an affine transformation.  The affine
transformation has possible additive disturbance input, which is
bounded.  The parameters of the affine transformation of a continuous
transition depend on the location in which the transition takes place.
In a discrete transition, there is a change in the discrete variable
accompanied by an affine transformation of the continuous variable.
The transition is has precondition specified by a linear constraint,
called a guard, while the post-condition is the staying condition in
the location reached after transition.  A set of edges specifies
the possible discrete transitions, vis a vis, the locations between
which a discrete transition takes place, the parameters of the affine
transformation and the guard.   

{\it Sub-parallelotopic guards and staying conditions}: In this paper,
we consider hybrid systems where the guards and staying conditions can
be specified by a sub-parallelotope with a common template.  We note
that the class of sub-parallelotopic constraints are quite general and
can be used in the specification of many practical affine hybrid
systems.  

{\bf Model.}  We specify the discrete time affine hybrid system
by a tuple 
 %
\[
\system = \lt(\locations,\qtemp,\stay,\linmap,\inputset,\edgeset,\Psi\rt).
\]
%
The finite set of locations is $\locations$.  The sub-parallelotopic
template for specifying the guards and staying conditions is
$\qtemp\in\mat{k}{n}{\reals}$.  The staying set in a location
$\loc\in\locations$ is a sub-parallelotope
$\ptope{\qtemp}{\lsys{\stay_\loc}}{\usys{\stay_\loc}}$, whose pair of
lower and upper interval bounds is
$\stay_\loc=\lt(\lsys{\stay_\loc},\usys{\stay_\loc}\rt)$ .  The
parameters affine transformation in a location $\loc\in\locations$
consist of a linear transformation, specified by a matrix
$\linmap_\loc\in\mat{n}{n}{\reals}$, and a bounded additive
disturbance input set $\inputset_\loc\subset\reals^n$.  The set of
edges is $E$.  An edge $\edge\in\edgeset$ is specified by a tuple
%
\[
\edge=\lt(\edge_1,\edge_2,\usys{\edge},\lsys{\edge},\linmap_\edge,\inputset_\edge\rt).
\]
%
The before and after locations of a discrete transition along an edge
$\edge$ are $\edge_1,\edge_2$.  The guard on the transition along the
edge $\edge$ is the sub-parallelotope
$\ptope{\qtemp}{\lsys{\edge}}{\usys{\edge}}$, whose pair of
lower and upper interval bounds is
$\lt({\lsys{\edge}},{\usys{\edge}}\rt)$.
The parameters of the affine transformation for the discrete
transition along the edge $\edge$ consists of a linear map specified
by the matrix $\linmap_\edge\in\mat{n}{n}{\reals}$ and a bounded
additive disturbance input set $\inputset_\edge\subset\reals^n$.  The
set of initial states of the system is $\Psi\subseteq\locations\times\reals^n$.

{\bf Dynamics.}  A state of the hybrid system is a pair $(x,\loc)$,
where $x\in\reals^n$, called the {\it continuous state}, and
$\loc\in\locations$, called the {\it discrete state}.  A {\it
  trajectory} specifies the evolution of the state of the system as a
function of discrete time instants.  A trajectory is a function
$\maphtrj:\integers_{\geq 0}\ra\reals^n\times\locations$, such that
$\forall t\in\integers_{\geq 0}$, one of the following conditions is
true.
%
\begin{enumerate}
\item todo
\item todo
\end{enumerate}
%































{\color{red} TODO.}