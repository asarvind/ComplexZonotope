To prove a linear invariance property, it follows from
Lemma~\ref{lem:pi-ver} that we can compute a positive invariant
containing the initial set and which satisfies the property.  We can
also infer from Lemma~\ref{lem:pi-ver} that the smallest positive
invariant is the union of reachable sets at different all time points.
However, the smallest positive invariant can be computationally
expensive to represent because reachable set at a time step is a union
of sets whose number can be exponential in the number of the time
step.  Alternatively, we can compute a positive invariant whose
continuous projection in any location is represented as an augmented
complex zonotope with a representation size that is amenable to
computation.  We have earlier ellucidated the advantage of using a
complex zonotope over other set representations, for computing a
positive invariant of an affine hybrid system.  We shall now discuss
how to compute the required augmented complex zonotopic positive
invariant for verifying a linear invariance property.

In the subsequent discussion, we consider a set of states $\Omega$ whose
projection onto continuous states in a location $\loc$ is an augmented
complex zonotope
%
\[
\Omega_\loc=\acztope{\ptemp_\loc}{\cen_\loc}{\sfact_\loc}{\pinv{\qtemp}}{\lb_\loc}{\ub_\loc}.
\]
%
We apriori fix the primary and secondary templates of the augmented
complex zonotope and compute a center, scaling factors and lower and
upper interval bounds such that the augmented complex zonotope
satisfies a given linear invariance property.  Above we have chosen
the secondary template as the pseudo-inverse of the sub-parallelotopic
template used to define the guards and staying conditions.  This is
because we can then compute an over-approximation of the intersection
between the augmented complex zonotope and the guards and staying
conditions, using Theorem~\ref{thm:acz-ptope-intersection}.  The choice
of the primary template will be explained later.  Firstly, we shall
derive convex conditions on the rest of variables for verifying a
linear invariance property.


For $\Omega$ to be a positive invariant, it is necessary that the set
of continuous reachable states after a continuous transition in a
location are contained within the continuous states of $\Omega$ in the
location.  In this regard, let us consider the following relation.  In
the below definition, recall the notation "$\order$" in
Definition~\ref{defn:inclusion-acz}, which is a sufficient condition
for checking inclusion between two augmented complex zonotopes.  We
extend meaning of the notation as follows.
%
\begin{definition}
We say that $\contreach{\loc}{\Omega}\order\Omega_\loc$ iff all of the
following is collectively true.
%
\begin{align*}

\end{align*}
%
\end{definition}
% 