By Lemma~\ref{lem:pi-ver}, to prove a linear invariance property, we
can compute a positive invariant containing the initial set and which
satisfies the property.  We shall derive a convex program to compute a
positively invariant satisfying a linear invariance property and
containing an initial set, whose continuous projection in any location
is represented as an augmented complex zonotope.  Our procedure
apriori fixes the templates of the augmented complex zonotope and
synthesizes the set of scaling factors, and lower and upper interval
bounds for verifying the property.  We shall discuss in a latter
section how to select a suitable primary template.  But the secondary
template has to be the pseudo-inverse of the sub-parallelotopic
template of the system, so that we can over-approximate the
intersection with the guards and staying conditions based on
Theorem~\ref{thm:main-intersection}.

Therefore, we consider a set of states $\Omega$ whose projection onto
continuous states in a location $\loc$ is an augmented complex
zonotope specified as
%
\[
\Omega_\loc=\acztope{\ptemp}{\cen_\loc}{\sfact_\loc}{\pinv{\qtemp}}{\lb_\loc}{\ub_\loc}.
\]
%
where $\ptemp\in\mat{n}{m}{\compnums}$.  Furthermore, we require a
condition the following condition on the templates for computing a
sound intersection with the guards and staying conditions, based on
Theorem~\ref{thm:main-intersection}.
%
\begin{equation}~\label{eqn:boxcenter}
\begin{split}
 & \forall
 i\in\set{1,....,k},~\forall \loc\in\locations\\
 & \tcztope{\compid{i}\qtemp\ptemp}{\compid{i}\qtemp\cen_\loc}{\sfact_\loc}\order\tcztope{\qtemp\ptemp}{\qtemp\cen_\loc}{\sfact_\loc}
 \end{split}
\end{equation}
%
We consider an
over-approximation of the input
disturbance set in any transition function, by an
augmented complex zonotope, as follows.

%
\begin{align}
& \forall
\loc\in\locations,~\inputset_\loc\subseteq\acztope{\inputtemp_\loc}{\inputcen_\loc}{\inputsfact_\loc}{\inputstemp_\loc}{\inputlb_\loc}{\inputub_\loc}\\
& \forall \edge\in\edgeset,~\inputset_\edge\subseteq\acztope{\inputtemp_\edge}{\inputcen_\edge}{\inputsfact_\edge}{\inputstemp_\edge}{\inputlb_\edge}{\inputub_\edge}.
\end{align}
%
{\bf Min and Max-approximation functions:} The over-approximation of
an intersection between an augmented complex zonotope and a
sub-parallelotope, given in Theorem~\ref{thm:main-intersection}
requires computing componentwise minimum (meet) and maximum (join) of
two real vectors.  The meet and join operations on real vectors are in
general not affine functions of the arguments.  Since we are
interested in deriving a convex program, we want to find an affine
expression for the meet and join operations.  In this regard, we
observe that under a certain affine constraint on the variables, the
meet and join operations can be expressed as affine expressions of the
variables.  We consider two affine functions
called \emph{min-approximation} and \emph{max-approximation}
functions, defined as follows.  A function
$\minapproxsymbol:\reals^k\times\lt(\reals\cup\infty\rt)^k$ is a
min-approximation function if for all $i\in\set{1,...,k}$
%
\[
\lt(\minapprox{\ub}{\pub}\rt)_i=
\lt\{
\begin{array}{l}
\ub_i~\text{if}~\pub_i=\infty\\
\pub_i~\text{if}~\pub_i<\infty
\end{array}
\rt.
\]
%
By the above definition, $\minapprox{\ub}{\pub}$ is an affine function of its
first argument $\ub$, and is a finite valued real vector.
%
\begin{equation}~\label{eqn:minapprox}
\minapprox{\ub}{\pub}\geq\meet{\ub}{\pub}.
\end{equation}
%
Similarly, a function
$\maxapproxsymbol:\reals^k\times\lt(\reals\cup\lt(-\infty\rt)^k\rt)$
is a max-approximation function if for all $i\in\set{1,...,k}$
%
\[
\lt(\maxapprox{\lb}{\plb}\rt)_i=
\lt\{
\begin{array}{l}
\lb_i~\text{if}~\plb_i=-\infty\\
\plb_i~\text{if}~\plb_i>-\infty
\end{array}
\rt.
\]
%
By the above definition, $\maxapprox{\lb}{\plb}$ is an affine function
of its first argument $\lb$, and is a finite valued real vector.
%
\begin{equation}~\label{eqn:minapprox}
\minapprox{\lb}{\plb}\geq\join{\lb}{\plb}.
\end{equation}
%
The following lemma states an affine condition when the meet and join
operations can be equivalently computed by min and max-approximation
functions, respectively.
%
\begin{lemma}~\label{lem:min-max-approximation}
Let us consider vectors $\lb,\ub\in\reals^k$ and
$\plb,\pub\in\set{\reals,-\infty,\infty}^k$.  If
$\lb\leq\maxapprox{\lb}{\plb}\leq\minapprox{\ub}{\pub}\leq\ub$, then
%
\begin{align*}
& \join{\lb}{\plb}=\maxapprox{\lb}{\plb}~\numberthis\label{eqn:max1}\\
& \meet{\ub}{\pub}=\minapprox{\ub}{\pub}.~\numberthis\label{eqn:max2}
\end{align*}
%
\end{lemma}
%
\begin{proof}
For any $i\in\set{1,...,k}$, we derive results for the following four cases

{\it Case 1: }  Let us consider that $\plb_i>-\infty$. So,
$\lt(\maxaffine{\lb}{\plb}\rt)_i=\plb_i$.  Then by
Equation~\ref{eqn:max1}, we get
%
\begin{align*}
& \lt(\join{\lb}{\plb}\rt)_i=\plb_i=\lt(\maxaffine{\lb}{\plb}\rt)_i.
\end{align*}
%

{Case 2: }  Let us consider that $\plb_i=-\infty$.  Then
$\lt(\maxaffine{\lb}{\plb}\rt)_i=\lb_i$.   Then by
Equation~\ref{eqn:max1}, we get
%
%
\begin{align*}
& \lt(\join{\lb}{\plb}\rt)_i=\lb_i=\lt(\maxaffine{\lb}{\plb}\rt)_i.
\end{align*}
%

{\it Case 3: }  Let us consider that $\pub_i<\infty$. So,
$\lt(\minaffine{\ub}{\pub}\rt)_i=\pub_i$.  Then by
Equation~\ref{eqn:max2}, we get
%
\begin{align*}
& \lt(\meet{\ub}{\pub}\rt)_i=\plb_i=\lt(\minaffine{\ub}{\pub}\rt)_i.
\end{align*}
%

{Case 4: }  Let us consider that $\pub_i=\infty$.  Then
$\lt(\minaffine{\ub}{\pub}\rt)_i=\ub_i$.  Then by
Equation~\ref{eqn:max2}, we get
%
\begin{align*}
& \lt(\meet{\ub}{\pub}\rt)_i=\ub_i=\lt(\minaffine{\ub}{\pub}\rt)_i.
\end{align*}
%
\end{proof}
%
{\bf Deriving sufficient conditions for positive invariance.}  We
introduce the following condition, which along with
Equation~\ref{eqn:boxcenter} is sufficient for the inclusion of the
set of continuous reachable states $\contreach{\loc}{\Omega}$ inside
$\Omega_\loc$.  We shall prove this inclusion in
Lemma~\ref{lem:pi-cont}.
%
\begin{definition}
  For $\loc\in\locations$, we say that
  $\posinv{\Omega}{q}$ iff  all of the following is
  collectively true.
%
\begin{align}
& \exists l^\pr,u^\pr,l^\dpr,u^\dpr\in\reals^k:\nonumber\\
& \lb_\loc\leq\maxapprox{\lb_\loc}{\lsys{\stay_\loc}}\leq\minapprox{\ub_\loc}{\usys{\stay_\loc}}\leq\ub_{\loc},~\label{eqn:continv1}\\
& l^\pr= \maxapprox{\lb_\loc}{\lsys{\stay_\loc}},~~u^\pr= \minapprox{\ub_\loc}{\usys{\stay_\loc}},~\label{eqn:continv2}\\
& \acztope{
\mymatrix{\linmap_\loc\ptemp_\loc & \inputtemp_\loc}
}
{
\cen_\loc+\inputcen_\loc
}
{
\mymatrix{\sfact_\loc\\ \inputsfact_\loc}
}
{
\mymatrix{\linmap_\loc\qtemp & \inputstemp_\loc}
}
{
\mymatrix{l^\pr\\\inputlb_\loc}
}
{
\mymatrix{u^\pr\\ \inputub_\loc}
},\nonumber\\
& \order 
\acztope{\ptemp_\loc}{\cen_\loc}{\sfact_\loc}{\qtemp}{l^\dpr}{u^\dpr},~\label{eqn:continv3}\\
&
l^\dpr\leq\maxapprox{l^\dpr}{\lsys{\stay_\loc}}\leq\minapprox{u^\dpr}{\usys{\stay_\loc}}\leq\ub^\dpr,~\label{eqn:continv4}\\
& \lb_\loc\leq\maxapprox{l^\dpr}{\lsys{\stay_\loc}},~~\minapprox{u^\dpr}{\usys{\stay_\loc}}\leq\ub_\loc~\label{eqn:continv5}.
\end{align}
%
\end{definition}
%
\begin{lemma}~\label{lem:pi-cont}
For a location $\loc\in\locations$, if $\posinv{\Omega}{\loc}$, and
Equation~\ref{eqn:boxcenter} holds, then
$\contreach{\loc}{\Omega}\subseteq\Omega$.
\end{lemma}
%
\begin{proof}
By Theorem~\ref{thm:main-intersection},
Equations~\ref{eqn:continv1},~\ref{eqn:continv2},~\ref{eqn:boxcenter}
and Lemma~\ref{lem:min-max-approximation}, we get
%
\begin{align*}
& \Omega_q\bigcap{\ptope{\qtemp}{\lsys{\stay_\loc}}{\usys{\stay_\loc}}}
  \subseteq \acztope{\ptemp_\loc}{\cen_\loc}{\sfact_\loc}{\qtemp}{l^\pr}{u^\pr}.
\end{align*}
%
Using the above over-approximation and the expressions for the linear
transformation and Minkowski sum of augmented complex zonotopes, we
get
%
\begin{align}
& \minsum{A\lt(\Omega_q\bigcap{\ptope{\qtemp}{\lsys{\stay_\loc}}{\usys{\stay_\loc}}}\rt)}{\inputset_\loc}\nonumber\\
& \subseteq \acztope{
\mymatrix{\linmap_\loc\ptemp_\loc & \inputtemp_\loc}
}
{
\cen_\loc+\inputcen_\loc
}
{
\mymatrix{\sfact_\loc\\ \inputsfact_\loc}
}
{
\mymatrix{\linmap_\loc\qtemp & \inputstemp_\loc}
}
{
\mymatrix{l^\pr\\\inputlb_\loc}
}
{
\mymatrix{u^\pr\\ \inputub_\loc}
}\nonumber\\
& \%\%~~\text{by Theorem~\ref{thm:acz-inclusion} and Equation~\ref{eqn:continv3}}\nonumber\\
& \subseteq \acztope{\ptemp_\loc}{\cen_\loc}{\sfact_\loc}{\qtemp}{l^\dpr}{u^\dpr}~\label{eqn:proof-continv1}.
\end{align}
%
Again by Theorem~\ref{thm:main-intersection},
Equations~\ref{eqn:continv4},~\ref{eqn:boxcenter}
and Lemma~\ref{lem:min-max-approximation}, we
get
%
\begin{align}
& \acztope{\ptemp_\loc}{\cen_\loc}{\sfact_\loc}{\qtemp}{l^\dpr}{u^\dpr}\bigcap\ptope{\qtemp}{\lsys{\stay_{\edge_2}}}{\usys{\stay_{\edge_2}}}\nonumber\\
& \subseteq \acztope{\ptemp_\loc}{\cen_\loc}{\sfact_\loc}{\qtemp}{\maxapprox{l^\dpr}{\lsys{\stay_\loc}}}{\minapprox{u^\dpr}{\usys{\stay_\loc}}}\nonumber\\
& \%\%~~\text{by Equation~\ref{eqn:continv5}}\nonumber\\
& \subseteq \acztope{\ptemp_\loc}{\cen_\loc}{\sfact_\loc}{\qtemp}{\lb_\loc}{\ub_\loc}=\Omega_\loc~\label{eqn:proof-continv2}.
\end{align}
%
Using Equations~\ref{eqn:contreach},~\ref{eqn:proof-continv1},
and~\ref{eqn:proof-continv2}, we get
$\contreach{\loc}{\Omega}\subseteq \Omega_\loc$.
\end{proof}
%
For the set of continuous states reached from $\Omega$ after a
discrete transition, $\contreach{\edge}{\Omega}$, to be contained
within 
$\Omega_{\edge_2}$, we introduce the following condition.  We shall
show in Lemma~\ref{lem:pi-dis} that this condition is sufficient for
$\contreach{\edge}{\Omega}\subseteq\Omega_{\edge_2}$.
%
\begin{definition}
For an edge $\edge\in\edgeset$, we say that $\posinv{\Omega}{\edge}$
iff all of the following is collectively true.
%
\begin{align}
& \exists l^\pr,u^\pr,l^\dpr,u^\dpr\in\reals^k:\nonumber\\
& \lb_{\edge_1}\leq \maxapprox{\lb_{\edge_1}}{\join{\lsys{\edge}}{\lsys{\stay_{\edge_1}}}}\leq
\minapprox{\ub_{\edge_1}}{\meet{\usys{\edge}}{\usys{\stay_{\edge_1}}}}\leq\ub_{\edge_1}~\label{eqn:disinv1}\\
&
l^\pr=\maxapprox{\lb_{\edge_1}}{\join{\lsys{\edge}}{\lsys{\stay_{\edge_1}}}},~~
u^\pr=\minapprox{\ub_{\edge_1}}{\meet{\usys{\edge}}{\usys{\stay_{\edge_1}}}}~\label{eqn:disinv2}\\
& \acztope{
\mymatrix{\linmap_{\edge}\ptemp_{\edge_1} & \inputtemp_{\edge}}
}
{
\cen_{\edge_1}+\inputcen_{\edge}
}
{
\mymatrix{\sfact_{\edge_1}\\ \inputsfact_{\edge}}
}
{
\mymatrix{\linmap_{\edge}\qtemp & \inputstemp_{\edge}}
}
{
\mymatrix{l^\pr\\\inputlb_{\edge}}
}
{
\mymatrix{u^\pr\\ \inputub_{\edge}}
},\nonumber\\
& \order 
\acztope{\ptemp_{\edge_2}}{\cen_{\edge_2}}{\sfact_{\edge_2}}{\qtemp}{l^\dpr}{u^\dpr},~\label{eqn:disinv3}\\
&
l^\dpr\leq\maxapprox{l^\dpr}{\lsys{\stay_{\edge_2}}}\leq\minapprox{u^\dpr}{\usys{\stay_{\edge_2}}}\leq
u^\dpr,~\label{eqn:disinv4}\\
& \lb_{\edge_2}\leq\maxapprox{l^\dpr}{\lsys{\stay_{\edge_2}}},~~\minapprox{u^\dpr}{\usys{\stay_{\edge_2}}}\leq\ub_{\edge_2}~\label{eqn:disinv5}.
\end{align}
%
\end{definition}
%
\begin{lemma}~\label{lem:pi-dis}
For an edge $\edge\in\edgeset$, if $\posinv{\Omega}{\edge}$, then $\contreach{\edge}{\Omega}\subseteq\Omega_{\edge_2}$.
\end{lemma}
%
\begin{proof}
By Theorem~\ref{thm:main-intersection},
Equations~\ref{eqn:disinv1},~\ref{eqn:disinv2},~\ref{eqn:boxcenter},
and Lemma~\ref{lem:min-max-approximation} we get
%
\begin{align*}
& \Omega_{\edge_1}\bigcap\ptope{\qtemp}{\join{\lsys{\edge}}{\lsys{\stay_{\edge_1}}}}{\meet{\usys{\edge}}{\usys{\stay_{\edge_1}}}}
\subseteq \acztope{\ptemp_{\edge_1}}{\cen_{\edge_1}}{\sfact_{\edge_1}}{\qtemp}{l^\pr}{u^\pr}.
\end{align*}
%
Using the above over-approximation and the expressions for the linear
transformation and Minkowski sum of augmented complex zonotopes, we
get
%
\begin{align}
& \minsum{A\lt(\Omega_{\edge_1}\bigcap\ptope{\qtemp}{\join{\lsys{\edge}}{\lsys{\stay_{\edge_1}}}}{\meet{\usys{\edge}}{\usys{\stay_{\edge_1}}}}\rt)}{\inputset_{\edge}}\nonumber\\
& \subseteq \acztope{
\mymatrix{\linmap_{\edge_1}\ptemp_{\edge_1} & \inputtemp_\edge}
}
{
\cen_{\edge_1}+\inputcen_\edge
}
{
\mymatrix{\sfact_{\edge_1}\\ \inputsfact_\edge}
}
{
\mymatrix{\linmap_{\edge_1}\qtemp & \inputstemp_\edge}
}
{
\mymatrix{l^\pr\\\inputlb_\edge}
}
{
\mymatrix{u^\pr\\ \inputub_\edge}
}\nonumber\\
& \%\%~~\text{by Theorem~\ref{thm:acz-inclusion} and Equation~\ref{eqn:disinv3}}\nonumber\\
& \subseteq \acztope{\ptemp_{\edge_2}}{\cen_{\edge_2}}{\sfact_{\edge_2}}{\qtemp}{l^\dpr}{u^\dpr}~\label{eqn:proof-dis1}.
\end{align}
%
Again by Theorem~\ref{thm:main-intersection},
Equations~\ref{eqn:disinv4},~\ref{eqn:boxcenter} and Lemma~\ref{lem:min-max-approximation}, we get
%
\begin{align}
& \acztope{\ptemp_{\edge_2}}{\cen_{\edge_2}}{\sfact_{\edge_2}}{\qtemp}{l^\dpr}{u^\dpr}\bigcap\ptope{\qtemp}{\lsys{\stay_{\edge_2}}}{\usys{\stay_{\edge_2}}}\nonumber\\
& \subseteq \acztope{\ptemp_{\edge_2}}{\cen_{\edge_2}}{\sfact_{\edge_2}}{\qtemp}{\maxapprox{l^\dpr}{\lsys{\stay_{\edge_2}}}}{\minapprox{u^\dpr}{\usys{\stay_{\edge_2}}}}\nonumber\\
& \%\%~~\text{by Equation~\ref{eqn:disinv5}}\nonumber\\
& \subseteq \acztope{\ptemp_{\edge_2}}{\cen_{\edge_2}}{\sfact_{\edge_2}}{\qtemp}{\lb_{\edge_2}}{\ub_{\edge_2}}=\Omega_{\edge_2}~\label{eqn:proof-dis2}.
\end{align}
%
Using Equations~\ref{eqn:disreach},~\ref{eqn:proof-dis1}
and~\ref{eqn:proof-dis2}, we get $\contreach{\edge}{\Psi}
\subseteq \Omega_{\edge_2}$.
\end{proof}
%
The following condition is sufficient to check a linear
invariance property.
%
\begin{theorem}~\label{thm:maintheorem}
We get $\lt(\system,\Psi\rt)\models\invariance{T}{d}$ if all of the
following is true.
%
\begin{align*}
& \forall
 i\in\set{1,....,k},~\forall \loc\in\locations,~\forall\edge\in\edgeset\\
& \tcztope{\compid{i}\qtemp\ptemp}{\compid{i}\qtemp\cen_\loc}{\sfact_\loc}\order\tcztope{\qtemp\ptemp}{\qtemp\cen_\loc}{\sfact_\loc},\\
& \posinv{\Omega}{\loc}~\wedge~\posinv{\Omega}{\edge},\\
& T\lt(\cen_\loc+\pinv{\qtemp}\frac{\lb_\loc+\ub_\loc}{2}\rt)+\absolute{T\mymatrix{\ptemp
& \pinv{\qtemp}}}\mymatrix{\sfact_\loc\\\frac{\ub_\loc-\lb_\loc}{2}}\leq d~\numberthis\label{eqn:mainsup}.
\end{align*}
%
\end{theorem}
%
\begin{proof}
By Lemmas~\ref{lem:pi-cont} and~\ref{lem:pi-dis}, we get that $\Omega$ is
a positive invariant.  According to the expression for support
function in Lemma~\ref{lem:support-acz} which matches the last part of
Equation~\ref{eqn:mainsup}, we get that all continuous states
$x\in\Omega_\loc$ for all the locations satisfy $Tx\leq d$.  By
Lemma~\ref{lem:pi-ver}, the linear invariance property is satisfied.
\end{proof}
%
{\bf Algorithm for verification: } For fixed primary template and the
secondary template chosen as explained previously, the verification
procedure is given in Algorithm~\ref{alg:ver}.  The algorithm if
succesful guarantees the verification of a linear invariance property.
But it can not invalidate a linear invariance property because it is
based on a sufficient but not necessary condition.  It can be
implemented by a single step of
\emph{second order conic programming}.  This follows from the fact
that min and max-approximation functions are affine and the relation
``$\order$'' between complex zonotopes is equivalent to a set of
second order conic constraints on the center, scaling factors and
lower and upper interval bounds.
%
 \begin{algorithm}
\caption{Verification of linear invariance property}
\label{alg:ver}
For all $\loc\in\locations$, solve for $\cen_\loc$,
$\sfact_\loc$, $\lb_\loc$ and $\ub_\loc$ satisfying Equation~\ref{eqn:mainsup}. 
 \end{algorithm}
%

{\bf Selecting the primary template:  }  We note that adding any
arbitrary vector to a primary template increases the accuracy of the
verification procedure because the scaling factors are adjusted by the
optimizer.  However, there the computational cost also increases by
adding more vectors to a template.  Therefore, we have to select the
primary template wisely.  In this regard, we provide some suggestions
for chosing the primary template, which are based on the properties of
complex zonotopes that we derived earlier.
%
\begin{enumerate}
\item \emph{Eigenvectors:  }  We can add eigenvectors of the linear
transformation matrices and their products.  This choice is based on
Lemma~\ref{lem:eig-scaling} which states that a complex zonotope can
capture the contraction by a linear transformation along the
eigenvectors of the transformation.
\item \emph{Orthonormal vectors and their projections:  }  We can add
the orthonormal vectors in the null space of the secondary template
and the projection of template vectors in the space orthogonal to the
null space.  This is based on Theorem~\ref{thm:main-intersection}
where the upper bound on the error in over-approximation of the intersection
between an augmented complex zonotope and sub-parallelotope is
proportional to the orientation between the primary template and the
sub-parallelotopic template.
\item \emph{Template of the input set and its transformations:  }  We can add the templates
used to over-approximate the disturbance input set and its
transformations by the system matrices.  This is based on
Proposition~\ref{prop:commin} which states that the template size of
the resultant complex zonotope from the Minkowski sum of two complex
zonotopes does not increase when their templates are the same.  Since
we take Minkowski sum with the disturbance input in our verification
procedure, we expect to increase accuracy by incorporating the input
template and its transformations in the primary template.
\item Adding any vector to the primary template will increase the
accuracy because the scaling factors can be adjusted by the
optimizer.
\end{enumerate}
