To prove a linear invariance property, it follows from
Lemma~\ref{lem:pi-ver} that we can compute a positive invariant
containing the initial set and which satisfies the property.  We can
also infer from Lemma~\ref{lem:pi-ver} that the smallest positive
invariant is the union of reachable sets at different all time points.
However, the smallest positive invariant can be computationally
expensive to represent because reachable set at a time step is a union
of sets whose number can be exponential in the number of the time
step.  Alternatively, we can compute a positive invariant whose
continuous projection in any location is represented as an augmented
complex zonotope with a representation size that is amenable to
computation.  We have earlier ellucidated the advantage of using a
complex zonotope over other set representations, for computing a
positive invariant of an affine hybrid system.  We shall now discuss
how to compute the required augmented complex zonotopic positive
invariant for verifying a linear invariance property.

In the subsequent discussion, we consider a set of states $\Omega$ whose
projection onto continuous states in a location $\loc$ is an augmented
complex zonotope
%
\[
\Omega_\loc=\acztope{\ptemp_\loc}{\cen_\loc}{\sfact_\loc}{\pinv{\qtemp}}{\lb_\loc}{\ub_\loc}.
\]
%
We apriori fix the primary and secondary templates of the augmented
complex zonotope and compute a center, scaling factors and lower and
upper interval bounds such that the augmented complex zonotope
satisfies a given linear invariance property.  Above we have chosen
the secondary template as the pseudo-inverse of the sub-parallelotopic
template used to define the guards and staying conditions.  This is
because we can then compute an over-approximation of the intersection
between the augmented complex zonotope and the guards and staying
conditions, using Theorem~\ref{thm:acz-ptope-intersection}.  The choice
of the primary template will be explained later.  Firstly, we shall
derive convex conditions on the rest of variables for verifying a
linear invariance property.  Furthermore, we consider an
over-approximation of the input
disturbance set for any transition by an
augmented complex zonotope as follows.
%
\begin{align}
& \forall
\loc\in\locations,~\inputset_\loc\subseteq\acztope{\inputtemp_\loc}{\inputcen_\loc}{\inputsfact_\loc}{\inputstemp_\loc}{\inputlb_\loc}{\inputub_\loc}\\
& \forall \edge\in\edgeset,~\inputset_\edge\subseteq\acztope{\inputtemp_\edge}{\inputcen_\edge}{\inputsfact_\edge}{\inputstemp_\edge}{\inputlb_\edge}{\inputub_\edge}.
\end{align}
%

{\bf Min and Max-approximation functions:} The over-approximation of
the intersection of an augmented complex zonotope and a
sub-parallelotope given in Theorem~\ref{thm:acz-ptope-intersection}
requires computing componentwise minimum (meet) and maximum (join) of
two real vectors.  The meet and join operations on real vectors are in
general not affine functions of the arguments.  However, under a
certain affine constraint on the variables that shall be explained
later in Lemma~\ref{lem:min-max-approximation}, the meet and join operations can be
expressed as affine expressions of the variables.  In the context, we
introduce the following functions.  A function
$\minapproxsymbol:\reals^k\times\lt(\reals\cup\infty\rt)^k$ is a
min-approximation function if for all $i\in\set{1,...,k}$
%
\[
\lt(\minapprox{\ub}{\pub}\rt)_i=
\lt\{
\begin{array}{l}
\ub_i~\text{if}~\pub_i=\infty\\
\pub_i~\text{if}~\pub_i<\infty
\end{array}
\rt.
\]
%
By the above definition, $\minapprox{\ub}{\pub}$ is an affine function of its
first argument $\ub$, which is a finite valued real vector.
%
\begin{equation}~\label{eqn:minapprox}
\minapprox{\ub}{\pub}\geq\meet{\ub}{\pub}.
\end{equation}
%
Similarly, a function
$\maxapproxsymbol:\reals^k\times\lt(\reals\cup\lt(-\infty\rt)^k\rt)$
is a max-approximation function if for all $i\in\set{1,...,k}$
%
\[
\lt(\maxapprox{\lb}{\plb}\rt)_i=
\lt\{
\begin{array}{l}
\lb_i~\text{if}~\plb_i=-\infty\\
\plb_i~\text{if}~\plb_i>-\infty
\end{array}
\rt.
\]
%
By the above definition, $\maxapprox{\lb}{\plb}$ is an affine function
of its first argument $\lb$, which is a finite valued real vector.
%
\begin{equation}~\label{eqn:minapprox}
\minapprox{\lb}{\plb}\geq\join{\lb}{\plb}.
\end{equation}
%
Then, we have the following lemma about representing the meet and join
operations using min and max-approximation functions under a certain
condition.
%
\begin{lemma}~\label{lem:min-max-approximation}
Let us consider vectors $\lb,\ub\in\reals^k$ and
$\plb,\pub\in\set{\reals,-\infty,\infty}^k$.  If
$\lb\leq\maxapprox{\lb}{\plb}\leq\minapprox{\ub}{\pub}\leq\ub$, then
%
\begin{align*}
& \join{\lb}{\plb}=\maxapprox{\lb}{\plb}\\
& \meet{\ub}{\pub}=\minapprox{\ub}{\pub}.
\end{align*}
%
\end{lemma}
%
{\bf Deriving sufficient conditions for positive invariance.}  For the
set of continuous reachable states $\contreach{\loc}{\Omega}$ in any
location to be contained within the continuous states of $\Omega$, we
introduce the following condition.  We shall show in
Lemma~\ref{lem:pi-cont} that this
condition is sufficient for
$\contreach{\loc}{\Omega}\subseteq\Omega_\loc$.
%
\begin{definition}
  For $\loc\in\locations$, we say that
  $\posinv{\Omega}{q}$ iff  all of the following is
  collectively true.
%
\begin{align}
& \exists l^\pr,u^\pr,l^\dpr,u^\dpr\in\reals^k:\nonumber\\
& \lb_\loc\leq\maxapprox{\lb_\loc}{\lsys{\stay_\loc}},~~\ub_\loc\geq\minapprox{\ub_\loc}{\usys{\stay_\loc}},~\label{eqn:continv1}\\
& l^\pr= \maxapprox{\lb_\loc}{\lsys{\stay_\loc}},~~u^\pr= \minapprox{\ub_\loc}{\usys{\stay_\loc}},~\label{eqn:continv2}\\
& \acztope{
\mymatrix{\linmap_\loc\ptemp_\loc & \inputtemp_\loc}
}
{
\cen_\loc+\inputcen_\loc
}
{
\mymatrix{\sfact_\loc\\ \inputsfact_\loc}
}
{
\mymatrix{\linmap_\loc\qtemp & \inputstemp_\loc}
}
{
\mymatrix{l^\pr\\\inputlb_\loc}
}
{
\mymatrix{u^\pr\\ \inputub_\loc}
},\nonumber\\
& \order 
\acztope{\ptemp_\loc}{\cen_\loc}{\sfact_\loc}{\qtemp}{l^\dpr}{u^\dpr},~\label{eqn:continv3}\\
&
l^\dpr\leq\maxapprox{l^\dpr}{\lsys{\stay_\loc}},~~u^\dpr\geq\minapprox{u^\dpr}{\usys{\stay_\loc}},~\label{eqn:continv4}\\
& \lb_\loc\leq\maxapprox{l^\dpr}{\lsys{\stay_\loc}},~~\minapprox{u^\dpr}{\usys{\stay_\loc}}\leq\ub_\loc~\label{eqn:continv5}.
\end{align}
%
\end{definition}
%
\begin{lemma}~\label{lem:pi-cont}
For a location $\loc\in\locations$, if $\posinv{\Omega}{\loc}$, then
$\contreach{\loc}{\Omega}\subseteq\Omega$.
\end{lemma}
%
\begin{proof}
Based on Equations~\ref{eqn:continv1},~\ref{eqn:continv2}, and Lemma~\ref{lem:min-max-approximation}, we get
%
\begin{align*}
& l^\pr=\join{\lb_\loc}{\lsys{\stay_\loc}},~~u^\pr=\meet{\ub_\loc}{\usys{\stay_\loc}}
\end{align*}
%
Then by Theorem~\ref{thm:acz-ptope-intersection}, we get
%
\begin{align*}
& \Omega_q\bigcap{\ptope{\qtemp}{\lsys{\stay_\loc}}{\usys{\stay_\loc}}}=\acztope{\ptemp_\loc}{\cen_\loc}{\sfact_\loc}{\qtemp}{\lb_\loc}{\ub_\loc}\bigcap\ptope{\qtemp}{\lsys{\stay_\loc}}{\usys{\stay_\loc}}\nonumber\\
& \subseteq \acztope{\ptemp_\loc}{\cen_\loc}{\sfact_\loc}{\qtemp}{l^\pr}{u^\pr}.
\end{align*}
%
Using the above over-approximation and the expressions for the linear
transformation and Minkowski sum of augmented complex zonotopes, we
get
%
\begin{align}
& \minsum{A\lt(\Omega_q\bigcap{\ptope{\qtemp}{\lsys{\stay_\loc}}{\usys{\stay_\loc}}}\rt)}{\inputset_\loc}\nonumber\\
& \subseteq \acztope{
\mymatrix{\linmap_\loc\ptemp_\loc & \inputtemp_\loc}
}
{
\cen_\loc+\inputcen_\loc
}
{
\mymatrix{\sfact_\loc\\ \inputsfact_\loc}
}
{
\mymatrix{\linmap_\loc\qtemp & \inputstemp_\loc}
}
{
\mymatrix{l^\pr\\\inputlb_\loc}
}
{
\mymatrix{u^\pr\\ \inputub_\loc}
}\nonumber\\
& \%\%~~\text{by Equation~\ref{eqn:continv3}}\nonumber\\
& \subseteq \acztope{\ptemp_\loc}{\cen_\loc}{\sfact_\loc}{\qtemp}{l^\dpr}{u^\dpr}~\label{eqn:proof-continv1}.
\end{align}
%
Based on Lemma~\ref{lem:min-max-approximation}, we get
%
\begin{align*}
& \join{l^\dpr}{\lsys{\stay_\loc}}=\maxapprox{l^\dpr}{\lsys{\stay_\loc}},~~
\meet{u^\dpr}{\usys{\stay_\loc}}=\minapprox{u^\dpr}{\usys{\stay_\loc}}.
\end{align*}
%
By Theorem~\ref{thm:acz-ptope-intersection} and the above equation, we get
%
\begin{align}
& \acztope{\ptemp_\loc}{\cen_\loc}{\sfact_\loc}{\qtemp}{l^\dpr}{u^\dpr}\bigcap\ptope{\qtemp}{\lsys{\stay_{\edge_2}}}{\usys{\stay_{\edge_2}}}\nonumber\\
& \subseteq \acztope{\ptemp_\loc}{\cen_\loc}{\sfact_\loc}{\qtemp}{\maxapprox{l^\dpr}{\lsys{\stay_\loc}}}{\minapprox{u^\dpr}{\usys{\stay_\loc}}}\nonumber\\
& \%\%~~\text{by Equation~\ref{eqn:continv5}}\nonumber\\
& \subseteq \acztope{\ptemp_\loc}{\cen_\loc}{\sfact_\loc}{\qtemp}{\lb_\loc}{\ub_\loc}=\Omega_\loc~\label{eqn:proof-continv2}.
\end{align}
%
Using Equations~\ref{eqn:proof-continv2} and~\ref{eqn:proof-continv1},
we get
\begin{align*}
& \contreach{\loc}{\Omega}=\lt(\minsum{A\lt(\Omega_q\bigcap{\ptope{\qtemp}{\lsys{\stay_\loc}}{\usys{\stay_\loc}}}\rt)}{\inputset_\loc}\rt)\bigcap
\ptope{\qtemp}{\lsys{\stay_{\edge_2}}}{\usys{\stay_{\edge_2}}}\\
& \subseteq \Omega_\loc.\hspace{25em}\qedhere
\end{align*}
%
\end{proof}
%
If we consider the templates of the augmented complex zonotope as
constants, then the above condition is equivalent to a set of second
order conic constraints on other parameters of the augmented complex
zonotope and some auxillary variables.  This is stated in the
following lemma.
%
\begin{lemma}
For constant $\ptemp_{\loc}$ and $\qtemp$, the relation
$\posinv{\Omega}{\loc}$ is equivalent to a set of second order conic
constraints on $\cen_\loc$, $\sfact_\loc$, $\lb_\loc$ and $\ub_\loc$
and some additional variables.
\end{lemma}
%
\begin{proof}
{\color{red} TODO}.
\end{proof}
%
For the set of continuous states reached from $\Omega$ after a
discrete transition, $\contreach{\edge}{\Omega}$, to be contained
within 
$\Omega_{\edge_2}$, we introduce the following condition.  We shall
show in Lemma~\ref{lem:pi-dis} that this condition is sufficient for
$\contreach{\edge}{\Omega}\subseteq\Omega_{\edge_2}$.
%
\begin{definition}
For an edge $\edge\in\edgeset$, we say that $\posinv{\Omega}{\edge}$
iff all of the following is collectively true.
%
\begin{align}
& \exists l^\pr,u^\pr,l^\dpr,u^\dpr\in\reals^k:\nonumber\\
& \lb_{\edge_1}\leq \maxapprox{\lb_{\edge_1}}{\join{\lsys{\edge}}{\lsys{\stay_{\edge_1}}}},~~
\ub_{\edge_1}\geq\minapprox{\ub_{\edge_1}}{\meet{\usys{\edge}}{\usys{\stay_{\edge_1}}}}~\label{eqn:disinv1}\\
&
l^\pr=\maxapprox{\lb_{\edge_1}}{\join{\lsys{\edge}}{\lsys{\stay_{\edge_1}}}},~~
u^\pr=\minapprox{\ub_{\edge_1}}{\meet{\usys{\edge}}{\usys{\stay_{\edge_1}}}}~\label{eqn:disinv2}\\
& \acztope{
\mymatrix{\linmap_{\edge}\ptemp_{\edge_1} & \inputtemp_{\edge}}
}
{
\cen_{\edge_1}+\inputcen_{\edge}
}
{
\mymatrix{\sfact_{\edge_1}\\ \inputsfact_{\edge}}
}
{
\mymatrix{\linmap_{\edge_1}\qtemp & \inputstemp_{\edge}}
}
{
\mymatrix{l^\pr\\\inputlb_{\edge}}
}
{
\mymatrix{u^\pr\\ \inputub_{\edge}}
},\nonumber\\
& \order 
\acztope{\ptemp_{\edge_2}}{\cen_{\edge_2}}{\sfact_{\edge_2}}{\qtemp}{l^\dpr}{u^\dpr},~\label{eqn:disinv3}\\
&
l^\dpr\leq\maxapprox{l^\dpr}{\lsys{\stay_{\edge_2}}},~~u^\dpr\geq\minapprox{u^\dpr}{\usys{\stay_{\edge_2}}},~\label{eqn:disinv4}\\
& \lb_{\edge_2}\leq\maxapprox{l^\dpr}{\lsys{\stay_{\edge_2}}},~~\minapprox{u^\dpr}{\usys{\stay_{\edge_2}}}\leq\ub_{\edge_2}~\label{eqn:disinv5}.
\end{align}
%
\end{definition}
%
\begin{lemma}~\label{lem:pi-dis}
For an edge $\edge\in\edgeset$, if $\posinv{\Omega}{\edge}$, then $\contreach{\edge}{\Omega}\subseteq\Omega_{\edge_2}$.
\end{lemma}
%
\begin{proof}
Based on Equations~\ref{eqn:disinv1},~\ref{eqn:disinv2}, and Lemma~\ref{lem:min-max-approximation}, we get
%
\begin{align*}
&
l^\pr=\join{\lb_{\edge_1}}{\join{\lsys{\edge}}{\lsys{\stay_{\edge_1}}}},
~~u^\pr=\meet{\ub_{\edge_1}}{\meet{\usys{\edge}}{\usys{\stay_{\edge_1}}}}
\end{align*}
%
Then by Theorem~\ref{thm:acz-ptope-intersection}, we get
%
\begin{align*}
& \Omega_{\edge_1}\bigcap\ptope{\qtemp}{\join{\lsys{\edge}}{\lsys{\stay_{\edge_1}}}}{\meet{\usys{\edge}}{\usys{\stay_{\edge_1}}}}\\
& =\acztope{\ptemp_{\edge_1}}{\cen_{\edge_1}}{\sfact_{\edge_1}}{\qtemp}{\lb_{\edge_1}}{\ub_{\edge_1}}\bigcap\ptope{\qtemp}{\join{\lsys{\edge}}{\lsys{\stay_{\edge_1}}}}{\meet{\usys{\edge}}{\usys{\stay_{\edge_1}}}}\\
& \subseteq \acztope{\ptemp_{\edge_1}}{\cen_{\edge_1}}{\sfact_{\edge_1}}{\qtemp}{l^\pr}{u^\pr}.
\end{align*}
%
Using the above over-approximation and the expressions for the linear
transformation and Minkowski sum of augmented complex zonotopes, we
get
%
\begin{align}
& \minsum{A\lt(\Omega_{\edge_1}\bigcap\ptope{\qtemp}{\join{\lsys{\edge}}{\lsys{\stay_{\edge_1}}}}{\meet{\usys{\edge}}{\usys{\stay_{\edge_1}}}}\rt)}{\inputset_{\edge}}\nonumber\\
& \subseteq \acztope{
\mymatrix{\linmap_{\edge_1}\ptemp_{\edge_1} & \inputtemp_\edge}
}
{
\cen_{\edge_1}+\inputcen_\edge
}
{
\mymatrix{\sfact_{\edge_1}\\ \inputsfact_\edge}
}
{
\mymatrix{\linmap_{\edge_1}\qtemp & \inputstemp_\edge}
}
{
\mymatrix{l^\pr\\\inputlb_\edge}
}
{
\mymatrix{u^\pr\\ \inputub_\edge}
}\nonumber\\
& \%\%~~\text{by Equation~\ref{eqn:continv3}}\nonumber\\
& \subseteq \acztope{\ptemp_{\edge_2}}{\cen_{\edge_2}}{\sfact_{\edge_2}}{\qtemp}{l^\dpr}{u^\dpr}~\label{eqn:proof-continv1}.
\end{align}
%
Based on Lemma~\ref{lem:min-max-approximation}, we get
%
\begin{align*}
& \join{l^\dpr}{\lsys{\stay_{\edge_2}}}=\maxapprox{l^\dpr}{\lsys{\stay_{\edge_2}}},~~
\meet{u^\dpr}{\usys{\stay_{\edge_2}}}=\minapprox{u^\dpr}{\usys{\stay_{\edge_2}}}.
\end{align*}
%
By Theorem~\ref{thm:acz-ptope-intersection} and the above equation, we get
%
\begin{align}
& \acztope{\ptemp_{\edge_2}}{\cen_{\edge_2}}{\sfact_{\edge_2}}{\qtemp}{l^\dpr}{u^\dpr}\bigcap\ptope{\qtemp}{\lsys{\stay_{\edge_2}}}{\usys{\stay_{\edge_2}}}\nonumber\\
& \subseteq \acztope{\ptemp_{\edge_2}}{\cen_{\edge_2}}{\sfact_{\edge_2}}{\qtemp}{\maxapprox{l^\dpr}{\lsys{\stay_{\edge_2}}}}{\minapprox{u^\dpr}{\usys{\stay_{\edge_2}}}}\nonumber\\
& \%\%~~\text{by Equation~\ref{eqn:continv5}}\nonumber\\
& \subseteq \acztope{\ptemp_{\edge_2}}{\cen_{\edge_2}}{\sfact_{\edge_2}}{\qtemp}{\lb_{\edge_2}}{\ub_{\edge_2}}=\Omega_{\edge_2}~\label{eqn:proof-continv2}.
\end{align}
%
Using Equations~\ref{eqn:proof-continv2} and~\ref{eqn:proof-continv1},
we get
\begin{align*}
& \contreach{\edge}{\Psi}\\
& =\lt(\minsum{\linmap_\edge\lt(\Psi_{\edge_1}\bigcap\ptope{\qtemp}{\join{\lsys{\edge}}{\lsys{\stay_{\edge_1}}}}{\meet{\usys{\edge}}{\usys{\stay_{\edge_1}}}}\rt)}{\inputset_{\edge}}\rt)\bigcap 
\ptope{\qtemp}{\lsys{\stay_{\edge_2}}}{\usys{\stay_{\edge_2}}}\\
& \subseteq \Omega_\loc.\hspace{25em}\qedhere
\end{align*}
%
\end{proof}
%
