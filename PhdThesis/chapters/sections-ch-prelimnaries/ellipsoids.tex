Ellipsoids are sub-level sets of positive semi-definite quadratic
forms.  In control theory, they are implicitly found while computing
quadratic lyapunov functions for proving stability or stabilizability of
systems~\cite{todo}.  An ellipsoid is represented by a pair of a
positive semi-definite symmetric matrix $Q\in\mat{n}{n}{\reals}$ and a center
$\cen\in\reals^n$, which is assigned to the following set.
%
\[
\concrete{Q,c}:=\cen\oplus\set{x\in\reals^n:~\transpose{x}Qx\leq 1}.
\]
%
{\bf Computing positive invariants} For stable linear systems,
ellipsoidal positive invariants can be efficiently computed using
semi-definite programming~\cite{todo}.  This is an advantage of using
ellipsoids for computing positive invariants for linear systems, and a
motivation for using them for other classes of systems.  There are
techniques that try to compute positive invariants for
non-linear~\cite{todo} and hybrid systems~\cite{todo}, although in the
latter case, we can not guarantee the existence of positively
invariant elliposids.  An alternative approach to semi-definite
programming for computing positively invariant ellipsoids was
presented in~\cite{todo}, which uses fixed point iteration based on
Lowner order.  In this work, each iteration involves only algebraic
computations, which possibly resulted in better computational
efficiency than semi-definite programming on the tested examples.

Ellipsoids are closed under linear transformations.  But they are not
closed under a Minkowski sum, mutual intersection or intersection with
half-spaces.  Recent work~\cite{todo} has addressed the problem of
over-approximating the Minkowski sum of ellipsoids and their
intersection with half-spaces, based on the Lowner order combined with
semi-definite programming.  Still, there can be a significant error in
the approximation.  An advantage of zonotopes and complex zonotopes
over ellipsoids is that the former are closed under Minkowski sum,
which can also be computed efficiently.

