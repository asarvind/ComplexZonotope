Ellipsoids are sub-level sets of positive semi-definite quadratic
forms.  In control theory, they are implicitly found while computing
quadratic Lyapunov functions for proving stability or stabilizability of
systems~\cite{blanchini2008set}.  An ellipsoid is represented by a pair of a
positive semi-definite symmetric matrix $Q\in\mat{n}{n}{\reals}$ and a center
$\cen\in\reals^n$, which is assigned to the following set.
%
\[
\concrete{Q,c}:=\cen\oplus\set{x\in\reals^n:~\transpose{x}Qx\leq 1}.
\]
%
The existence of ellipsoidal positive invariants is guaranteed for
specific classes of affine hybrid systems~\cite{shorten2003result}.  Many techniques
for computing ellipsoidal approximations of reachable set for affine
hybrid system use mathematical
optimization~\cite{blanchini2008set,kurzhanskiy2006ellipsoidal,DBLP:conf/hybrid/RouxJGF12}.  An
alternative approach for computing positively invariant ellipsoids was
presented in~\cite{DBLP:journals/tecs/AllamigeonGSGP16}, which uses
fixed point iteration based on L\"owner order.  In this work, each
iteration involves only algebraic computations, which possibly
resulted in better computational efficiency than semi-definite
programming on the tested examples.

Ellipsoids are closed under linear transformations.  But they are not
closed under Minkowski sum and intersection with half-spaces.  The
ellipsoidal toolbox~\cite{kurzhanskiy2006ellipsoidal} and a recent
approach based on L\"owner order~\cite{allamigeon2017fast} have addressed
the problem of over-approximating the Minkowski sum of ellipsoids and
their intersection with half-spaces.  For Minkowski sum, the
ellipsoidal toolbox can compute an approximation that is tight along a
specified direction.  But a single ellipsoid can not provide a tight
approximation of a Minkowski sum along all the directions.  Whereas,
zonotopes and complex zonotopes have the advantage that they are
closed under Minkowski sum, which can also be computed efficiently.

