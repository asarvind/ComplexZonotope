
{\color{red}Mention related work here.}
A simple zonotope is a linear combination of real valued vectors plus a
center, such that the absolute values of the real valued combining
co-efficients are bounded within unity.  Geometrically speaking, a
simple zonotope is a Minkowski sum of line segments.  The
representation of a simple zonotope is given below.
%
\begin{definition}[Simple zonotope]
Let us consider $\ptemp\in\reals^m$, whose column vectors are called
{\it generators}, and $\cen\in\reals^n$.  The following is the
representation of a real zonotope.
%
\begin{align*}
\rztope{\ptemp}{\cen}=\set{\cen+\ptemp\zeta:~\zeta\in\reals^m,~\infnorm{\zeta}\leq 1}.
\end{align*}
%
\end{definition}
%
A main advantage of the zonotope set representation is not only that
they are closed under linear transformation and Minkowski sum, but
these can also be computed efficiently.  The linear transformation of
a zonotope is computed as follows.  For the rest of this chapter, we
use the notation $\ptemp\in\reals^m$ and $\cen\in\reals^n$, unless
otherwise sepcified.
%
\begin{lemma}[Linear transformation]
Let $A\in\mat{n}{n}{\reals}$.  Then
%
\begin{align*}
A\rztope{\ptemp}{\cen}=\rztope{A\ptemp}{A\cen}
\end{align*}
%
\end{lemma} 
%
\begin{proof}
We derive the following.
%
\begin{align*}
&
  A\rztope{\ptemp}{\cen}=A\set{\cen+\ptemp\zeta:~\zeta\in\reals^m,~\infnorm{\zeta}\leq
    1}\\
& = \set{A\cen+A\ptemp\zeta:~\zeta\in\reals^m,~\infnorm{\zeta}\leq 1}=\rztope{A\ptemp}{A\cen}.\hspace{3em}\qedhere
\end{align*}
%
\end{proof}
%
The Minkowski sum of two zonotopes can be computed as follows.
%
\begin{lemma}[Minkowski sum]
Let us consider two matrices $\ptemp\in\mat{n}{m}{\reals}$ and
$\ptemp^\pr\in\mat{n}{r}{\reals}$.  Then the following is true.
%
\begin{align*}
& \rztope{\ptemp}{\cen}\oplus\rztope{\ptemp^\pr}{\cen^\pr}
 \rztope{\mymatrix{\ptemp & \ptemp^\pr}}{\cen+\cen^\pr}.  
\end{align*}
%
\end{lemma}
%
\begin{proof}
We derive the following.
%
\begin{align*}
& \rztope{\ptemp}{\cen}\oplus\rztope{\ptemp^\pr}{\cen^\pr}\\
& = \set{\cen+\ptemp\zeta:~\zeta\in\reals^m,~\infnorm{\zeta}\leq
    1}\oplus\set{\cen^\pr+\ptemp^\pr\zeta^\pr:~\zeta^\pr\in\reals^r,~\infnorm{\zeta^\pr}\leq
    1}\\
& =
  \set{\lt(\cen+\cen^\pr\rt)+\ptemp\zeta+\ptemp\zeta^\pr:~\zeta\in\reals^m,\zeta^\pr\in\reals^r,~\infnorm{\zeta}\leq
    1,\infnorm{\zeta^\pr}\leq 1}\\
& = \set{\lt(\cen+\cen^\pr\rt)+\mymatrix{\ptemp
      &\ptemp^\pr}\mymatrix{\zeta\\\zeta^\pr}:~\mymatrix{\zeta\\\zeta^\pr}\in\reals^{m+r},~\infnorm{\mymatrix{\zeta\\\zeta^\pr}}\leq
    1}\\
& =  \rztope{\mymatrix{\ptemp & \ptemp^\pr}}{\cen+\cen^\pr}.\hspace{3em}\qedhere
\end{align*}
%
\end{proof}
%
Because of efficient computation of linear transformation and
Minkowski sum operations, simple zonotopes have been sucessfully
applied for bounded time computation of uncertain linear systems and
some hybrid systems with simple switching conditions~\cite{todo}.
However, for unbounded time computation of reachable sets, which
generally requires computing positive invariants, zonotopes have the
following drawback.  For the simple case of a stable linear system, we
do not know if a positively invariant non-zero zonotope exists.  For a
special case when a stable linear system (say discrete
time) has real eigenvalues, there exists a positively invariant
zonotope zonotope based on the following proposition.
%
\begin{proposition}
Let us consider $\ptemp\in\mat{n}{n}{\reals}$ consists
of the real eigenvectors of a matrix $A\in\mat{n}{n}{\reals}$ as
its column vectors and $\mu\in\reals^n$ be the vector of complex
eigenvalues, i.e., $A\ptemp
= \ptemp\diagonal{\mu}$.  Then \[A\lt(\rztope{\ptemp}{0}\rt)
= \rztope{\ptemp\diagonal{\mu}}{0}.\] If
$\infnorm{\mu}\leq 1$, then
$A\lt(\rztope{\ptemp}{0}\rt)\subseteq \rztope{\ptemp}{0}$.
\end{proposition}
% 
\begin{proof}
We derive
  %
\begin{align*}
& A\lt(\rztope{\ptemp}{0}\rt) =
A\set{\ptemp\zeta:~\zeta\in\reals^n,\infnorm{\zeta}\leq 1}\\
& =\set{A\ptemp\zeta:~\zeta\in\reals^n,\infnorm{\zeta}\leq 1}
= \rztope{A\ptemp}{0}=\rztope{\ptemp\diagonal{\mu}}{0}.
\end{align*}
%
which proves the first part of the
Proposition.

For the second part, we are given that $\infnorm{\mu}\leq 1$.
Consider a point
%
\begin{align*}
  & y\in \rztope{\ptemp}{0}=\rztope{\ptemp\diagonal{\mu}}{0}~~\text{where}\\
  &y = \ptemp\diagonal{\mu}\delta:\infnorm{\delta}\leq
1.
\end{align*}
%
Let $\zeta = \diagonal{\mu}\delta$. Then $\infnorm{\zeta} \leq
\infnorm{\mu}\infnorm{\delta} \leq 1$.  So,
%
\begin{align*}
  & y=\ptemp\zeta~~\text{ where }~
  \infnorm{\zeta}\leq 1.
\end{align*}
%
So, we get $y\in \rztope{\ptemp}{0}$.  As this is true for all $y\in
A\rztope{\ptemp}{0}$, we have
$A\lt(\rztope{\ptemp}{0}\rt)\subseteq
\rztope{\ptemp}{0}$ when $\infnorm{\mu}\leq 1$.
\end{proof}
%
However, the eigenvalues of a linear matrix can have non-zero
imaginary part, in addition to the real part.  Whereas, simple
zonotopes are defined on real numbers.  Therefore, in the next chapter
we introduce complex zonotopes that can capture contraction along
complex vectors.

Another drawback of simple zonotopes is that they not closed under
mutual intersection and intersection with half-spaces.  The
computation of intersection with half-spaces and hyperplanes is
required in reachability analysis of hybrid systems having linear
guards (pre-conditions) for switching.  To compute the intersection,
we can try to convert a zonotope to a polytope, compute the
intersection with the polytope and over-approximate the intersection
by another zonotope.  However, a zonotope with $m$ generators in an
$n$ dimensional space can have as many as $^{2m}C_{n-1}$
faces~\cite{todo}, which is exponential in $n$.  Therefore, the
conversion of a zonotope to a polytope can be costly in higher
dimensions.  To address this problem, an algorithm for the tight
over-approximation of the intersection between a zonotope and a
hyperplane is proposed in~\cite{todo}, that is based on the projection
of the zonotope onto 2-dimensional hyperplanes.  Another approach is
to extend the zonotope to a more general set representation in which
the intersection with a half-space is a closed operation and can be
computed efficiently.  Contrained zonotopes and constrained affine
sets are examples of such extensions, which are closed under
intersection with hyperplanes and halfspaces, respectively.

In a constrained zonotope, in addition to the bound on the absolute
values of the combining coefficients, we can have linear equality
constraints.  Therefore, the intersection of a hyperplane and a constrained
zonotope can be represented another constrained zonotope.
%
\begin{definition}[Constrained zonotope]

\end{definition}
%
