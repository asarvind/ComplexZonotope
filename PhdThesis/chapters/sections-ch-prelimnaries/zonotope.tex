Simple zonotopes are linear projections of higher dimensional
hypercubes onto lower dimensional spaces, therefore a sub-class of
polytopes.  A simple zonotope is represented by a linear combination
of real valued vectors plus a center, such that the real valued
combining coefficients are contained in a hypercube.  Geometrically
speaking, a simple zonotope is a Minkowski sum of line segments.
%
\begin{definition}[Simple zonotope]~\label{defn:rztope}
Let us consider $\ptemp\in\reals^m$, whose column vectors are called
\emph{generators}, and $\cen\in\reals^n$ called the \emph{center}.  The following is the
representation of a real zonotope.
%
\begin{align*}
\rztope{\ptemp}{\cen}:=\set{\cen+\ptemp\zeta:~\zeta\in[-1,1]^m}.
\end{align*}
%
\end{definition}
%
For the rest of this chapter, we
use the notation $\ptemp\in\reals^m$ and $\cen\in\reals^n$, unless
otherwise specified.

A main advantage of the zonotope set representation is that besides
being closed under linear transformation and Minkowski sum, these
operations can be computed efficiently.  The computation of linear
transformation of a zonotope is given below.
%
\begin{lemma}[Linear transformation]~\label{lem:lin-rz}
Let $A\in\mat{n}{n}{\reals}$.  Then
%
\begin{align*}
A\rztope{\ptemp}{\cen}=\rztope{A\ptemp}{A\cen}.
\end{align*}
%
\end{lemma} 
%
\begin{proof}
We derive the following.
%
\begin{align*}
&
  A\rztope{\ptemp}{\cen}=A\set{\cen+\ptemp\zeta:~\zeta\in[-1,1]^m}\\
& = \set{A\cen+A\ptemp\zeta:~\zeta\in[-1,1]^m}=\rztope{A\ptemp}{A\cen}.\hspace{3em}\qedhere
\end{align*}
%
\end{proof}
%
The Minkowski sum of two zonotopes can be computed as follows.
%
\begin{lemma}[Minkowski sum]~\label{lem:min-rz}
Let us consider two matrices $\ptemp\in\mat{n}{m}{\reals}$ and
$\ptemp^\pr\in\mat{n}{r}{\reals}$.  Then the following is true.
%
\begin{align*}
& \rztope{\ptemp}{\cen}\oplus\rztope{\ptemp^\pr}{\cen^\pr}
= \rztope{\mymatrix{\ptemp & \ptemp^\pr}}{\cen+\cen^\pr}.  
\end{align*}
%
\end{lemma}
%
\begin{proof}
We derive the following.
%
\begin{align*}
& \rztope{\ptemp}{\cen}\oplus\rztope{\ptemp^\pr}{\cen^\pr}\\
& = \set{\cen+\ptemp\zeta:~\zeta\in[-1,1]^m}\oplus\set{\cen^\pr+\ptemp^\pr\zeta^\pr:~\zeta^\pr\in[-1,1]^r}\\
& =
  \set{\lt(\cen+\cen^\pr\rt)+\ptemp\zeta+\ptemp\zeta^\pr:~\zeta\in[-1,1]^m,~\zeta^\pr\in[-1,1]^r}\\
& = \set{\lt(\cen+\cen^\pr\rt)+\mymatrix{\ptemp
      &\ptemp^\pr}\mymatrix{\zeta\\\zeta^\pr}:~{\mymatrix{\zeta\\\zeta^\pr}}\in[-1,1]^{m+r}}\\
& =  \rztope{\mymatrix{\ptemp & \ptemp^\pr}}{\cen+\cen^\pr}.\hspace{3em}\qedhere
\end{align*}
%
\end{proof}
%
We have earlier discussed that for a $V$-representation of a polytope,
the complexity of Minkowski sum operation is quadratic, while for a
$H$-representation, the complexity of Minkowski sum can be
exponential.  On the other hand, the complexity of both Minkowski sum
and linear transformation operations for a zonotope is only linear.
This is an advantage of the zonotope representation compared to a
general convex polytope representation.  Therefore, simple zonotopes
have been successfully applied for reachability analysis of uncertain
linear systems and some affine hybrid systems with simple switching
conditions~\cite{makhlouf2014networked,Girard05reachabilityof,girard2008zonotope,HybridFluctuat}.
Still, zonotopes have the following drawback while computing positive
invariants of affine hybrid systems.  Even for the simple case of a
stable linear system having complex valued eigenvectors, we do not
know if a positively invariant non-zero zonotope exists.  When a stable linear system has
 real eigenvectors, the computation of a non-zero positively
invariant zonotope is guaranteed by the
following proposition.
%
\begin{proposition}~\label{prop:eig-rztope}
Let us consider $\ptemp\in\mat{n}{n}{\reals}$ consists
of the real eigenvectors of a matrix $A\in\mat{n}{n}{\reals}$ as
its column vectors, where $\mu\in[0,1]^n$ is the vector of real
eigenvalues, i.e., $A\ptemp
= \ptemp\diagonal{\mu}$.  Then,
$A\lt(\rztope{\ptemp}{0}\rt)\subseteq \rztope{\ptemp}{0}$.
\end{proposition}
% 
\begin{proof}
We derive
  %
\begin{align*}
& A\lt(\rztope{\ptemp}{0}\rt) = A\set{\ptemp\zeta:~\zeta\in[-1, 1]^n}\\
&
=\set{A\ptemp\zeta:~\zeta\in[-1,1]^n}=\set{V\diagonal{\mu}\zeta:~\zeta\in[-1,1]^n}.~\numberthis\label{proof-eigzonotope}
\end{align*}
%
Let $y$ be a point
in $\rztope{\ptemp}{0}$.  By Equation~\ref{proof-eigzonotope}, we get
%
\begin{align*}
  &\exists\delta\in[-1,1]^n:~y = \ptemp\diagonal{\mu}\delta.
\end{align*}
%
Let $\zeta = \diagonal{\mu}\delta$. Since $\mu\in[0,1]^n$ and
$\delta\in[-1,1]^n$, we get $\zeta=\diagonal{\mu}\delta\in[-1,1]^n$.  So,
%
\begin{align*}
  & y=\ptemp\zeta~~\text{ where }~
  \zeta\in[-1,1]^n\\
  &\therefore~y\in\rztope{\ptemp}{0}.
\end{align*}
%
As the above is true for any $y\in
A\rztope{\ptemp}{0}$, we get
$A\lt(\rztope{\ptemp}{0}\rt)\subseteq
\rztope{\ptemp}{0}$ when $\infnorm{\mu}\leq 1$.
\end{proof}
%
However, the eigenvalues of a linear matrix can have non-zero
imaginary part, in addition to the real part.  Whereas, simple
zonotopes are defined on real numbers.  So, when the eigenvalues are
complex valued, we can not rely on the above proposition to find a
positive invariant.  Therefore, in the next chapter we introduce
complex zonotopes that can capture contraction along complex vectors.

To approximate non-linear transformations of zonotopes, they have been
extended to quadratic zonotopes~\cite{DBLP:conf/aplas/AdjeGW15}, and
more generally, polynomial
zonotopes~\cite{DBLP:conf/hybrid/Althoff13}.  Although a polynomial
zonotope and the complex zonotope that we introduce are both
non-polytopic sets, these representations are geometrically
different.  While a polynomial zonotope is described by a polynomial function of
real valued intervals, a complex zonotope shall be described as a
complex valued linear function of circles in the complex plane.
%
\begin{definition}~\cite{DBLP:conf/hybrid/Althoff13}
Let $f\in\polyring{x_1,\ldots,x_n}{\reals}$.  Then
$\set{f(x):~x\in[-1,1]^n}$ is a polynomial zonotope.
\end{definition}
%
Another drawback of simple zonotopes is that they not closed under
intersection with half-spaces.  The computation of intersection with
half-spaces and hyperplanes is required in reachability analysis of
hybrid systems having linear guards (pre-conditions) for switching.
To compute the intersection, we can try to convert a zonotope to a
polytope, compute the intersection with the polytope and
over-approximate the intersection by another zonotope.  However, a
zonotope with $m$ generators in an $n$ dimensional space can have as
many as $2\lt(\begin{matrix}m\\n-1\end{matrix}\rt)$
  faces~\cite{zaslavsky1975facing}, which is exponential in $n$.  It
  means that the conversion of a zonotope to a polytope can be
  intractable in higher dimensions.  To address this problem, an
  algorithm for the tight over-approximation of the intersection
  between a zonotope and a hyperplane is proposed
  in~\cite{girard2008zonotope}, that is based on the projection of the
  zonotope onto 2-dimensional hyperplanes.  Alternatively, there are approaches to extend
  the zonotope to a more general set representation in which the
  intersection with a half-space is a closed operation and can be
  computed efficiently.  Constrained
  zonotopes~\cite{scott2016constrained}, constrained affine
  sets~\cite{Ghorbal2010} and zonotopic
  bundles~\cite{althoff2011zonotope} are examples of such extensions.

In a constrained zonotope, there are linear equality constraints in
addition to the unit interval bounds on the combining coefficients.
Therefore, the intersection of a hyperplane and a constrained zonotope
can be represented by another constrained zonotope.  Moreover, they
retain the closure property under linear transformation and Minkowski
sum, which can also be computed efficiently.
%
\begin{definition}[Constrained zonotope]~\cite{scott2016constrained}
A constrained zonotope is represented by a tuple
$\lt(\ptemp,\cen,A,b\rt)$, where $\ptemp\in\mat{n}{m}{\reals}$,
$A\in\mat{k}{m}{\reals}$, $\cen\in\reals^n$ and $b\in\reals^m$, and
the tuple is assigned to the set
%
\[\set{\cen+\ptemp\zeta:\zeta\in[-1,1]^m,A\zeta=b}.\]
%
\end{definition}
%
An even more general representation is a constrained affine
set~\cite{Ghorbal2010}, where the combining coefficients can range in some
abstract domain.  To represent the intersection of a constrained affine
set with half-spaces, the abstract domain constraining the combining
coefficients can be chosen such that it has linear relations.  Another
generalization is a zonotopic bundle set representation~\cite{althoff2011zonotope},
which is closed under mutual intersection.


