A sub-level set of a polynomial is a natural extension the
$H$-representation of the polytope to non-polytopic sets to provide
better approximation quality.  Mathematically, a polynomial sub-level
set is defined as follows.
%
\begin{definition}
A polynomial sub-level set is represented by a tuple $\lt(f,d\rt)$,
where 
$f\in\lt(\polyring{x_1,\ldots,x_n}{\reals}\rt)^r$ and $d\in\reals^r$,
which is assigned to the following set.
%
\[
\concrete{f,d}=\set{x\in\reals^n:~\forall
  i\in\set{1,\ldots,r},~f_i(x)\leq d_i}
\]
%
\end{definition}
%
Although polynomial sub-level sets are more expressive than
$H$-representation, approximating reachable sets using the former is
computationally more expensive.  To alleviate the complexity of
computating positively invariant polynomial sub-level sets, a
semi-definite relaxation combined with policy iteration has been
proposed in~\cite{todo}.  Still, this approach uses a recursion of
convex optimization steps.  So, the convergence of the recursion can
be slower than polyhedral abstract domains~\cite{todo}, which use
linear programming in the recursion.  In contrast, our approach using
complex zonotopes tries to compute a positive invariant in only a
single step of convex optimization.
