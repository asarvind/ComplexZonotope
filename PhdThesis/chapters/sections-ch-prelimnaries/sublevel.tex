A polynomial sub-level set encodes a set of points that satisfy
a collection of multi-variate polynomials.  It
is a natural extension of the $H$-representation of the polytope to
non-polytopic sets for better approximation.  Mathematically, a
polynomial sub-level set is defined as follows.
%
\begin{definition}
A polynomial sub-level set is represented by a tuple $\lt(f,d\rt)$,
where 
$f\in\lt(\polyring{x_1,\ldots,x_n}{\reals}\rt)^r$ and $d\in\reals^r$,
which is assigned to the following set.
%
\[
\concrete{f,d}=\set{x\in\reals^n:~\forall
  i\in\set{1,\ldots,r},~f_i(x)\leq d_i}
\]
%
\end{definition}
%
In~\cite{duggirala2013safety}, it was shown that the safety
verification problem for linear systems is decidable when the initial
set is a sub-level set of eigen-functions, which can be quadratic
polynomials.  Our extension of real zonotopes to complex zonotopes
inspired by their work, which analyzes the eigenstructure of linear
systems.  In~\cite{Sriram}, positive invariants represented by
polynomial equalities are computed for a general class of hybrid
systems, where the polynomial is a linear combination of pre-chosen
templates that have to be guessed.  On the other hand, the method
of~\cite{tiwariRodriguezCarbonellPolynomialInvariants} computes
polynomial equality invariants for affine hybrid systems without
having to guess a template polynomial.  But~\cite{Sriram,tiwariRodriguezCarbonellPolynomialInvariants}
can not handle an additive disturbance input set having a non-empty
interior, because polynomial equalities represent hyper-surfaces which
have an empty interior.

On the other hand, polynomial sub-level sets specified by polynomial
inequalities can over-approximate sets having a non-empty interior.  A
sub-level set of a single multi-variate polynomial is implicitly found
while computing barrier certificates by sum of squares
programming~\cite{prajna2004safety}.  For certain classes of hybrid
systems~\cite{prajna2005necessity}, the existence of a barrier certificate is both
necessary and sufficient for safety.  However, the accuracy of
over-approximation provided by a single multi-variate polynomial
naturally depends on the degree of the polynomial.  But high
dimensional spaces, representing polynomials of higher degree can be
computationally expensive.  Alternatively, we can use multiple number
of easily encodable multi-variate polynomials for better accuracy.
But computing tight approximations of reachable sets by intersection
of polynomial sub-level sets requires non-convex optimization, which
is expensive.  To alleviate complexity of non-convex optimization
problem in the case of multiple polynomials, an approach using
semi-definite relaxation coupled with policy iterations has been
developed in~\cite{DBLP:conf/esop/AdjeGG10}.  However, this approach
uses a recursion of convex optimization steps.  So, the convergence of
the recursion can be slower than polyhedral abstract domains, which
generally use linear programming in the recursion.  In contrast, our
algorithm for computing positive invariants using complex zonotopes
has only a single step of convex optimization.
