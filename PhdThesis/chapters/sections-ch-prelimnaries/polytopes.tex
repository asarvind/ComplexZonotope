Polytopes are sets that satisfy Boolean combinations of linear
inequalities.  Polytopes have been used in many tools for
verification~\cite{kvasnica2004multi,silva2000modeling,asarin2002d,bagnara2008parma,frehse2008phaver}.  A
convex polytope is a special case where a polytope is convex.
Generally, a half-space representation of a convex polytope, also
called a $H$-representation, is used in the verification algorithms
for hybrid systems, which is defined as follows.

%%Various representations of a convex polytope can
%% be broadly classified into two main categories, half-space
%% representation ($H$-representation) and the vertex representation
%% ($V$-representation).  There can be other possible representations for
%% some sub-classes of polytopes, like zonotopes.  We shall discuss
%% zonotopes in a separate section.

%% Convex polytopes are closed under linear transformation, Minkowski
%% sum, checking inclusion, mutual intersection and intersection with
%% half-spaces.  


%% However, the complexity of computing these operations
%% depends on the way the polytope is represented.  For some operations
%% like Minkowski sum and linear transformation, the $V$-representation
%% is more efficient.  Whereas, for mutual intersection and intersection
%% with half-spaces, the $H$-representation is more efficient.  We shall
%% discuss the $H$ and the $V$-representations in two separate sections.


%
\begin{definition}[$H$-representation]
Let us consider a matrix $T\in\mat{r}{n}{\reals}$ and a vector $d\in\reals^r$.
The $H$-representation of a convex polytope is a tuple $\lt(T,d\rt)$
that is assigned to the set
%
\[
\concrete{T,d}=\set{x\in\reals^n:~Tx\leq d}.
\]
%
\end{definition}
%
The row vectors of $T$ are sometimes called \emph{support vectors}~\cite{FLD+11}.
An invertible linear transformation of a $H$-representation can be
computed by the following formula.  Let $A$ be an invertible matrix.
Then
%
\[
A\concrete{T,d}=\concrete{TA^{-1},d}.
\]
%
But when the matrix $A$ is not invertible, the computation of the
transformed polytope can be expensive.  Computing the Minkowski sum of
two $H$-representations in an $n$-dimensional
space can require elimination of at least $n$-variables in a system of
more than $2n$ linear inequalities, which is discussed
in~\cite{kvasnica2005minkowski}.  The complexity of variable
elimination by known algorithms is more than exponential.  Therefore,
computation of Minkowski sum of two $H$-representations can be costly
in higher dimensions.
The intersection of two $H$-representations having the same set of
supporting hyperplanes is easy to compute, as follows.  
%
\begin{align*}
& \concrete{T,d}\bigcap\concrete{T,e}=\concrete{T,\meet{d}{e}}.~\numberthis\label{eqn:half-intersection}
\end{align*}
%
The above operation also generalizes to the intersection with half-spaces
by including the support vectors of a half-spaces among
the support vectors of a $H$-representation.  Therefore, a
$H$-representation is efficient while computing intersections with
linear guards of hybrid systems.

%% Examples of such representations with a fixed set
%% of support vectors include template polyhedra~\cite{todo} and
%% octagons~\cite{todo}.

%% \subsubsection{$V$-representation}
%% A $V$-representation is a convex hull of a
%% finite set of points, i.e.,
%% %
%% \[
%% \convexhull{\set{v_1,\ldots,v_m}}:~v_1,\ldots,v_n\in\reals^n.
%% \]
%% %
%% The linear transformation and Mikowski sum of $V$-representations can
%% be more efficiently computed than the $H$-representation, as follows.
%% %
%% \begin{align*}
%% &
%%   A\convexhull{\set{v_1,\ldots,v_m}}=\convexhull{\set{Av_1,\ldots,Av_m}}.\numberthis\\
%% &
%%   \convexhull{\set{v_1,\ldots,v_m}}\oplus\convexhull{\set{w_1,\ldots,w_k}}\\&=\convexhull{\set{v_i+w_j:1\leq
%%       i\leq m,~1\leq j\leq k}}.\numberthis\label{eqn:min-hull}
%% \end{align*}
%% %
%% However, computing intersection of a $V$-representation with a half-space
%% can require coverting the $V$-representation to the $H$-representation
%% and then using Equation~\ref{eqn:half-intersection}.  But conversion
%% from a $V$-representation to a $H$-representation by known methods has
%% exponential complexity in the dimension of the space.  Therefore,
%% computing the intersection of a $V$-representation with a half-space
%% can be costly in higher dimensions.

%% By Equation~\ref{eqn:min-hull}, the complexity of Minkowski sums of
%% $V$-representations is quadratic.  We have explained earlier that the
%% complexity of Minkowski sums of $H$-representations can be
%% exponential.  On the other hand, as we shall see later, the complexity
%% of Minkowski sum and linear transformations of zonotopes is only
%% linear.

In~\cite{rakovic2004computation}, a method for computing the maximal
polytopic positive invariant set for a piecewise affine hybrid system
is implemented using the MPT
toolbox\footnote{\url{http://people.ee.ethz.ch/~mpt/3/}}.  However,
the method involves partitioning of the state space, which can be
computationally expensive in higher dimensions.  Some sub-classes of
convex polytopes that have been used for computing positive invariants
include template
polyhedra~\cite{Gawlitza,Sankaranarayanan+Dang+Ivancic-08-Symbolic},
hypercube~\cite{cousot1976static,tiwari2008generating},
parallelotopes~\cite{amato2012abstract},
octagons~\cite{DBLP:journals/lisp/Mine06} and support
vectors~\cite{FLD+11}.  In these set-representations, generally
mathematical optimization is used for performing necessary operations
required for computing a positive invariant~(for
example~\cite{Gawlitza,dang1998reachability}) .  In simpler abstract
domains~\cite{DBLP:journals/lisp/Mine06,cousot1976static,amato2012abstract},
algebraic methods can be used to compute positive invariants.
In~\cite{Gawlitza}, the normals to the bounding hyperplanes are fixed
and the algorithm tries to find the tightest possible bounds
of a positive invariant.  Our template based approach to
compute complex zonotopic positive invariants is in a similar spirit.
We fix a set directions for the generators and optimize their
magnitudes.  However, in our approach we can choose complex valued
templates that can capture contraction along the complex eigenvectors
of the transformation matrices.  But such complex valued templates can
not be selected for polytopes.

It is known that for any stable linear system, there exists a non-zero
positively invariant polytope given by a polyhedral lyapunov
function~\cite{blanchini2008set}.  However, to our knowledge, there is
no known upper bound on the smallest possible representation size of a
polytopic invariant for stable linear systems.  In contrast, for a stable and
invertible linear transformation, a complex zonotopic positive
invariant containing the origin in its interior can be computed by
just having the $n$-eigenvectors as its generators.

Real zonotopes are also a sub-class of polytopes, which we discuss in
a separate section.

          
