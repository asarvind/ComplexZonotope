Polytopes are sets that can be represented by boolean combinations of
linear inequalities.  Their advantage is that in general they are easy
to manipulate, either by algebraic means or algorithmically by linear
programming.  Polytopes have been used in a number of tools and
algorithms~\cite{todo}.  A convex polytope is a special case where a
polytope is convex.  Various representations of a convex polytope can
be broadly classified into two main categories, half-space
representation ($H$-representation) and the vertex representation
($V$-representation).  There can be other possible representations for
some sub-classes of polytopes, like zonotopes.  We shall discuss
zonotopes in a separate section.

Convex polytopes are closed under linear transformation, Minkowski
sum, checking inclusion, mutual intersection and intersection with
half-spaces.  However, the complexity of computing these operations
depends on the way the polytope is represented.  For some operations
like Minkowski sum and linear transformation, the $V$-representation
is more efficient.  Whereas, for mutual intersection and intersection
with half-spaces, the $H$-representation is more efficient.  We shall
discuss the $H$ and the $V$-representations in two separate sections.

\subsubsection{$H$-representation}
A $H$-representation of a polytope is an intersection of a number of
half-spaces.  It is mathematically defined as follows.
%
\begin{definition}[$H$-representation]
Let us consider a matrix $T\in\mat{r}{n}{\reals}$ and $d\in\reals^r$.
The $H$-representation of a convex polytope is a tuple $\lt(T,d\rt)$
that is assigned to the set
%
\[
\concrete{T,d}=\set{x\in\reals^n:~Tx\leq d}.
\]
%
\end{definition}
%
The row vectors of $T$ are sometimes called \emph{support vectors}~\cite{todo}.

An invertible linear transformation of a $H$-representation can be
computed by the following formula.  Let $A$ be an inverible matrix.
Then
%
\[
A\concrete{T,d}=\concrete{TA^{-1},d}.
\]
%
But when the matrix $A$ is not invertible, the computation of the
transformed polytope may require conversion to the $V$ representation,
which can be computationally expensive in higher dimensions.

The Minkowski sum of two bounded $H$-representations in an
$n$-dimensional space can require ellimination of at least
$n$-variables in a system of more than $2n$ linear inequalities, which
is discussed in~\cite{kvasnica2005minkowski}.  The complexity of
variable ellimination by known algorithms is more than exponential.
Therefore, computation of Minkowski sum of two $H$-representations can
be costly in higher dimensions.

The intersection of two $H$-representations having the same set of
supporting hyperplanes is easy to compute, as follows.  
%
\begin{align*}
& \concrete{T,d}\bigcap\concrete{T,e}=\concrete{T,\meet{d}{e}}.~\numberthis\label{eqn:half-intersection}
\end{align*}
%
The above operation also generalizes to the intersection with half-spaces
by including the support vectors of a half-spaces among
the support vectors of a $H$-representation.  Therefore, a
$H$-representation is efficient while computing intersections with
linear guards of hybrid systems.

For a fixed set of support vectors, it
has been shown that the tightest possible approximation of a Minkowski
sums or linear transformation can be computed by linear
programming~\cite{todo}.  In some cases, the tigtest possible
approximation can be computed even more efficiently by algebraic
means~\cite{todo}.  Examples of such representations with a fixed set
of support vectors include template polyhedra~\cite{todo} and
octagons~\cite{todo}.

\subsubsection{$V$-representation}
In a $V$-representation, a convex polytope is a convex hull of a
finite set of points, i.e.,
%
\[
\convexhull{\set{v_1,\ldots,v_m}}:~v_1,\ldots,v_n\in\reals^n.
\]
%
The linear transformation and Mikowski sum of $V$-representations can
be more efficiently computed than the $H$-representation, as follows.
%
\begin{align*}
&
  A\convexhull{\set{v_1,\ldots,v_m}}=\convexhull{\set{Av_1,\ldots,Av_m}}.\numberthis\\
&
  \convexhull{\set{v_1,\ldots,v_m}}\oplus\convexhull{\set{w_1,\ldots,w_k}}\\&=\convexhull{\set{v_i+w_j:1\leq
      i\leq m,~1\leq j\leq k}}.\numberthis\label{eqn:min-hull}
\end{align*}
%
By Equation~\ref{eqn:min-hull}, the complexity of Minkowski sums of
$V$-representations is quadratic.  We have explained earlier that the
complexity of Minkowski sums of $H$-representations can be
exponential.  On the other hand, as we shall see later, the complexity
of Minkowski sum and linear transformations of zonotopes is only
linear.

Computing intersection of a $V$-representation with a half-space can
be performed by coverting the $V$-representation to the
$H$-representation and then using
Equation~\ref{eqn:half-intersection}.  But conversion from a
$V$-representation to a $H$-representation by known methods has
exponential complexity in the dimension of the space.  Therefore,
computing the intersection of a $V$-representation with a half-space
can be costly in higher dimensions.

\subsubsection{Related work on computing positive invariants}
It has been shown that for any stable linear system, there exists a
non-zero positively invariant polytope~\cite{todo}.  However, for any
dimension, the complexity of computing the positive invariant can be
arbitrarily high~\cite{todo}.  This is a drawback of polytopes.  In
comparison, our complex zonotope representation can easily compute a
non-zero positive invariant for a stable linear system using the
eigenstructure.

Most of the approaches for computing polytopic positive invariants for
affine hybrid systems use the $H$-representation.  Set representations
based on the $H$-representation include template
polyhedra~\cite{todo}, hypercubes~\cite{todo}, octagons~\cite{todo},
and support vectors~\cite{todo}.  Some of these approaches fix a
template of directions for the bounding hyperplanes and compute a
positive invariant~\cite{todo} by mathematical optimization.  In
simpler abstract domains, algebraic methods can be used to compute
positive invariants.  We also introduce a template based approach to
compute complex zonotopic positive invariants where we fix a set
directions for the generators and optimize their magnitudes.  However,
in our approach we can choose complex valued templates based on the
complex eigenstructure of the transformation matrices.  Such complex
valued templates can not be selected for polytopes.






