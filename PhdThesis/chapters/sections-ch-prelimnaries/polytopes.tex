Polytopes are sets that can be represented by boolean combinations of
linear inequalities.  Their popularity in reachability analysis is
mainly because they are easy to manipulate, either through algebraic
means or algorithmically by linear programming.  They have been used
in a number of tools and algorithms~\cite{todo}.  A convex polytope is
a special case where a polytope is convex.  A general polytope can be
seen as a union of a finite number of convex polytopes.  The
representations of a convex polytope can be broadly classified into
two main categories, i.e., half-space representation
($H$-representation) and the vertex representation
($V$-representation).  There are other possible representations for
some sub-classes of polytopes, like zonotopes.  We shall discuss
zonotopes in a separate section.

Convex polytopes are closed under linear transformation, Minkowski
sum, checking inclusion, mutual intersection and intersection with
half-spaces.  However, the complexity of computing these operations
depends on the way the polytope is represented.  For some operations,
like Minkowski sum and linear transformation, the $V$-representation
is more efficient.  Whereas, for the intersection with half-spaces and
mutual intersection, the $H$-representation is more efficient.

\paragraph*{$H$-representation}
A $H$-representation of a polytope is an intersection of a number of
half-spaces.  It is mathematically defined as follows.
%
\begin{definition}[$H$-representation]
Let us consider a matrix $T\in\mat{r}{n}{\reals}$ and $d\in\reals^r$.
The $H$-representation of a convex polytope is a tuple $\lt(T,d\rt)$
that is assigned to the set
%
\[
\concrete{T,d}=\set{x\in\reals^n:~Tx\leq d}.
\]
%
\end{definition}
%
The row vectors of $T$ are sometimes called \emph{support vectors} of
the convex polytope~\cite{todo}.

We can easily compute an invertible linear transformation in a
$H$-representation by the following formula.  Let $A$ be an inverible
matrix.  Then
%
\[
A\concrete{T,d}=\concrete{TA^{-1},d}.
\]
%
But when the matrix $A$ is not invertible, the computation of the
transformed polytope may require conversion to the $V$ representation,
which can be computationally expensive in higher dimensions.

The Minkowski sum of two bounded polytopes in an $n$-dimensional
space, specified by the $H$-representation, can require ellimination
of at least $n$-variables in a system of more than $2n$ linear
inequalities, as described in~\cite{kvasnica2005minkowski}.  The
complexity of variable ellimination by known algorithms is more than
exponential.  Therefore, computation of Minkowski sum of polytopes via the
$H$-representation can be costly in higher dimensions.  

The intersection of two $H$-representations having the same set of
supporting hyperplanes can be be computed easily as follows.  
%
\begin{align*}
& \concrete{T,d}\bigcap\concrete{T,e}=\concrete{T,\meet{d}{e}}.~\numberthis\label{eqn:half-intersection}
\end{align*}
%
The above operation also generalizes to intersection with half-spaces
as follows.  We can include the support vector of a half-space among
the support vectors of a $H$-representation and compute the
intersection using Equation~\ref{eqn:half-intersection}.  Therefore, a
$H$-representation is efficient for computing intersections with
linear guards of hybrid systems.

As we have explained earlier, computing Minkowski sum and
transformation by singular matrices can be costly in a
$H$-representation.  However, for a fixed set of support vectors, it
has been shown that the tightest possible approximations of Minkowski
sums and linear transformations can be computed by linear
programming~\cite{todo}.  In some cases, the tigtest possible
approximation can be computed even more efficiently by algebraic
means~\cite{todo}.  Examples of such representations with a fixed set
of support vectors include template polyhedra~\cite{todo} and
octagons~\cite{todo}.

\paragraph*{$V$-representation}
In a $V$-representation, a convex polytope is a convex hull of a
finite set of points.
%
\[
\convexhull{\set{v_1,\ldots,v_m}}:~v_1,\ldots,v_n\in\reals^n.
\]
%
The linear transformation and Mikowski sum of $V$-representations can
be more efficiently computed than the $H$-representation, as follows.
%
\begin{align*}
&
  A\convexhull{\set{v_1,\ldots,v_m}}=\convexhull{\set{Av_1,\ldots,Av_m}}.\\
&
  \convexhull{\set{v_1,\ldots,v_m}}\oplus\convexhull{\set{w_1,\ldots,w_k}}\\&=\convexhull{\set{v_i+w_j:1\leq
      i\leq m,~1\leq j\leq k}}.
\end{align*}
%
As we have described above, the complexity of computing a Minkowski sum
in a $V$-representation can be quadratic, while in a
$H$-representation, it can be exponential.  On the other hand, as we
shall see later, zonotopes have the advantage that the complexity
of Minkowski sum and linear transformations is only linear.

Computing intersection of a $V$-representation with a half-space can
be performed by coverting the $V$-representation to the
$H$-representation and then using
Equation~\ref{eqn:half-intersection}.  But conversion from a
$V$-representation to a $H$-representation by known methods has
exponential complexity in the dimension of the space.  Therefore,
computing the intersection of a $V$-representation with a half-space
can be costly in higher dimensions.

\paragraph*{Computing positive invariant for linear systems}
It has been shown that for any stable linear system, there exists a
positively invariant polytope~\cite{todo}.  However, for any dimension, the
complexity of computing the positive invariant can be arbitrarily
high~\cite{todo}.  This is a drawback of polytopes.  This drawback is overcome by
our complex zonotope representation introduced later, where positive
invariants for linear systems can be easily found using the possibly
complex eigenstructure.
