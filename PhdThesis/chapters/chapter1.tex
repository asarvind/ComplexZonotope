In the previous chapter, we described a set representation called
simple zonotope, which is a type of polytope represented as a linear
combination of real valued vectors with bounded combining
coefficients.  The advantage of simple zonotopes in set based
computations is that they are closed under matrix multiplication and
Minkowski sum operations and these can be computed efficiently.
However, for computing a positive invariant of a linear system, there
is no known procedure to choose a suitable set of generators for a
simple zonotope so as to compute an invariant.  In this chapter, we
extend simple zonotope a new set representation called \emph{complex
zonotope} by which we can easily specify invariants of a linear system
using the eigenstructure of the system.  Complex zonotopes have
complex valued generators complex combining coefficients bounded in
their absolute values, as opposed to the real valued generators and
real combining coefficients of simple zonotopes.  The real projections
of complex zonotopes can describe some non-polytopic sets in addition
to the polytopic zonotopes, which are Minkowski sums of some
ellipsoids along with line segments.  Furthermore, complex zonotopes
are also closed under matrix multiplications and Minkowski sums and
these can be computed efficiently.

Apart from discussing operations on complex zonotopes like linear
transformations and Minkowski sums, in this chapter we derive a convex
program for checking inclusion between two complex zonotopes.  The
inclusion relation is a key ingredient for efficient invariant
computation, as we shall see in the latter chapters.  The organization
of this chapter is as follows.  PUT ORGANIZATION.

\section{Basic representation of a complex zonotope}
The basic representation of a complex zonotope is a generalization of
the representation of a simple zonotope given in Definition~\ref{todo}
to a space of complex valued vectors.  A complex zonotope is
represented as a linear combination of complex valued vectors, called
generators, such that the combining coefficients are bounded in their
absolute values by unity.  However, this representation can be
extended to more general forms for efficient computation, as we shall
discuss later.
%
\begin{definition}[Complex zonotope]
Let $\ptemp\in\mat{m}{n}{\compnums}$ be a complex valued matrix
whose columns are called generators and $\cen\in\compnums^n$ be a
complex valued vector called the center.  Then the following is a
complex zonotope.
%
\begin{equation}
\cztope{\ptemp}{\cen} = \set{\ptemp\zeta+\cen:~\zeta\in\compnums^m,~\infnorm{\zeta}\leq 1}.
\end{equation}
%
\end{definition}
%
Like a simple zonotope, a complex zonotope is also symmetric around
the center.  To see this, consider a point in a complex zonotope
centered at the origin, written as $y=\ptemp\zeta$ where $\ptemp$
defines the generator set and $\zeta$ is the vector of combining
coefficients.  Since, $\infnorm{-\zeta}=\infnorm{\zeta}\leq 1$, so
even $-y=\ptemp(-\zeta)$ belongs to the complex zonotope.  However, unlike
simple zonotopes, the real projection of a complex zonotope can
describe some non-polytopic sets in addition to polytopic zonotopes,
since geometrically speaking the real projection of a complex zonotope
can be a Minkowski sum of line segments as well as some ellipsoids.
An example of such a non-polyhedral complex zonotope projection in
real space is illustrated with Figure~\ref{fig:cz}, where the
generators are $\ptemp=\lt(\begin{array}{lll}(1+2i) & 1 & (2+i)\\(1-2i) & 1
& (2-i)\end{array}\rt)$ and the center is the origin.

A motivation for extending simple zonotopes to complex zonotopes is
that a complex zonotope with its generators as the complex
eigenvectors of a discrete time linear system will be a positive
invariant if the complex eigenvalues corresponding to the generators
are bounded within unity in their absolute values.  This property is
described mathematically in the following proposition.
%
\begin{proposition}[Invariance based on eigenstructure]
Let us consider that $\cztope{\ptemp}{0}$ is a complex zonotope centered at
the origin, where the column vectors of $\ptemp\in\mat{m}{n}{\compnums}$
are the complex eigenvectors of a matrix $A\in\mat{n}{n}{\reals}$ such
that $A\ptemp = \ptemp\diagonal{\mu}$ and $\mu\in\compnums^m$ is a vector of
complex eigenvalues corresponding to the column vectors of $\ptemp$.
Then \[A\lt(\cztope{\ptemp}{0}\rt) = \cztope{\ptemp\diagonal{\mu}}{0}.\]
Therefore, if $\infnorm{\mu}\leq 1$, then
$A\lt(\rztope{\ptemp}{0}\rt)\subseteq \rztope{\ptemp}{0}$.
\end{proposition}
% 
\begin{proof}
We have $A\lt(\cztope{\ptemp}{0}\rt) =
A\set{\ptemp\zeta:~\zeta\in\compnums^m,\infnorm{\zeta}\leq 1}$\\
$=\set{A\ptemp\zeta:~\zeta\in\compnums^m,\infnorm{\zeta}\leq 1}
= \cztope{A\ptemp}{0}$.  Next $\cztope{A\ptemp}{0}
= \cztope{\ptemp\diagonal{\mu}}{0}$, which proves the first part of the
Proposition.  For the second part, we are given that
$\infnorm{\mu}\leq 1$.  Consider a point $y\in \cztope{\ptemp\diagonal{\mu}}{0}$,
described by the generators as
$y = \ptemp\diagonal{\mu}\delta:\infnorm{\delta}\leq 1$.
Let $\zeta = \diagonal{\mu}\delta$. Then
$\infnorm{\zeta} \leq \infnorm{\mu}\infnorm{\delta} \leq 1$.  So, 
$y=\ptemp\zeta$ where $\infnorm{\zeta}\leq 1$, and this implies $y\in \cztope{\ptemp}{0}$.
Therefore, $A\lt(\cztope{\ptemp}{0}\rt)=\cztope{A\ptemp}{0}\subseteq \cztope{\ptemp}{0}$.
\end{proof}
%
\section{Template complex zonotope}
In reachability analysis based verification algorithms, the quality of
over-approximation of the reachable set computed by the algorithm, i.e,
the difference between the actual reachable set and the
over-approximation, affects the accuracy of verification.  So, for any
set representation used in reachability analysis, it is desirable that
there is an efficient way of refining a set to obtain smaller sets
that over-approximate the reachable set.  However, in the case of the
basic representation of a complex zonotope as given in the above
definition, adding a generator to the representation increases the
size of the complex zonotope.  Moreover, adding a generator can
violate positive invariance, as shown in Figure~\ref{todo}.  Instead,
to refine a complex zonotope, we may try to adjust the magnitude of
contribution of each generator to the size of the set in a way that
does not increase the size of the set and also preserves positive
invariance.  Therefore, in addition to the generators in the
representation of a complex zonotope, we can consider the magnitude of
contribution of each generator as a part of the representation.  This
would be a more general representation of a complex zonotope, which we
shall call as a template complex zonotope, and is defined as follows.

A template complex zonotope is a linear combination of complex valued
vectors that constitute the columns of a matrix, which we call as the
template, such that the magnitude of each combining coefficient is
bounded in its absolute value by a positive real, called a scaling factor.
%
\begin{definition}[Template complex zonotope]
Let us consider $\ptemp\in\mat{n}{m}{\compnums}$ called the template,
$\sfact\in\reals^m_{\geq 0}$ called scaling factors and
$\cen\in\compnums^n$ called the center.  Then the following is a template
complex zonotope.
%
\begin{equation}
\tcztope{\ptemp}{\cen}{\sfact}
= \set{\ptemp\zeta+\cen:~\absolute{\zeta_i}\leq \sfact_i~\forall
i\in\set{1,...,m}}.
\end{equation}
\end{definition}
%
The scaling factors of a template complex zonotope can be treated as
variables that can be adjusted to find better over-approximations of a
reachable set.  For example, Figure~\ref{toref} illustrates the
different over-approximations of a convex hull of five points by a
complex zonotopes with a fixed template but different scaling factors.
%
\section{Operations on a template complex zonotope}
In the reachability analysis of affine hybrid systems, linear
transformation, Minkowski sum, inclusion-checking and intersection
with linear constraints are the essential operations used in the
computation of reachable sets.  In this section, we describe the
computation of linear transformation, Minkowski sum and
inclusion-checking for template complex zonotopes.  Complex zonotopes
are not closed under intersection with linear constraints.  We address
the issue of intersection with linear constraints in a latter chapter
by using a more general representation of complex zonotopes.

Template complex zonotopes are closed under linear transformation and
Minkowski sum operations and these are straightforward algebraic
computations, just like in the case of simple (real) zonotopes.
%
\begin{proposition}[Linear transformation: template complex zonotope]
Let $A\in\mat{n}{n}{\reals}$ and $\ptemp\in\mat{n}{m}{\compnums}$.
Given a template complex zonotope $\tcztope{\ptemp}{\cen}{\sfact}$, we
get another template complex zonotope after multiplying the former
template complex zonotope by the matrix $A$, which is computed as
follows. 
%
\begin{equation}
A\tcztope{\ptemp}{\cen}{\sfact}=\tcztope{A\ptemp}{A\cen}{\sfact}.
\end{equation}
%
\end{proposition}
%
\begin{proof}
First we shall show
$A\tcztope{\ptemp}{\cen}{\sfact}\subseteq\tcztope{A\ptemp}{A\cen}{\sfact}.$
Let $y\in\tcztope{\ptemp}{\cen}{\sfact}$.  Then $y=\ptemp\zeta+\cen$
for some $\zeta\in\compnums^m$ such that
$\absolute{\zeta}\leq \sfact$.  We have $Ay=A\ptemp\zeta+A\cen$.  So,
$Ay\in\tcztope{A\ptemp}{Ac}{\sfact}$.  Therefore,
$A\tcztope{\ptemp}{\cen}{\sfact}\subseteq\tcztope{A\ptemp}{A\cen}{\sfact}.$
 
Now we have to show $\tcztope{A\ptemp}{A\cen}{\sfact}\subseteq
A\tcztope{\ptemp}{\cen}{\sfact}$.  Let
$z\in\tcztope{A\ptemp}{Ac}{\sfact}$.  Then $z=A\ptemp\zeta^\pr+A\cen$
for some $\zeta^\pr\in\compnums^m$ such that
$\absolute{\zeta^\pr}\leq \sfact$.  Equivalently, $z=
A(\ptemp\zeta^\pr+\cen)$.  So, $z\in
A\tcztope{\ptemp}{\cen}{\sfact}$.  Therefore, $\tcztope{A\ptemp}{A\cen}{\sfact}\subseteq
A\tcztope{\ptemp}{\cen}{\sfact}$.
From the previous two conclusions, we have
$A\tcztope{\ptemp}{\cen}{\sfact}=\tcztope{A\ptemp}{A\cen}{\sfact}.$
\end{proof}
%
\begin{proposition}[Minkowski sum: template complex zonotopes]
Let $\tcztope{\ptemp}{\cen}{\sfact}$ and
$\tcztope{\ptemp^\pr}{\cen^\pr}{\sfact^\pr}$ be two template complex
zonotopes which are subsets of $\compnums^n$, i.e.,
$\cen,\cen^\pr\in\compnums^n$.  The Minkowski sum of the two
template complex zonotopes is another template complex zonotope,
which is computed as follows.
%
\begin{equation}
\minsum{\tcztope{\ptemp}{\cen}{\sfact}}{\tcztope{\ptemp^\pr}{\cen^\pr}{\sfact^\pr}}
= \tcztope{\begin{bmatrix}\ptemp & \ptemp^\pr\end{bmatrix}}{\cen+\cen^\pr}{\begin{bmatrix}\sfact\\\sfact^\pr\end{bmatrix}}
\end{equation}
\end{proposition}
%
\begin{proof}
First we shall show that \\$\minsum{\tcztope{\ptemp}{\cen}{\sfact}}{\tcztope{\ptemp^\pr}{\cen^\pr}{\sfact^\pr}}
\subseteq \tcztope{\begin{bmatrix}\ptemp
& \ptemp^\pr\end{bmatrix}}{\cen+\cen^\pr}{\begin{bmatrix}\sfact\\\sfact^\pr\end{bmatrix}}$.
Let $y\in\tcztope{\ptemp}{\cen}{\sfact}$ and
$y^\pr\in\tcztope{\ptemp^\pr}{\cen^\pr}{\sfact^\pr}$ such that
$y=\ptemp\zeta+\cen:\absolute{\zeta}\leq \sfact$ and
$y^\pr=\ptemp^\pr\zeta^\pr+\cen^\pr:\absolute{\zeta^\pr}\leq \sfact^\pr$
for some complex valued vectors $\zeta$ and $\zeta^\pr$.  Then
$y+y^\pr = \ptemp\zeta+\ptemp^\pr\zeta^\pr = \begin{bmatrix} \ptemp
& \ptemp^\pr\end{bmatrix}\begin{bmatrix} \zeta\\\zeta^\pr\end{bmatrix}$.
We have
$\absolute{\begin{matrix} \zeta\\\zeta^\pr\end{matrix}}\leq\begin{bmatrix}\sfact\\\sfact^\pr\end{bmatrix}$.
Therefore,
$\minsum{\tcztope{\ptemp}{\cen}{\sfact}}{\tcztope{\ptemp^\pr}{\cen^\pr}{\sfact^\pr}}
\subseteq \tcztope{\begin{bmatrix}\ptemp
& \ptemp^\pr\end{bmatrix}}{\cen+\cen^\pr}{\begin{bmatrix}\sfact\\\sfact^\pr\end{bmatrix}}$.
Next we have to show $\tcztope{\begin{bmatrix}\ptemp
& \ptemp^\pr\end{bmatrix}}{\cen+\cen^\pr}{\begin{bmatrix}\sfact\\\sfact^\pr\end{bmatrix}} \subseteq \minsum{\tcztope{\ptemp}{\cen}{\sfact}}{\tcztope{\ptemp^\pr}{\cen^\pr}{\sfact^\pr}}$.
Consider that $y\in\tcztope{\begin{bmatrix}\ptemp
& \ptemp^\pr\end{bmatrix}}{\cen+\cen^\pr}{\begin{bmatrix}\sfact\\\sfact^\pr\end{bmatrix}}$.
Then $y=\begin{bmatrix}\ptemp
&\ptemp^\pr\end{bmatrix}\begin{bmatrix}\zeta\\ \zeta^\pr\end{bmatrix}=\ptemp\zeta+\ptemp\zeta^\pr$
for some complex valued vectors $\zeta$ and $\zeta^\pr$ such that $\absolute{\zeta}\leq \sfact$ and
$\absolute{\zeta^\pr}\leq\sfact^\pr$.  So,
$y\in\minsum{\tcztope{\ptemp}{\cen}{\sfact}}{\tcztope{\ptemp^\pr}{\cen^\pr}{\sfact^\pr}}$.
Therefore, $\tcztope{\begin{bmatrix}\ptemp
& \ptemp^\pr\end{bmatrix}}{\cen+\cen^\pr}{\begin{bmatrix}\sfact\\\sfact^\pr\end{bmatrix}} \subseteq \minsum{\tcztope{\ptemp}{\cen}{\sfact}}{\tcztope{\ptemp^\pr}{\cen^\pr}{\sfact^\pr}}$.
From the previous two conclusions, we get
$\minsum{\tcztope{\ptemp}{\cen}{\sfact}}{\tcztope{\ptemp^\pr}{\cen^\pr}{\sfact^\pr}}
= \tcztope{\begin{bmatrix}\ptemp
& \ptemp^\pr\end{bmatrix}}{\cen+\cen^\pr}{\begin{bmatrix}\sfact\\\sfact^\pr\end{bmatrix}}$.
\end{proof}
%
While computing positive invariants using a set representation,
deciding the inclusion of a given set inside another set is necessary
for ascertaining the positive invariance of the latter.  In the case
of complex zonotopes, checking the exact inclusion is a non-convex
optimization problem.  However, we shall later propose a convex
relaxation, which is a sufficient condition for checking the
inclusion.  But first we show the non-convexity of the exact inclusion
checking problem.  Deciding the inclusion between any two complex
zonotopes amounts to solving the following optimization problem.
%
\begin{lemma}[Exact inclusion: template complex zonotopes]
Consider two template complex zonotopes
$\tcztope{\ptemp}{\cen}{\sfact}$ and
$\tcztope{\ptemp^\pr}{\cen^\pr}{\sfact^\pr}$ such that
$\ptemp\in\mat{n}{m}{\compnums}$ and $\ptemp^\pr\in\mat{n}{r}{\compnums}$.  The inclusion
$\tcztope{\ptemp^\pr}{\cen^\pr}{\sfact^\pr}\subseteq\tcztope{\ptemp}{\cen}{\sfact}$
holds if and only if
\begin{equation}\label{eqn:exact-inclusion}
\max_{\set{\zeta^\pr\in\compnums^{r}:\absolute{\zeta^\pr}\leq \sfact^\pr}}\min_{\set{\zeta\in\compnums^m:\ptemp\zeta=\ptemp^\pr\zeta^\pr+\cen^\pr-\cen}}\max_{i=1}^m\lt(\absolute{\zeta_i}-s_i\rt)\leq 0
\end{equation}
\end{lemma}
%
\begin{proof}
We have
$\tcztope{\ptemp^\pr}{\cen^\pr}{\sfact^\pr}\subseteq\tcztope{\ptemp}{\cen}{\sfact}$
iff for every $\zeta^\pr\in\compnums^r:\absolute{\zeta^\pr}\leq \sfact^\pr$,
there exists
$\zeta\in\compnums^m:\ptemp\zeta+\cen=\ptemp^\pr\zeta^\pr+\cen^\pr~\wedge~\absolute{\zeta}\leq
\sfact$.  The former statement is expressed in terms of optimization
as \vspace{-2.2em}\[\hspace{10em}\max_{\set{\zeta^\pr\in\compnums^{r}:\absolute{\zeta^\pr}\leq \sfact^\pr}}\min_{\set{\zeta\in\compnums^m:\ptemp\zeta=\ptemp^\pr\zeta^\pr+\cen-\cen^\pr}}\max_{i=1}^m\lt(\absolute{\zeta_i}-s_i\rt)\leq
0. \]
\end{proof}
%
In the above Equation~\ref{eqn:exact-inclusion}, if
$\ptemp\zeta=\ptemp\zeta^\pr+\cen^\pr+\cen$ has a solution $\zeta^*$, then any other
solution can be written as $\zeta^*+v$ such that $v$ belongs to the
complex valued null-space of $\ptemp$.  So, the value computed by the
term \[\min_{\set{\zeta\in\compnums:\ptemp\zeta=\ptemp^\pr\zeta^\pr+\cen^\pr-\cen}}\max_{i=1}^m\lt(\absolute{\zeta_i}-s_i\rt)\]
is a point-wise minimum of a set of quadratic convex functions, which
is consequently neither convex nor concave in general.  As the
above maximization is defined over a non-concave function, it does
not belong to the class of convex optimization.

Now we shall discuss a set of lemmas that lead to a sufficient
condition for inclusion between two template complex zonotopes, which
is expressed as a set of second order conic constraints (SOCC) over
the scaling factors and some auxillary variables.  {\color{blue} More
  explanation required.}

\begin{lemma}
  Let $\sfact\in\reals^m{\geq 0}$, $\sfact^\pr\in\reals^r_{\geq 0}$
  and $\zeta^\pr\in\compnums^r$ such that
  $\absolute{\zeta^\pr}\leq\sfact^\pr$ and
  $\cen,\cen^\pr\in\compnums^n$.  Let $\ptemp\in\mat{n}{m}{\compnums}$
  and $\ptemp^\pr\in\mat{r}{m}{\compnums}$ be two complex valued
  matrices such that $\ptemp^\pr\diagonal{\sfact^\pr}=\ptemp\tmat$
  where $\tmat$ is a complex valued matrix.  Let $y\in\compnums^m$
  such that $\ptemp y=\cen^\pr-\cen$.  Consider a function
  $\func{\nu}{\reals_{\geq 0}^r}{\reals_{>0}^r}$ as $\forall
  j\in\set{1,...,m}$, $\nu_j(\sfact^\pr)=1$ if $\sfact^\pr_j=0$ and
  $\nu_j(\sfact^\pr)=\sfact^\pr_i$ otherwise.  We have the following
  inequality.
\begin{equation}\label{eqn:transfer-matrix}
\min_{\set{\zeta\in\compnums:\ptemp\zeta=\ptemp^\pr\zeta^\pr+\cen^\pr-\cen}}\max_{i=1}^m\lt(\absolute{\zeta_i}-\sfact_i\rt)\leq \max_{i=1}^m\lt(\sum_{j=1}^r\absolute{\tmat_{ij}}-\sfact_i\rt).
\end{equation}
{\color{red} correct lemma}.
\end{lemma}
\begin{proof}
{\color{red}{Correct proof}}
  Firstly, note that $\diagonal{\nu\lt(\sfact^\pr\rt)}$ is an
  invertible matrix because $\nu\lt(\sfact^\pr\rt)$ is a vector of
  positive reals by definition.  We can write
  \[\ptemp^\pr\zeta^\pr+\cen^\pr-\cen =
  \ptemp^\pr\diagonal{\nu\lt(\sfact^\pr\rt)}\lt(\diagonal{\nu\lt(\sfact^\pr\rt)}\rt)^{-1}\zeta^\pr+\cen^\pr-\cen.\]
  Since $\ptemp^\pr\diagonal{\sfact^\pr}=\ptemp\tmat$ and $\ptemp
  y=\cen^\pr-\cen$, by substitution in the above we get
  \[\ptemp^\pr\zeta^\pr+\cen^\pr-\cen = \ptemp\lt(\tmat\lt(\diagonal{\nu\lt(\sfact^\pr\rt)}\rt)^{-1}\zeta^\pr+y\rt).\]
  This gives
\begin{eqnarray}\label{eqn:ref1}
\begin{split}
&
\min_{\set{\zeta\in\compnums:\ptemp\zeta=\ptemp^\pr\zeta^\pr+\cen^\pr-\cen}}\max_{i=1}^m\lt(\absolute{\zeta_i}-\sfact_i\rt)\\
& \leq
\max_{i=1}^m\lt(\absolute{\lt(\tmat\lt(\diagonal{\nu\lt(\sfact^\pr\rt)}\rt)^{-1}\zeta^\pr\rt)_i}+\absolute{y_i}-\sfact_i\rt).
\end{split}
\end{eqnarray}
Since $\absolute{\zeta^\pr}\leq\sfact^\pr$ as given and
$\sfact^\pr\leq \nu\lt(\sfact^\pr\rt)$ by definition, so
$\infnorm{\lt(\diagonal{\nu\lt(\sfact^\pr\rt)}\rt)^{-1}\zeta^\pr}\leq
1$.  Using this inequality, we get that
$\max_{i=1}^m\lt(\sum_{j=1}^r\absolute{\tmat_{ij}}-\sfact_i\rt)$ is an
upper bound on the R.H.S of Equation~\ref{eqn:ref1}.
\end{proof}
%
Based on the above lemma, we define a relation between two complex
zonotopes which is a sufficient condition for the inclusion relation.
We also show that the relation is a partial order.
%
\begin{definition}
We define a relation $\order$ between two complex zonotopes as
follows.  Let $\tcztope{\ptemp}{\cen}{\sfact}$ and
$\tcztope{\ptemp^\pr}{\cen^\pr}{\sfact^\pr}$ be two complex
zonotopes such that $\ptemp\in\mat{n}{m}{\compnums}$ and
$\ptemp^\pr\in\mat{n}{r}{\compnums}$.  Then
$\tcztope{\ptemp}{\cen}{\sfact}\order\tcztope{\ptemp^\pr}{\cen^\pr}{\sfact^\pr}$
\end{definition}
