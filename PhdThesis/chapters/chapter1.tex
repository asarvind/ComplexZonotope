In the previous chapter, we described a set representation called
simple zonotope, which is type of polytope represented as a linear
combination of real-valued vectors with bounded combining
co-efficients.  The advantage of simple zonotopes in set based
computations is that they are closed under matrix multiplication and
Minkowski sum operations and these can be efficiently.  However, for
invariant computation, there is no known procedure to choose a
suitable set of generators for a simple zonotope so as to compute an
invariant.  In this chapter, we extend simple zonotope a new set
representation called \emph{complex zonotope} by which we can easily
specify invariants of a linear system using the eigenstructure of the
system.  Complex zonotopes have complex valued generators complex
combining co-efficients bounded in their absolute values, as opposed
to the real valued generators and real combining co-efficients of
simple zonotopes.  The real projections of complex zonotopes can
describe some non-polytopic sets in addition to the polytopic  
zonotopes, which are Minkowski sums of some ellipsoids along with line
segments.  Furthermore, complex zonotopes are also closed under matrix
multiplications and Minkowski sums and these can be computed
efficiently.

Apart from discussing operations on complex zonotopes like linear
transformations and Minkowski sums, in this chapter we derive a convex
program for checking inclusion between two complex zonotopes.  The
inclusion relation is a key ingredient for efficient invariant
computation, as we shall see in the latter chapters.  The organization
of this chapter is as follows.  PUT ORGANIZATION.

\section{Basic representation of a complex zonotope}
The basic representation of a complex zonotope is a generalization of
the Definition~\ref{} of simple zonotope to complex numbers, which is
a linear combination of complex valued vectors, called generators,
where the combining coefficients are bounded in absolute value by
unity.  However, this representation can be extended to more general
forms for efficient computation, as we shall discuss later.
%
\begin{definition}[Complex zonotope]
Let $V\in\mat{m}{n}{\compnums}$, which is a complex valued matrix,
whose columns are called generators, and $c\in\compnums^n$ be a
complex valued vector called the center.  Then the following is a
complex zonotope.
%
\begin{equation}
\cztope{V}{c} = \set{V\zeta:~\zeta\in\compnums^m,~\infnorm{\zeta}\leq 1}.
\end{equation}
%
\end{definition}
%
Like a simple zonotope, a complex zonotope is also symmetric around
the center.  To see this, consider a point in a complex zonotope
centered at the origin, written as $y=V\zeta$ where $V$ defines the
generator set and $\zeta$ is the vector of combining coefficients.
Since, $\infnorm{-\zeta}=\infnorm{\zeta}\leq 1$, so even
$-y=V(-\zeta)$ belongs to the complex zonotope.  However, unlike
simple zonotopes, the real projection of a complex zonotope can
describe some non-polytopic sets in addition to polytopic zonotopes,
since geometrically speaking the real projection of a complex zonotope
can be a Minkowski sum of line segements as well as some ellipsoids.
An example of such a non-polyhedral complex zonotope projection in
real space is illustrated with Figure~\ref{fig:cz}, where the
generators are $V=\lt(\begin{array}{lll}(1+2i) & 1 & (2+i)\\(1-2i) & 1
& (2-i)\end{array}\rt)$ and the center is the origin.

A motivation for extending simple zonotopes to complex zonotopes is
that a complex zonotope with its generators as the complex
eigenvectors of a discrete time linear system will be a positive
invariant if the complex eigenvalues corresponding to the generators
are bounded within unity in their absolute values.  This property is
described mathematically in the following proposition.
%
\begin{proposition}[Invariance based on eigenstructure]
Let us consider that $\cztope{V}{0}$ is a complex zonotope centered at
the origin, where the column vectors of $V\in\mat{m}{n}{\compnums}$
are the complex eigenvectors of a matrix $A\in\mat{n}{n}{\reals}$ such
that $AV = V\diagonal{\mu}$ and $\mu\in\compnums^m$ is a vector of
complex eigenvalues corresponding to the column vectors of $V$.
Then \[A\lt(\cztope{V}{0}\rt) = \cztope{V\diagonal{\mu}}{0}.\]
Therefore, if $\infnorm{\mu}\leq 1$, then
$A\lt(\rztope{V}{0}\rt)\subseteq \rztope{V}{0}$.
\end{proposition}
% 
\begin{proof}
We have $A\lt(\cztope{V}{0}\rt) =
A\set{V\zeta:~\zeta\in\compnums^m,\infnorm{\zeta}\leq 1}$\\
$=\set{AV\zeta:~\zeta\in\compnums^m,\infnorm{\zeta}\leq 1}
= \cztope{AV}{0}$.  Next $\cztope{AV}{0}
= \cztope{V\diagonal{\mu}}{0}$, which proves the first part of the
Proposition.  For the second part, we are given that
$\infnorm{\mu}\leq 1$.  Consider a point $y\in \cztope{V\diagonal{\mu}}{0}$,
described by the generators as
$y = V\diagonal{\mu}\delta:\infnorm{\delta}\leq 1$.
Let $\zeta = \diagonal{\mu}\delta$. Then
$\infnorm{\zeta} \leq \infnorm{\mu}\infnorm{\delta} \leq 1$.  So, 
$y=V\zeta$ where $\infnorm{\zeta}\leq 1$, and this implies $y\in \cztope{V}{0}$.
Therefore, $A\lt(\cztope{V}{0}\rt)=\cztope{AV}{0}\subseteq \cztope{V}{0}$.
\end{proof}
%
In reachability analysis based verification algorithms, the quality of
overapproximation of the reachable set computed by the algorithm, i.e,
the difference in some measure between the actual reachable set and
the overapproximation, affects the accuracy of verification.  So, for
any set representation used in reachability analysis, it is desirable
that there is an efficient way of refining a set so as to obtain
smaller sets that overapproximate the reachable set.  In this context,
we note that for a basic representation of a complex zonotope as given
in the above definition, adding a generator to the representation
increases the size of the complex zonotope.  Moreover, adding a
generator can distort the positive invariance of the set itself, as
illustrated in Figure~\ref{todo}.  Instead, we should be able to
change and also add generators to the complex zonotope in a way that
we obtain smaller overapproximations of during reachable set
computations.  In this regard, we introduce a more general
representation of a complex zonotope, called a template complex
zonotope, where in addition to the set of generators, called a
template, we have a vector of scaling factors that defines the
magnitude of contribution of each generator to the size of the set.
The template complex zonotope representation and its advantage
compared to the previous basic representation of a complex zonotope is
discussed in the following section.
%
\section{Template complex zonotope}
