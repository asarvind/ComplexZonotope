In the previous chapter, we described the set representation called
simple zonotopes, a sub-class of polytopes, which are geometrically
Minkowski sums of line segments.  The advantage simple zonotopes have
is that they are closed the very commonly used operations in reachability analysisunder, matrix multiplication and Minkowski sum, and also 
these operations can be computed exactly and efficiently.  Simple zonotopes are
represented as a linear combination of real-valued vectors, where the
combining co-efficients are bounded inside real-valued intervals.
However, for linear hybrid systems, the possibly complex-valued
eigenstructure of matrices, i.e., having both real and imaginary
parts, can be very useful in computing invariants and stability
verification, as we shall discuss later.  In this respect, we
introduce a new set representation called \emph{complex zonotope} in
this chapter, which extends simple zonotopes by having complex-valued
generators and complex-valued combining coefficients.  This allows us
to utilize the possibly complex eigenstructure of matrices for
efficient invariant computation.  Their real projections describe a
richer class of sets, which are Minkowski sums of line segments and
also some ellipsoids.  Still, like simple zonotopes, complex zonotopes
are also closed under matrix multiplications and Minkowski sums and
these can be computed efficiently.

This chapter is organized as follows.  Before introducing complex
zonotopes, we first discuss in detail the motivation behind
introducing complex zonotopes.  Then we introduce the set
representation of complex zonotopes and discuss operations like linear
transformation, Minkowski sum, eigenstructure based invariant
computation and inclusion checking.  Next, we draw comparison of
complex zonotope with other well known set representations, in terms of
their geometry and the efficiency in computing relevant set operations.
