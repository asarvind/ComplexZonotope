In the previous chapter, we described the set representation called
simple zonotopes, which are geometrically Minkowski sums of line
segments.  Henceforth, they are a subclass of polytopes.  They have
the advantage that common operations in reachability analysis like
Minkowski sums and linear transformations can be computed exactly and
efficiently.  Simple zonotopes are represented as a linear combination
of real-valued vectors, where the combining co-efficients are bounded
inside real-valued intervals.  However, for linear hybrid systems, the
eigenstructure of matrices, which can be complex-valued, i.e., have
both real and imaginary parts, is very useful in computing invariants
and stability verification, as we shall discuss later.  In this
respect, we introduce a new set representation called \emph{complex
  zonotope} in this chapter, which extends simple zonotopes by having
complex-valued generators and complex-valued combining coefficients,
so as to utilize the possibly complex eigenstructure in for efficient
invariant computation.  On the other hand, complex zonotopes still
retain the advantage of simple zonotopes that linear transformations
and Minkowski sums are are computed exactly and efficiently.
Furthermore, they describe a richer class of sets, since geometrically
their real projections are Minkowski sums of line segments and also
some ellipsoids which are non-polyhedral.

This chapter is organized as follows.  Before introducing complex
zonotopes, we first discuss in detail the motivation behind
introducing complex zonotopes.  Then we introduce the set
representation of complex zonotopes and discuss operations like linear
transformation, Minkowski sum, eigenstructure based invariant
computation and inclusion checking.  Next, we draw comparison of
complex zonotope with other well known set representations, in terms of
their geometry and the efficiency in computing relevant set operations.
