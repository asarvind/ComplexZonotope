In the previous chapter, we described the set representation called
simple zonotopes, a sub-class of polytopes, which are geometrically
Minkowski sums of line segments.  Simple zonotopes are represented as
a linear combination of real-valued vectors, where the combining
co-efficients are bounded inside real-valued intervals.  The advantage
of simple zonotopes in reachability analysis of linear systems is that
they are closed under matrix multiplication and Minkowski sum
operations and these can be efficiently.  However, for linear hybrid
systems, the possibly complex-valued eigenstructure of matrices, i.e.,
having both real and imaginary parts, can be very useful in computing
invariants and stability verification, as we shall discuss later.  In
this respect, this chapter introduces a new set representation
called \emph{complex zonotope}, which extends simple zonotopes by
having complex-valued generators and complex-valued combining
coefficients, and allows specification of invariants based on the
complex eigenstructure.  The real projections of complex zonotopes
describe a richer class of sets than simple zonotopes, which are
Minkowski sums of some ellipsoids along with line segments.  Still,
like simple zonotopes, complex zonotopes are also closed under matrix
multiplications and Minkowski sums and these can be computed
efficiently.

This chapter is organized as follows.  PUT ORGANIZATION.  

\section{Motivation}
When the eigenstructure of a linear system is real, it can be used to
specify a real zonotopic positive invariant for the system.  This is
based on the following result.
%
\begin{proposition}
Let us consider a simple zonotope centered at the origin
$\rztope{V}{0}$, where the column vectors of $V\in\mat{m}{n}{\reals}$
are the eigenvectors of a matrix $A\in\mat{n}{n}{\reals}$ such that
$AV = V\diagonal(\mu)$ where $\mu\in\reals^m$ is a vector of eigenvalues
corresponding to the eigenvectors among the columns of $V$.
Then \[A\lt(\rztope{V}{0}\rt) = \rztope{V\diagonal{\mu}}{0}.\] If
$\infnorm{\mu}\leq 1$, then
$A\lt(\rztope{V}{0}\rt)\subseteq \rztope{V}{0}$.
\end{proposition}
\begin{proof}
We have $A\lt(\rztope{V}{0}\rt) = \rztope{AV}{0}$ according to
Proposition~\ref{toref}.  Next $\rztope{AV}{0}
= \rztope{V\diagonal{\mu}}{0}$.  This proves the first part of the
Proposition.  For the second part, we are given that
$\infnorm{\mu}\leq 1$.  Let $y\in A\lt(\rztope{V}{0}\rt)
= \rztope{V\diagonal{\mu}}{0}$.  So, we can write $y$ as $y =
V\diagonal{\mu}\delta:\infnorm{\delta}\leq 1$.  Let $\epsilon
= \diagonal{\mu}\delta$. Then
$\infnorm{\epsilon} \leq \infnorm{\mu}\infnorm{\delta} \leq 1$.  So,
$y=V\epsilon$ where $\infnorm{\epsilon}\leq 1$,
and hence $y\in \rztope{V}{0}$.  Therefore,
$A\lt(\rztope{V}{0}\rt)\subseteq \rztope{V}{0}$.
\end{proof}
%
The above proposition means that we can easily specify a positively
invariant simple zonotope for a stable discrete time linear system
with real eigenstructure by having only the (real) eigenvectors of the linear
system as the generators of the zonotope.

[INCLUDE FIGURE]

%%========================================================================================
%% In the following, we try to derive the same result for a polytope,
%% but additional details are required for it to be correct.  So,
%% we commented it out.

%% \begin{proposition}
%% Consider a polytope $\polytope{J}{d}: J\in\mat{n}{n}{\reals}$ where the
%% rows of $T$ are the left eigenvectors of a matrix $A\in\mat{n}{n}{R}$, i.e.,
%% $\forall i\in\{1,...,n\}$, $J_iA
%% = \mu_iJ_i:~\mu_i\in\reals$.  We have
%% $A\lt(\polytope{J}{d}\rt)\subseteq\polytope{J}{\diagonal(\mu)d}$.
%% \end{proposition}
%% \begin{proof}
%% We have $A\lt(\polytope{J}{d}\rt)=A\lt\{x\in\reals^n:Jx\leq
%% d\rt\}=\lt\{Ax:x\in\reals^n,Jx\leq d\rt\}$.  Consider any $y=Ax\in
%% A\lt(\polytope{J}{d}\rt)$.  Then $(Jy)_i=J_iAx=\mu_iJ_ix\leq\mu_id_i$.  
%% \end{proof}

%%========================================================================================
