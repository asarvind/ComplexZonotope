This dissertation contains five main chapters and a conclusive
chapter.  In the first chapter, we have briefly discussed the
desirable characteristics of a good set representation for computing
accurate positive invariants of hybrid systems.  In light of these
characteristics, we provided a short review of some set
representations and discussed their advantages and drawbacks.  We have
discussed simple zonotopes, which are defined over real numbers, and
the linear transformation and Minkowski sum operations on them. We
explained the shortcoming of real zonotopes that they can not capture
contraction along complex valued eigenvectors, which is closely
related to computing positive invariants.  We also reviewed some of
the previously known extensions of real zonotopes.

In Chapter~\ref{ch:tcz}, we introduce the complex zonotope set
representation and discuss set operations on them required for
reachability analysis.  We first define the basic representation of a
complex zonotope.  We derive a result that a complex zonotope can
efficiently represent a non-zero positive invariant for a stable
linear transformation based on the eigenstructure.  Next we introduce
the template based representation of a complex zonotope, called
\emph{template complex zonotope}.  It allows adding more generators to
a template for increasing the quality of approximation by the complex
zonotope.  Later we discuss set operations on template complex
zonotopes and derive a second order conic program for checking
inclusion between two template complex zonotopes.

In Chapter~\ref{ch:acz}, we generalize complex zonotope to an
\emph{augmented complex zonotope} representation for computing
over-approximation of the intersection with a class of sub-level sets
of linear inequalities called \emph{sub-parallelotopes}.  We first
introduce a representation called \emph{interval zonotope}, which is a
more general but geometrically equivalent representation of a simple
zonotope.  An interval zonotope has variable interval bounds on the
combining coefficients.  We show that the intersection between an
interval zonotope and a suitably aligned sub-parallelotope is another
interval zonotope which can be computed algebraically.  Motivated by
this result, we define an augmented complex zonotope so as to
efficiently compute the over-approximation of its intersection with a
sub-parallelotope.  Latter we discuss other set operations on
augmented complex zonotopes.  We extend the inclusion-checking
relation between template complex zonotopes to augmented complex
zonotopes.

In Chapter~\ref{ch:invariance}, we describe a discrete time affine
hybrid system, its positive invariants and the problem of verifying a
\emph{linear invariance property}.  We develop a convex program to
verify a linear invariance property based on computing positively
invariant augmented complex zonotopes.  We implement the method on
some benchmark examples to demonstrate its efficiency.

In Chapter~\ref{ch:lis}, we develop an algorithm for exponential
stability verification of linear impulsive systems with sampling
uncertainty using template complex zonotopes.  The algorithm uses the
eigenstructure of reachability operators to compute a contractive set.
We implement the algorithm for stability verification of two benchmark
examples and compare the results with state-of-the-art methods.

The concluding remarks are given in Chapter~\ref{ch:conclusion}.


