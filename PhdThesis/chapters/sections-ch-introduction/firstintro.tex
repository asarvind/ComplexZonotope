
Physical systems controlled by digital logic exhibit a mix of
continuous dynamics, described by differential or difference
equations, and discrete dynamics described by switching of discrete
valued variables.  Mathematical models describing the behavior of such
systems are called \emph{hybrid systems}.  Formal verification of
hybrid systems generally requires computing the set of reachable
states of the system.  But exactly computing the reachable states of a
hybrid system is either undecidable or computationally expensive.  For
hybrid systems that have differential equations, exact computation of
the reachable set is undecidable~\cite{alur1995algorithmic}, except in
particular cases~\cite{lafferriere1998decidable}.  Even in a simpler
case like a discrete time affine hybrid system, the exact reachable
set at any time is a union of sets whose number is exponential in the
number of time steps.  An alternative is find a sufficiently accurate
over-approximation of the infinite set of reachable states.  To do so,
we need to represent an infinite set of states symbolically by a {\it
set representation}, which can be efficiently manipulated for the
computation of the desired over-approximation.


Verification of some properties of hybrid systems requires
over-approximating the unbounded time reachable set of a system.  The
unbounded time reachable set can be approximated by a \emph{positive
invariant}, which is a set of states whose the set of successor states is
contained within itself.  The efficiency, in terms of accuracy and
computational speed, for computing positive invariants using a set
representation is related to the following characteristics of the set
representation.
%
\begin{enumerate}
\item Closure under set operations used in reachability analysis and computational complexity of
performing them.  Some of these operations are linear transformation,
Minkowski sum, intersection, computation of support function and
inclusion-checking.
\item Efficient encoding of positive invariants in the set representation.
\end{enumerate}

Well-known set representations have some of the above characteristics,
but not all.  For example, polytopes have the advantage that they are
closed under linear transformation, Minkowski sum and intersection.
However, for a half-space representation of a polytope, the Minkowski
sum operation is computationally expensive in higher dimensions.
Moreover, to our knowledge, there is no upper bound on the
representation size of a non-zero polytopic positive invariant for a
stable linear system.  Ellipsoids have the advantage that they are
closed under linear transformation and also efficiently encode the
positive invariant of a stable linear system.  But ellipsoids are not
closed under Minkowski sum and intersection with half-spaces.
Although there has been work on over-approximation of the Minkowski
sum and intersection with half-spaces for
ellipsoids~\cite{allamigeon2017fast,kurzhanskiy2006ellipsoidal}, still
there can be a significant approximation
error. Zonotope~\cite{DBLP:conf/hybrid/Girard05} is yet another set
representation, which is a type of polytope specified as a linear
combination of real vectors whose combining coefficients are bounded
in the interval $[0,1]$.  Geometrically, they are Minkowski sums of
line segments.  

Zonotopes have the advantage that they are closed under linear
transformation and Minkowski sum operations, which can also be
computed efficiently.  Therefore, they have been successfully applied
to the computation of bounded time reachable sets of uncertain
continuous linear systems~\cite{DBLP:conf/hybrid/Girard05} and affine
hybrid systems with simple
switching~\cite{makhlouf2014networked,girard2008zonotope}.  But for
over-approximation of the {\it unbounded time reachable set by a
positive invariant}, a drawback of real zonotopes can be explained as follows.  
The accuracy of over-approximation by a positive invariant
is closely related to the capturing the directions for convergence of
the states to an equilibrium.  In affine hybrid systems, some of these
directions can be encoded by the complex eigenvectors of a linear
transformation.  In case of real zonotopes, we shall show that when
the eigenvectors of a stable linear transformation are real valued,
then collecting the eigenvectors of the matrix among the generators of
the zonotope gives a positively invariant real zonotope.  However,
this result does not hold when the eigenvectors have complex values,
i.e., have non-zero imaginary and real parts.  In other words, real
zonotopes can not exploit the complex valued eigenstructure of the
transformation matrices for efficient computation of a positive
invariant, because they have real valued generators.

\begin{table}
\resizebox{\textwidth}{!}{
\begin{tabular}{|l|c|c|c|c|}
\hline
Set & Linear & Minkowski & {Intersection } & {Positive Invariant: }\\
representation & transformation & sum & with half-space & stable
linear system\\
\hline
\multirow{2}{*}{Convex polytope} & {Efficient } & More than
& \multirow{3}{*}{Efficient}  &  {Maximum complexity}\\
\multirow{2}{*}{$H$-representation} & only for & exponential  &  & of encoding  \\
& invertible matrix & complexity & & not bounded \\
\hline
\multirow{2}{*}{Zonotope} & \multirow{2}{*}{Efficient}
& \multirow{2}{*}{Efficient} & Not & May not\\
& & & closed & exist\\
\hline
\multirow{2}{*}{Ellipsoid} & \multirow{2}{*}{Efficient} & Not & Not &
Efficient\\
& & closed & closed & encoding\\
\hline
\multirow{2}{*}{Polynomial} & More than & More than & \multirow{3}{*}{Efficient} &
\multirow{2}{*}{Efficient} \\
\multirow{2}{*}{sub-level set} & exponential & exponential &
& \multirow{2}{*}{encoding}\\
& complexity & complexity & & \\
\hline
{Complex} & \multirow{2}{*}{Efficient} & \multirow{2}{*}{Efficient} &
Not & Efficient\\
Zonotope & & & closed & encoding\\
\hline
\end{tabular}
}
\caption{Comparison of set representations}~\label{tab:compset}
\end{table}


To capture contraction along complex valued vectors, we extend real
zonotopes to the complex valued domain to yield a set representation
called {\it complex zonotope}.  A complex zonotope is a linear
combination of complex valued vectors with absolute valued bounds on
the combining coefficients.  It can capture contraction along the
complex eigenvectors of a linear system, due to which they can
efficiently encode non-zero positive invariants of a linear system.
It is also geometrically more expressive than real zonotope because
its real projection can represent Minkowski sums of some ellipsoids in
addition to line segments.  They are also different from other
extensions of complex zonotopes, like quadratic
zonotopes~\cite{DBLP:conf/aplas/AdjeGW15}, which we shall explain
later in a separate section.  Still, complex zonotopes retain the
merit of real zonotopes that they are closed under linear
transformation and Minkowski sum, which can also be computed
efficiently.  In Table~\ref{tab:compset}, we draw comparison of
complex zonotopes with four categories of set representations
comprising convex polytopes, zonotopes, ellipsoids and polynomial
sub-level sets.  A review of the other set representations and related
work on computing positive invariants will be presented in a separate
section.

We apply complex zonotopes to two problems in the domain of hybrid
systems that require computing accurate positive invariants.  One
problem is verifying \emph{linear invariance properties} of discrete
time affine hybrid systems.  A linear invariance property is a linear
constraint on the state of the system that is satisfied by all the
reachable states.  We derive a convex program based for computing
positively invariant complex zonotopes that verify a linear invariance
property.  We demonstrate the efficient of our method by experiments
on some benchmark examples.  The other problem is verifying stability
of linear impulsive systems with sampling uncertainty.  We develop an
algorithm that finds a contractive complex zonotope using the
eigenstructure of the system.  Our experiments on two benchmark
examples show either better or comparable performance compared to
other approaches.




