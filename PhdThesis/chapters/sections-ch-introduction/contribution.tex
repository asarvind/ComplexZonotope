The main contributions of our work is summarized below.
%
\begin{itemize}
\item \emph{Complex zonotope: } Our most important contribution is a
  new set representation called complex zonotope for verification of
  hybrid systems, that can capture contraction along complex
  eigen-vectors but still is computationally as efficient as real
  zonotopes.  Just like real zonotopes, a complex zonotope is closed
  under Minkowski sum, linear transformation and their computation is
  also efficient.  Similar to a simple (polytopic) zonotope, the
  support function of a complex zonotope can be computed by a simple
  algebraic expression.  But complex zonotopes can represent
  non-polytopic sets in addition to simple zonotopes.  They can
  capture contractive along complex valued eigen-vectors when there
  are non-zero real and imaginary parts, which real zonotopes can not.
\item \emph{Template based representation: }  We introduce a template
  based representation of a complex zonotope, by which we can
  add generators to a complex zonotope to find better approximations.
  In a real zonotope, adding a generator to the zonotope can increase
  the size of the denoted set.  This problem is addressed by the
  template based representation of a complex zonotope, which has a set
  of scaling factors that determine the amount of contribution of each
  \emph{generator} to the size of the set.  Henceforth, when we add
  more generators to a complex zonotope, the scaling factors can be
  suitably modified to find a better approximation than that of the
  former complex zonotope.
\item \emph{Intersection with half-spaces: } Similar to real
  zonotopes, complex zonotopes are also not closed under intersection
  with half-spaces.  Previous
  approaches~\cite{scott2016constrained,Ghorbal2010} have addressed
  this problem for real zonotopes by allowing more linear contraints
  on the combining coefficients in addition to the interval bounds.
  For real zonotopes, even with addition of arbitrary linear
  constraints, the support function can be computed efficiently by
  linear programming.  But in the case of complex zonotopes, if we add
  more constraints than the quadratic absolute value bounds on the
  combining coefficients, accurate computation of the support function
  becomes intractable.  Alternatively, we generalize complex zonotopes
  to a set representation called \emph{augmented complex zonotopes} to
  over-approximate the intersection with a particular class of linear
  sub-level sets called \emph{sub-parallelotopes}.  An augmented
  complex zonotope is geometrically equivalent to a complex zonotope,
  where the computation of support function is just as efficient as
  the latter.  However, we show that an over-approximation of the
  intersection of an augmented complex zonotope with a
  sub-parallelotope can be efficiently encoded as another augmented
  complex zonotope.  Furthermore, the error in
  over-approximation can be regulated by adjusting the scaling
  factors.
\item \emph{Convex program for verifying linear invariance property:
}  We develop a convex program based on computing positively invariant
  augmented complex zonotopes to verify linear invariance properties
  of discrete time affine hybrid systems.  We perform experiments on
  three benchmark examples that demonstrate the efficiency of our
  approach. 
\item \emph{Stability verification of linear impulsive systems: } We
  develop an algorithm to find contractive complex zonotopes that
  verify global exponential stability of linear impulsive systems with
  sampling uncertainty.  The novelty of our algorithm lies in using
  the eigenstructure of reachability operators for stability
  verification, i.e., to find contractive complex zonotopes. Our
  experiments on two benchmark examples demonstrate either better or
  competitive performance compared to state of the art approaches.
\end{itemize}
%
