%The main contributions of our work is summarized below.
%
\begin{itemize}
\item \emph{Complex zonotope: } Our most important contribution is the
  extension of real zonotopes to the complex valued domain by
  complex zonotopes, which can capture contraction along complex
  vectors, but still are computationally as efficient as a real
  zonotope.  Just like real zonotopes, a complex zonotope is closed
  under Minkowski sum, linear transformation and their computation is
  also efficient.  The support function of a complex zonotope can also
  be computed by a simple algebraic expression.  But additionally, a
  complex zonotope can efficiently encode positive invariants for
  linear transformations by incorporating complex eigenvectors as
  generators.  On the other hand, eigenvectors having non-zero real
  and imaginary parts can not be used as generators in real zonotopes.
  Moreover, complex zonotopes are geometrically more expressive since
  their real projections can represent non-polytopic sets in addition
  to polytopic zonotopes.
\item \emph{Template based representation: } We introduced a template
  based representation of a complex zonotope, by which we can add
  generators to a complex zonotope to find better approximations.  In
  a real zonotope, adding a generator can increase the size of the
  denoted set.  This problem is addressed by the template based
  representation of a complex zonotope, which has a set of scaling
  factors that determine the amount of contribution of each generator
  to the size of the set.  Henceforth, to find a better
  approximations, we can add more generators to a complex zonotope and
  adjust the scaling factors.
\item \emph{Set operations: } Apart from the simpler computations like
  linear transformation, Minkowski sum and support function, we
  developed a convex program to efficiently check-inclusion between
  template complex zonotopes.  Then, we developed the augmented
  complex zonotope representation to efficiently over-approximate the
  intersection with sub-parallelotopes.  The error in
  over-approximation can be regulated by adjusting the scaling
  factors.  The main advantage of augmented complex zonotope is that
  we can still compute the support function efficiently.  It
  provided an alternative to the previous known variations of real
  zonotopes for computing the intersection, because their extension to
  complex zonotope makes computation of the support function
  intractable.
%% \item \emph{Intersection with half-spaces: } Similar to real
%%   zonotopes, complex zonotopes are also not closed under intersection
%%   with half-spaces.  Previous approaches have addressed this problem
%%   for real zonotopes by allowing more linear constraints on the
%%   combining coefficients in addition to the interval bounds.  For real
%%   zonotopes, even with addition of arbitrary linear constraints, the
%%   support function can be computed efficiently by linear programming.
%%   But in the case of complex zonotopes, if we add more constraints
%%   than the quadratic absolute value bounds on the combining
%%   coefficients, accurate computation of the support function becomes
%%   intractable.  Alternatively, we generalized complex zonotopes to a
%%   set representation called \emph{augmented complex zonotopes} to
%%   over-approximate the intersection with a particular class of linear
%%   sub-level sets called \emph{sub-parallelotopes}.  An augmented
%%   complex zonotope is geometrically equivalent to a complex zonotope,
%%   where the computation of support function is just as efficient as
%%   the latter.  However, we showed that an over-approximation of the
%%   intersection of an augmented complex zonotope with a
%%   sub-parallelotope can be efficiently encoded as another augmented
%%   complex zonotope.  Furthermore, the error in over-approximation can
%%   be regulated by adjusting the scaling factors.
\item \emph{Verification of discrete time affine hybrid systems:
}  We developed a convex program based on computing positively invariant
  augmented complex zonotopes to verify linear invariance properties
  of discrete time affine hybrid systems.  We performed experiments on
  three benchmark examples that demonstrate the efficiency of our
  approach. 
\item \emph{Stability verification of linear impulsive systems: } We
  developed an algorithm to find contractive complex zonotopes that
  verify global exponential stability of linear impulsive systems with
  sampling uncertainty.  The novelty of our algorithm lies in using
  the eigenstructure of reachability operators for stability
  verification, i.e., to find contractive complex zonotopes. Our
  experiments on two benchmark examples demonstrate either better or
  competitive performance compared to state of the art approaches.
\end{itemize}
%
