In hybrid dynamical systems, a transition canxs be controlled by
constraints on the state variables which act as preconditions for the
transition.  In an affine hybrid system, the preconditions are linear
constraints on the state variables.  So, computing an
overapproximation of a reachable set in a set representation would
involve computing accurate overapproximation for the intersection with
linear constraints.  In case of complex zonotopes, similar to simple
zonotopes they are not closed under intersection with sub-level sets
of linear inequalities.  To address this problem, we shall introduce a
more general representation of a complex zonotopic set, called
augmented complex zonotope, by which we can over-approximate the
intersection with a particular class of linear constraints, called
\emph{sub-parallelotopic}.  Furthermore, we can regulate the error in
overapproximation by adjusting the scaling factors.  Before describing
an augmented complex zonotope, we briefly discuss related work on
zonotopes and compare it with the augmented complex zonotopes.  

{\bf Related work: } To account for the intersection operation with
real zonotopes, these sets have been extended to constrained
zonotopes~\cite{scott2016constrained} or more generally constrained
affine sets~\cite{Ghorbal2010}.  Constrained zonotopes are briefly described
in the introductory chapter.  In these representations, in addition to
the interval constraints on the combining coefficients, there can be
more general linear constraints.  Intersection with hyperplanes, or
more generally half-spaces, can be computed efficiently in the
representation.

{\bf Drawback in extending earlier approaches to complex zonotopes: }
In a constrained zonotope or a constrained affine set, the support
function, i.e., projection along a direction, can be computed
efficiently by linear programming because there are only linear
constraints on the combining coefficients.  Also, in our complex
zonotope representation, although there are quadratic constraints
(absolute value bounds) on the combining coefficients, the support
function can be computed by a simple affine expression given in
Lemma~\ref{lem:support-tcz}.  However, if we have linear constraints on the
complex combining coefficients in addition to the quadratic absolute
value bounds, computing the support function becomes intractable.  But
accurate computation of the support function is required in verifying
bounds on the reachable set.  In other words, extending the idea of
constrained zonotope or constrained affine set to a complex zonotope
will make the computation of support function intractable.

Henceforth, we want to develop a set representation for
overapproximating the intersection such that computation of the
support function is still efficient.  In this regard, we generalize
complex zonotope to an augmented complex zonotope representation.  We
shall observe that for particular cases, intersection of a version of
simple zonotope, called interval zonotope, with sub-parallelotopic
constraints can be computed efficiently.  Motivated by this, we define
an augmented complex zonotope as a Minkowski sum of a complex zonotope
and an interval zonotope.  However, the real projection of an
augmented complex zonotope is geometrically equivalent to a real
zonotope, which allows efficient computation of the support function.
Furthermore, we can use the interval zonotope part of augmented
complex zonotope to over-approximate the intersection.  We show that
the error in overapproximation can be regulated by adjusting the
scaling factors and the center.  The intersection operation will be
explained later in detail.


The chapter has three main sections.  In Section~\ref{sec:iztope}, we
introduce the interval zonotope and sub-parallelotope set
representations and describe the intersection between them.
Furthermore, we discuss other operations on interval zonotopes like
linear transformation, Minkowski sum and computation of the support
function.  In Section~\ref{sec:acz-it}, we introduce the augmented
complex zonotope and described its intersection with a
sub-parallelotope.  We compute a bound on over-approximation of the
intersection between them, which is dependent on the scaling factors
and the center.  In Section~\ref{sec:operations-acz}, we shall discuss
other operations on augmented complex zonotopes like linear
transformation, Minkowski sum and computation of the support function.
