In hybrid dynamical systems, a transition can be controlled by
constraints on the state variables which act as preconditions for the
transition.  In an affine hybrid system, the preconditions are linear
constraints on the state variables.  So, computing an
overapproximation of a reachable set in a set representation would
involve computing accurate overapproximation for the intersection with
linear constraints.  In case of complex zonotopes, similar to simple
zonotopes they are not closed under intersection with sub-level sets
of linear inequalities.  To address this problem, we shall introduce a
more general representation of a complex zonotopic set, called
augmented complex zonotope, by which we can over-approximate the
intersection with a particular class of linear constraints, called
\emph{sub-parallelotopic}.  Furthermore, we can control the error in
overapproximation by adjusting the scaling factors.  Before describing
an augmented complex zonotope, we briefly discuss related work on
zonotopes and compare it with the augmented complex zonotopes.  

{\bf Related set representations and problem with their extension to complex
zonotopes: } To accurately represent the intersection with linear
sub-level sets, real zonotopes have been extended to constrained
zonotopes~\cite{scott2016constrained} or more generally constrained
affine sets~\cite{Ghorbal2010}.  Constrained zonotopes are briefly
described in the introductory chapter.  In these representations, in
addition to the interval constraints on the combining coefficients,
there can be more general linear constraints.  This permits exact
representation of the intersection with hyperplanes, in the case of
constrained zonotopes, and half-spaces, in the case of constrained
affine sets.  In these extended representations, the support function
can still be computed efficiently by linear programming because there are
only linear constraints on the combining coefficients.  Similarly, in
our complex zonotope representation, although there are quadratic
constraints (absolute value bounds) on the combining coefficients, the
support function can be computed by a simple affine expression given
in Lemma~\ref{lem:support-tcz}.  However, if we introduce linear
constraints on the complex combining coefficients in addition to the
quadratic absolute value bounds, computing the support function
becomes intractable.  But accurate computation of the support function
is required in verifying bounds on the reachable set.  In other words,
extending the idea of constrained zonotope or constrained affine set
to a complex zonotope will make the computation of support function
intractable.

{\bf Our solution: } We observe that in certain cases, intersection of
a version of the real zonotope representation, called \emph{interval
zonotope}, with a particular class of linear sub-level sets,
called \emph{sub-parallelotopes}, can be computed efficiently.
Motivated by this, we introduce a more general representation of
complex zonotope, called \emph{augmented complex zonotope}, which
denotes the Minkowski sum of a complex zonotope and an interval
zonotope.  We show that the interval zonotope part of an augmented
complex zonotope can be used to over-approximate the intersection.  On
the other hand, we can still compute the support function efficiently
because an augmented complex zonotope is geometrically equivalent to a
template complex zonotope.  Furthermore, we show that the error in
overapproximation can be regulated by adjusting the scaling factors
and the center of the template complex zonotope part.

This chapter has three main sections.  In Section~\ref{sec:iztope}, we
introduce the interval zonotope and sub-parallelotope set
representations, and describe the intersection between them.
Furthermore, we discuss other operations on interval zonotopes like
linear transformation, Minkowski sum and computation of the support
function.  In Section~\ref{sec:acz-it}, we introduce the augmented
complex zonotope and described its intersection with a
sub-parallelotope.  We compute a bound on the over-approximation of
the intersection between them, which is dependent on the scaling
factors and the center.  In Section~\ref{sec:operations-acz}, we shall
discuss other operations on augmented complex zonotopes like linear
transformation, Minkowski sum and computation of the support function.
