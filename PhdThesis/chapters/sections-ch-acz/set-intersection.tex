\section{Augmented Complex Zonotope and Intersection with Sub-parallelotope}
In Lemma~\ref{lem:motivation}, we have shown that the intersection of
a possibly translated interval zonotope and a sub-parallelotope, with
aligned templates, can be represented as another translated interval
zonotope that can be computed by a simple algebraic formula.
Motivated by this, we extend template complex zonotope to a more
general representation called {\it augmented complex zonotope},
which is specified as a Minkowski sum of a template complex zonotope
and an interval zonotope.  The idea behind such a representation is
that the interval zonotope part is used to approximate the
intersection with a sub-parallelotope, while the template complex
zonotope may capture positive invariance using a complex
eigenstructure.  An augmented complex zonotope is
mathematically represented as follows.
%
\begin{definition}
Let us consider a template complex zonotope
$\tcztope{\ptemp}{\cen}{\sfact}\subseteq\compnums^n$ and an interval
zonotope $\iztope{\stemp}{\lb}{\ub}\subseteq\reals^n$.  Then the
following is the representation of an augmented complex zonotope.
%
\[
\acztope{\ptemp}{\cen}{\sfact}{\stemp}{\lb}{\ub}=\minsum{\tcztope{\ptemp}{\cen}{\sfact}}{\iztope{\stemp}{\lb}{\ub}}.
\]
%
\end{definition}
%
The following lemma gives a formula for the intersection between
an augmented complex zonotope and a sub-parallelotope, which is later
used to derive an over-approximation of the intersection by another
augmented complex zonotope.  In the below result, we require that the
secondary template of the augmented complex zonotope is aligned with
the sub-parallelotopic template, similar the case of Lemma~\ref{lem:motivation}.
%
\begin{lemma}
Let $\qtemp$ be a sub-paralleotopic template.  Then,
%
\begin{align*}
&
\acztope{\ptemp}{\cen}{\sfact}{\pinv{\qtemp}}{\lb}{\ub}\bigcap\ptope{\qtemp}{\plb}{\pub}=\\
&
\bigcup_{x\in\tcztope{\ptemp}{\cen}{\sfact}}\lt(x\oplus\iztope{\pinv{\qtemp}}{\join{\lb}{\lt(\plb-\qtemp
  x\rt)}}{\meet{\ub}{\lt(\pub-\qtemp x}\rt)}\rt).
\end{align*}
%
\end{lemma}
%
\begin{proof}
We get $\acztope{\ptemp}{\cen}{\sfact}{\pinv{\qtemp}}{\lb}{\ub}\bigcap\ptope{\qtemp}{\plb}{\pub}=$
%
\begin{align*}
& \lt(\tcztope{\ptemp}{\cen}{\sfact}\oplus\iztope{\pinv{\qtemp}}{\lb}{\ub}\rt)\bigcap\ptope{\qtemp}{\plb}{\pub}\\
& =
\bigcup_{x\in\tcztope{\ptemp}{\cen}{\sfact}}\lt(\lt(x\oplus\iztope{\pinv{\qtemp}}{\lb}{\ub}\rt)\bigcap\ptope{\qtemp}{\plb}{\pub}\rt)\\
& =\bigcup_{x\in\tcztope{\ptemp}{\cen}{\sfact}}\lt(\lt(x\oplus\iztope{\pinv{\qtemp}}{\lb}{\ub}\rt)\bigcap\ptope{\qtemp}{\plb}{\pub}\rt)\\
& \%\%~~\text{by Lemma~\ref{lem:motivation}}\\
& = \bigcup_{x\in\tcztope{\ptemp}{\sfact}{\cen}}\lt(x\oplus\iztope{\pinv{\qtemp}}{\join{\lb}{\lt(\plb-\qtemp
  x}\rt)}{\meet{\ub}{\lt(\pub-\qtemp x\rt)}}\rt).\hspace{4em}\qedhere
\end{align*}
%
\end{proof}
%
\begin{theorem}[Over-approximation of intersection]~\label{thm:acz-ptope-intersection}
Let us consider a sub-parallelotope $\ptope{\qtemp}{\plb}{\pub}\subseteq\reals^n$ and an
augmented complex zonotope
$\acztope{\ptemp}{\cen}{\sfact}{\pinv{\qtemp}}{\lb}{\ub}\subset\compnums^n$. Let us denote
$S_1=\real\lt(\intersection{\acztope{\ptemp}{\cen}{\sfact}{\pinv{\qtemp}}{\lb}{\ub}}{\ptope{\qtemp}{\plb}{\pub}}\rt)$ and\\
$S_2=\real\lt(\acztope{\ptemp}{\cen}{\sfact}{\pinv{\qtemp}}{\join{\ub}{\pub}}{\meet{\lb}{\plb}}\rt)$.
Let us consider the Hausdorff distance between $S_1$ and $S_2$, given by
%
\[
\hausdorff{S_1}{S_2}=\max\lt(\sup_{w\in S_1}\inf_{v\in S_2}\sqnorm{w-v},\sup_{w\in S_2}\inf_{v\in S_1}\sqnorm{w-v}\rt).
\]
Then $S_1\subseteq S_2$ and 
%
\[
\hausdorff{S_1}{S_2}\leq \sup_{x\in \tcztope{\ptemp}{\cen}{\sfact}}\hausdorff{\iztope{\qtemp{}{}}{}
\]
%
\end{theorem}
%
\begin{proof}
Let $S_1=\tcztope{\ptemp}{\cen}{\sfact}$,
$S_2=\iztope{\pinv{\qtemp}}{\lb}{\ub}$ and
$S_3=\ptope{\qtemp}{\plb}{\pub}$.  We get
%
\[
0=\cen-\cen=\cen-\ptemp z:~~\absolute{-z}\leq\sfact~~\text{(as given)}.
\]
%
So, $0\in \tcztope{\ptemp}{\lb}{\ub}=S_1$.  Since by definition $\lb$ and $\ub$ are finite real
vectors and by what is given
%
\begin{equation}~\label{proof:acz-ptope-intersection1}
  \lb\leq\join{\lb}{\plb}\leq\meet{\ub}{\pub}\leq \ub,
\end{equation}
%
 we get that $\join{\lb}{\plb}$ is a finite vector.  Let
 $w=\pinv{\qtemp}\lt(\join{\lb}{\plb}\rt)$.  Then by Equation~\ref{proof:acz-ptope-intersection1},
 we get that $w\in\iztope{\pinv{\qtemp}}{\lb}{\ub}=S_1$.  Also,
%
 \begin{align*}
   &\%\%~~\text{Since}~~\join{\lb}{\plb}\leq\meet{\ub}{\pub}\\
& \plb\leq\join{\lb}{\plb}=\qtemp\lt(\pinv{\qtemp}\lt(\join{\lb}\plb\rt)\rt)=\qtemp
w
\leq\meet{\ub}{\pub}.
\end{align*}
%
So, $w\in\ptope{\qtemp}{\plb}{\pub}=S_2$.  Also, we previously showed
that $w\in\ptope{\qtemp}{\plb}{\pub}=S_2$.  Therefore, $w\in
\intersection{S_1}{S_2}$ and as previously shown, $0\in S_1$.
\begin{enumerate}
\item {\it Valid over-approximation}:
%
\begin{align*}
  & \intersection{\acztope{\ptemp}{\cen}{\sfact}{\pinv{\qtemp}}{\lb}{\ub}}{\ptope{\qtemp}{\plb}{\pub}}=
  \intersection{\lt(\minsum{S_1}{S_2}\rt)}{S_3}\\
  &\%\%~~ \text{by Lemma~\ref{lem:gen-int}}\\
  & \subseteq\minsum{S_1}{\lt(\intersection{S_2}{S_3}\rt)}=
  \minsum{S_1}{\lt(\intersection{\iztope{\pinv{\qtemp}}{\lb}{\ub}}{\ptope{\qtemp}{\plb}{\pub}}\rt)}\\
  & \%\%~~\text{by Lemma~\ref{lem:motivation}}\\
  &
  = \minsum{S_1}{\iztope{\pinv{\qtemp}}{\join{\lb}{\plb}}{\meet{\ub}{\pub}}}\\
  & =
  \minsum{\tcztope{\ptemp}{\cen}{\sfact}}{\iztope{\pinv{\qtemp}}{\join{\lb}{\plb}}{\meet{\ub}{\pub}}}\\
  & = \acztope{\ptemp}{\cen}{\sfact}{\pinv{\qtemp}}{\join{\lb}{\plb}}{\meet{\ub}{\pub}}.
\end{align*}
\item {\it Error in over-approximation}: For any $w,v\in\reals^n$, let
  us denote
  %
  \begin{align*}
    & h(w,v)\\ & =\support{w}{v}{\real\lt(S_1\rt)\oplus\lt(S_2\bigcap
  S_3\rt)}-\support{w}{v}{\lt(\real\lt(S_1\rt)\oplus S_2\rt)\bigcap
      S_3}.
  \end{align*}
  %
  Then,
  %
  \begin{align*}
&
    \approxerror{\Psi_1}{\Psi_2}=\approxerror{\minsum{\real\lt(S_1\rt)}{\lt(\intersection{S_2}{S_3}\rt)}}{\intersection{\lt(\minsum{\real\lt(S_1\rt)}{S_2}\rt)}{S_3}}\\
    & = \max_{w,v\in\reals^n:~\sqnorm{w-v}\leq 1}\lt(\support{w}{v}{\real\lt(S_1\rt)\oplus\lt(S_2\bigcap
      S_3\rt)}\rt.\\
    & \lt.\hspace{9em}-\support{w}{v}{\lt(\real\lt(S_1\rt)\oplus S_2\rt)\bigcap
      S_3}\rt)\\
    & = \max_{w,v\in\reals^n:~\sqnorm{w-v}\leq 1}h(w,v).
  \end{align*}
  %
  Let us consider a vector 
  $v\in\reals^n$, such that $\sqnorm{v-w}\leq 1$.  
  By the rank-nullity theorem,
  the vector $w-v$ can be written as
  %
  \[
w-z=(\qtemp^*)^Te+u;
  \]
  %
  such that $\qtemp^* u=0$.  Since $\orthnorm{\qtemp}$ is an
  orthornormal basis, so $\sqnorm{w-v}=\sqnorm{e}+\sqnorm{u}\leq 1$.
  We analyze the following two cases:

  {\it Case 1}:  Let us consider the case where $u=0$.
  \begin{align*}
& h(w,z) = \\
    & \%\%~~\text{by Lemma~\ref{lem:gen-int}},\\
    &\leq \absolute{\support{w}{v}{\real\lt(S_1\rt)}}\\
    & \%\%~~\text{by Lemma~\ref{lem:support-tcz} and the fact that}~u=0,\\
    & = \absolute{\lt(\absolute{e^T\qtemp^*\ptemp\sfact}+e^T\qtemp^*\cen\rt)}\\
    & \%\%~\text{since}~\sqnorm{e}\leq 1\\
    & \leq \sqnorm{\qtemp^*\ptemp\sfact}+\sqnorm{\qtemp^*\cen}.
  \end{align*}
  %
      {\it Case 2}: Let us consider the case where $u\neq 0$.  Let us
  consider a point $y\in\ptope{\qtemp}{\plb}{\pub}$.
  %
  \begin{align*}
    & \forall\lambda\in\reals,\nonumber\\
    & \qtemp(y+\lambda u)=\qtemp y +\lambda\qtemp u=\qtemp y.\nonumber\\
    & \%\%~\text{since}~y\in\ptope{\qtemp}{\plb}{\pub}\nonumber\\
    & \implies~\plb\leq\qtemp(y+\lambda u)=\qtemp y\leq \pub\nonumber\\
    & \implies y+\lambda u\in\ptope{\qtemp}{\plb}{\pub}.\\
  \end{align*}
  %
  So we get,
  \begin{align*}
    & \support{w}{v}{S_3}=\max_{x\in\ptope{\qtemp}{\plb}{\pub}}\lt(e^T\qtemp^*+u^T\rt)x\\
    & \%\%~\text{since}~\forall\lambda\in\reals:~y+\lambda u\in\ptope{\qtemp}{\plb}{\pub}\\
    & \geq \max_{\lambda\in\reals}\lt(e^T\qtemp^*+u^T\rt)\lt(y+\lambda u\rt)\\
    & \geq \lt(e^T\qtemp^*+u^T\rt)y+\max_{\lambda\in\reals}\lambda e^T\qtemp^*u+\lambda\sqnorm{u}\\
    & \%\%~\text{since}~\qtemp^* u=0\\
    & \geq \lt(e^T\qtemp^*+u^T\rt)y+\max_{\lambda\in\reals}\lambda\sqnorm{u}\\
    & = \infty.
  \end{align*}
  %
  Since $S_1$ and $S_2$ are bounded sets, we get
  %
  \begin{equation}~\label{eqn:approx-error1}
    \support{0}{v}{\real\lt(S_1\rt)}+\support{w}{v}{S_2}<\infty=\support{w}{v}{S_3}.
  \end{equation}
  %
  Then by Lemma~\ref{lem:gen-int} and
  Equation~\ref{eqn:approx-error1}, we get $h(w,z)=0$.

  By the results in both the above cases, we conclude that
  %
  \begin{align*}
& \forall w,v\in\reals^n:~\sqnorm{v-w}\leq 1:~~h(w,z)\leq
    \sqnorm{\qtemp^*\ptemp\sfact}+\sqnorm{\qtemp^*\cen}.\\
& \implies \approxerror{\Psi_1}{\Psi_2}=\max_{w,v\in\reals^n:~\sqnorm{v-w}\leq
      1:~~h(w,z)}h(w,z)\\
&  \leq \sqnorm{\qtemp^*\ptemp\sfact}+\sqnorm{\qtemp^*\cen}.~~~~~~~~~~~\qedhere
  \end{align*}
  %
\end{enumerate}
%
\end{proof}
%

