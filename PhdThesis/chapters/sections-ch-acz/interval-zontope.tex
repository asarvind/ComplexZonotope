We introduce a set representation called interval zonotope, which is
geometrically the same as a simple (real) zonotope, but the difference
is in their respective specifications.  In an interval zonotope, we
specify the intervals in which the combining coefficients are
constrained, without explicitly specifying the center.
%
\begin{definition}[Interval Zonotope]
Let us consider $\stemp\in\mat{n}{k}{\reals}$, called the {\it
  template}, and $\lb,\ub\in\reals^k$, called the upper and lower
interval bounds, respectively, such that $\lb\leq \ub$.  The following
is the representation of an interval zonotope.
%
\[
\iztope{\stemp}{\lb}{\ub}=\set{\stemp\zeta:~\zeta\in\reals^k,~\lb\leq\zeta\leq\ub}.
\]
%
\end{definition}
%
In general, interval zonotopes are not closed under intersection with
the sub-level sets of linear inequalities. But we shall describe an
observation latter that for a class of sub-level sets of linear
inequalities, called \emph{sub-parallelotopes}, which are oriented in
a certain way with the interval zonotope, the aforementioned
intersection is a closed operation on interval zonotopes and computed
by a simple algebraic expression. A sub-parallelotope can be seen as a
generalization of parallelotopes to possibly unbounded sets, defined
as follows.
%
\begin{definition}[Sub-parallelotope]
Let us consider ${\qtemp}\in\mat{k}{n}{\reals}$ such that
$\lt(\qtemp\transpose{\qtemp}\rt)$ is an invertible square matrix.  We
call such a matrix $\qtemp$ as a sub-paralleotopic template.  Let us
consider a pair of vectors with possibly unbounded components,
$\plb,\pub\in\set{\reals,-\infty,\infty}^k$ such that $\plb\leq\pub$,
called {\it lower and upper offsets}, respectively.  The following is
the representation of a sub-parallelotope.
%
\[
\ptope{\qtemp}{\plb}{\pub}=\set{x\in\reals^n:~\plb\leq\qtemp x\leq\pub}.
\]
%
\end{definition}
%
In other words, a sub-level set of a set of linear inequalities is a
sub-parallelotope if and only if the linear functions on which the
inequalities are defined are linearly independent. For example, the
sub-level set of the linear inequalities
%
\begin{align*}
-1\leq x+y-z\leq 1\\
x-y+z\leq 3
\end{align*}
%
is a sub-parallelotope
%
\[
\ptope{\lt[\begin{matrix}
    1 & 1 & -1\\
    1 & 1 & 1
  \end{matrix}
  \rt]}
{\lt[
    \begin{matrix}
      -1\\
      -\infty
    \end{matrix}
    \rt]
}
{\left[
    \begin{matrix}
      1\\
      3
    \end{matrix}
    \right]
},
\]
because the row vectors $\lt[\begin{matrix}1 & 1 &
    -1 \end{matrix}\rt]$ and $\lt[\begin{matrix}1 & 1 &
    1 \end{matrix}\rt]$, which correspond to the linear functions in the
above inequalities, are linearly independent.  On the other hand, the
sub-level set of
%
\begin{align*}
  -1\leq x+y-z\leq 1\\
  x+y+z\leq 2\\
  -1\leq x+y
\end{align*}
%
is not a sub-parallelotope, because there is linear dependence among
the row vectors $\lt[\begin{matrix} 1 & 1 & -1\end{matrix}\rt]$,
$\lt[\begin{matrix} 1 & 1 & 1\end{matrix}\rt]$, and
$\lt[\begin{matrix} 1 & 1 & 0 \end{matrix}\rt]$.

Sub-parallelotopes have a similarity to interval zonotopes, in the
sense that we can express a sub-parallelotope as a linear combination
of real vectors with relevant interval bounds on the combining
coefficients.  This is stated in the following proposition.  Recall
our notation that
$\pinv{\qtemp}=\transpose{\qtemp}\inv{\lt(\qtemp\transpose{\qtemp}\rt)}$,
which exists for a sub-parallelotopic template $\qtemp$ because
$\lt(\qtemp\transpose{\qtemp}\rt)$ is required to be invertible.
%
\begin{proposition}~\label{lem:ptope-iz-conversion}
  Consider a sub-parallelotope
  $\ptope{\qtemp}{\plb}{\pub}$ where $\qtemp\in\mat{k}{n}{\reals}$.
  Then,
  %
  \[
  \ptope{\qtemp}{\plb}{\pub}=\set{\cen+\pinv{\qtemp}\zeta:~c\in\reals^n,\zeta\in\reals^k,~\qtemp
  \cen=0,~\plb\leq
  \zeta\leq \pub
  }.
  \]
%
\end{proposition}
%
\begin{proof}
Let $\zeta\in\reals^k$ and $x^*=\pinv{\qtemp}\zeta$.  Then
%
\[
\qtemp x^*=\qtemp\pinv{\qtemp}\zeta = \qtemp\qtemp^T\inv{\lt(\qtemp\qtemp^T\rt)}\zeta=\zeta.
\]
%
By the rank-nullity theorem, $\qtemp x^* = \zeta$ implies
%
\begin{equation}
\set{x\in\reals^n:\qtemp x=\zeta}=\set{c+x^*:\qtemp c=0}.
\end{equation}
%
Combining the above two resuls we get,
%
\begin{equation}
\set{x\in\reals^n:\qtemp x=\zeta}=\set{c+\pinv{\qtemp}\zeta:~\qtemp c=0}.
\end{equation}
%
Based on the above equation, we get
%
\begin{align*}
& \ptope{\qtemp}{\plb}{\pub} = \set{x\in\reals^n:~\plb\leq \qtemp x\leq\pub}\nonumber\\
& = \set{x\in\reals^n:\qtemp
  x=\zeta, \plb\leq \zeta\leq\pub}\nonumber\\
& = \set{c+\pinv{\qtemp}\zeta:\qtemp c=0,~\plb\leq\zeta\leq\pub}.~~~\qedhere
\end{align*}
%
\end{proof}
%
The above similarity between a sub-parallelotope and an interval
zonotope is a motivation for considering the intersection of interval
zonotopes with sub-parallelotopes.  We observe that when a
sub-parallelotope has its template aligned with that of an interval
zonotope, their intersection can be exactly represented by another
interval zontope.  As an example, the intersection of
%
\[
\iztope{\lt[\begin{array}{l l}1 & 0 \\ 0 &
      1\end{array}\rt]}{\lt[\begin{array}{c}-1\\ -1\end{array}\rt]}{\lt[\begin{array}{c}2\\ 2\end{array}\rt]}
\]
with a sub-parallelotope defined by the sub-level sets of
%
\begin{align*}
  x_1\leq 1,\\
  x_2\geq 0.5
\end{align*}
%
gives
%
\[
\iztope{\lt[\begin{array}{l
        l}1 & 0 \\ 0 &
      1\end{array}\rt]}{\lt[\begin{array}{c}-1\\ 0.5\end{array}\rt]}{\lt[\begin{array}{c}1\\ 2\end{array}\rt]}.
\]
%
In the general case, we compute compute the intersection between a
possibly a sub-parallelotope and a possibly translated interval
zonotope by a simple algebraic expression as follows.
%
\begin{lemma}~\label{lem:motivation}
  Let us consider a sub-parallelotopic template
  $\qtemp\in\mat{k}{n}{R}$ and a real vector
  $c\in\reals^n$.  Then
%
\begin{align}~\label{eqn:motivation}
& \lt(\cen+\iztope{\pinv{\qtemp}}{\lb}{\ub}\rt)\bigcap\ptope{\qtemp}{\plb}{\pub}\nonumber\\
& =\cen+\iztope{\pinv{\qtemp}}{\join{\lb}{\lt(\plb-\qtemp\cen\rt)}}{\meet{\ub}{\lt(\pub-\qtemp\cen\rt)}}.
\end{align}
%
\end{lemma}
%
\begin{proof}
Let us denote
$S_1=\lt(\cen+\iztope{\pinv{\qtemp}}{\lb}{\ub}\rt)\bigcap\ptope{\qtemp}{\plb}{\pub}$
and\\
$S_2=\cen+\iztope{\pinv{\qtemp}}{\join{\lb}{\lt(\plb-\qtemp\cen\rt)}}{\meet{\ub}{\lt(\pub-\qtemp\cen\rt)}}$.
We shall first prove that $S_1\subseteq S_2$.  Let us consider that
$x\in S_1$.  So, $\exists
\zeta\in\reals^k:~x=\cen+\pinv{\qtemp}\zeta,~\lb\leq\zeta\leq\ub$
and $x\in\ptope{\qtemp}{\plb}{\pub}$. Then we get
%
\begin{align*}
& \plb\leq\qtemp x\leq \pub\\
& \equivalent \plb\leq \qtemp\lt(\cen+\pinv{\qtemp}\zeta\rt)\leq \pub\\
& \equivalent \plb-\qtemp\cen\leq\zeta\leq\pub-\qtemp\zeta.
\end{align*}
%
Also, we have $\lb\leq \zeta\leq \ub$ as given previously.
Therefore,
\[\join{\lb}{\lt(\plb-\qtemp\cen\rt)}\leq\zeta\leq\meet{\ub}{\lt(\pub-\qtemp\cen\rt)},\]
which implies that $x=\cen+\pinv{\qtemp}\zeta\in S_2$.  This shows that $S_1\subseteq S_2$.

Now, we shall prove $S_2\subseteq S_1$.  Since
\begin{align}
&\lb\leq\join{\lb}{\lt(\plb-\qtemp\cen\rt)}~~\text{and}~~\meet{\ub}{\lt(\pub-\qtemp\cen\rt)}\leq
\ub,~~\text{we get}\nonumber\\
&
S_2=\cen+\iztope{\pinv{\qtemp}}{\join{\lb}{\lt(\plb-\qtemp\cen\rt)}}{\meet{\ub}{\lt(\pub-\qtemp\cen\rt)}}
\subseteq \cen+\iztope{\pinv{\qtemp}}{\lb}{\ub}~\label{eqn:proof-motivation1}
\end{align}
%
Next, let us condier that $y\in S_2$.  So, $\exists
\zeta^\pr\in\reals^k:$ \[y=\cen+\pinv{\qtemp}\zeta^\pr,~\join{\lb}{\lt(\plb-\cen\rt)}\leq\zeta^\pr
\leq\meet{\ub}{\lt(\pub-\cen\rt)}.\]
Then
%
\begin{align*}
& \qtemp y=\qtemp\lt(\cen+\pinv{\qtemp}\zeta^\pr\rt)=\qtemp\cen+\zeta^\pr\\
& \%\%~~\text{since}~\join{\lb}{\lt(\plb-\cen\rt)}\leq\zeta^\pr\leq
\meet{\ub}{\lt(\pub-\cen\rt)}\\
& \implies \qtemp\cen+\lt(\join{\lb}{\lt(\plb-\cen\rt)}\rt)\leq \qtemp
y\leq \qtemp\cen+\lt(\meet{\ub}{\lt(\pub-\qtemp\cen\rt)}\rt)\\
& \implies \qtemp\cen+\plb-\qtemp\cen\leq\qtemp y\leq
\qtemp\cen+\pub-\qtemp\cen\\
& \implies \plb\leq\qtemp y\leq\pub.
\end{align*}
%
Therefore, $y\in\ptope{\qtemp}{\plb}{\pub}$.  This shows that
$S_2\subseteq \ptope{\qtemp}{\plb}{\pub}$.  This combined with the
containment in Equation~\ref{eqn:proof-motivation1}, we get that
$S_2\subseteq S_1$.  Earlier, we have shown that $S_1\subseteq S_2$.
Therefore, $S_1=S_2$.
\end{proof}
%
An interval zonotope can be equivalently specified as the real
projection of the following template complex zonotope.  This result is
latter used in Lemma~\ref{lem:acz-tcz-conversion} to convert between
reals projections of an augmented complex zonotope and a template
complex zonotope, which is in turn used in
Theorem~\ref{thm:acz-inclusion} to derive a convex program for checking
inclusion between augmented complex zonotopes.
%
\begin{lemma}~\label{lem:iz-tcz-conversion}
We have the following equivalence between an interval zonotope and the
real projection of a template complex zonotope.
%
\[
\iztope{\stemp}{\lb}{\ub}=\real\lt(\tcztope{\stemp}{\stemp\frac{\ub+\lb}{2}}{\frac{\ub-\lb}{2}}\rt).
\]
%
\end{lemma}
%
\begin{proof}
Let us consider a point $x\in\iztope{\stemp}{\lb}{\ub}$ expressed as
%
\[
x=\stemp\zeta:\zeta\in\reals^n,\lb\leq\zeta\leq\ub.
\]
%
  Let $\zeta^\pr=
\zeta-\frac{\ub+\lb}{2}$.
Then we get,
%
\begin{align*}
& x=\stemp\zeta=\stemp\frac{\ub+\lb}{2}+\stemp\lt(\zeta-\frac{\ub+\lb}{2}\rt)\\
& = \stemp\frac{\ub+\lb}{2}+\stemp\zeta^\pr~~\text{and}\\
& \absolute{\zeta^\pr}=\absolute{\zeta-\frac{\ub+\lb}{2}}\\
& \%\%~\text{As $\zeta$ is a real vector and $\lb\leq\zeta\leq\ub$}\\
& \leq \join{\absolute{\lb-\frac{\ub+\lb}{2}}}{\absolute{\ub-\frac{\ub+\lb}{2}}}
= \absolute{\frac{\ub-\lb}{2}}.
\end{align*}
%
So,
$x\in\real\lt(\tcztope{\stemp}{\stemp\frac{\ub+\lb}{2}}{\frac{\ub-\lb}{2}}\rt)$.  Therefore,
%
\[
\iztope{\stemp}{\lb}{\ub}\subseteq\real\lt(\tcztope{\stemp}{\stemp\frac{\ub+\lb}{2}}{\frac{\ub-\lb}{2}}\rt).
\]
%
Next consider $y\in\real\lt(\tcztope{\stemp}{\stemp\frac{\ub+\lb}{2}}{\frac{\ub-\lb}{2}}\rt)$,
expressed as
\[
y=\stemp\zeta+\stemp\frac{\ub+\lb}{2}:~\absolute{\zeta}\leq\frac{\ub-\lb}{2}.
\]
%
Let $\zeta^\dpr=\zeta+\frac{\ub+\lb}{2}$.
As
$\absolute{\zeta}\leq\frac{\ub-\lb}{2}$, so we get
%
\begin{align*}
&
\frac{\ub+\lb}{2}-\frac{\ub-\lb}{2}\leq\zeta^\dpr\leq\frac{\ub-\lb}{2}+\frac{\ub+\lb}{2}\\
& \equivalent \lb\leq\zeta^\dpr\leq\ub
\end{align*}
%
Furthermore, $y=\stemp\zeta+\stemp\frac{\ub+\lb}{2}=\stemp\zeta^\dpr$.
So, $y\in\iztope{\stemp}{\lb}{\ub}$.  Therefore,
%
\[
\iztope{\stemp}{\lb}{\ub}\supseteq\real\lt(\tcztope{\stemp}{\stemp\frac{\ub+\lb}{2}}{\frac{\ub-\lb}{2}}\rt).
\]
%
Combining the previous two conclusions about the set inclusions, we get
%
\[
\iztope{\stemp}{\lb}{\ub}=\real\lt(\tcztope{\stemp}{\stemp\frac{\ub+\lb}{2}}{\frac{\ub-\lb}{2}}\rt).~~~~~~\qedhere
\]
%
\end{proof}
%
An interval zonotope can be equivalently represented as a simple
zonotope, which is stated in the following lemma.
%
\begin{lemma}~\label{lem:iz-rz-conversion}
We have
%
\[
\iztope{\stemp}{\lb}{\ub}=\rztope{\stemp\diagonal{\frac{\ub-\lb}{2}}}{\stemp\frac{\ub+\lb}{2}}.
\]
%
\end{lemma}
%
\begin{proof}
Based on Lemma~\ref{lem:iz-tcz-conversion}, we have 
%
\[
\iztope{\stemp}{\lb}{\ub}=\real\lt(\tcztope{\stemp}{\stemp\frac{\ub+\lb}{2}}{\frac{\ub-\lb}{2}}\rt).
\]
%
Next based on Lemma~\ref{lem:normalization}, we have 
%
\[
\tcztope{\stemp}{\stemp\frac{\ub+\lb}{2}}{\frac{\ub-\lb}{2}}=\cztope{\ptemp\diagonal{\frac{\ub-\lb}{2}}}{\stemp\frac{\ub-\lb}{2}}.
\]
%
As $\ptemp$ is real, so
%
\[
\real\lt(\cztope{\ptemp\diagonal{\frac{\ub-\lb}{2}}}{\stemp\frac{\ub-\lb}{2}}\rt)=\rztope{\ptemp\diagonal{\frac{\ub-\lb}{2}}}{\stemp\frac{\ub-\lb}{2}}.
\]
%
Combining the above results, we get
\[
\iztope{\stemp}{\lb}{\ub}=\rztope{\stemp\diagonal{\frac{\ub-\lb}{2}}}{\stemp\frac{\ub+\lb}{2}}.~~~~\qedhere
\]
%
\end{proof}
%
Interval zonotopes are closed under linear transformation and
Minkowski sum operations.  The parameters of the resultant interval
zonotopes are an affine transformation of the original parameters.
This is described in the following lemmas.
%
\begin{lemma}[Linear transformation]~\label{lem:iz-lin-transform}
Let us consider an interval zonotope $\iztope{\stemp}{\lb}{\ub}$ and a
matrix $A\in\mat{n}{n}{\reals^n}$.  Then,
%
\[
A\iztope{\stemp}{\lb}{\ub}=\iztope{A\stemp}{\lb}{\ub}.
\]
%
\end{lemma}
%
\begin{proof}
Based on Lemma~\ref{lem:iz-rz-conversion}, we get
%
\begin{align*}
&  A\iztope{\stemp}{\lb}{\ub}\\
  & =
  A\rztope{\stemp\diag{\frac{\ub-\lb}{2}}}{\frac{\ub+\lb}{2}}\\
  & ~~\%\%~~\text{by Lemma~\ref{todo}}\\
  & = \rztope{A\stemp\diag{\frac{\ub-\lb}{2}}}{\frac{\ub+\lb}{2}}\\
  & ~~\%\%~~\text{by Lemma~\ref{lem:iz-rz-conversion}}\\
  & = \iztope{A\stemp}{\lb}{\ub}.~~~~~\qedhere
\end{align*}
%
\end{proof}
%
\begin{lemma}[Minkowski sum]~\label{lem:iz-min-sum}
  Let us consider two interval zonotopes\\
  $\iztope{\stemp}{\lb}{\ub},~\iztope{\stemp^\pr}{\lb^\pr}{\ub^\pr}\subset\compnums^n$.
  Then 
%
\[
\minsum{\iztope{\stemp}{\lb}{\ub}}{\iztope{\stemp^\pr}{\lb^\pr}{\ub^\pr}}
= \iztope{\mymatrix{\stemp &
    \stemp^\pr}}{\mymatrix{\lb\\\lb^\pr}}{\mymatrix{\ub\\\ub^\pr}}. 
\]
%
\end{lemma}
%
\begin{proof}
We get
%
\begin{align*}
  & \%\%~~\text{by Lemma~\ref{lem:iz-rz-conversion}}\\
  &
  \minsum{\iztope{\stemp}{\lb}{\ub}}{\iztope{\stemp^\pr}{\lb^\pr}{\ub^\pr}}\\
  & =
  \minsum{\rztope{\stemp\diag{\frac{\ub-\lb}{2}}}{\frac{\ub+\lb}{2}}}{\rztope{\stemp\diag{\frac{\ub^\pr-\lb^\pr}{2}}}{\frac{\ub^\pr+\lb^\pr}{2}}}\\
  & \%\%~~\text{by Lemma~\ref{todo}}\\
  &= \rztope{\mymatrix{\stemp\diag{\frac{\ub-\lb}{2}} &
      \stemp^\pr\diag{\frac{\ub^\pr-\lb^\pr}{2}}}}{\frac{\ub+\lb}{2}+\frac{\ub^\pr+\lb^\pr}{2}}\\
  & = \rztope{\mymatrix{
      \stemp & \stemp^\pr
      }
    \diag{
      \frac{
        \mymatrix{
\ub\\\ub^\pr
          }-\mymatrix{
\lb\\\lb^\pr
          }
      }{2}
    }
    }{\frac{
        \mymatrix{
\ub\\\ub^\pr
          }+\mymatrix{
\lb\\\lb^\pr
          }
    }{2}}\\
  & \%\%~~\text{by Lemma~\ref{lem:iz-rz-conversion}}\\
  & = \iztope{\mymatrix{\stemp &
    \stemp^\pr}}{\mymatrix{\lb\\\lb^\pr}}{\mymatrix{\ub\\\ub^\pr}}.~~~~~\qedhere
\end{align*}
\end{proof}
%
The support of an interval zonotope along a vector is an affine
function of the lower and upper interval bounds, which is stated in
the following proposition.
%
\begin{lemma}[Support of a vector]
Let us consider $\iztope{\stemp}{\lb}{\ub}\subset\reals^n$ and
two real vectors $w,v\in\reals^n$.  Then
%
\[
\support{w}{v}{\iztope{\stemp}{\lb}{\ub}}=v^T\stemp\lt(\frac{\ub+\lb}{2}\rt)+\absolute{v^T\stemp}\absolute{\frac{\ub-\lb}{2}}.
\]
%
\end{lemma}
%
\begin{proof}
We get
%
\begin{align*}
  & \support{w}{v}{\iztope{\stemp}{\lb}{\ub}} = \\
  & \%\%~~\text{by Lemma~\ref{lem:iz-tcz-conversion}}\\
  & \support{w}{v}{\rztope{\diagonal{\frac{\ub-\lb}{2}}}{\frac{\ub+\lb}{2}}}=\\
  & \%\%~~\text{by Lemma~\ref{todo}}\\
  & v^T\stemp\lt(\frac{\ub+\lb}{2}\rt)+\absolute{v^T\stemp}\absolute{\frac{\ub-\lb}{2}}.
\end{align*}
%
\end{proof}
%

























































