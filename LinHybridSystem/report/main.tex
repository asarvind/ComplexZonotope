\documentclass{llncs}
\usepackage{times}
\usepackage{amsmath}
\usepackage{amssymb} 
\usepackage{algorithm}
\usepackage{algpseudocode}
\usepackage{multirow}
\usepackage{url}
\usepackage{tikz} 
\usetikzlibrary{arrows,backgrounds,decorations,decorations.pathmorphing,positioning,fit,automata,shapes,snakes,patterns,plotmarks,calc}
\usepackage{graphicx,color} 
\usepackage{enumitem}
\usepackage{float}
\usepackage{bm}
\usepackage{array}
\usepackage{caption}
\usepackage{cite}
\usepackage{relsize}
%\usepackage{natbib}

\setlength{\itemsep}{0ex}
\setlength{\parsep}{0ex}
\setlength{\parskip}{0ex}
\setlength{\partopsep}{0ex}
\setlength{\parindent}{0pt} 
%\setlength{\bibsep}{0.0pt}


% Input macros
% Symbols
\newcommand{\ra}{\rightarrow}
\newcommand{\ov}{\overline}
\newcommand{\ur}{\underline}
\newcommand{\pr}{\prime}
\newcommand{\tbf}{\textbf}
\newcommand{\omg}{\Omega}
\newcommand{\mc}{\mathcal}
\newcommand{\lt}{\left}
\newcommand{\rt}{\right}
\newcommand{\mb}{\mathbb}
\newcommand{\imp}{\implies}
\newcommand{\dimp}{\Leftrightarrow}
\newcommand{\wh}{\widehat}
\newcommand{\dg}{\mc{D}}


% Sets
\newcommand{\realset}{\mb{R}}
\newcommand{\comprealset}{\mb{\ov{R}}}
\newcommand{\pint}{\mb{Z}_{> 0}}
\newcommand{\nzrl}{\mb{R}_{\geq 0}}
\newcommand{\prl}{\mb{R}_{>0}}
\newcommand{\rl}{\mb{R}}
\newcommand{\fin}{\forall i\in\{1,...,n\}}
\newcommand{\tup}[1]{\{1,...,#1\}}
\newcommand{\seq}[2]{_{#1=1}^#2}
\newcommand{\mCnn}{\mb{M}_{n\times n}(\mb{C})}
\newcommand{\mCmm}{\mb{M}_{m\times m}(\mb{C})}
\newcommand{\mCpn}{\mb{M}_{p\times n}(\mb{C})}
\newcommand{\mCnm}{\mb{M}_{n\times m}(\mb{C})}
\newcommand{\mRnn}{\mb{M}_{n\times n}(\mb{R})}
\newcommand{\mRpn}{\mb{M}_{p\times n}(\mb{R})}
\newcommand{\mRnm}{\mb{M}_{n\times m}(\mb{R})}
\newcommand{\mRno}{{{\mb{R}}}^n_{\geq 0}}
\newcommand{\mRmo}{{{\mb{R}}}^m_{\geq 0}}
\newcommand{\mRo}{\mb{R}_{\geq 0}}
\newcommand{\mRn}{{\mb{R}}^n}
\newcommand{\mRm}{{\mb{R}}^m}
\newcommand{\mCn}{\mb{C}^n}
\newcommand{\mCm}{\mb{C}^m}
\newcommand{\inv}{\mb{GL}_n(\mb{R})}
\newcommand{\id}[1]{\mb{I}_{#1\times #1}}
\newcommand{\mat}[3]{\mb{M}_{#1\times #2}(\mb{#3})}


% matrix operations
\newcommand{\ColumnJoin}[2]{\left[\begin{array}{l}{#1}\\{#2} \end{array}\right]}
\newcommand{\minaffine}[2]{\Lambda^{\min}\lt(#1,#2\rt)}
\newcommand{\maxaffine}[2]{\Lambda^{\max}\lt(#1,#2\rt)}

% local macros
\DeclareMathOperator{\real}{\operatorname{Re}}
\newcommand{\CZ}{\lt(V,c,s\rt)}
\newcommand{\GCZ}{\gcz{V}{c}{s}{W}{l}{u}}
\newcommand{\cz}[3]{\mc{C}\lt(#1,#2,#3\rt)}
\newcommand{\CZO}{\lt(V,0,s\rt)}
\newcommand{\czo}[2]{\mc{Z}\lt(#1,0,#2\rt)}
\newcommand{\trj}[2]{{\bf #1}(#2)}
\newcommand{\IncTcz}[6]{\mc{T}\lt(#1,#2,#3,#4,#5,#6\rt)}
\newcommand{\IncGcz}[6]{\mc{G}\lt(#1,#2,#3,#4,#5,#6\rt)}
\newcommand{\Ptope}[3]{\mc{P}\left(#1,#2,#3\right)}
\newcommand{\gcz}[6]{\mathcal{Z}\lt(#1,#2,#3,#4,#5,#6\rt)}
\newcommand{\sptope}[3]{\mathcal{P}\lt(#1,#2,#3\rt)}
\newcommand{\system}{\mb{H}}
\newcommand{\locationset}{Q}
\newcommand{\edgeset}{E}
\newcommand{\stay}{\gamma}
\newcommand{\linearmapset}{\mc{A}}
\newcommand{\inputset}{U}
\newcommand{\initialset}{\Omega}
\newcommand{\edge}{\sigma}
\newcommand{\loc}{q}
\newcommand{\map}{\linearmapset}
\newcommand{\inp}{\inputset}
\newcommand{\ptemplate}{\mc{K}}
\newcommand{\systrj}[2]{\lt({\bf #1},{\bf #2}\rt)}
\newcommand{\wholenums}{\mb{Z}_\geq 0}
\newcommand{\preloc}[1]{#1_{1}}
\newcommand{\postloc}[1]{#1_{2}}
\newcommand{\upperedgebound}[1]{#1^+}
\newcommand{\loweredgebound}[1]{#1^-}
\newcommand{\reset}[1]{#1_r}
\newcommand{\locationtransition}[1]{R_{#1}}
\newcommand{\edgetransition}[1]{R_{#1}}
\newcommand{\staysptope}[1]{\sptope{\ptemplate\lt(#1\rt)}{\stay^-\lt(#1\rt)}{\stay^+\lt(#1\rt)}}
\newcommand{\guardsptope}[1]{\sptope{\ptemplate\lt(\preloc{#1}\rt)}{\max\lt(\loweredgebound{#1},\stay^-\lt(\preloc{#1}\rt)\rt)}{\min\lt(\upperedgebound{#1},\stay^+\lt(\preloc{#1}\rt) \rt)}}
\newcommand{\hybridset}{\Gamma}
\newcommand{\transfer}[4]{#1#4 = #2\dg\lt(#3\rt)}
\newcommand{\centertransfer}[4]{#1#4 = #3-#2}
\newcommand{\scalebound}[5]{\max_{i=1}^{#4}\lt(\lt(\sum_{j=1}^{#5}\lt|#1\rt|_j\rt)-#3_i+#2_i\rt)}
\newcommand{\pseudoinverse}[1]{#1\lt(#1#1^T\rt)^{-1}}


% Headers
\title{Augmented complex zonotopes for computing invariants of affine hybrid systems\thanks{This research work is partially supported by ANR project MALTHY.}
} 

\author{Arvind\ Adimoolam and Thao\ Dang
} 
\institute{\ Verimag,~Grenoble, France\\ \url{{santosh.adimoolam,thao.dang}@univ-grenoble-alpes.fr}.
}


\begin{document}

\maketitle

\begin{abstract}
Zonotopes are a useful set representation for bounded time reach set
computation of affine hybrid systems because of their closure under
Minkowski sum and matrix multiplication operations.  For unbounded
time reach set approximation of arbitrarily switched affine hybrid
systems, template complex zonotopes and a corresponding invariant
computation procedure were introduced, which utilized the possibly
complex eigenstructure of the affine maps.  But a major hurdle in
extending the technique for computing invariants of more general
affine hybrid systems, where switching is state dependent and
controlled by linear constraints, is that the class of template
complex zonotopes is not closed under intersection with linear
constraints.  In this paper, we use a more expressive set
representation called augmented complex zonotopes, for which we
propose an algebraic over-approximation of the intersection with
linear constraints.  This over-approximation is then used to derive a
set of second order conic constraints for computing an augmented
complex zonotopic positive invariant for discrete time affine hybrid
systems with additive disturbance input and linear safety constraints.
We demonstrate the efficiency of this approach by experimenting on
some benchmark examples.
\end{abstract}

\section{Introduction}
In the design of embedded and cyber-physical systems, one of the most
important requirements is safety, which can be roughly stated as that
the system will never enter a bad state. Safety verification for such
systems are known to be computationally challenging due to the
complexity resulting from the interactions among heterogenous
components, having mixed (continuous and discrete) dynamics. In this
paper, we focus on the problem of finding invariants for hybrid
systems, which are widely recognized as appropriate for modelling
embedded and cyber-physical systems. An invariant is a property that
is satisfied in every state that the system can reach. Therefore a
common approach for proving a safety property is to find an invariant
that implies the safety property. Invariant computation has been
studied extensively in the context of verification of transition
systems and program analysis (see for
example~\cite{CousotHalbwachs78,DBLP:journals/fmsd/BensalemL99,DBLP:conf/tacas/TiwariRSS01,DBLP:conf/cav/ColonSS03,DBLP:conf/sas/Goubault13}
and the developed techniques have been extended to continuous and
hybrid
systems \cite{DBLP:conf/hybrid/SankaranarayananSM04,%% jeannet2009apron,
DBLP:conf/hybrid/Rodriguez-CarbonellT05,DBLP:conf/cdc/SassiGS14,%% DBLP:journals/tecs/AllamigeonGSGP16,
HybridFluctuat,DBLP:conf/vmcai/SogokonGJP16,DBLP:conf/aplas/DangG11}. Barrier
certificates \cite{prajna2004safety} are closely related to invariants
in the sense that they describe a boundary that the system starting
from a given initial set will never cross to enter a region containing
bad states. Another common approach to safety verification is to
compute or over-approximate the reachable set of the system, and these
reachability computation techniques have been developed for continuous
and hybrid systems.  Many such techniques are based on iterative
approximation of the reachable state on a step-by-step basis, which
can be thought of as a set-based extension of numerical integration. A
major drawback of this approach, inherent to undecidability of general
hybrid systems with non-trivial dynamics, is that such an iterative
procedure may not terminate and thus can only be used for bounded-time
safety verification (except when the over-approximation error
accumulation is not too bad that the safety can be decided). In
contrast, invariants and barrier certificates are based
on conditions that are satisfied at all times. Although solving these
conditions often involves fixed point computation, by exploiting the
structure of the dynamics (such as eigenstructures of linear systems),
one can derive meaningful conditions which can significantly reduce
the number of iterations until convergence.



For discrete time affine hybrid systems, the eigenvectors of the
products of linear matrices related to the affine dynamics of
different subsystems can possibly capture some of the stable
directions for the overall hybrid dynamics.  As such, for invariant
computation, template complex zonotopes have the advantage that they
can include the possibly complex eigenvectors among the generators,
while usual (real) zonotopes can not.  In an earlier
work~\cite{tcz2017}, numerically efficiently solvable conditions for
computing a template complex zonotopic invariant subject to linear
safety constraints were obtained for a limited class of hybrid
systems, i.e., having uncontrolled switching.  However, a formidable
hurdle in extending the approach for more general affine hybrid
systems, where switching is controlled by linear constraints, is that
we have to handle the intersection of template complex zonotopes with
the linear constraints.  In this regard, template complex zonotopes
share the drawback of usual zonotopes that these classes of sets are
not closed under intersection with linear constraints.

In this paper, we circumvent this problem as follows.  We observe that
it is possible to compute or reasonably overapproximate the
intersection of a template complex zonotope with a class of linear
constraints, called subparallelotpic, by appropriately choosing the
template of the complex zonotope.  We use a slightly more general set
representation, called augmented complex zonotope, with which the
intersection operation can be succinctly presented.  %% Geometrically
%% speaking, augmented complex zonotopes and template complex zonotopes
%% describe the same classes of sets in terms of their real valued
%% projections.  However, it is easier and more succinct, using the
%% representation of an augmented complex zonotope instead of a template
%% complex zonotope, to represent the resultant intersection with linear
%% constraints. 
Then, we derive a numerically efficiently solvable
sufficient condition for computing an augmented complex zonotopic
invaraint satisfying linear safety constraints, for a discrete time
affine hybrid system with subparallelotopic switching constraints and
bounded additive disturbance input.  The sufficient condition is
expressed as a set of second order conic constraints.  We also note
that the class of sub-parallelotopic constraints that we consider are
quite general and can be used in the specification of many examples of 
affine hybrid systems.  To corroborate our approach by presenting the
experimental results for three benchmark examples from literature.
%% We implemented our approach on three benchmark examples from
%% literature and compared the results with that of the SpaceEx tool
%% on the same discrete time models, and also the reported benchmark
%% results from literature.  In some experiments, we could verify
%% finite safety bounds when SpaceEx could not even find an
%% invariant. In other experiments, we could verify competitive safety
%% bounds.  Also, our computation time is quite reasonable in all the
%% experiments, depending on the size of the specification.



\emph{Related work.} For hybrid systems verification, convex polyhedra~\cite{CousotHalbwachs78,jeannet2009apron}, and their special classes such as
%% zones~\cite{DBLP:conf/pado/Mine01},
octagons~\cite{DBLP:journals/lisp/Mine06} and
zonotopes~\cite{DBLP:conf/hybrid/Girard05,DBLP:conf/eucc/MaigaCRT14}
and tropical polyhedra~\cite{DBLP:conf/sas/AllamigeonGG08} are the
most commonly used set representations. During reachability analysis,
which requires operations under which a set representation is not
closed (such as the union or join operations for convex polyhedra and
additionally intersection for zonotopes), the complexity of generated
sets increases rapidly in order to guarantee a desired error
bound. One way to control this complexity increase is to fix the face
normal vectors or generators, which leads to template convex
polyhedra \cite{Sankaranarayanan+Dang+Ivancic-08-Symbolic,DBLP:conf/aplas/DangG11}. Although
our template complex zonotopes proposed in~\cite{tcz2017} do not
belong to the class of convex polyhedra, they follow the same spirit
of controlling the complexity using templates. Set representations
defined by non-linear constraints include
ellipsoids~\cite{Kurzhanski2000201}, polynomial
inequalities\cite{DBLP:conf/sas/BagnaraRZ05} and
equalities~\cite{Rodriguez-Carbonell:2007}, quadratic templates and
piecewise quadratic templates~\cite{%% DBLP:conf/esop/AdjeGG10,
DBLP:conf/hybrid/RouxJGF12,DBLP:conf/fm/RouxG14,DBLP:conf/hybrid/Adje17},
which are used for computing non-linear invariants. A major problem of
template based approaches finding good templates.  In this regard,
using template complex zonotopes and the augmented version introduced
in this paper, we can exploit eigen-structures of linear dynamics
which reflect the contraction or expansion of a set by the dynamics,
and define good templates for efficient convergence to an invariant
(see Proposition 4.3 of~\cite{adimoolamACC2016}).

The extension to complex zonotope~\cite{adimoolamACC2016} is
very similar in spirit to quadratic
zonotopes~\cite{DBLP:conf/aplas/AdjeGW15} and more generally
polynomial zonotopes~\cite{DBLP:conf/hybrid/Althoff13}. Nevertheless,
while a polynomial zonotope is a set-valued polynomial function
of \emph{intervals}, a complex zonotope is a set-valued function of
unit \emph{circles} in the complex plane.  Our idea in this paper of coupling
additional linear constraints with complex zonotopes is inspired by the work
on constrained zonotopes proposed
in~\cite{DBLP:conf/cav/GhorbalGP09,scott2016constrained} for computing
intersection with linear constraints.  But
while~\cite{DBLP:conf/cav/GhorbalGP09,scott2016constrained} compute
the intersection or its overapproximation, algorithmically, we instead
derive a simple algebraic expression to overapproximate the
intersection.  This algebraic expression is latter used to obtain
second order conic (convex) constraints, for invariant computation in
a single step of convex optimization.

\emph{Organization.}  The rest of the paper is organized as follows.  Firstly, we explain
some of the mathematical notation used in this paper.  Then in
Section~\ref{sec:system}, we describe the model of a discrete-time
affine hybrid system, controlled by sub-parallelotopic switching
conditions and having a bounded additive disturbance input. In
Section~\ref{sec:acz}, we present the set representation of augmented
complex zonotopes and discuss some important operations and relations,
in particular intersection with sub-parallelotopic constraints,
projection in any direction, linear transformation, Minkowski sum and
inclusion checking.  In Section~\ref{sec:invcomp}, we derive a set of
second order conic constraints to compute an augmented complex
zonotopic invariant, satisfying linear safety constraints and
containing an initial set.  Furthermore, we explain how to choose the
template.  In Section~\ref{sec:exp}, we report some experimental
results.  The conclusion and future work are given in
Section~\ref{sec:conclusion}.
%% We annex proofs of some lemmas presented in the paper in an
%% Appendix.

\emph{Notation.} Some notations for which we
consider explanation may be required is described below.  We denote
$\comprealset = \realset\bigcup\lt\{-\infty,\infty\rt\}$.  If $S$ is a
set of complex numbers, then $\real(S)$ and $\img(S)$ represent the
real and imaginary projections of $S$, respectively.  If $z$ is a
complex number, then $|z|$ denotes the absolute value of $z$.  On the
other hand, if $X$ is a complex matrix (or vector), then $\lt|X\rt|$
denotes the matrix (or vector) containing the absolute values of the
elements of $X$.  The diagonal square matrix containing the entries of
a complex vector $z$ along the diagonal is denoted by $\dg(z)$.  The
conjugate transpose of a matrix $V\in\mat{m}{n}{\complexset}$ is
denoted $\conjtranspose{V} = \lt(\real(V)-\iu\img(V)\rt)^T.$ If
$V\conjtranspose{V}$ is invertible, then
$\pinv{V}= \conjtranspose{V}\lt(V\conjtranspose{V}\rt)^{-1}$, which is
the pseudo-inverse of $V$.  Given two vectors $l,u\in\realset^k$ and
any relation $\bowtie$ between numbers in $\comprealset$, we say
$l\bowtie u$ if $l_i\bowtie u_i,~\forall i\in\tup{k}$.  The meet of
the two vectors $l$ and $u$ is denoted $l\bigwedge u$, defined as
$\lt(l\bigwedge u\rt)_i=\min\lt(l_i,u_i\rt)~\forall i\in\tup{k}$.  The
join is denoted $l\bigvee u$, defined as $\lt(l\bigvee
u\rt)_i=\max\lt(l_i,u_i\rt)~\forall i\in\tup{k}$.


\section{Hybrid systems and positive invariants}~\label{sec:system}

In a discrete-time affine hybrid system, we have a finite set of
discrete variables, called locations, and a finite set of continuous
variables whose valuation is in the real Euclidean space of dimension
$n\in\pint$.  In each location, there are a set of linear constraints,
called \emph{staying conditions}, within which the continuous state of
the system in that location is constrained.  Furthermore, there is an
affine transition map with (possibly) additive uncertain but bounded
disturbance input specifying the evolution of the continuous
variables. A set of labeled directed edges specify possible discrete
transitions between locations, accompanied by affine reset map on
continuous variables with a bounded additive disturbance input. Each
edge transition is controlled by a set of linear constraints on the
continuous variables, called guards.

In this paper, we consider a specific class of linear constraints
called, sub-parallelotopic, for defining guards and staying
conditions, such that their intersection with the reachable set
represented by augmented complex zonotopes (introduced later) can be
effectively computed. The sets corresponding to sub-parallelotopic
constraints can be seen as a generalization of parallelotopes to
possibly unbounded sets.  We discuss the aforementioned intersection
operation later after defining augmented complex zonotopes.
%
\begin{definition}[Sub-parallelotope]~\label{defn:sub-parallelotope} Let
  $K\in\mat{k}{n}{R}$ such that $k\leq n$ and $\lt(KK^T\rt)$ is
  non-singular.  We call such a matrix $K$ a
  \emph{sub-parallelotopic template}.  Let
  $\wh{u},\wh{l}\in\comprealset^n$ such that $\wh{u}\leq\wh{l}$.  Then
  a sub-parallelotopic set is $\sptope{K}{\wh{l}}{\wh{u}} = \lt\{x\in\realset^n: \wh{l}\leq Kx \leq \wh{u}\rt\}$.
\end{definition}
%
For example, the set of linear constraints $-1\leq x+y-z\leq
1~\wedge~~ x-y+z\leq 3$ is equivalent to a sub-parallelotope
$$\sptope{\ColumnJoin{\lt[1~~~1~-1\rt]}{\lt[1~-1~1\rt]}}{\ColumnJoin{-1}{-\infty}}{\ColumnJoin{1}{3}},$$
because the rows of the sub-parallelotopic template are linearly
independent.  On the other hand, the set of constraints $-1\leq
x+y-z\leq 1~\wedge~~x+y+z\leq 2\wedge~~-1\leq x+y$ do not constitute a
sub-parallelotope, because the three row vectors $\lt[\begin{array}{c
c c}1 & 1 & -1\end{array}\rt]$, $\lt[\begin{array}{c c c}1 & 1 &
1\end{array}\rt]$, and $\lt[\begin{array}{c c c}1 & 1 &
0\end{array}\rt]$ together are linearly dependent.  Sub-parallelotopic
constraints are algebraically related to a generator representation.
We can express $\sptope{K_{n\times
k}}{l}{u}=\lt\{\pinv{K}\zeta+w:~Kw=0\wedge\zeta\in\realset^k\rt\}$.
Because of the above, we can express the intersection of
sub-parallelotope with a suitably aligned zonotope as a simple
algebraic expression, as we will see latter.

%Nevertheless, for many examples of affine hybrid systems, the guards and staying conditions can be specified by sub-parallelotopes.


\tbf{System model.}
We consider discrete-time affine hybrid systems defined by a tuple $\system =
\lt(\locationset,\ptemplate,\stay,\linearmapset,\inputset,\edgeset\rt)$. Here, $\locationset$ is a finite set of locations.  For each location
$\loc\in\locationset$, a sub-parallelotopic template
$\ptemplate_\loc\in\mat{k_q}{n}{\realset}$, i.e.,
$\ptemplate_\loc\lt(\ptemplate_\loc\rt)^T$ is non-singular, and $k(q)$
is the number of rows of the template, is used for defining the
staying conditions and the guards on edges emanating from the
location.  Then, a pair of upper and lower bounds
$\stay_\loc=\lt(\stay^-_\loc,\stay^+_\loc\rt)\in\mb{R}^{k_q}\times\mb{R}^{k_q}:
~~\stay^-_\loc\leq\stay^+_\loc$ together with the sub-parallelotopic
template define the sub-parallelotopic staying set as
$\sptope{\ptemplate_\loc}{\stay^-_\loc}{\stay^+_\loc}$.  The matrix
$A_\loc$ and a bounded set $\inp_\loc\subseteq\realset^n$ define the
linear transformation and the additive input set in the location.  The
set of edges is $\edgeset$, where $\edge\in\edgeset$ is a tuple $\edge
= \lt(\preloc{\edge},\postloc{\edge},\loweredgebound{\edge},\upperedgebound{\edge},\edgemap_\edge,\edgeinp_\edge\rt)$.
The pre and post locations of the edge are
$\preloc{\edge}\in\locationset$ and $\postloc{\edge}\in\locationset$,
respectively.  The pair of upper and lower bounds
$\lt(\edge^-,\edge^+\rt)\in\realset^{k_{\preloc{\edge}}}\times\realset^{k_{\preloc{\edge}}}:~~\edge^-\leq\edge^+$,
gives the sub-parallelotopic guard set
$\sptope{\ptemplate_{\preloc{\edge}}}{\edge^-}{\edge^+}$, which
is a precondition on the edge transition.  The matrix $\edgemap_\edge$
and a bounded set $\edgeinp_\edge\subseteq\realset^n$, respectively,
give the linear transfomation and the additive input set for all edge
(interlocation) transitions.

\tbf{Dynamics.}
The state of the hybrid system is a pair $(x,\loc)$, where
$x\in\realset^n$ is called the continuous state and
$\loc\in\locationset$ is called the discrete state.  The
evolution of the state of the system in time is called a
\emph{trajectory} of the system.  The trajectory is a function
$\systrj{x}{\loc}:\wholenums\ra\realset^n\times\locationset$, such
that for all $t\in\wholenums$, one of the following is true.

\begin{enumerate}
\item Continuous transition.
\begin{align}~\label{eqn:intralocation}
\begin{split}
& \exists u\in\inputset_{\trj{\loc}{t}}~~\text{such that all of
    the following  are collectively true.}\\
& \trj{x}{t+1} = \map_{\trj{\loc}{t}}\trj{x}{t}+u,~~~\trj{\loc}{t+1} = \trj{\loc}{t}~~
\text{and}\\
& \trj{x}{t},~\trj{x}{t+1}\in\sptope{\ptemplate_{\trj{\loc}{t}}}{\stay^-_{\trj{\loc}{t}}}{\stay^+_{\trj{\loc}{t}}}.
\end{split}
\end{align}
\item Discrete transition.
\begin{align} 
\begin{split}
& \exists \edge\in\edgeset~\text{and}~u\in\edgeinp_{\edge}~\text{such
that all of the following are collectively true.}\\
& \trj{\loc}{t}=\preloc{\edge},~~~\trj{x}{t}\in\sptope{\ptemplate_{\preloc{\edge}}}{\loweredgebound{\edge}\bigvee\stay^-_{\preloc{\edge}}}{\upperedgebound{\edge}\bigwedge\stay^+_{\preloc{\edge}}} \\
& \trj{x}{t+1} = \edgemap_{\trj{\loc}{t}}\trj{x}{t}+u,~~~\trj{\loc}{t+1}
= \postloc{\edge}\\
& \trj{x}{t+1}\in \sptope{\ptemplate_{\postloc{\edge}}}{\stay^-_{\postloc{\edge}}}{\stay^+_{\postloc{\edge}}}.
\end{split}
\end{align}
\end{enumerate}

Given a set of continuous states $S\in\realset^n$, we compute the set
of reachable continuous states in the next time step of intralocation transition
in a location $\loc\in\locationset$ or interlocation transition along
an edge $\edge\in\edgeset$, by the functions
$\locationtransition{\loc}:2^{\realset^n}\ra 2^{\realset^n}$ or
$\edgetransition{\edge}:2^{\realset^n}\ra 2^{\realset^n}$,
respectively, defined as
\begin{align*}
&\locationtransition{q}\lt(S\rt) = \lt\{\Calign{\lt(\map_{\loc}\lt(S\bigcap\staysptope{\loc}\rt)\oplus
\inputset_\loc                   \rt)}{~~\bigcap~~\staysptope{\loc}}\rt..\\
&\edgetransition{\edge}\lt(S\rt) =  \lt\{\Calign{\lt(\edgemap_\edge\lt(S\bigcap
\guardsptope{\edge}\rt)\oplus\edgeinp_\edge\rt)}{~~\bigcap~~\staysptope{\postloc{\edge}}}\rt..
\end{align*}

We shall identify a set of states by a mapping of the kind
$\hybridset:\locationset\ra 2^{\realset^n}$, called a \emph{state
set}, which corresponds to the set of states
$\lt\{\lt(x,\loc\rt):x\in\hybridset\lt(\loc\rt)\rt\}$.  For notational
convenience, we shall denote $\Gamma_\loc$ as the set of continuous
states of $\Gamma$ in a location $\loc$.  A \emph{positive invariant}
is a set of states of the system such that all trajectories beginning
at any state in the positive invariant remain within the positive
invariant.  Equivalently, a state set is a positive invariant if the
reachable set in one time step by both the intralocation and
interlocation dynamics is contained within the original state set.
\begin{definition}
A state set $\hybridset$ is a positive invariant if
$\forall\loc\in\locationset,~~\locationtransition{q}\lt(\hybridset_\loc\rt) \subseteq \hybridset_\loc~\label{eqn:pi1}~\text{and}~~
 \forall\edge\in\edgeset,~~\edgetransition{\edge}\lt(\hybridset_{\preloc{\edge}}\rt) \subseteq
  \hybridset_{\postloc{\edge}}$.
\end{definition}


%\section{Review of some set representations}~\label{sec:review}


\section{Augmented complex zonotopes}~\label{sec:acz}

Before we introduce augmented complex zonotopes, we briefly review the
related set representations that are used in this paper.  Firstly,
polytopes can be defined in terms of halfspace representation.
%
%\begin{definition}
Let $T\in\mat{n}{k}{\mb{R}}$ and $d\in\mb{R}^k$.  Then a (possibly
unbounded) \emph{polytope}, denoted $\polytope{T}{d}$, is defined as
$\polytope{T}{d} = \lt\{x\in\comprealset^k: Tx\leq d\rt\}$.
%\end{definition}
%
Usual zonotopes form a subclass of polytopes, which are geometrically
Minkowski sums of line segments. They are represented as a linear
combination of real vectors, called \emph{generators}, whose combining
coefficients are bounded in real valued intervals.
%\begin{definition}[Real zonotope]
Let $W\in\mat{n}{k}{\mb{R}}$ and $l,u\in\mb{R}^m: l\leq u$.  Then 
 a \emph{real zonotope} is
%\begin{equation*}
$\zon{W}{l}{u} = \lt\{W\zeta: \zeta\in\mb{R}^k,~\zeta_i\in[l_i,u_i]~\forall i\in \tup{k}\rt\}.$
%\end{equation*}
%\end{definition}
%
For simple examples of zonotopes like boxes and octagons, efficient
interconversion between the zonotopic representation and halfspace
polytopic representation is possible.  However, in general, zonotopes
do not have efficient halfspace representation as a polytope.  The
reason is that a zonotope with $m$ generators in an $n$ dimensional
space has ${m}\choose{n}$ faces (bounding hyperplanes), if all
combinations of $n$ generators are linearly independent.  That is, the
halfspace representation of a zonotope can be exponentially large,
compared to the above generator representation.

Zonotopes are closed under linear transformations and Minkowski sums, which can be computed efficiently.  Hence, zonotopes are
considered efficient for reachability analysis of linear systems.  Nevertheless,
a major drawback of zonotopes is that their intersection with sets defined by linear
constraints need not be zonotopes.  Also, there is no unique smallest
zonotope that overapproximates such intersections.  However, we observe that when the linear constraints
constitute a sub-parallelotope with a template aligned with that of
the zonotope, their intersection can be exactly computed.  This is
also the reason we considered the case of staying conditions and guards specified as sub-parallelotopes 
in the class of affine hybrid systems under study.  As a simple example, the intersection of
$\zon{\lt[\begin{array}{l l}1 & 0 \\ 0 &
      1\end{array}\rt]}{\lt[\begin{array}{c}-1\\ -1\end{array}\rt]}{\lt[\begin{array}{c}2\\ 2\end{array}\rt]}$
with $x_1\leq 1~\wedge~x_2\geq 0.5$ gives $\zon{\lt[\begin{array}{l
        l}1 & 0 \\ 0 &
      1\end{array}\rt]}{\lt[\begin{array}{c}-1\\ 0.5\end{array}\rt]}{\lt[\begin{array}{c}1\\ 2\end{array}\rt]}$.
The general case is described in the following lemma.
%
\begin{lemma}~\label{lem:motivation}
Let $K\in\mat{k}{n}{R}$ such that $k\leq n$ and $\lt(KK^T\rt)$ is
non-singular.  Then
\[
\zon{\pinv{K}}{l}{u} \bigcap \sptope{K}{\wh{l}}{\wh{u}}
= \zon{K}{l\bigvee \wh{l}}{u\bigwedge \wh{u}}
\]
\end{lemma}
%
To incorporate the possibly complex (having real and imaginary parts)
eigenstructure of linear maps while computing invariants, the complex
zonotope set representation and its generalization to the template
complex zontope were introduced in~\cite{adimoolam2016using,tcz2017}.
A template complex zonotope has complex valued vectors as generators,
whose combining coefficients are complex and bounded in their absolute
values.
%
\begin{definition}[Template complex zonotope]
Let $V\in\mat{n}{m}{\mb{C}}$ (template) and $s\in\mb{R}^m_{\geq 0}$ (scaling factors) and
$c\in\realset^n$ (center).  Then the following is a template complex zonotope:
%\begin{equation*}
$\cz{V}{c}{s} =
\lt\{V\epsilon:\epsilon\in\mb{C}^m,~\lt|\epsilon_i\rt|\leq s_i~\forall
i\in\tup{m}\rt\}.$
%\end{equation*}
\end{definition}
%
Unlike real zonotopes, a template complex zonotope can have a
non-polyhedral real projection.  Therefore, in general, checking the
exact inclusion between two template complex zonotopes amounts to
solving a non-convex optimization problem, which could be
computationally intractable.  Instead, a convex condition was proposed
in~\cite{tcz2017}, which is sufficient to guarantee the inclusion
between template complex zonotopes.  Here, we present this condition
as a relation between template complex zonotopes.  
%
\begin{definition}
We define a relation ``$\order$'' between template complex zonotopes
as\\ $\cz{V^\pr_{n\times m^\pr}}{c^\pr}{s^\pr}\order \cz{V_{n\times
    m}}{c}{s}$ if all of the below statements are collectively true.
\begin{align}~\label{eqn:tcz-inc}
\begin{split}
& \exists X\in\mat{m}{m^\pr}{\mb{C}}~\text{and}~y\in\mb{C}^{m}~\text{s.t.}\\
& \transfer{V}{V^\pr}{s^\pr}{X},~~~\centertransfer{V}{c}{c^\pr}{y}\\
& \scalebound{X}{y}{s}{m}{m^\pr}\leq 0\\
\end{split}
\end{align}
\end{definition}
%
\begin{lemma}[Inclusion: template complex
  zonotopes]~\label{lem:zon-zon} The
inclusion $\cz{V^\pr}{c^\pr}{s^\pr}\subseteq \cz{V}{c}{s}$ holds if
the relation $\cz{{V^\pr}}{c^\pr}{s^\pr}\order \cz{V}{c}{s}$ is true.
\end{lemma}
\emph{Proof idea.}
We relate the combining
coefficients of the two template complex zonotopes by a linear
transformation, with appropriate bounds on the transformation matrix
such that the inclusion holds.

For fixed $V$ and $V^\pr$, we observe that~(\ref{eqn:tcz-inc}) is
equivalent to as a set of convex constraints called second order
conic constraints.  A second order conic constraint (SOCC)
is defined as follows.
%
%\begin{definition}[SOCC]
A constraint of the form $\|Ax\|_{2}+Fx+b\leq 0$ on an $n$-dimensional
variable $x$, given $A,F\in\mat{n}{k}{\realset}$ and $b\in\mb{R}^k$, is
a second order conic constraint.
%\end{definition}
%
We also note that linear inequalities and equalities can be expressed
in the form of SOCC described above.  Our aforementioned observation
about~(\ref{eqn:tcz-inc}) is formalized below.
%
\begin{proposition}~\label{lem:zon-socc}
For fixed $V$,$V^\pr$, the relation
$\cz{V^\pr}{c^\pr}{s^\pr}\order \cz{V}{c}{s}$ is equivalent to a set
of second order conic constraints on the variables
$c,c^\pr,s,s^\pr,l,l^\pr$ and some additional variables.
\end{proposition}
%
There are many convex optimization tools that can efficiently solve
SOCC upto a high numerical precision.  

An augmented complex zonotope is the Minkowski sum of a template
complex zonotope and a real zonotope. In terms of expressivity, an
augmented complex zonotope is slightly more general than template
complex zonotopes.  But geometrically, the sets that can be described
as real projections of augmented complex zonotopes can also be
described as real projections of template complex zonotopes.  However,
with augmented complex zonotopes, the intersection with
subparallelotopic constraints can be succinctly specified, as we will
see latter. Consequently, this representation is more convenient to
derive conditions for computing invariants for the affine hybrid
system.
%
\begin{definition}[Augmented complex zonotope]
Let $V\in\mat{n}{m}{C}$ called primary template, $W\in\mat{n}{k}{R}$
called secondary template, $c\in\mb{R}^n$ called primary offset,
$s\in\mb{R}^m$ called scaling factors, $u,l\in\mb{R}^k$ called lower
and upper interval bounds, respectively, such that $l\leq u$.  The
following is an augmented complex
zonotope.
\begin{equation*}
=\gcz{V}{c}{s}{W}{l}{u} = \cz{V}{c}{s}\oplus\zon{W}{l}{u}.
\end{equation*}
\end{definition}
%
\subsection{Intersection with sub-parallelotope}
In invariant computation, we have to overapproximate the intersection
between the augmented complex zonotope and sub-parallelotopic
constraints.  We first discuss the intersection operation before
discussing other operations on augmented complex zonotope.  For
deriving a formula for the intersection, we require some results about
intersection among convex sets.  

Let us define the support of a vector $v\in\realset^n$ in a set
$S\in\realset^n$ relative to a point $w\in\realset^n$ as
$\support{w}{v}{S}=\max_{x\in S}v^T\lt(x-w\rt)$.   Then the following
lemma states a relationship between support of vectors and inclusion
between sets.
%
\begin{lemma}~\label{supp-inclusion}
Let $S_1,S_2\subseteq \realset^n$ be two closed convex sets such that
$S_1\bigcap S_2\neq \emptyset$.  Let $w\in S_1\bigcap S_2$.
Then $S_1\subseteq S_2$ iff $\forall
v\in\realset^n:~\support{w}{v}{S_1}\leq \support{w}{v}{S_2}$.
\end{lemma}
Let us say that two convex and closed sets $S_1$ and $S_2$ have
non-empty intersection and $w$ is a common point, i.e., inside the
sets.  According to the above lemma, the fact that $S_1$ is contained
inside $S_2$, is equivalent to saying that the maximum possible
displacement in $S_1$ from $w$ along the direction of any
vector $v$ is less than the maximum possible displacement in $S_2$
from $w$ along the direction of the vector $v$.

We defined an augmented complex zonotope as a Minkowski sum of a
complex zonotope and a real zonotope, i.e.,
$\cz{V}{c}{s}\oplus\zon{W}{l}{u}$.  In Lemma~\ref{lem:motivation}, we
have seen that the intersection of a sub-parallelotope
$\sptope{K}{\wh{l}}{\wh{u}}$ with a zonotope $\zon{W}{l}{u}$ can be
computed when $W=\pinv{K}$.  This being the motivation, we want to
find a condition under which we can overapproximate the intersection
of an augmented complex zonotope with a sub-parallelotope
$\lt(\cz{V}{c}{s}\oplus\zon{W}{l}{u}\rt)\bigcap\sptope{K}{\wh{l}}{\wh{u}}$
by the intersection
$\cz{V}{c}{s}\oplus\lt(\zon{W}{l}{u}\bigcap\sptope{K}{\wh{l}}{\wh{u}}\rt)$.
More generally, we find the required condition in the case of any
three closed convex sets sets $S_1,S_2,S_3$ and apply the result to
augmented complex zonotope.  That is, find a condition under which
$\lt(S_1\oplus S_2\rt)\bigcap S_3$ can be overapproximated by
$S_1\oplus\lt(S_2\bigcap S_3\rt)$.  We state and derive the required
condition as follows.
%
\begin{lemma}~\label{gen-int}
Let $S_1,S_2,S_3\in\complexset^n$ be closed convex sets such that
$S_2\cap S_3\neq \emptyset$ and $0\in S_1$.  Then $\lt(S_1\oplus S_2\rt)\bigcap
S_3\subseteq S_1\oplus\lt(S_2\cap S_3\rt)$.
\end{lemma}
\begin{proof}
 Let $w\in S_2\bigcap S_3$.  Then, since $0\in S_1$, so $w=w+0\in
 S_1\oplus S_2$.  Therefore, $w\in \lt(S_1\oplus S_2\rt)\bigcap S_3$.
 So, based on Lemma~\ref{supp-inclusion}, it sufficient to prove that
 for any $v\in\realset^n$, $\support{w}{v}{\lt(S_1\oplus
   S_2\rt)\bigcap S_3}\leq \support{w}{v}{S_1\oplus\lt(S_2\cap
   S_3\rt)}$.  Let us define $a = \support{0}{v}{S_1}$,
 $b=\support{w}{v}{S_2}$ and $c = \support{w}{v}{S_3}$.  It is easy to
 see that if a point belongs to a set, then relative to that point,
 the support in any direction is always greater than zero.  Since
 $w\in S_2\bigcap S_3$, so $b\geq 0$ and $c\geq 0$.  Since, $0\in
 S_1$, so also $a\geq 0$.  Also, the support of the intersection of
 two closed sets relative to $w$ is the minimum of the support of the
 two sets relative to $w$.  So, $\support{w}{v}{\lt(S_1\oplus
   S_2\rt)\bigcap S_3}$ $= \min\lt(\support{w}{v}{S_1\oplus
   S_2},\support{w}{v}{S_3}\rt)$.  As $w$ can be written as $w+0$, so
 the above equals
 $\min\lt(\support{0}{v}{S_1}+\support{w}{v}{S_2},\support{w}{v}{S_3}\rt)
 = \min(a+b,c)$.  On the other hand, by a similar computation, we can
 show $\support{w}{v}{S_1\oplus\lt(S_2\cap S_3\rt)} = a+min(b,c)$.
 So, we need to show $\min(a+b,c)\leq a+min(b,c)$.  We prove this in
 two cases. Case 1: Let $b\geq c$.  In this case, since $a\geq 0$ and
 $b\geq c$, so $\min(a+b,c) = c \leq a+c = a+\min(b,c)$.  Case 2: Let
 $b\leq c$.  In this case, $\min(a+b,c)\leq a+b = a+\min(b,c)$, since
 $b=min(b,c)$.
\end{proof}

Now, we define the following functions which are used latter to
express the overapproximation of the intersection between an augmented
complex zonotope and a sub-parallelotope.
%
Let us define a binary function
$\minaffinefunc:\realset^k\times\comprealset^k$,
called \emph{min-approximation} function, as follows.  For
$u\in\realset^k$ and $\wh{u}\in\comprealset^k$,
$\lt(\minaffine{u}{\wh{u}}\rt)_i = \left\{
\begin{array}{l}
\wh{u}_i~~\text{if}~\wh{u}_i<\infty\\
u_i~~\text{if}~\wh{u}_i=\infty
\end{array}
\right..$
Similarly, let us define another binary function       
$\maxaffinefunc:\realset^k\times\comprealset^k$,
called \emph{max-approximation} function, as follows.  For
$l\in\realset^k$ and $\wh{l}\in\comprealset^k$,
$\lt(\maxaffine{l}{\wh{l}}\rt)_i = \left\{
\begin{array}{l}
\wh{l}_i~~\text{if}~\wh{l}_i>\infty\\
l_i~~\text{if}~\wh{l}_i=-\infty
\end{array}
\right..$
%
Then, the following lemma is used to overapproximate the intersection
of an augmented complex zonotope with a sub-parallelotope, which is
expressed in terms of min and max approximation function.
\begin{theorem}~\label{lem:acz-int}
Consider a sub-parallelotope $\sptope{\ptemplate}{\wh{l}}{\wh{u}}$ an
augmented complex zonotope $\gcz{V}{c}{s}{\pinv{\ptemplate}}{l}{u}$
such that $VV^T$ is non-singular, $\lt|\pinv{V}c\rt|\leq s$, $l\leq
\maxaffine{l}{\wh{l}}$ and $u\geq \minaffine{l}{\wh{l}}$.  Then
$\gcz{V}{c}{s}{\pinv{\ptemplate}}{l}{u}\bigcap\sptope{\ptemplate}{\wh{l}}{\wh{u}}\subseteq
\gcz{V}{c}{s}{\pinv{\ptemplate}}{l\bigvee\wh{l}}{u\bigwedge\wh{u}}$.
\end{theorem}
\begin{proof}
  Since, $\maxaffine{l}{\wh{l}}\leq l$ as given, while we know
  $\maxaffine{l}{\wh{l}}\leq l\bigvee\wh{{l}}$, so
  $l\bigvee\wh{l}=\maxaffine{l}{\wh{l}}$. Similary, we can show
  $u\bigwedge \wh{u}=\minaffine{u}{\wh{u}}$.  We recall that
  $\gcz{V}{c}{s}{\pinv{\ptemplate}}{l}{u}=\cz{V}{c}{s}\oplus\zon{\pinv{\ptemplate}}{l}{u}$.
  Furthermore, by Lemma~\ref{lem:motivation}, we get
  $\zon{\pinv{\ptemplate}}{\wh{l}}{\wh{u}}\bigcap\sptope{\ptemplate}{\wh{l}}{\wh{u}}$
  is equal to
  $\zon{\pinv{\ptemplate}}{l\bigvee\wh{l}}{u\bigwedge{\wh{l}}}$, which
  equals
  $\zon{\pinv{\ptemplate}}{\maxaffine{l}{\wh{l}}}{\minaffine{u}{\wh{u}}}$
  by what we derived above.  So, provided the conditions in Lemma 4
  are satisfied, i.e., $0\in S_1$ and $S_2\bigcap S_3\neq 0$, the
  proof is accomplished by substituting $S_1=\cz{V}{c}{s}$,
  $S_2=\zon{\pinv{\ptemplate}}{l}{u}$ and
  $S_3=\sptope{\ptemplate}{\wh{l}}{\wh{u}}$ in the Lemma.  The prove
  that the conditions are satisfied as follows.  We write $0
  = c-V\pinv{V}c$, and since $\lt|\pinv{V}c\rt|\leq s$ as given, so
  $0\in \cz{V}{c}{s}=S_1$.  Next, since consider
  $w=\pinv{\ptemplate}\maxaffine{l}{\wh{l}}$.  Since $l\leq
  \maxaffine{l}{\wh{l}}\leq \minaffine{u}{\wh{u}}\leq u$, so
  $w\in\zon{\pinv{\ptemplate}}{l}{u}= S_2$.  Also, $\ptemplate
  w=\ptemplate\pinv{\ptemplate}\maxaffine{l}{\wh{l}} =
  \maxaffine{l}{\wh{l}}$ and $\wh{l}\leq \maxaffine{l}{\wh{l}}\leq
  \minaffine{u}{\wh{u}}\leq \wh{u}$, so $w\in
  \sptope{\ptemplate}{\wh{l}}{\wh{u}}=S_3$.  So, $S_2\bigcap S_3\neq
  0$, thus the required condition for Lemma~\ref{gen-int} to be
  applied is satisfied.
\end{proof}
%
%% To illustrate, the intersection of $\gcz{\lt[\begin{array}{cc}1+2i &
%% 2+i\\1-2i & 2-i\\0 &
%% 0\end{array}\rt]}{\lt[\begin{array}{c}1\\1\\0\end{array}\rt]}{\lt[\begin{array}{c}1\\1\\1\end{array}\rt]}
%% {\lt[\begin{array}{c}0\\0\\1\end{array}\rt]}{-2}{2}$. with a
%% constraint on the third coordinate $-1\leq x_3\leq 1$ is exactly\\
%% $\gcz{\lt[\begin{array}{cc}1+2i & 2+i\\1-2i & 2-i\\0 &
%% 0\end{array}\rt]}{\lt[\begin{array}{c}1\\1\\0\end{array}\rt]}{\lt[\begin{array}{c}1\\1\\1\end{array}\rt]}
%% {\lt[\begin{array}{c}0\\0\\1\end{array}\rt]}{-1}{1}$.  We observe that
%% the center $\lt(\begin{array}{c}1\\1\\0\end{array}\rt)$ is
%% perpendicular to the third axis, i.e., the vector
%% $\lt[\begin{array}{ccc}0 & 0 & 1\end{array}\rt]$, which is a required
%% condition in the above lemma.  Furthermore, since the primary template
%% is orthogonal to the subparallelotopic template in this example, i.e.
%% $\lt[\begin{array}{ccc}0 & 0 &
%% 1\end{array}\rt]\lt[\begin{array}{cc}1+2i & 2+i\\1-2i & 2-i\\0 &
%% 0\end{array}\rt]=0$, the resultant intersection is exactly an
%% augmented complex zonotope.  On the other hand, if the primary
%% template is not orthogonal with the secondary template, like in the
%% case of $\gcz{\lt[\begin{array}{cc}1+2i & 2+i\\1-2i & 2-i\\1 &
%% 1\end{array}\rt]}{\lt[\begin{array}{c}1\\1\\0\end{array}\rt]}{\lt[\begin{array}{c}1\\1\\1\end{array}\rt]}
%% {\lt[\begin{array}{c}0\\0\\1\end{array}\rt]}{-2}{2}$, then the
%% intersection with $-1\leq x_3\leq 1$ is overapproximated by
%% $\gcz{\lt[\begin{array}{cc}1+2i & 2+i\\1-2i & 2-i\\1 &
%% 1\end{array}\rt]}{\lt[\begin{array}{c}1\\1\\0\end{array}\rt]}{\lt[\begin{array}{c}1\\1\\1\end{array}\rt]}
%% {\lt[\begin{array}{c}0\\0\\1\end{array}\rt]}{-1}{1}$, but this is not
%% the exact intersected set.


Similar to usual zonotopes, augmented
complex zonotopes are closed under Minkowski sums and linear
transformations, and their computations are also similar. The computation of some important operations are summarized as follows.

\begin{enumerate}
\item $A\gcz{V}{c}{s}{W}{l}{u} = \gcz{AV}{Ac}{s}{AW}{l}{u}$.
\item $\gcz{V}{c}{s}{W}{l}{u}\oplus
  \gcz{V^\pr}{c^\pr}{s^\pr}{W^\pr}{l^\pr}{u^\pr}$\\
$= \gcz{\lt[V~~V^\pr\rt]}{c+c^\pr}{\ColumnJoin{s}{s^\pr}}{\lt[W~~W^\pr\rt]}{\ColumnJoin{l}{l^\pr}}{\ColumnJoin{u}{u^\pr}}$

%
\item The limits of the projection of an augmented complex zonotope along
any direction can be computed as follows. For $v\in\realset^n$,
\begin{equation}\label{lem:polylimits-acz}
\max_{x\in\gcz{V}{c}{s}{W}{l}{u}}v^Tx = v^T\lt(c+W\frac{l+u}{2}\rt)+\lt|v^T[V~~W]\rt|\lt(\ColumnJoin{s}{\frac{u-l}{2}}\rt)
\end{equation}
\end{enumerate}
%
%\begin{lemma}[Linear transformation and Minkowski sum]
%\begin{enumerate}
%\item $A\gcz{V}{c}{s}{W}{l}{u} = \gcz{AV}{Ac}{s}{AW}{l}{u}$.
%\item $\gcz{V}{c}{s}{W}{l}{u}\oplus
%  \gcz{V^\pr}{c^\pr}{s^\pr}{W^\pr}{l^\pr}{u^\pr}$\\
%$= \gcz{\lt[V~~V^\pr\rt]}{c+c^\pr}{\ColumnJoin{s}{s^\pr}}{\lt[W~~W^\pr\rt]}{\ColumnJoin{l}{l^\pr}}{\ColumnJoin{u}{u^\pr}}$
%\end{enumerate}
%\end{lemma}
%%
%The limits of the projection of an augmented complex zonotope along
%any direction are stated in the following lemma.
%%
%\begin{lemma}~\label{lem:polylimits-acz}
%Let $V\in\mat{n}{m}{\mb{C}}$ and $v\in\realset^n$.  Then,
%\[
%\max_{x\in\gcz{V}{c}{s}{W}{l}{u}}v^Tx = v^T\lt(c+W\frac{l+u}{2}\rt)+\lt|v^T[V~~W]\rt|\lt(\ColumnJoin{s}{\frac{u-l}{2}}\rt)
%\]
%\end{lemma}
%%%%%%%%%%%


The real projection of an augmented complex zonotope can be
equivalently transformed as the real projection of a template complex
zonotope, as follows.
%
\begin{lemma}~\label{lem:conversion}
$\real\lt(\gcz{V}{c}{s}{W}{l}{u}\rt) = \real\lt(\cz{\lt[V~W\rt]}{c+W\lt(\frac{u+l}{2}\rt)}{\ColumnJoin{s}{\frac{u-l}{2}}}\rt)$.
\end{lemma}
%
Because of the above relationship, checking the inclusion between the
real projections of two augmented complex zonotopes amounts to
checking the inclusion between real projections of two template
complex zonotopes.  Recall the relation between template complex
zonotopes that was a sufficient condition for inclusion.  We extend
the relation to augmented complex zonotopes as follows.
%
\begin{definition}~\label{defn:gcz-order}
We say that $\gcz{V^\pr}{c^\pr}{s^\pr}{W^\pr}{l^\pr}{u^\pr}\order
\gcz{V}{c}{s}{W}{l}{u}$ if\\ $\cz{\lt[V^\pr~W^\pr\rt]}{c^\pr+W^\pr\lt(\frac{u^\pr+l^\pr}{2}\rt)}{\ColumnJoin{s^\pr}{\frac{u-l}{2}}}
\order
\cz{\lt[V~W\rt]}{c+W\lt(\frac{u+l}{2}\rt)}{\ColumnJoin{s}{\frac{u-l}{2}}}.$
\end{definition}
%
\begin{lemma}[Inclusion: augmented complex
    zonotopes]~\label{lem:gcz-gcz} The real inclusion\\
$\real\lt(\gcz{V^\pr}{c^\pr}{s^\pr}{W^\pr}{l^\pr}{u^\pr}\rt)\subseteq \real\lt(\gcz{V}{c}{s}{W_{n\times
k}}{l}{u}\rt)$ holds if the relation\\
$\gcz{V^\pr}{c^\pr}{s^\pr}{W^\pr}{l^\pr}{u^\pr}\order \gcz{V}{c}{s}{W_{n\times
k}}{l}{u}$ is true.
\end{lemma}

The intersection of an augmented complex zonotope with a
subparallelotope involves the meet and join operations, as stated in
Lemma~\ref{lem:acz-int}.  These operations are piecewise affine
functions of their arguments, but not affine.  Hence, their
composition with a convex function may be non-convex.  But since we
are interested in deriving convex conditions for finding an invariant,
in this regard, we define the following upper and lower bound
functions for the join and meet operations, respectively.










\section{Computation of positive invariants}~\label{sec:invcomp}
For computing the possible reachable state from a given state set in
one transition, we define the following functions related to the
intralocation and interlocation dynamics of the hybrid system.
\begin{enumerate}
\item For any location $\loc\in\locationset$, $\locationtransition{\loc}:2^{\realset^n}\ra 2^{\realset^n}$, defined as
\begin{multline*}
\locationtransition{q}\lt(S\rt) = \lt(\map\lt(\loc\rt)\lt(S\bigcap\staysptope{\loc}\rt)\oplus \inputset\lt(\loc\rt)\rt)\\  \bigcap \staysptope{\loc}.
\end{multline*}
\item For any edge $\edge\in\edgeset$, $\edgetransition{\edge}:2^{\realset^n}\ra 2^{\realset^n}$, defined as
\begin{multline*}
\edgetransition{\edge}\lt(S\rt) =  \reset{\edge}\lt(S\bigcap \guardsptope{\edge}\rt) \\ \bigcap\staysptope{\postloc{\edge}}.
\end{multline*}
\end{enumerate}

A positively invariant set is a set of state of the hybrid system such
that all trajectories beginning at any state in the positive invariant
remain withing the positive invariant.  Since we identity set of
states of the hybrid system by a mapping called state set,
equivalently, we can define positively invariant state sets as
follows. This is based on the the reachable states by
intralocation and interlocation dynamics of the system in one step,
as follows.

\begin{definition}
A state set $\hybridset$ is a positive invariant if
both the following are true.
\begin{align}
& \forall\loc\in\locationset,~~\locationtransition{q}\lt(\hybridset(\loc)\rt) \subseteq \hybridset\lt(\loc\rt)\\
& \forall\edge\in\edgeset,~~\edgetransition{\edge}\lt(\hybridset\lt(\preloc{\edge}\rt)\rt) \subseteq
  \hybridset\lt(\postloc{\edge}\rt).
\end{align}
\end{definition}


Our objective is to find positive invariants whose projection in any
location is a generalized complex zonotope.  Firstly, for each
location $\loc\in\locationset$, we fix the primary template as
$V(\loc)\in\mat{n}{m(q)}{\mb{C}}$ and the secondary tempalte as
$\lt[\sectemp{\loc}\rt]$.  The secondary template
is chosen as such so that Lemma~\ref{lem:intersection} can be used to
compute the intersection with the sub-parallelotopic staying and guard
conditions.  Next, we derive sufficient conditions for finding the
primary offsets, scaling factors and sub-parallelotopic bounds of the
generalized complex zonotopic projections in all locations such that
the state set so defined is positively invariant.

\begin{lemma}
  Let  
  us consider a state set $\hybridset$ whose projection in in each
  location $\loc\in\locationset$ is $\hybridset(\loc) =
  \gcz{V(\loc)}{c(\loc)}{s(\loc)}{W(\loc)}{l(\loc)}{u(\loc)}$ where
  $W(\loc)={\lt[\sectemp{\loc}\rt]}$.  Let us consider that the additive input in each location
  $\loc\in\locationset$ can be overapproximated as $\inputset(\loc) = \cz{B_{n\times
      m^\pr(q)}(\loc)}{b(\loc)}{r(\loc)}$.  Then
   $\locationtransition{\loc}\lt(\hybridset(\loc)\rt)
  \subseteq \hybridset\lt(\loc\rt)$ holds if,
\begin{align}
\begin{split}
& \exists X(\loc)\in\mat{\lt(m(\loc)+k(\loc)\rt)}{\lt(m(\loc)+k(\loc)+m^\pr(\loc)\rt)}{\mb{C}},~
y(\loc)\in\mb{C}^{m(\loc)+k(\loc)}~\text{and}\\&~u^\pr(\loc),l^\pr(\loc)\in\realset^{k(\loc)}~\text{such
  that}:\\
\end{split}\\
\begin{split}
& \left[V(\loc),~W(\loc)\rt]X(\loc) =
  [AV(\loc),~AW(\loc),~B(\loc)]\dg\lt(\cjoin{\hspace{3em}~~~~~~~~s(\loc)}{\frac{\minaffine{\stay^+(\loc)}{u(\loc)}
      -\maxaffine{\stay^-(\loc)}{l(\loc)}}{2}}{\hspace{3em}~~~~~~~~r(\loc)}\rt),\\
\end{split}\\
\begin{split}
  & \left[V(\loc),~~W(\loc)\rt]y(\loc) = 
    \map(\loc)\lt(c(\loc)+\frac{\minaffine{\stay^+(\loc)}{u(\loc)}
     +\maxaffine{\stay^-(\loc)}{l(\loc)}}{2}\rt)\\ & \hspace{10em} +
  b(\loc)-c(\loc)-\frac{u^\pr(\loc)+l^\pr(\loc)}{2},
\end{split}\\
\begin{split}
& \forall
i\in\tup{m(\loc)},~\lt|y_i(\loc)\rt|+\sum_{i=1}^{m(\loc)+k(\loc)+m^\pr(\loc)}\lt|X_{ij}(\loc)\rt|\leq
s_i(\loc),
\end{split}\\
\begin{split}
& \forall
i\in\lt\{m(\loc)+1,...,m(\loc)+k(\loc)\rt\},\\
&\lt|y_i(\loc)\rt|+\sum_{i=1}^{m(\loc)+k(\loc)+m^\pr(\loc)}\lt|X_{ij}(\loc)\rt|\leq
\frac{u^\pr(\loc)-l^\pr(\loc)}{2},
\end{split}\\
& l^\pr(\loc)\leq u^\pr(\loc),\\
\begin{split}
& \maxaffine{l^\pr(\loc)}{\stay^-(\loc)}\geq
l(\loc)~\text{and}~\minaffine{u^\pr(\loc)}{\stay^+(\loc)}\leq u(\loc).
\end{split}
\end{align}
\end{lemma}

\begin{lemma}
  Let us consider a state set $\hybridset$ whose projection in each location
  $\loc\in\locationset$ is $\hybridset(\loc) =
  \gcz{V(\loc)}{c(\loc)}{s(\loc)}{W(\loc)}{l(\loc)}{u(\loc)}$ where
  $W(\loc)={\lt[\sectemp{\loc}\rt]}$.  Then for any edge
  $\edge\in\edgeset$, $\edgetransition{\edge}\lt(\hybridset\lt(\preloc{\edge}\rt)\rt) \subseteq
  \hybridset\lt(\postloc{\edge}\rt)$
    holds if the following is true.
\begin{align}
\begin{split}
& \exists X(\edge)\in
\mat{\lt(m\lt(\postloc{\edge}\rt)+k\lt(\postloc{\edge}\rt)\rt)}{\lt(m\lt(\preloc{\edge}\rt)+k\lt(\preloc{\edge}\rt)\rt)}{\mb{C}}~\text{and}~y(\edge)\in
\mb{C}^{m\lt(\preloc{\edge}\rt)+k\lt(\preloc{\edge}\rt)},\\
& \exists u^\pr(\edge),l^\pr(\edge)\in \realset^{k\lt(\postloc{\edge}\rt)}~\text{and}~s^\pr(\edge)\in\realset^{k\lt(\preloc{\edge}\rt)}~\text{s.t.}\\
\end{split}\\
\begin{split}
& \lt[V\lt(\postloc{\edge}\rt),~W\lt(\postloc{\edge}\rt)\rt]X(\edge) = \reset{\edge}\lt[V\lt(\preloc{\edge}\rt),~W\lt(\preloc{\edge}\rt)\rt]\dg
\ColumnJoin{s\lt(\preloc{\edge}\rt)}{s^\pr(\edge)}
\end{split}\\
\begin{split}
s^\pr(\edge) = \minaffine{u\lt(\preloc{\edge}\rt)}{\min\lt(\stay^+\lt(\preloc{\edge},\edge^+\rt)\rt)}-
\maxaffine{l\lt(\preloc{\edge}\rt)}{\max\lt(\stay^-\lt(\preloc{\edge},\edge^-\rt)\rt)},\\
\end{split}\\
\end{align}
\end{lemma}

\vspace{-1em}
\section{Experiments}~\label{sec:exp}
We performed experiments on three benchmark examples from literature
and compared the results with that obtained by the tool SpaceEx. [Add
  configuration here]. [Add floating point error here]


\subsection{Robot with a saturated controller}   Our first example is a benchmark
model of a self-balancing two wheeled robot called NXTway-GS1 by
Yorihisa Yamamoto, presented in the ARCH
workshop~\cite{heinz2014benchmark}.  The model is a networked control
system, i.e. a plant interacting with a controller.  The controller
has a hole, which is an unknown input to the controller and is modeled
as an additive disturbance input.  The controller input received by
the plant has a saturation limit.  Due to the saturation, the
composite system is modeled as a hybrid system.  Three different
models of the controller are proposed in the benchmark: continuous
linear, sampled data (discrete time) linear and non-linear.  In our
experiment, we consider the sampled data linear controller, with two
kinds of interaction with the plant: saturated, i.e., hybrid system
and unsaturated, i.e., linear system.  The sampling time given in the
benchmark is $4 ms$.  The safety requirement is that the \emph{body
  pitch angle} of the robot, denoted $\psi$, should be bounded within
some value. In the benchmark, it was suggested that
$\psi\in\lt[-\frac{\pi}{2}+\epsilon,\frac{\pi}{2}-\epsilon\rt]$ for
any $\epsilon>0$, is given as a safe set.  For the linear system
model, $\psi\in\lt[\frac{-\pi}{2.26},~\frac{\pi}{2.26}\rt]$ is given
as a safe set.


In discrete time, the composite sampled data system of the plant and
controller could be modeled using thirteen continuous state variables
and four uncertain input variables.  The model, however, had unbounded
trajectories in some directions.  Therefore, we decoupled some bounded
directions from the unbounded directions by an appropriate linear
transformation of the co-ordinates, such that the body pitch angle and
the controller inputs belong to the bounded directions.  The latter
model has ten continuous state variables and four uncertain input
variables. The controller input received by the plant is two
dimensional, which we denote by $u_1$ and $u_2$, respectively.  The
saturation limit on $u_i$ is $v_i=\delta d_p$, where $\delta=100$ and
$d_p=0.0807$.  Then, the saturated input is computed as $sat(u_i) =
max\lt(-v_i,min\lt(u_i,v_i\rt)\rt)$.  Thus, the two dimensional
controller input can be divided into nine regions such that in each
region, the saturation function is affine.

\tbf{Modeling}.  We model the saturated system (after transformation)
as a ten dimensional hybrid system using one location and nine self
edges with appropriate guards, such that all possible transitions
occur only along the edges.  For the unsaturated model, we have one
location and no edges, and the only system transition is specified by
the intralocation affine map.  The initial set is the origin.

\tbf{Size of model}: 10 dimensional, 1 location and 9 edges.

\tbf{Implementation.}  For the hybrid system, we choose the secondary
template as the pseudoinverse of the guarding hyperplane normals, in
conformity with Theroem~\ref{thm:main}.  The primary template for the
hybrid system is chosen as the collection of the (complex) eigenvectors of
linear matrices of all affine maps for the edge transitions, the
orthonormal vectors to the guarding hyperplane normals and the
projections of the eigenvectors on the subspace spanned by the
orthonormal vectors.  For the linear system, we only have a primary
template, which is constituted by the eigenvectors of the linear map,
the input set template and its multiplication by the linear matrix
(related to affine map) and square of the linear matrix.  For the
SpaceEx implementation, we tested with the octagon template and a
template with 400 uniformly sampled support vectors.

%% For the hybrid system, we computed a single augmented complex
%% zonotopic invariant satisfying both the upper and lower safety bounds.
%% But for the linear system, we computed two different invariants, one
%% each satisfying the upper and lower bounds, respectively.

\tbf{Results.}  For both the hybrid and the linear systems, we could
verify smaller magnitudes for the bounds on the pitch angle than what
is proposed in the benchmark~\cite{heinz2014benchmark}.  But the
SpaceEx tool could not find a finite bound for either of the above
systems.  The results are reported in the
Tables~\ref{tab:robot-unsaturated} and~\ref{tab:robot-saturated}.

\begin{table}
\center{UB: $>$1000, ~~NT: Not terminating in more than 180s, \newline
  n/a: Not applicable/not available, ~~ACZ: Augmented complex
  zonotope.\vspace{1em} }
\begin{minipage}{0.48\textwidth}
\centering
\begin{tabular}{|l|c|c|c|}
\hline
\multicolumn{2}{|c|}{\multirow{2}{*}{Method}} &
\multirow{2}{*}{$\lt|\psi\rt|\leq$} & Comp.\\
\multicolumn{2}{|c|}{} & & time (s)\\
\hline
\multirow{4}{*}{SpaceEx} & octagon & \multirow{2}{*}{UB} & \multirow{2}{*}{NT}\\
& template & & \\
\cline{2-4}
& 400 support & \multirow{2}{*}{UB} & \multirow{2}{*}{NT}\\
& vectors & &\\
\hline
\multicolumn{2}{|c|}{\multirow{2}{*}{Suggested in~\cite{heinz2014benchmark}}} &
\multirow{2}{*}{$1.39$} & \multirow{2}{*}{n/a}\\
\multicolumn{2}{|c|}{} & &\\
\hline
\multicolumn{2}{|c|}{\multirow{2}{*}{ACZ invariant}} & \multirow{2}{*}{$1.29$} &
\multirow{2}{*}{$4$}\\
\multicolumn{2}{|c|}{} & & \\
\hline
\end{tabular}
\caption{Unsaturated robot model: results}
~\label{tab:robot-unsaturated}
\end{minipage}
\hspace{0em}
\begin{minipage}{0.48\textwidth}
\centering
\begin{tabular}{|l|c|c|c|}
\hline
\multicolumn{2}{|c|}{\multirow{2}{*}{Method}} &
\multirow{2}{*}{$\lt|\psi\rt|\leq$} & Comp.\\
\multicolumn{2}{|c|}{} & & time (s)\\
\hline
\multirow{4}{*}{SpaceEx} & octagon & \multirow{2}{*}{UB} &
\multirow{2}{*}{NT}\\
& template & & \\
\cline{2-4}
& 400 support & \multirow{2}{*}{UB} & \multirow{2}{*}{NT}\\
& vectors & & \\
\hline
\multicolumn{2}{|c|}{\multirow{2}{*}{Suggested in~\cite{heinz2014benchmark}}} &
$1.571-\epsilon:$ & \multirow{2}{*}{n/a}\\
\multicolumn{2}{|c|}{} & $\epsilon>0$ &\\
\hline
\multicolumn{2}{|c|}{\multirow{2}{*}{ACZ invariant}} & \multirow{2}{*}{$1.13$} &
\multirow{2}{*}{45}\\
\multicolumn{2}{|c|}{} & &\\
\hline
\end{tabular}
\caption{Saturated robot model: results}
~\label{tab:robot-saturated}
\end{minipage}
\begin{minipage}{0.45\textwidth}
\begin{tabular}{|l|c|c|c|c|}
\hline
\multicolumn{2}{|c|}{\multirow{2}{*}{Method}} &
\multirow{2}{*}{$\lt|x_1\rt|\leq$} & \multirow{2}{*}{$\lt|x_2\rt|\leq$} & Comp.\\
\multicolumn{2}{|c|}{} & & & time (s) \\
\hline
\multirow{4}{*}{SpaceEx} & octagon & \multirow{2}{*}{0.38} &
\multirow{2}{*}{0.43} & \multirow{2}{*}{1.7}\\
& template & & &\\
\cline{2-5}
& 100 support & \multirow{2}{*}{0.38} & \multirow{2}{*}{0.43} & \multirow{2}{*}{23.6}\\
& vectors & & &\\
\hline
\multicolumn{2}{|c|}{\multirow{2}{*}{ACZ invariant}} &
\multirow{2}{*}{0.38} & \multirow{2}{*}{0.36} & 
\multirow{2}{*}{5.1}\\
\multicolumn{2}{|c|}{} & & &\\
\hline
\end{tabular}
\caption{Small invariant computation:\newline Perturbed double
  integrator}
~\label{tab:smallinv-pdi}
\vspace{1em}
\end{minipage}
\hspace{4em}
\begin{minipage}{0.4\textwidth}
\begin{tabular}{|c|c|}
\hline
\multirow{2}{*}{Method} & Comp.\\
& time (s)\\
\hline
\multirow{2}{*}{MPT tool~\cite{rakovic2004computation}} & \multirow{2}{*}{107}\\
& \\
\hline
\multirow{2}{*}{ACZ} & \multirow{2}{*}{12}\\
& \\
\hline
\end{tabular}
\caption{Large invariant computation: Perturbed double integrator}
~\label{tab:largeinv-pdi}
\vspace{1em}
\end{minipage}
%
\begin{tabular}{|l|c|c|c|c|c|}
\hline
\multicolumn{2}{|c|}{\multirow{2}{*}{Method}} &
\multirow{2}{*}{$-e_1\leq$} & \multirow{2}{*}{$-e_2\leq$} & \multirow{2}{*}{$-e_3\leq$} & Comp.\\
\multicolumn{2}{|c|}{} & & & & time (s)\\
\hline
\multirow{4}{*}{SpaceEx} & octagon & \multirow{2}{*}{28} &
\multirow{2}{*}{27} & \multirow{2}{*}{10} &
\multirow{2}{*}{NT}\\
& template & & & & \\
\cline{2-6}
& 100 support & \multirow{2}{*}{28} & \multirow{2}{*}{25} &
\multirow{2}{*}{13} & \multirow{2}{*}{1.3}\\
& vectors & & & & \\
\hline
\multicolumn{2}{|c|}{\multirow{2}{*}{Real zonotope~\cite{makhlouf2014networked}}} &
\multirow{2}{*}{25} & \multirow{2}{*}{25} & \multirow{2}{*}{10}
 & \multirow{2}{*}{n/a}\\
\multicolumn{2}{|c|}{} & & & & \\
\hline
\multicolumn{2}{|c|}{\multirow{2}{*}{ACZ invariant}} &
\multirow{2}{*}{28} & \multirow{2}{*}{26} &
\multirow{2}{*}{12} & \multirow{2}{*}{12}\\
\multicolumn{2}{|c|}{} & & & &\\
\hline
\end{tabular}
\caption{Large minimum dwell time (20s) model of networked
  vehicle platoon: results}
~\label{tab:largedwell-platoon}
 $~$\\
\begin{tabular}{|l|c|c|c|c|c|}
\hline
\multicolumn{2}{|c|}{\multirow{2}{*}{Method}} &
\multirow{2}{*}{$-e_1\leq$} & \multirow{2}{*}{$-e_2\leq$} & \multirow{2}{*}{$-e_3\leq$} & Comp.\\
\multicolumn{2}{|c|}{} & & & & time (s)\\
\hline
\multirow{4}{*}{SpaceEx} & octagon & \multirow{2}{*}{UB} &
\multirow{2}{*}{UB} & \multirow{2}{*}{UB} &
\multirow{2}{*}{NT}\\
& template & & & & \\
\cline{2-6}
& 100 support & \multirow{2}{*}{UB} & \multirow{2}{*}{UB} &
\multirow{2}{*}{UB} & \multirow{2}{*}{NT}\\
& vectors & & & & \\
\hline
\multicolumn{2}{|c|}{\multirow{2}{*}{ACZ invariant}} &
\multirow{2}{*}{46} & \multirow{2}{*}{54} &
\multirow{2}{*}{57} & \multirow{2}{*}{12.6}\\
\multicolumn{2}{|c|}{} & & & &\\
\hline
\end{tabular}
\caption{Small minimum dwell time (1s) model of networked vehicle
  platoon: results}
~\label{tab:smalldwell-platoon}
\end{table}
%
\subsection{Perturbed double integrator}
Our second example is a perturbed double integrator system that is
described in~\cite{rakovic2004computation}.  The closed loop system
with a feedback control is piecewise affine, having four different
affine dynamics in four different regions of space.  The system is two
dimensional and has a bounded additive disturbance input.  We perform
two different experiments on this system.  In the first experiment, we
try to verify the smallest possible magnitude of bounds on the two
coordinates, denoted $x_1$ and $x_2$. We compare these bounds with
that found by the SpaceEx tool.

%% \begin{equation}~\label{eqn:pwa-regions}
%% \trj{x}{t+1}=\lt(A_i+B_iK_i\rt)\trj{x}{t}+w,~\text{where}~
%% i=\left\{\begin{array}{l}
%% 1,~\text{if}~x_1\geq 0~\text{and}~x_2\geq 0\\
%% 2,~\text{if}~x_1\leq 0~\text{and}~x_2\leq 0\\
%% 3,~\text{if}~x_1\leq 0~\text{and}~x_2\geq 0\\
%% 4,~\text{if}~x_1\geq 0~\text{and}~x_2\leq 0\\
%% \end{array} \rt.
%% \end{equation}
%% %
%% \begin{align*}
%% & A_1 =\lt[\begin{array}{ll}
%% 1 & 1\\
%% 0 & 1
%% \end{array}\rt],~B_1 = \lt[\Calign{1}{0.5}\rt],~K_1 = \lt[-0.5897~
%%   -0.9347\rt]\\
%% & A_2 = \lt[\begin{array}{ll}
%% 1 & 1\\
%% 1 & 0
%% \end{array}
%% \rt],~B_2 = \lt[\Calign{-1}{-0.5}\rt],~K_2 = \lt[0.5897~~0.9387\rt]\\
%% & A_3 = \lt[\begin{array}{ll}
%% 1 & -1\\
%% 0 & 1
%% \end{array}
%% \rt],~B_3 = \lt[\Calign{-1}{0.5}\rt],~K_3 = \lt[0.5897~-0.9387\rt]\\
%% & A_4 = \lt[\begin{array}{ll}
%% 1 & -1\\
%% 0 & 1
%% \end{array}
%% \rt],~B_4 = \lt[\Calign{1}{-0.5}\rt],~K_4 = \lt[-0.5897~~0.9387\rt].\\
%% \end{align*}
%% %  
%% The additive disturbance input $w$ is bounded as $\|w\|_{\infty}\leq
%% 0.2$.  

In the second experiment, we try to quickly compute a large invariant
for the system under the safety constraints given
in~\cite{rakovic2004computation}.  %% The
%% given safety
%% constraints are $\|x\|_{\infty}\leq 5$ and $\lt|K_i(x)\leq
%% 1\rt|~\forall i\in\lt\{1,2\rt\}$.
In the latter case, we maximize the sum of the scaling factors and
differences of the upper and lower interval bounds of the augmented
complex zonotopic invaraint.  Furthermore, we decompose the given
safety constraints as the intersection of four different sets of
safety constraints.  For each set of safety constraints, we compute a
large augmented complex zonotopic invariant.  Then the desired
invariant is the intersection of four augmented complex zonotopic
invaraints.  Although we may not find the largest possible (maximal)
invariant by this approach, still the optimizer would find a large
invaraint.  We draw comparison in terms of the computation time with
the reported result for the MPT tool~\cite{rakovic2004computation}.

\tbf{Modeling.}  In our formalism, we model the system as two
dimensional with four locations and twelve edges.  Appropriate staying
conditions are specified in each location, reflecting the division of
the state space into different regions where the dynamics is affine.
The edges constitute all possible interlocation transitions.  The
initial set is the origin.  The same model is specified in SpaceEx.

\tbf{Size of model}: 2 dimensional, 4 locations and 12 edges.

\tbf{Implementation}.  We choose the secondary template in each
location as the pseudoinverse (in this case equal to) the hyperplane
normals of the staying conditions in that location, so that
Theorem~\ref{thm:main} is applicable.  For the primary template, we
collected the (complex) eigenvectors of all linear matrices of the
affine maps and their binary products. For the SpaceEx tool, we
experimented with two different templates, the octagon template and a
template with 100 uniformly sampled support vectors.

\tbf{Results.}  In the first experiment on this example, the bounds
verified bounds for the first coordinate by our method is equal to
that of SpaceEx. But for the second coordinate, we verfied smaller
bounds than that of SpaceEx.  In our second experiment on this
example, the computation time for finding a large invariant by our
method is significantly smaller than that of the reported result for
the MPT tool.  The results are summarized in the Tables~\ref{tab:smallinv-pdi}
and~\ref{tab:largeinv-pdi}.



\subsection{Networked platoon of vehicles}
Our third example is a model of a networked cooperative platoon of
vehicles, which is presented as a benchmark in the ARCH
workshop~\cite{makhlouf2014networked}.  The platoon consists of three
vehicles $M_1$, $M_2$ and $M_3$ along with a leader board ahead.  Each
vehicle has a reference distance to the vehicle ahead of it.  The
difference between the actual distance of a vehicle $M_i$ to its
successor and the reference distance is denoted as $e_i$.  Any upper
bound on $-e_i$ is a safe lower limit on the reference distance above
which the platoon is guaranteed not to collide.

The movement of the vehicles is dependent on the communication between
them.  In the benchmark proposal, the dynamics of the vechicles is
described as a hybrid system with two locations having different
dynamics.  In one location, there is communication between all the
vehicles, while in another location, there is complete communication
failure.  In the general model described in the benchmark, there can
be staying conditions for each location and time constraints on the
switching time.  The benchmark then considers a specific case where
the minimum dwell time is 20 seconds (also specified in the
distributed SpaceEx
implementation\footnote{http://cps-vo.org/node/15096}).  In this
paper, we consider two cases, minimum dwell times of 20 seconds and 1
second, respectively.


\tbf{Modeling.}  Since our method is applicable to discrete time
hybrid systems, we need to find a discrete time system whose reachable
set overapproximates that of the given continuous time system.  For
the large minimum dwell time of $20s$, it is possible efficiently
discretize the system, satisfying the aforementioned requirement.  We
do not, however, explain the discretization procedure here, because it
is beyond the scope of this paper.  But for the small minimum dwell
time of $1s$, a similar discretization of the system would lead to a
very high complexity model.  So, for our experiment in the latter
case, we alternatively considered a model in which the switching can
only happen at integer instants of time.  With this assumption,
efficient discretization was possible for the $1s$ minimum dwell
model.  Henceforth, we model the above two systems in the formalism
described in this paper. The same models are also used for the
discrete time SpaceEx implementation.  The size of both the models is
given below.

\tbf{Size of large dwell time model}: 9 dimensional, 2
locations and 4 edges.

\tbf{Size of small dwell time model}: 9 dimensional, 2
locations, 2 edges.

\tbf{Implementation.}  In our approach, we choose the primary template
as the collection of the (complex) eigenvectors of linear matrices of
the affine maps in the the two locations and their binary products,
the axis alligned box template and the templates used for
overapproximating the input sets.  The secondary template is set to
the zero vector since there are no linear guards or staying conditions
in this example.  For the SpaceEx tool, we experimented with two
templates, octagon and hundred uniformly sampled support vectors.

\tbf{Results.}  For the large minimum dwell time of $20s$, the
discrete time SpaceEx implementation and also a method based on using
real zonotopes~\cite{makhlouf2014networked} could verify slightly
smaller bounds on $-e_1$, $-e_2$ and $-e_3$, compared to our approach.
But for the small minimum dwell time ($1s$) model, SpaceEx could not
even find a finite set of bounds, whereas our approach could verify a
finite set of bounds.  These results are reported in the
Tables~\ref{tab:largedwell-platoon} and~\ref{tab:smalldwell-platoon}.

\subsection{Discussion of experimental results}
We observe that for four out of five experiments,
i.e. Tables~\ref{tab:robot-unsaturated},~\ref{tab:robot-saturated},~\ref{tab:smallinv-pdi},
and~\ref{tab:smalldwell-platoon}, where comparison is drawn with
SpaceEx, our approach produced more accurate results than the latter.
In terms of computation time, SpaceEx showed better performance only
if it managed to find an invariant. However, we note that out of the
five experiments with SpaceEx, only two experiments yeilded an
invariant.  We have also demonstrated in an experiment on the
perturbed double integrator that our method can be used to quickly
compute large invaraint.  Our method took significantly smaller
computation time than that of the MPT tool.  Furthermore, we draw the
following inference about the performance of the SpaceEx tool and a
real zonotope based implementation~\cite{makhlouf2014networked}, and
the advantage of our approach.

SpaceEx may show good performace for strongly stable systems, i.e.,
systems that converge quickly to a stable state.  For instance,
SpaceEx performed well on the networked platoon model with a large
minimum dwell time ($20s$).  This is because, both the dynamics in the
locations being stable, the lyapunov exponent (measure of stability)
tends to be higher for larger dwell times.  But SpaceEx could not
compute an invariant when the minimum dwell time is small, i.e., $1s$.
The reason being that in the latter case, the lyapunov exponent could
be small.  The same reasoning pertains to the case of the real
zonotope based set simulation~\cite{makhlouf2014networked}, which
showed better performance when the dwell time is larger.  If the dwell
time is smaller, then a zonotope based set simulation method would be
difficult to implement.  Also for high dimensional systems having
complex (non-real) eigenvalues, it can be difficult to find an
appropriate polytopic template for which SpaceEx could work well.  For
instance, SpaceEx with as many as 400 uniformly sampled support
vectors did not find an invariant for even the linear (unsaturated)
model of the NXT-robot.  Although in theory, a linear system has a
polytopic invariant, but the number of faces of such a polytope can be
arbitrarily large for a fixed dimension.  In case of NXT model, the
system is 10 dimensional with some eigenvalues whose magnitude is
close to one and it has complex (with non-negative imaginary part)
eigenvalues.  This could be the reason even a large number of support
vectors did not suffice to find the SpaceEx invariant for NXT-robot
model.


In contrast, since our approach can incorporate complex valued vectors
as generators, we are guaranteed to find finite invaraints for any
stable linear system by using the eigenvectors as generators.  The
overapproximation accuracy can, however, be increased by adding more
generators to the template.  Furthermore, by using the theory for
computing the intersection of augmented complex zonotopes with
sub-parallelotopes, we could handle models having linear guards and
staying conditions.  For instance, the saturated robot and the
perturbed double integrator models  have either linear guards or
staying conditions.






\vspace{-0.5em}

\section{Conclusion}~\label{sec:conclusion}
%\vspace{-em}
\section{Conclusion}
\vspace{0em}
We extended complex zonotopes to template complex zonotopes in order to
improve the efficiency of the computation of contractive sets and
positive invariants.  Template complex zonotopes retain a useful feature of complex zonotopes, which is the scope to incorporate the
eigenvectors of linear dynamics among the generators because the
eigenstructure is related to existence of positive
invariants.  In addition, compared to complex zonotopes, the advantage
template complex zonotopes have is the ability to regulate the
contribution of each generator to the set by using the scaling factors.
This resulted in significantly improved verification of stability of
nearly periodic impulsive systems, and also extending its application
to switched systems for verification of linear invariance properties.
The advantage of this new set representation is attested
by the experimental results that are better or competitive, compared
to the state-of-the-art methods and tools on benchmark examples. This
work also contributes a method for exploiting the eigenstructure of
linear dynamics to algorithmically determine template directions,
required by most verification approaches using template sub-polyhedral
sets. A number of directions for future research can be
identified. First, we intend to extend these techniques to analysis to
switched systems under constrained switching laws. Also
computationally speaking, our approach is close in spirit to abstract
interpretation. Indeed the operations used to find positive invariants and contractive sets
can be extended to invariant computation for more general hybrid
systems with state-dependent discrete transitions.
%to the way invariant sets are computed using some zonotopic

%and hybrid systems analysis. Indeed the way we find contractive sets using such zonotopes is similar 
%to the way invariant sets are computed using some zonotopic
%\cite{Girard05reachabilityof,Althoff2011,DBLP:conf/sas/GoubaultPV12} and
%template-polyhedral abstract domains \cite{S riram2008,jeannet2009apron}.


\newpage
%\bibliographystyle{plain}
\bibliographystyle{abbrv}
\bibliography{ref}

\appendix
%\section*{Proofs in section}

%\section{Proofs in Section~\ref{}}
%
\subsection*{Proof of Lemma~\ref{lem:motivation}}
\begin{proof}
Firstly, we prove $\zon{\pinv{K}}{l}{u} \bigcap
\sptope{K}{\wh{l}}{\wh{u}} \subseteq \zon{K}{l\bigvee
  \wh{l}}{u\bigwedge \wh{u}}$.  Let $x\in\zon{\pinv{K}}{l}{u} \bigcap
\sptope{K}{\wh{l}}{\wh{u}}$.  Then, $x=\pinv{K}\zeta:
\wh{l}\leq\zeta\leq \wh{u}$.  Since $x\in\sptope{K}{\wh{l}}{\wh{u}}$, so
$l\leq K\lt(\pinv{K}\zeta\rt)\leq u$ $\dimp$ $l\leq \zeta\leq u$.  So,
$l\bigvee \wh{l}\leq \zeta\leq u\bigwedge\wh{u}$.  Hence, $x\in\zon{K}{l\bigvee
  \wh{l}}{u\bigwedge \wh{u}}$.

Next, we show $\zon{\pinv{K}}{l}{u} \bigcap \sptope{K}{\wh{l}}{\wh{u}}
\supseteq \zon{K}{l\bigvee \wh{l}}{u\bigwedge \wh{u}}$. Let
$x=\pinv{K}\zeta\in\zon{K}{l\bigvee \wh{l}}{u\bigwedge \wh{u}}$.
Then, $l\bigvee\wh{l}\leq \zeta = K\lt(\pinv{K}\zeta\rt)=Kx\leq
u\bigwedge\wh{u}$.  Since $l\leq l\bigvee\wh{l} \zeta\leq u\bigwedge
\wh{u}\leq u$, so $x\in \zon{\pinv{K}}{l}{u}$.  Since $\wh{l}\leq
l\bigvee\wh{l} Kx\leq u\bigwedge \wh{u}\leq \wh{u}$, $x\in
\zon{\pinv{K}}{l}{u}$, so $x\in\sptope{K}{\wh{l}}{\wh{u}}$.
\end{proof}

\subsection*{Matrices in the first benchmark (robot with a saturated controller)}
{\scriptsize 
\begin{align*}
& F_1  = \lt[\begin{matrix}
3.6929   &      0  &  0.7302  &  7.9715 &  14.5019 &   -0.0072 &
0.0720 &   -2.7354\\
    3.6929   &      0  &  0.7302  &  7.9715 &  14.5019 &  -0.0072  &  0.0720  & -2.7354\\
    0.9562    &     0  &  0.0019 &  -0.0021 &  -0.0022 &   -0.0000 &  -0.0001 &  -0.0002\\
         0 &   0.6910    &     0    &     0  &       0     &    0   &      0    &     0\\
    0.8833     &    0  & -0.1154 &  -1.2943 &  -2.3520  &  0.0012 &  -0.0118  &  0.4427\\
   -0.4712    &     0 &  -0.0812 &    0.1151  & -1.4845  &  0.0007 &  -0.0071  &  0.2819\\
   -0.1560     &    0 &  -0.0459 &  -0.3173  &  0.3650  &  0.0003  & -0.0023  &  0.1162\\
   -0.7719   &      0 &  -0.1248  & -1.4264 &  -2.5901  &  0.9973 &  -0.0131  &  0.4869\\
   -0.7544  &       0  & -0.1243 &  -1.4204 &  -2.5792  &  0.0013 &   0.9825 &   0.4796\\
   -0.1905   &      0  & -0.0148  & -0.2081 &  -0.3751  &  0.0002  &  0.0033  &  1.0651
\end{matrix}\rt]\\
& F_2 = \lt[\begin{matrix}
0.2543  &  0.2543\\
    0.2543  &  0.2543\\
   -0.0001 &  -0.0001\\
         0 &        0\\
   -0.0413 &  -0.0413\\
    0.0219  &  0.0219\\
    0.0102 &   0.0102\\
    0.0431 &   0.0431\\
    0.0428 &   0.0428\\
    0.0065 &   0.0065\\
\end{matrix}\rt],
~F_3 = 10^{-2}\times\lt[\begin{matrix}
 0.0000    &     0  & -0.0330 &   2.0218\\
    0   &      0  & -0.0330 &  -2.0218\\
    0  &       0 &   -0  &  0\\
   -0  &       0  &  0 &   0.0109\\
   -0.0118 &        0  &  0.0172  &  0 \\
    0.0436  &       0 &   0.0003 &  0 \\
   -0.0478   &      0  &  0.0034 &   0 \\
  -13.3924 &        0 &   0.0062 &   0 \\
    0.0909     &    0  &  0.0061 &  0\\
   -0.0798  &       0 &   0.0017  &  0\\
\end{matrix}\rt]
\end{align*}}

\subsection*{Matrices in the second benchmark (perturbed double integrator)}
{\scriptsize
\begin{align*}~\label{eqn:pwa-regions}
i=\left\{\begin{array}{l}
1,~\text{if}~x_1\geq 0~\text{and}~x_2\geq 0\\
2,~\text{if}~x_1\leq 0~\text{and}~x_2\leq 0\\
3,~\text{if}~x_1\leq 0~\text{and}~x_2\geq 0\\
4,~\text{if}~x_1\geq 0~\text{and}~x_2\leq 0\\
\end{array} \rt.,
~M_1=M_2=\lt[\begin{matrix}
0.4103  &  0.0653\\
   -0.2949  &  0.5327
\end{matrix}\rt],~M_3=M_4=\lt[\begin{matrix}
0.4103  &  -0.0653\\
   0.2949  &  0.5327
\end{matrix}\rt]
\end{align*}}

\end{document}
