
Before we introduce augmented complex zonotopes, we briefly review
related set representations and discuss our motivation for the
current extension.

Usual (real) zonotopes are Minkowski sums of
line segemnts, represented as a linear combination of real vectors,
called \emph{generators}, whose combining coefficients
are bounded in real valued intervals.
\begin{definition}[Real zonotope]
Let $W\in\mat{n}{k}{\mb{R}}$ and $l,u\in\mb{R}^m: l\leq u$.  Then the
following is a real zonotope.
\begin{equation*}
\zon{W}{l}{u} = \lt\{W\zeta: \zeta\in\mb{R}^k,~\zeta_j\in[l_j,u_j]~\forall j\in \tup{k}\rt\}
\end{equation*}
\end{definition}
%
The linear transformations and Minkowski sums of zonotopes are also
zonotopes, which can be computed efficiently.  Hence, zonotopes are
efficient for reachability analysis of linear systems and some special
cases of affine hybrid systems.  We will discuss these operations
latter in the general case of augmented complex zonotopes.

Zonotopes are a sub-class of polytopes.  First, we define a polytope
in terms of hyperplane representation.  Latter we explain that
zonotopes do not have efficient hyperplane representation. 
%
\begin{definition}
Let $T\in\mat{n}{k}{\mb{R}}$ and $d\in\mb{R}^k$.  Then a (possibly
unbounded) polytope, denoted $\polytope{T}{d}$, is defined as
$\polytope{T}{d} = \lt\{x\in\comprealset^k: Tx\leq d\rt\}$.
\end{definition}
%
Although zonotopes are a special class of polytopes, they do not have
efficient hyperplane representation as a polytope.  A zonotope with
$m$ generators in an $n$ dimensional space can have as many as
${m}\choose{n}$ faces (bounding hyperplanes).  [GIVE EXAMPLE HERE].

Generally, the intersection of a real zonotope with linear constraints
is not a zonotope.  However, if the linear constraints constitute a
sub-parallelotope, then the intersection with a real zonotope having a
corresponding template can be exactly computed.  This is described in
the following lemma.
%
\begin{lemma}\label{lem:motivation}
Let $K\in\mat{n}{k}{R}$ such that $k\leq n$ and $\lt(K^TK\rt)$ is
non-singular.  Then
\[
\zon{K}{l}{u} \bigcap \sptope{\pinv{K}}{\wh{l}}{\wh{u}}
= \zon{K}{l\bigvee \wh{l}}{u\bigwedge \wh{u}}
\]
\end{lemma}
%

To incorporate the possibly complex eigenstructure of linear maps while
computing invariants, the complex zonotope set representation and its
generalization to the template complex zontope were introduced
in~\cite{todo}.  A template complex zonotope has complex valued
vectors as generators, whose combining coefficients are complex and
bounded in their absolute values.

\begin{definition}[Template complex zonotope]
Let $V\in\mat{n}{m}{\mb{C}}$ (template) and $s\in\mb{R}^m_{\geq 0}$ (scaling factors) and
$c\in\mb{C}^n$ (center).  Then the following is a template complex zonotope.
\begin{equation*}
\cz{V}{c}{s} =
\lt\{V\epsilon:\epsilon\in\mb{C}^m,~\lt|\epsilon_i\rt|\leq s_i~\forall
i\in\tup{m}\rt\}
\end{equation*}
\end{definition}

When the template vectors of a complex zonotope are real, then the
real projection of the complex zonotope is equivalent to a usual
zonotope and has a polytopic shape.  However, when some of the
template vectors have both real and imaginary parts, then the real
projection of a complex zonotope is a Minkowski sum of ellipsoids and
line segments, hence not polyhedral.  An example of the non-polyhedral
real projection of a complex zonotope is given in Figure~\ref{TODO}.

%% A major issue with using zonotopes for
%% analysis of affine hybrid systems is that there is no efficient way to
%% compute a reasonable zonotopic overapproximation of intersection with
%% linear guard conditions.  In case of usual zonotopes, a method for
%% computing the former was proposed in [TODO]~\cite{} that relied on
%% computing polytopic projections on different hyperplanes.  However,
%% this method is not efficient for high dimensions.  Furthermore, in case of
%% non-polyheadral complex zonotopes, to our knowledge there is no existing method to
%% overapproximate with reasonable accuracy the intersection with
%% linear guards as a complex zonotope.

In general, checking the exact inclusion between two template complex
zonotopes amounts to solving a non-convex optimization problem, which
could be inefficient.  Therefore, a convex condition was proposed
in~[CITE], which is sufficient to guarantee inclusion between template
complex zonotopes.  Here, we present this condition as a
partial order between template complex zonotopes.

%
\begin{definition}
We define a relation ``$\order$'' between template complex zonotopes
as\\ $\cz{V^\pr_{n\times m^\pr}}{c^\pr}{s^\pr}\order \cz{V_{n\times
    m}}{c}{s}$ if all of the following is true.
\begin{align}
\begin{split}
& \exists X\in\mat{m}{m^\pr}{\mb{C}}~\text{and}~y\in\mb{C}^{m}~\text{s.t.}\\
& \transfer{V}{V^\pr}{s^\pr}{X},~~~\centertransfer{V}{c}{c^\pr}{y}\\
& \scalebound{X}{y}{s}{m}{m^\pr}\leq 0\\
\end{split}
\end{align}
\end{definition}
%
The following lemma states that the relation defined above is a partial
order and is sufficient to
guarantee inclusion between template complex zonotopes.

\begin{lemma}[Ordering: template complex
  zonotopes]~\label{lem:zon-zon}
\begin{enumerate}
\item The relation ``$\order$'' between template complex zonotopes is
  a partial order.
\item The inclusion $\cz{V^\pr}{c^\pr}{s^\pr}\subseteq
  \cz{V}{c}{s}$ holds if the
  relation\\ $\cz{{V^\pr}}{c^\pr}{s^\pr}\order
  \cz{V}{c}{s}$ holds.
\end{enumerate}
\end{lemma}



