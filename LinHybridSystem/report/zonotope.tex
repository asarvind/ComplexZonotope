
Before we introduce augmented complex zonotopes, we briefly review
related set representations and discuss our motivation for the
current extension.

\subsection{Related set representations}
Usual zonotopes are Minkowski sums of
line segemnts, represented as a linear combination of real vectors,
called \emph{generators}, whose combining coefficients
are bounded in real valued intervals.
\begin{definition}[Real zonotope]
Let $W\in\mat{n}{k}{\mb{R}}$ and $l,u\in\mb{R}^m: l\leq u$.  Then the
following is a real zonotope.
\begin{equation*}
\zon{W}{l}{u} = \lt\{W\zeta: \zeta\in\mb{R}^k,~\zeta_j\in[l_j,u_j]~\forall j\in \tup{k}\rt\}
\end{equation*}
\end{definition}

Zonotopes are a special class of polytopes.  In terms of hyperplane
representation, we define a polytope as follows.
%
\begin{definition}
Let $T\in\mat{n}{k}{\mb{R}}$ and $d\in\mb{R}^k$.  Then a (possibly
unbounded) polytope, denoted $\polytope{T}{d}$, is defined as
$\polytope{T}{d} = \lt\{x\in\comprealset^k: Tx\leq d\rt\}$.

\end{definition}

Although zonotopes are a special class of polytopes, they do not have
efficient hyperplane representation as a polytope.  A zonotope with
$m$ generators in an $n$ dimensional space can have as many as
${m}\choose{n}$ faces (bounding hyperplanes).  [GIVE EXAMPLE HERE].
The linear transformations and Minkowski sums of zonotopes are also
zonotopes, which can be computed efficiently.  Hence, zonotopes are
efficient for reachability analysis of linear systems and some special
cases of affine hybrid systems.  We will discuss these operations
latter in the general case of augmented complex zonotopes.

To incorporate possibly complex eigenstructure of linear maps while
finding invariants, the complex zonotope set representation and its
generalization to template complex zontopes was introduced
in~\cite{todo}.  A template complex zonotope has complex valued
vectors as generators, whose combining coefficients are complex and
bounded in their absolute values.

\begin{definition}[Template complex zonotope]
Let $V\in\mat{n}{m}{\mb{C}}$ (template) and $s\in\mb{R}^m_{\geq 0}$ (scaling factors) and
$c\in\mb{C}^n$ (center).  Then the following is a template complex zonotope.
\begin{equation*}
\cz{V}{c}{s} =
\lt\{V\epsilon:\epsilon\in\mb{C}^m,~\lt|\epsilon_i\rt|\leq s_i~\forall
i\in\tup{m}\rt\}
\end{equation*}
\end{definition}

When the template vectors of a complex zonotope are real, then the
real projection of the complex zonotope is equivalent to a usual
zonotope and has a polytopic shape.  However, when some of the
template vectors have both real and imaginary parts, then the real
projection of a complex zonotope is a Minkowski sum of ellipsoids and
line segments, hence not polyheadral.  An example of the non-polyheadral
real projection of a complex zonotope is given in Figure~\ref{TODO}.

%% A major issue with using zonotopes for
%% analysis of affine hybrid systems is that there is no efficient way to
%% compute a reasonable zonotopic overapproximation of intersection with
%% linear guard conditions.  In case of usual zonotopes, a method for
%% computing the former was proposed in [TODO]~\cite{} that relied on
%% computing polytopic projections on different hyperplanes.  However,
%% this method is not efficient for high dimensions.  Furthermore, in case of
%% non-polyheadral complex zonotopes, to our knowledge there is no existing method to
%% overapproximate with reasonable accuracy the intersection with
%% linear guards as a complex zonotope. 

However, certain polytopes like parallelotopes are equivalent to
zonotopes and the coversion between the representations only requires
computing the inverse of some matrix.  Firstly, we define a class of
polytopes that can be efficiently converted to zonotopes and are
somewhat more general than parallelotopes. We call them as
sub-parallelotopes, whose hyperplane representation is defined as
follows.
%
\begin{definition}[Sub-parallelotope]~\label{defn:sub-parallelotope} Let
  $K\in\mat{k}{n}{R}$ such that $k\leq n$ and $\lt(KK^T\rt)$ is
  non-singular.  We call $K$ as a ñ\emph{sub-parallelotopic template}.
  Let $\wh{u},\wh{l}\in\comprealset^n$ such that $u\leq l$.  Then the
  following is a sub-parallelotopic set.
\[
\sptope{K}{\wh{l}}{\wh{u}} = \lt\{x\in\realset^n: \wh{l}\leq Kx \leq \wh{u}\rt\}
\]
\end{definition}
Note that in the above definition, the lower and upper limits on the
sub-parallelotope can be unbounded in some directions.  Therefore we
defined that the limits along each direction belong to
$\realset\bigcup\{\inf\}\bigcup\{-\inf\}$. We say that a
sub-parallelotope $\sptope{K}{\wh{u}}{\wh{l}}$ is bounded if $\wh{u}$
and $\wh{l}$ are finite, i.e. $\wh{u},\wh{l}\in\realset^k$, and
unbounded otherwise.  [GIVE EXAMPLE HERE].

A bounded sub-parallelotope has an equivalent representation as a
zonotope, which can be computed efficiently as follows.
\begin{lemma}
We have the equivalence $\sptope{K}{l}{u}=\zon{\pinv{K}}{l}{u}$.
\end{lemma}
[GIVE EXAMPLE HERE]


%% %
%% \begin{lemma}[Linear transformation]
%% Let $A\in\mat{n}{n}{\mb{R}}$, $W\in\mat{n}{k}{\mb{R}}$ and
%% $V\in\mat{n}{m}{\mb{C}}$.  We have 
%% \begin{align}
%% & A\zon{W}{l}{u} = \zon{AW}{l}{u}.\\
%% & A\cz{V}{c}{s} = \cz{AV}{Ac}{s}.
%% \end{align}
%% \end{lemma}
%% %
%% \begin{lemma}[Minkowski sum]
%% We have
%% \begin{align}
%% & \zon{W}{l}{u}\oplus\zon{W^\pr}{l^\pr}{u^\pr} = \zon{[W~~
%% W^\pr]}{\ColumnJoin{l}{l^\pr}}{\ColumnJoin{u}{u^\pr}}.\\
%% & \cz{V}{c}{s}\oplus\cz{V^\pr}{c^\pr}{s^\pr} = \cz{[V~~V^\pr]}{c+c^\pr}{\ColumnJoin{s}{s^\pr}}.
%% \end{align}
%% \end{lemma}
%% %

For affine transitions, it is known~[CITE] that the reachable set of a
real or complex zonotope after one transition can be efficiently
computed.  This is an advantage of using zonotopes for reachable set
computation of linear systems.  But in case of affine hybrid
systems, the transition between discrete states may be controlled by
linear guards (linear constraints).  Reachability analysis of such
systems would require computing the intersection of the set
representation being used with the linear guards.  In case of real
zonotopes, existing approaches to compute reasonably accurate
zonotopic overapproximations of the intersections with linear
guards~[CITE] are not efficient in high dimensions.  In case of
complex zonotopes, to our knowledge no method has yet been proposed
for computing a reasonable overapproximation of the intersection with
linear guards.

However, if the zonotope is also a sub-parallelotope, then the
intersection of 
the zonotope with a sub-parallelotope with a corresponding template can be
exactly computed as follows. The below observation is useful since for
the discrete time hybrid systems that we consider, the linear guards
are represented as sub-parallelotopes.
%
\begin{lemma}\label{lem:motivation}
Let $K\in\mat{n}{k}{R}$ such that $k\leq n$ and $\lt(K^TK\rt)$ is
non-singular.  We have
\[
\zon{K}{l}{u} \bigcap \sptope{\pinv{K}}{\wh{l}}{\wh{u}}
= \zon{K}{l\bigvee \wh{l}}{u\bigwedge \wh{u}}
\]
\end{lemma}
%

\subsection{Augmented complex zonotopes}
Motivated by the possibility of efficiently computing intersection
between a zonotope and a sub-parallelotope under some assumptions
(Lemma~\ref{lem:motivation}), we introduce a new set representation
called \emph{augmented complex zonotope}, which is a Minkowski sum of
template complex zonotope and real zonotope.

\begin{definition}[Augmented complex zonotope]
Let $V\in\mat{n}{m}{C}$ called primary template, $W\in\mat{n}{k}{R}$
called secondary template, $c\in\mb{R}^n$ called primary offset,
$s\in\mb{R}^m$ called scaling factors, $u,l\in\mb{R}^k$ called lower
and upper interval bounds, respectively, such that $l\leq u$.  The
following is an augmented complex
zonotope.
\begin{multline}
\gcz{V}{c}{s}{W}{l}{u} =
\lt\{
  c+V\epsilon+W\zeta:\epsilon\in\mb{C}^m,\zeta\in\mb{R}^k,\rt.\\ \lt.  \lt|\epsilon_i\rt|\leq
 s_i~\forall i\in\tup{m},~\zeta_j\in[l_j,u_j]~\forall j\in \tup{k}
\rt\}
\end{multline}
\end{definition}

We observe that the real zonotopic part of an augmented complex
zonotope can be used to represent bounded sub-parallelotopes.  Then
under suitable assumptions, intersection with sub-parallelotopes can
be exactly represented as an augmented complex zonotope and computed
easily.  These assumptions will be discussed in the next section.
Furthermore, we shall also state some relaxed assumptions under which
the intersection can be overapproximated, although not exactly
computed.



