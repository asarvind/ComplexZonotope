One of the most important requirements in the design of embedded and cyber-physical systems is safety which can be roughly stated as the system never enters a bad state. Safety verification for such systems are known to be computationally challenging due to the complexity in the interactions between their heterogenous components with mixed (continuous and discrete) dynamics. In this paper, we focus on the problem of finding invariants for hybrid systems which are mathematical widely recognized as suitable for modelling embedded and cyber-physical systems. An invariant is a property that is satisfied in every state that the system can reach. 
Therefore a common method for proving a safety property is to find an invariant that implies the safety property. Invariant computation has been studied extensively for program verification (see for example \cite{CousotHalbwachs1978,Bensalem2000,Tiwari2001,SriramSipma2004,colonSriramSipma2003,Fluctuat} and the techniques developed for generating invariants of programs have been extended to hybrid systems \cite{Sriram,Jeannet,tiwariRodriguezCarbonellPolynomialInvariants,Goubault,HybridFluctuat,differentialInvariantPlatzer,Gawlitza}. Barrier certificates \cite{prajna2004safety} are closely related to invariants in the sense that they describe a boundary that the system starting from a given initial set will never cross. Another common method for safety verification is to compute or over-approximate the reachable set of the system. Reachability computation techniques have been developed for continuous and hybrid systems, and many of such techniques are based on iterative approximation on a step-by-step basis and can be thought of as a set-extension numerical integration. A major drawback of this approach, inherent to undecidability of general hybrid systems with non-trivial dynamics, is that such an iterative procedure may not terminate and thus can only be used for bounded-time safety properties (when the over-approximation error accumulation is not too serious that the safety can be decided). Invariant and barrier certificate computations are by contrast consider conditions that invariants or barrier certificates should satisfy at any time. Although solving these conditions often involves fixed point computation, by exploiting the structure of the dynamics (such as eigenstructures of linear systems) one can derive meaningful conditions which reduce the number of iterations until convergence.

Comparison is needed (thao'll do)

The main contributions of our work is summarized below.
%
\begin{itemize}
\item \emph{Complex zonotope: } Our most important contribution is the
  extension of real zonotopes to the complex valued domain by
  complex zonotopes, which can capture contraction along complex
  vectors, but still are computationally as efficient as a real
  zonotope.  Just like real zonotopes, a complex zonotope is closed
  under Minkowski sum, linear transformation and their computation is
  also efficient.  The support function of a complex zonotope can also
  be computed by a simple algebraic expression.  But additionally, a
  complex zonotope can efficiently encode positive invariants for
  linear transformations by incorporating complex eigenvectors as
  generators.  On the other hand, eigenvectors having non-zero real
  and imaginary parts can not be used as generators in real zonotopes.
  Morover, complex zonotopes are geometrically more expressive since
  their real projections can represent non-polytopic sets in addition
  to polytopic zonotopes.
\item \emph{Template based representation: } We introduce a template
  based representation of a complex zonotope, by which we can add
  generators to a complex zonotope to find better approximations.  In
  a real zonotope, adding a generator to the zonotope can increase the
  size of the denoted set.  This problem is addressed by the template
  based representation of a complex zonotope, which has a set of
  scaling factors that determine the amount of contribution of each
  \emph{generator} to the size of the set.  Henceforth, when we can
  add more generators to a complex zonotope and adjust the scaling
  factors to find a better approximations.
\item \emph{Intersection with half-spaces: } Similar to real
  zonotopes, complex zonotopes are also not closed under intersection
  with half-spaces.  Previous
  approaches~\cite{scott2016constrained,Ghorbal2010} have addressed
  this problem for real zonotopes by allowing more linear contraints
  on the combining coefficients in addition to the interval bounds.
  For real zonotopes, even with addition of arbitrary linear
  constraints, the support function can be computed efficiently by
  linear programming.  But in the case of complex zonotopes, if we add
  more constraints than the quadratic absolute value bounds on the
  combining coefficients, accurate computation of the support function
  becomes intractable.  Alternatively, we generalize complex zonotopes
  to a set representation called \emph{augmented complex zonotopes} to
  over-approximate the intersection with a particular class of linear
  sub-level sets called \emph{sub-parallelotopes}.  An augmented
  complex zonotope is geometrically equivalent to a complex zonotope,
  where the computation of support function is just as efficient as
  the latter.  However, we show that an over-approximation of the
  intersection of an augmented complex zonotope with a
  sub-parallelotope can be efficiently encoded as another augmented
  complex zonotope.  Furthermore, the error in
  over-approximation can be regulated by adjusting the scaling
  factors.
\item \emph{Convex program for verifying linear invariance property:
}  We develop a convex program based on computing positively invariant
  augmented complex zonotopes to verify linear invariance properties
  of discrete time affine hybrid systems.  We perform experiments on
  three benchmark examples that demonstrate the efficiency of our
  approach. 
\item \emph{Stability verification of linear impulsive systems: } We
  develop an algorithm to find contractive complex zonotopes that
  verify global exponential stability of linear impulsive systems with
  sampling uncertainty.  The novelty of our algorithm lies in using
  the eigenstructure of reachability operators for stability
  verification, i.e., to find contractive complex zonotopes. Our
  experiments on two benchmark examples demonstrate either better or
  competitive performance compared to state of the art approaches.
\end{itemize}
%

