In the design of embedded and cyber-physical systems, one of the most
important requirements is safety, which can be roughly stated as that
the system will never enter a bad state. Safety verification for such
systems are known to be computationally challenging due to the
complexity resulting from the interactions among heterogenous
components, having mixed (continuous and discrete) dynamics. In this
paper, we focus on the problem of finding invariants for hybrid
systems, which are widely recognized as appropriate for modelling
embedded and cyber-physical systems. An invariant is a property that
is satisfied in every state that the system can reach. Therefore a
common approach for proving a safety property is to find an invariant
that implies the safety property. Invariant computation has been
studied extensively in the context of verification of transition
systems and program analysis (see for
example~\cite{CousotHalbwachs78,DBLP:journals/fmsd/BensalemL99,DBLP:conf/tacas/TiwariRSS01,DBLP:conf/cav/ColonSS03,DBLP:conf/sas/Goubault13}
and the developed techniques have been extended to continuous and
hybrid
systems \cite{DBLP:conf/hybrid/SankaranarayananSM04,jeannet2009apron,DBLP:conf/hybrid/Rodriguez-CarbonellT05,DBLP:conf/cdc/SassiGS14,DBLP:journals/tecs/AllamigeonGSGP16,HybridFluctuat,DBLP:conf/vmcai/SogokonGJP16,DBLP:conf/aplas/DangG11}. Barrier
certificates \cite{prajna2004safety} are closely related to invariants
in the sense that they describe a boundary that the system starting
from a given initial set will never cross to enter a region containing
bad states. Another common approach to safety verification is to
compute or over-approximate the reachable set of the system, and these
reachability computation techniques have been developed for continuous
and hybrid systems.  Many such techniques are based on iterative
approximation of the reachable state on a step-by-step basis, which
can be thought of as a set-based extension of numerical integration. A
major drawback of this approach, inherent to undecidability of general
hybrid systems with non-trivial dynamics, is that such an iterative
procedure may not terminate and thus can only be used for bounded-time
safety verification (except when the over-approximation error
accumulation is not too bad that the safety can be decided). In
contrast, invariants and barrier certificates are based
on conditions that are satisfied at all times. Although solving these
conditions often involves fixed point computation, by exploiting the
structure of the dynamics (such as eigenstructures of linear systems),
one can derive meaningful conditions which can significantly reduce
the number of iterations until convergence.



For discrete time affine hybrid systems, the eigenvectors of the
products of linear matrices related to the affine dynamics of
different subsystems can possibly capture some of the stable
directions for the overall hybrid dynamics.  As such, for invariant
computation, template complex zonotopes have the advantage that they
can include the possibly complex eigenvectors among the generators,
while usual (real) zonotopes can not.  In an earlier
work~\cite{tcz2017}, numerically efficiently solvable conditions for
computing a template complex zonotopic invariant subject to linear
safety constraints were obtained for a limited class of hybrid
systems, i.e., having uncontrolled switching.  However, a formidable
hurdle in extending the approach for more general affine hybrid
systems, where switching is controlled by linear constraints, is that
we have to handle the intersection of template complex zonotopes with
the linear constraints.  In this regard, template complex zonotopes
share the drawback of usual zonotopes that these classes of sets are
not closed under intersection with linear constraints.

In this paper, we circumvent this problem as follows.  We observe that
it is possible to compute or reasonably overapproximate the
intersection of a template complex zonotope with a class of linear
constraints, called subparallelotpic, by appropriately choosing the
template of the complex zonotope.  We use a slightly more general set
representation, called augmented complex zonotope, with which the
intersection operation can be succinctly presented.  %% Geometrically
%% speaking, augmented complex zonotopes and template complex zonotopes
%% describe the same classes of sets in terms of their real valued
%% projections.  However, it is easier and more succinct, using the
%% representation of an augmented complex zonotope instead of a template
%% complex zonotope, to represent the resultant intersection with linear
%% constraints. 
Then, we derive a numerically efficiently solvable
sufficient condition for computing an augmented complex zonotopic
invaraint satisfying linear safety constraints, for a discrete time
affine hybrid system with subparallelotopic switching constraints and
bounded additive disturbance input.  The sufficient condition is
expressed as a set of second order conic constraints.  We also note
that the class of sub-parallelotopic constraints that we consider are
quite general and can be used in the specification of many examples of 
affine hybrid systems.  To corroborate our approach by presenting the
experimental results for three benchmark examples from literature.
%% We implemented our approach on three benchmark examples from
%% literature and compared the results with that of the SpaceEx tool
%% on the same discrete time models, and also the reported benchmark
%% results from literature.  In some experiments, we could verify
%% finite safety bounds when SpaceEx could not even find an
%% invariant. In other experiments, we could verify competitive safety
%% bounds.  Also, our computation time is quite reasonable in all the
%% experiments, depending on the size of the specification.



\emph{Related work.} For hybrid systems verification, convex polyhedra~\cite{CousotHalbwachs78,jeannet2009apron}, and their special classes such as
zones~\cite{DBLP:conf/pado/Mine01},
octagons~\cite{DBLP:journals/lisp/Mine06}, zonotopes~\cite{DBLP:conf/hybrid/Girard05,DBLP:conf/eucc/MaigaCRT14} and
tropical polyhedra~\cite{DBLP:conf/sas/AllamigeonGG08} are the most
commonly used set representations. During the analysis which requires
operations under which a set representation is not closed (such as the
union or join operations for convex polyhedra and additionally
intersection for zonotopes) the complexity of generated sets increases
rapidly in order to guarantee a desired error bound. One way to
control this complexity increase, face normal vectors or generators
are fixed, which leads to template convex
polyhedra \cite{Sankaranarayanan+Dang+Ivancic-08-Symbolic,DBLP:conf/aplas/DangG11}. Although
our template complex zonotopes proposed in~\cite{tcz2017} do not
belong to the class of convex polyhedra, they follow the same
spirit of controlling the complexity. Set representations defined by non-linear constraints include
ellipsoids~\cite{Kurzhanski2000201}, polynomial
inequalities\cite{DBLP:conf/sas/BagnaraRZ05} et
equalities~\cite{Rodriguez-Carbonell:2007}, quadratic templates and
piecewise quadratic
templates~\cite{DBLP:conf/esop/AdjeGG10,DBLP:conf/hybrid/RouxJGF12,DBLP:conf/fm/RouxG14,DBLP:conf/hybrid/Adje17},
which are used for computing non-linear invariants. A major problem
that the template based approach faces is finding good templates. As
it will become clear later, using template complex zonotopes and the
augmented version introduced in this paper and exploiting
eigen-structures of linear dynamics which reflect the contraction or
expansion of a set by the dynamics, allows requiring only few of steps
until convergence to an invariant.

The complex zonotopes we proposed in~\cite{adimoolamACC2016} extend usual zonotopes to the complex domain, and geometrically speaking they are Minkowski sum of line segments and some ellipsoids. This extension is very similar in spirit to quadratic zonotopes~\cite{DBLP:conf/aplas/AdjeGW15} and more generally polynomial zonotopes~\cite{DBLP:conf/hybrid/Althoff13}. Nevertheless, while a polynomial zonotope is a set-valued polynomial function of \emph{intervals}, a complex zonotope is a set-valued function of unit \emph{circles} in the complex plane. Our idea of coupling additional linear constraints with zonotopes is inspired by the work on constrained zonotopes proposed in~\cite{DBLP:conf/cav/GhorbalGP09} for intersection computation. \\

\emph{Organization.}  The rest of the paper is organized as follows.  Firstly, we explain
some of the mathematical notation used in this paper.  Then in
Section~\ref{sec:system}, we describe the model of a discrete-time
affine hybrid system, controlled by sub-parallelotopic switching
conditions and having a bounded additive disturbance input. In
Section~\ref{sec:acz}, we present the set representation of augmented
complex zonotopes and discuss some important operations and relations, in particular intersection with sub-parallelotopic constraints, projection in any direction, linear transformation, Minkowski sum and inclusion checking.  In
Section~\ref{sec:invcomp}, we derive a set of second order conic
constraints to compute an augmented complex zonotopic invariant,
satisfying linear safety constraints and containing an initial set.
Furthermore, we explain how to choose the template.  In
Section~\ref{sec:exp}, we report some experimental results.  The
conclusion and future work are given in Section~\ref{sec:conclusion}.
We annex proofs of some lemmas presented in the paper in an
Appendix.

\emph{Notation.} Some notations, for which we
consider explanation may be required, is described below.  If $S$ is a
set of complex numbers, then $\real(S)$ and $\img(S)$ represent the
real and imaginary projections of $S$, respectively.  If $z$ is a
complex number, then $|z|$ denotes the absolute value of $z$.  On the
other hand, if $X$ is a complex matrix (or vector), then $\lt|X\rt|$
denotes the matrix (or vector) containing the absolute values of the
elements of $X$.  The diagonal square matrix containing the entries of
a complex vector $z$ along the diagonal is denoted by $\dg(z)$.  The
conjugate transpose of a matrix $V\in\mat{m}{n}{\complexset}$ is
denoted $\conjtranspose{V} = \lt(\real(V)-\iu\img(V)\rt)^T.$ If
$V\conjtranspose{V}$ is invertible, then
$\pinv{V}= \conjtranspose{V}\lt(V\conjtranspose{V}\rt)^{-1}$, which is
the pseudo-inverse of $V$.  Given two vectors $l,u\in\realset^k$, the
meet of the two vectors is denoted $l\bigwedge u$, defined as
$\lt(l\bigwedge u\rt)_i=\min\lt(l_i,u_i\rt)~\forall i\in\tup{k}$.  The
join is denoted $l\bigvee u$, defined as $\lt(l\bigvee
u\rt)_i=\max\lt(l_i,u_i\rt)~\forall i\in\tup{k}$.
