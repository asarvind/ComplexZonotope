One of the most important requirements in the design of embedded and cyber-physical systems is safety which can be roughly stated as the system never enters a bad state. Safety verification for such systems are known to be computationally challenging since the complexity in the interactions between their heterogenous components with mixed (continuous and discrete) dynamics. In this paper, we focus on the problem of finding invariants for hybrid systems which are mathematical widely recognized as suitable for modelling embedded and cyber-physical systems. An invariant is a property that is satisfied in every state that the system can reach. 
Therefore a common method for proving a safety property is to find an invariant that implies the safety property. Invariant computation has been studied extensively for program verification (see for example \cite{CousotHalbwachs1978,Bensalem2000,Tiwari2001,SriramSipma2004,colonSriramSipma2003,Fluctuate} and the techniques developed for generating invariants of programs have been extended to hybrid systems \cite{Sriram,Jeannet,tiwariRodriguezCarbonellPolynomialInvariants,Goubault,HybridFluctuate,differentialInvariantPlatzer,Gawlitza}. Barrier certificates \cite{Parillo...} are closely related to invariants in the sense that they describe a boundary that the system starting from a given initial set will never cross. Another common method for safety verification is to compute or over-approximate the reachable set of the system. Reachability computation techniques have been developed for continuous and hybrid systems, and many of such techniques are based on iterative approximation on a step-by-step basis and can be thought of as a set-extension numerical integration. A major drawback of this approach, inherent to undecidability of general hybrid systems with non-trivial dynamics, is that such an iterative procedure may not terminate and thus can only be used for bounded-time safety properties (when the over-approximation error accumulation is not too serious that the safety can be decided). Invariant and barrier certificate computations are by contrast consider conditions that invariants or barrier certificates should satisfy at any time. Although solving these conditions often involves fixed point computation, by exploiting the structure of the dynamics (such as eigenstructures of linear systems) one can derive meaningful conditions which reduce the number of iterations until convergence.