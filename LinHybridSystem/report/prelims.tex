
In a discrete time affine hybrid system, we have a finite set of
discrete variables, called locations, and a finite set of continuous
variables whose valuation is in the real euclidean space of dimension
$n\in\pint$.  In each location, there are a set of linear constraints,
called \emph{staying conditions}, within which the continuous state of
the system in that location is contrained.  Furthermore, there is
an affine transition map with (possibly) additive uncertain but
bounded disturbance input specifying the transition of the continuous
variables.  A set of labeled directed edges specify possible
transitions between different locations, accompanied by affine
transitions on continuous variables with a bounded additive
disturbance input.  Each edge transition is controlled by a set of
linear constraints on the continous variables, called guards.

In this paper, we consider a specific class of linear constraints
called, subparallelotopes, for defining guards and staying conditions,
such that intersection with augmented complex zonotopes (introduced
latter) can be conveniently computed.  These sets are a
generalization of parallelotopes to possibly unbounded sets.  We
discuss the aforementioned intersection operation latter after
defining augmented complex zonotopes.
%
\begin{definition}[Sub-parallelotope]~\label{defn:sub-parallelotope} Let
  $K\in\mat{k}{n}{R}$ such that $k\leq n$ and $\lt(KK^T\rt)$ is
  non-singular.  We call such a matrix $K$ as a
  \emph{sub-parallelotopic template}.  Let
  $\wh{u},\wh{l}\in\comprealset^n$ such that $\wh{u}\leq\wh{l}$.  Then
  the following is a sub-parallelotopic set.
\[
\sptope{K}{\wh{l}}{\wh{u}} = \lt\{x\in\realset^n: \wh{l}\leq Kx \leq \wh{u}\rt\}
\]
\end{definition}
%
For example, the set of linear constraints $-1\leq x+y-z\leq
1~\wedge~~ x-y+z\leq 3$ is equivalent to a sub-parallelotope
$\sptope{\ColumnJoin{\lt[1~~~1~-1\rt]}{\lt[1~-1~1\rt]}}{\ColumnJoin{-1}{-\infty}}{\ColumnJoin{1}{3}},$
because the rows of the sub-parallelotopic template are linearly
independent.  On the other hand, the set of constraints $-1\leq
x+y-z\leq 1~\wedge~~x+y+z\leq 2\wedge~~-1\leq x+y$ do not constitute a
sub-parallelotope, because the three row vectors $\lt[\begin{array}{c
c c}1 & 1 & -1\end{array}\rt]$, $\lt[\begin{array}{c c c}1 & 1 &
1\end{array}\rt]$, and $\lt[\begin{array}{c c c}1 & 1 &
0\end{array}\rt]$ together are linearly dependent.  Nevertheless, for
many examples of affine hybrid systems, the guards and staying
conditions can be specified by sub-parallelotopes.


\paragraph{Specification.}
We specify the discrete time affine hybrid system by a tuple 
%
\[
\system =
\lt(\locationset,\ptemplate,\stay,\edgeset,\linearmapset,\inputset\rt).
\]
%
Here, $\locationset$ is a finite set of locations.  For each location
$\loc\in\locationset$, a sub-parallelotopic template
$\ptemplate(\loc)\in\mat{k(q)}{n}{\realset}$ is used for defining the
the staying conditions and the guards on edges emanating from the
location.  A pair of upper and lower bounds
$\stay(\loc)=\lt(\stay^-(\loc),\stay^+(\loc)\rt)\in\mb{R}^n\times\mb{R}^n:
~~\stay^-(\loc)\leq\stay^+(\loc)$ together with the sub-parallelotopic
template $\ptemplate\lt(\loc\rt)$ define the
sub-parallelotopic staying set
$\sptope{\ptemplate\lt(\loc\rt)}{\stay^-(\loc)}{\stay^+(\loc)}$ in the
location.  The maps
$\linearmapset:\lt(\locationset\bigcup\edgeset\rt)\ra\mat{n}{n}{R}$
and $\inputset:\lt(\locationset\bigcup\edgeset\rt)\ra 2^{\mb{R}^n}$,
respectively, give the linear transfomation and additive input set for
all intralocation transitions and edge transitions.  The set of edges
is $\edgeset$, where $\edge\in\edgeset$ is a tuple $\edge =
\lt(\preloc{\edge},\postloc{\edge},\loweredgebound{\edge},\upperedgebound{\edge},\map(\edge),\inp(\edge)\rt)$.  The pre and post locations of the edge are
$\preloc{\edge}\in\locationset$ and $\postloc{\edge}\in\locationset$,
respectively.  The pair of upper and lower bounds
$\lt(\edge^-,\edge^+\rt)\in\realset^{k\lt(\preloc{\edge}\rt)}\times\realset^{k\lt(\preloc{\edge}\rt)}:~~\edge^-\leq\edge^+$,
relate to the sub-parallelotopic guard set
$\sptope{\ptemplate\lt(\preloc{\edge}\rt)}{\edge^-}{\edge^+}$, which
is a precondition on the edge transition.  The affine transition along
the edge is given by the linear map $\map(\edge)$ and additive
uncertain input set $\inp(\edge)$.

\paragraph{Dynamics.}
The state of the hybrid system is a pair $(x,\loc)$, where
$x\in\realset^n$ is called the continuous state and
$\loc\in\locationset$ is called the discrete state.  The
evolution of the state of the system in time is called a
\emph{trajectory}.  The trajectory is a function
$\systrj{x}{\loc}:\wholenums\ra\realset^n\times\locationset$, such
that for all $t\in\wholenums$, one of the following is true.

\begin{enumerate}
\item Intralocation dynamics.
\begin{align}~\label{eqn:intralocation}
\begin{split}
& \exists u\in\inputset\lt(\trj{\loc}{t}\rt)~~\text{such that all of
    the below conditions  are satisfied.}\\
& \trj{x}{t+1} = \map(\trj{\loc}{t})+u,~~~\trj{\loc}{t+1} = \trj{\loc}{t}~~
\text{and}\\
& \trj{x}{t},~\trj{x}{t+1}\in\sptope{\ptemplate\lt(\trj{\loc}{t}\rt)}{\stay^-\lt(\trj{\loc}{t}\rt)}{\stay^+\lt(\trj{\loc}{t}\rt)}.
\end{split}
\end{align}
\item Interlocation dynamics.
\begin{align} 
\begin{split}
& \exists \edge\in\edgeset~~\text{such that all of the below
    conditions are satisfied.}\\
&
  \trj{\loc}{t}=\preloc{\edge},~~~\trj{x}{t}\in\sptope{\ptemplate\lt(\preloc{\edge}\rt)}{\loweredgebound{\edge}\bigvee\stay^-\lt(\preloc{\edge}\rt)}{\upperedgebound{\edge}\bigwedge\stay^+\lt(\preloc{\edge}\rt)} \\
& \trj{x}{t+1} = \map(\trj{\loc}{t})+u,~~~\trj{\loc}{t+1} = \postloc{\edge}\\
& \trj{x}{t+1}\in \sptope{\ptemplate\lt(\postloc{\edge}\rt)}{\stay^-\lt(\postloc{\edge}\rt)}{\stay^+\lt(\postloc{\edge}\rt)}.
\end{split}
\end{align}
\end{enumerate}

The set of reachable continuous states in one time step of
intralocation and interlocation dynamics, respectively, from a given
set of continuous states is given by the following two functions.
\begin{enumerate}
\item For any location $\loc\in\locationset$, define $\locationtransition{\loc}:2^{\realset^n}\ra 2^{\realset^n}$ as
\begin{equation*}
\locationtransition{q}\lt(S\rt) = \lt\{\Calign{\lt(\map\lt(\loc\rt)\lt(S\bigcap\staysptope{\loc}\rt)\oplus
\inputset\lt(\loc\rt)\rt)}{~~\bigcap~~\staysptope{\loc}}\rt..
\end{equation*}
\item For any edge $\edge\in\edgeset$, define
  $\edgetransition{\edge}:2^{\realset^n}\ra 2^{\realset^n}$ as
\begin{multline*}
\edgetransition{\edge}\lt(S\rt) =  \lt\{\Calign{\lt(\map(\edge)\lt(S\bigcap
\guardsptope{\edge}\rt)\oplus\inp(\edge)\rt)}{~~\bigcap~~\staysptope{\postloc{\edge}}}\rt..
\end{multline*}
\end{enumerate}

We shall identify a set of states by a mapping of the kind
$\hybridset:\locationset\ra 2^{\realset^n}$, called a \emph{state
  set}, which corresponds to the set of states
$\lt\{\lt(x,\loc\rt):x\in\hybridset\lt(\loc\rt)\rt\}$.  A positive
invariant is a set of states of the system such that all trajectories
beginning at any state in the positive invariant remain withing the
positive invariant.  Equivalently, a state set is a positive invariant
if the reachable set in one time step by both the intralocation and
interlocation dynamics is contained within the original state set.
\begin{definition}
A state set $\hybridset$ is a positive invariant if
the following is true.
\[
 \forall\loc\in\locationset,~~\locationtransition{q}\lt(\hybridset(\loc)\rt) \subseteq \hybridset\lt(\loc\rt)~\label{eqn:pi1}~\text{and}~~
 \forall\edge\in\edgeset,~~\edgetransition{\edge}\lt(\hybridset\lt(\preloc{\edge}\rt)\rt) \subseteq
  \hybridset\lt(\postloc{\edge}\rt).
\]
\end{definition}
