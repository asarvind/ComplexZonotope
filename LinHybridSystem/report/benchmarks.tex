We performed experiments on three benchmark examples from literature
and compared the results with that obtained by the tool SpaceEx. [Add
  configuration here]. [Add floating point error here]

\subsection{Robot controller model with saturation}   We consider the benchmark
model of a self-balancing two wheeled robot called NXTway-GS1 by
Yorihisa Yamamoto, presented in the ARCH workshop~\cite{TODO}.  The
model is a networked control system, i.e. a plant interacting with a
controller.  The controller has a hole, which is an unknown input to
the controller and is modeled as an additive disturbance input.  The
controller send input to the plant, which has a saturation limit.  Due
to saturation, the system is modeled as a hybrid system.  Different
models of the controller are presented like continuous linear, sampled
data (discrete time) linear and non-linear.

In our experiment, we consider the discrete time affine hybrid system
model, i.e., with the saturation limits, as well as the unsaturated
discrete time affine system model.  The sampling time given in the
benchmark is $4 ms$.  The safety requirement is that the \emph{body
  pitch angle} of the robot, denoted $\psi$, should be bounded within
some value.  For the unsaturated affine system model, a the following
reasonable bounds were suggested:
$\psi\in\lt[-\frac{\pi}{2.26},\frac{\pi}{2.26}\rt]$.  Whereas, for the
affine hybrid system model, i.e. with saturation, the bounds were
suggested as
$\psi\in\lt[-\frac{\pi}{2}-\epsilon,\frac{\pi}{2}+\epsilon\rt]$ such
that $\epsilon>0$.

In discrete time, the composite system with the plant and controller
together can be modeled using thirteen continuous state variables and
four uncertain input variables.  This model, however, has unbounded
trajectories in some directions.  But we could decouple some bounded
directions from the unbounded directions by an appropriate linear
transformation of the co-ordinates, such that the body pitch angle and
the controller inputs belong to the bounded directions.  So, we
experimented with the transformed model.  The latter model has ten
continuous state variables and also four uncertain input variables
representing the additive disturbance input.  The controller input
received by the plant is two dimensional, which we denote by $u_1$ and
$u_2$, respectively.  The saturation limit on $u_i$ is $v_i=\delta
d_p$, where $\delta=100$ and $d_p=0.0807$.  Then, the saturated input
is computed as $sat(u_i) = max\lt(-v_i,min\lt(u_i,v_i\rt)\rt)$.  So,
the two dimensional controller input can be divided into nine regions
such that the saturation function is piecewise affine with respect to
these regions.  

\emph{Modeling}.  We model the hybrid system using one location and
nine self edges with appropriate guards, such that all possible
transitions occur only along the edges.  For the unsaturated model, we
have one location and no edges, where the system transition is given
by the intralocation affine map.  The initial set is the origin.

\emph{Implementation.}  We tried to find the smallest possible
magnitude of the safety bounds based on our approach and the SpaceEx
tool.  In our approach, we choose the secondary template as the
pseudoinverse of the normal for the hyperplane guards in the hybrid
model.  The primary template is choosen as the collection of (complex)
eigenvectors of all affine maps for the edge transitions, orthonormal
vectors to the normals of the hyperplane guards and projections of the
eigenvectors on the subspace orthogonal to the normal of hyperplane
guards.  For the SpaceEx implementation, we tested with the octagon
template and also a uniformly sampled template with 400 directions.
The performance of the octagon template is better than the latter,
hence we report the results for the octagon template.

\emph{Results.}  We could prove smaller magnitude of safety bounds
than what is proposed in the benchmark~\cite{TODO}..  In comparison,
the SpaceEx tool exceeded the proposed limits by a large margin in just
[TODO] iterations and could not find a fix point.  These results are
reported in Table~\ref{TODO}.  The computation times are reported in
Table~\ref{TODO}.

\subsection{Perturbed double integrator}
We consider the model of a perturbed double integrator, given
in~\cite{TODO}.  The closed loop system with feedback control is piecewise affine described as
\[\trj{x}{t+1}=\lt(A_i+B_iK_i\rt)\trj{x}{t}+w,~\text{where}~
i=\left\{\begin{array}{l}
1,~\text{if}~x_1\geq 0~\text{and}~x_2\geq 0\\
2,~\text{if}~x_1\leq 0~\text{and}~x_2\leq 0\\
3,~\text{if}~x_1\leq 0~\text{and}~x_2\geq 0\\
4,~\text{if}~x_1\geq 0~\text{and}~x_2\leq 0\\
\end{array} \rt.\]
%
\begin{align*}
& A_1 =\lt[\begin{array}{ll}
1 & 1\\
0 & 1
\end{array}\rt],~B_1 = \lt[\Calign{1}{0.5}\rt],~K_1 = \lt[-0.5897~
  -0.9347\rt]\\
& A_2 = \lt[\begin{array}{ll}
1 & 1\\
1 & 0
\end{array}
\rt],~B_2 = \lt[\Calign{-1}{-0.5}\rt],~K_2 = \lt[0.5897~~0.9387\rt]\\
& A_3 = \lt[\begin{array}{ll}
1 & -1\\
0 & 1
\end{array}
\rt],~B_3 = \lt[\Calign{-1}{0.5}\rt],~K_3 = \lt[0.5897~-0.9387\rt]\\
& A_4 = \lt[\begin{array}{ll}
1 & -1\\
0 & 1
\end{array}
\rt],~B_4 = \lt[\Calign{1}{-0.5}\rt],~K_4 = \lt[-0.5897~~0.9387\rt].\\
\end{align*}
%  
The additive disturbance input $w$ is bounded as $\|w\|_{\infty}\leq
0.2$.  

\emph{Modeling.}  In our formalism, the system is modeled by four
locations with the staying conditions corresponding to the divisions
of the state space described above.  Every location is connected to
every other location by an edge with a corresponding affine map.  In
this model, the linear constraints on transitions are comprised of the
staying conditions in the locations.  The same is modeled in SpaceEx.
The initial set is the origin.

\emph{Implementation}.  We choose the secondary template as
$\lt[\begin{array}{ll}1 & 0\\0 & 1\end{array}\rt]$, whose columns are
the pseudoinverse (in this case equal to) the normals of the
hyperplane guards.  For the primary template, we collected the
eigenvectors of all the affine maps as well as the eigenvectors of the
binary products of the different affine maps. For the SpaceEx tool, we
choose the octagon template.

\emph{Results.}  Firstly, using our approach and also SpaceEx, we
tried to find the smallest magnitudes of the bounds for the reachable
set along the two coordinates $x1$ and $x2$.  Our method found smaller
magnitudes of the bounds than that of SpaceEx for the second
coordinate, while these bounds were equal to that of SpaceEx for the
first coordinate.  The results are reported in Table~\ref{TODO}.

Next, we performed an exeriment to draw comparison, in terms of the
computation time, with the reported results of the MPT
tool~\cite{TODO} for finding a large invariant.  The method
of~\cite{TODO} computes the maximal invariant for the system under the
safety constraints $\|x\|_{\infty}\leq 5$ and $\lt|K_i(x)\leq
1\rt|~\forall i\in\lt\{1,2\rt\}$.  Our method may not be able to
compute the maximal invariant.  Still, we can optimize the size of the
invariant in terms of the largeness by maximizing an objective function
defined as the sum of scaling factors and differences between the
upper and lower interval bounds of the augmented complex zonotope.
The maximization of this objective function is done by second order
conic optimization in the CVX tool.  In our implementation, we first
decompose the above safety constraints as an intersection of four
different safety constraints.  The actual invariant is then computed
as the intersection of different invaraints satisfying each of the
four different safety constraints.  For each of the four safety
constraints, our method took less than $3$ seconds to compute an
invariant.  So, the total computation time is less than $4\times 3$
seconds $=12$ seconds.  In comparison, the reported computation time
of the MPT tool is 107 seconds.  This is summarized in
Table~\ref{TODO}.

\subsection{Networked platoon of vehicles}
We consider the example of a networked cooperative platoon, presented
as a benchmark in the ARCH workshop~\cite{TODO}.  The platoon consists
of three vehicles $M_1$, $M_2$ and $M_3$ along with a leader board
ahead.  Each vehicle has a reference distance to the vehicle ahead of
it.  The difference between the actual distance of a vehicle $M_i$ to
the successor and the reference distance is denoted as $e_i$.  Then,
any upper bound on $-e_i$ is a lower limit on the reference
distance, above which the platoon is guaranteed not to collide.

The movement of the vehicles is dependent on the communication between
them.  In the benchmark proposal, the dynamics of the vechicles is
described as a hybrid system with two locations having different
dynamics.  One location dynamics is exhibited when there is
communication among all the vehicles.  The other location dynamics is
exhibited when there is complete communication failure.  Furthermore,
in the SpaceEx implementation that is available online, a minimum time
lag of 20 seconds is assumed before switching from one location to
another.  The specification of the dynamics in each location can be
found in the paper~\cite{}.

\emph{Modeling.}  In our experiment, we specified a discrete time
system whose reachable states are an overapproximation of the
continuous time system.  There are nine continuous state variables in
the model, two locations and four edges in the discrete time
approximation.  Note that although in the continuous time model, there
are only two edges, two additional edges were added in the discrete
time model to account for the discretization error.  We do not explain
the discretization procedure for the continuous system since it is
beyond the scope of this paper.  The same discrete time model is
implemented in SpaceEx.  Furthermore, the paper~\cite{TODO} report the
continuous time implementation in SpaceEx as well.

\emph{Implementation.}  We tried to verify upper
bounds on $-e_1$, $-e_2$ and $-e_3$, respectively, that are as small
as possible.  In our implementation, we choose the primary
template as the collection of the eigenvectors of the affine maps in
the two locations.  The secondary template is set to the zero vector
since there are no linear guards or staying conditions in this
example.  For the SpaceEx implementation, we choose the octagon
template.

\emph{Results.}  We verified the integer upper bounds on $-e_1$,
$-e_2$ and $-e_3$ as $28$, $26$ and $12$, respectively.  In
comparision, the reported upper bounds for the continuous time SpaceEx
implementation are $30$, $30$ and $16$, respectively.  The discrete
time implementation in SpaceEx did not find a fix point in two hunderd
iterations.  Within these iterations, the bounds found by the discrete
time SpaceEx implementation are $28$, $27$ and $10$, respectively.
But the latter bounds are not reliable because a fixpoint is not
found.  Furthermore, the computation time until the end of two hundred
iterations for SpaceEx is TODO, while our computation time is less
than $19$ seconds.  We verified each bound separately, where each such
verification took around $6$ seconds, amounting to a total time of
less than $19$ seconds.  The results are reported in the
Tables~\ref{TODO} and~\ref{TODO}.

More accurate results are found by the (real) zonotope set based
simulation approach as reported in~\cite{TODO}.  These bounds are
$25$, $25$ and $10$, respectively.  For this particular example, the
set based simulation approach using zonotopes is feasible because
there are no guards or staying conditions on the state variables.
However, in general when there are linear guards and staying
conditions on the state variables, to our knowledge (usual) there is
no efficient way to compute, in each simulation step, the intersection
of the zonotope with the linear constraints.


