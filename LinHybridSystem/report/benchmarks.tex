We performed experiments on three benchmark examples from literature
and compared the results with that obtained by the tool SpaceEx. [Add
  configuration here]. [Add floating point error here]

\subsection{Robot controller model with saturation}   We consider the benchmark
model of a self-balancing two wheeled robot called NXTway-GS1 by
Yorihisa Yamamoto, presented in the ARCH workshop~\cite{TODO}.  The
model is a networked control system, i.e. a plant interacting with a
controller.  The controller has a hole, which is an unknown input to
the controller and is modeled as an additive disturbance input.  The
controller input received by the plant has a saturation limit.  Due to
saturation, the system is modeled as a hybrid system.  Three different
models of the controller are proposed in the benchmark: continuous
linear, sampled data (discrete time) linear and non-linear.  In our
experiment, we consider the sampled data hybrid system model, i.e.,
with the saturation limits.  The sampling time given in the benchmark
is $4 ms$.  The safety requirement is that the \emph{body pitch angle}
of the robot, denoted $\psi$, should be bounded within some value. In
the benchmark, the following bounds were suggested as reasonable
limits for the pitch angle:
$\psi\in\lt[-\frac{\pi}{2}+\epsilon,\frac{\pi}{2}-\epsilon\rt]$ such
that $\epsilon>0$.

\emph{Modeling}.  In discrete time, the composite sampled data system
of the plant and controller can be modeled using thirteen continuous
state variables and four uncertain input variables.  This model,
however, has unbounded trajectories in some directions.  But we could
decouple some bounded directions from the unbounded directions by an
appropriate linear transformation of the co-ordinates, such that the
body pitch angle and the controller inputs belong to the bounded
directions.  So, we experimented with the transformed model.  The
latter model has ten continuous state variables and four uncertain
input variables. The controller input received by the plant is two
dimensional, which we denote by $u_1$ and $u_2$, respectively.  The
saturation limit on $u_i$ is $v_i=\delta d_p$, where $\delta=100$ and
$d_p=0.0807$.  Then, the saturated input is computed as $sat(u_i) =
max\lt(-v_i,min\lt(u_i,v_i\rt)\rt)$.  So, the two dimensional
controller input can be divided into nine regions such that the
saturation function is piecewise affine with respect to these regions.
We model the hybrid system using one location and nine self edges with
appropriate guards, such that all possible transitions occur only
along the edges.  For the unsaturated model, we have one location and
no edges, where the system transition is given by the intralocation
affine map.  The initial set is the origin.

\emph{Implementation.}  In our approach, we choose the secondary
template as the pseudoinverse of the guarding hyperplane normals.  The
primary template is choosen as the collection of the (complex)
eigenvectors of all affine maps for the edge transitions, orthonormal
vectors to guarding hyperplane normals and projections of the
eigenvectors on the subspace spanned by the orthonormal vectors.  For
the SpaceEx implementation, we tested with the octagon template and
also a uniformly sampled template with 400 directions.  The
performance of the octagon template is better than the latter, hence
we report the results for the octagon template.

\emph{Results.}  We could prove smaller magnitude of safety bounds
than what is proposed in the benchmark~\cite{TODO}.  In comparison,
the SpaceEx tool exceeded the proposed limits by a large margin in just
[TODO] iterations and could not find a fix point.  These results are
reported in Table~\ref{TODO}.  The computation times are reported in
Table~\ref{TODO}.

\subsection{Perturbed double integrator}
We consider the model of a perturbed double integrator, given
in~\cite{TODO}.  The closed loop system with feedback control is piecewise affine described as
\[\trj{x}{t+1}=\lt(A_i+B_iK_i\rt)\trj{x}{t}+w,~\text{where}~
i=\left\{\begin{array}{l}
1,~\text{if}~x_1\geq 0~\text{and}~x_2\geq 0\\
2,~\text{if}~x_1\leq 0~\text{and}~x_2\leq 0\\
3,~\text{if}~x_1\leq 0~\text{and}~x_2\geq 0\\
4,~\text{if}~x_1\geq 0~\text{and}~x_2\leq 0\\
\end{array} \rt.\]
%
\begin{align*}
& A_1 =\lt[\begin{array}{ll}
1 & 1\\
0 & 1
\end{array}\rt],~B_1 = \lt[\Calign{1}{0.5}\rt],~K_1 = \lt[-0.5897~
  -0.9347\rt]\\
& A_2 = \lt[\begin{array}{ll}
1 & 1\\
1 & 0
\end{array}
\rt],~B_2 = \lt[\Calign{-1}{-0.5}\rt],~K_2 = \lt[0.5897~~0.9387\rt]\\
& A_3 = \lt[\begin{array}{ll}
1 & -1\\
0 & 1
\end{array}
\rt],~B_3 = \lt[\Calign{-1}{0.5}\rt],~K_3 = \lt[0.5897~-0.9387\rt]\\
& A_4 = \lt[\begin{array}{ll}
1 & -1\\
0 & 1
\end{array}
\rt],~B_4 = \lt[\Calign{1}{-0.5}\rt],~K_4 = \lt[-0.5897~~0.9387\rt].\\
\end{align*}
%  
The additive disturbance input $w$ is bounded as $\|w\|_{\infty}\leq
0.2$.  

\emph{Modeling.}  In our formalism, the system is modeled by four
locations with the staying conditions corresponding to the divisions
of the state space described above.  Every location is connected to
every other location by an edge with a corresponding affine map.  In
this model, the linear constraints on transitions are comprised of the
staying conditions in the locations.  The same is modeled in SpaceEx.
The initial set is the origin.

\emph{Implementation}.  We choose the secondary template as
$\lt[\begin{array}{ll}1 & 0\\0 & 1\end{array}\rt]$, whose columns are
the pseudoinverse (in this case equal to) the normals of the
hyperplane guards.  For the primary template, we collected the
eigenvectors of all the affine maps as well as the eigenvectors of the
binary products of the different affine maps. For the SpaceEx tool, we
choose the octagon template.

\emph{Results.}  Firstly, using our approach and also SpaceEx, we
tried to find the smallest magnitudes of the bounds for the reachable
set along the two coordinates $x1$ and $x2$.  Our method found smaller
magnitudes of the bounds than that of SpaceEx for the second
coordinate, while these bounds were equal to that of SpaceEx for the
first coordinate.  The results are reported in Table~\ref{TODO}.

Next, we performed an exeriment to draw comparison, in terms of the
computation time, with the reported results of the MPT
tool~\cite{TODO} for finding a large invariant.  The method
of~\cite{TODO} computes the maximal invariant for the system under the
safety constraints $\|x\|_{\infty}\leq 5$ and $\lt|K_i(x)\leq
1\rt|~\forall i\in\lt\{1,2\rt\}$.  Our method may not be able to
compute the maximal invariant.  Still, we can optimize the size of the
invariant in terms of the largeness by maximizing an objective function
defined as the sum of scaling factors and differences between the
upper and lower interval bounds of the augmented complex zonotope.
The maximization of this objective function is done by second order
conic optimization in the CVX tool.  In our implementation, we first
decompose the above safety constraints as an intersection of four
different safety constraints.  The actual invariant is then computed
as the intersection of different invaraints satisfying each of the
four different safety constraints.  For each of the four safety
constraints, our method took less than $3$ seconds to compute an
invariant.  So, the total computation time is less than $4\times 3$
seconds $=12$ seconds.  In comparison, the reported computation time
of the MPT tool is 107 seconds.  This is summarized in
Table~\ref{TODO}.

\begin{table}
\caption{Pitch angle bounds for the robot model}
\centering
\begin{tabular}{|c|c|}
\hline
Method & Bounds\\
\hline
\multirow{2}{9.8em}{Reasonable bounds suggested in~\cite{TODO}} &
$[-1.57+\epsilon~~~1.57-\epsilon]:$\\
& $\epsilon>0$\\
\hline
\multirow{2}{9.8em}{SpaceEx: Discrete time (Octagon template)} &
\multirow{2}{*}{$[-9.65~~9.65]$}\\
& \\
\hline
\multirow{2}{9.8em}{Augmented complex zonotope invariant}
&  \multirow{2}{*}{$[-1.13~~1.13]$}\\
& \\
\hline
\end{tabular}
\end{table}


\subsection{Networked platoon of vehicles}
We consider the example of a networked cooperative platoon, presented
as a benchmark in the ARCH workshop~\cite{TODO}.  The platoon consists
of three vehicles $M_1$, $M_2$ and $M_3$ along with a leader board
ahead.  Each vehicle has a reference distance to the vehicle ahead of
it.  The difference between the actual distance of a vehicle $M_i$ to
the successor and the reference distance is denoted as $e_i$.  Then,
any upper bound on $-e_i$ is a lower limit on the reference
distance, above which the platoon is guaranteed not to collide.

The movement of the vehicles is dependent on the communication between
them.  In the benchmark proposal, the dynamics of the vechicles is
described as a hybrid system with two locations having different
dynamics.  In one location, there is communication between all the
vehicles, while there is complete communication failure in the other
location.  In the general model described in the benchmark, there can
be staying conditions for each location and time constraints on the
switching time.  The paper then considers a specific case where the
switching time is greater than 20 seconds, (the minimum dwell time is
also specified in the distributed implementation~\cite{TODO} of SpaceEx).


\emph{Modeling.}  Since our implementation is concerned with discrete
time hybrid systems, we need to find a discrete time system whose
reachable set overapproximates that of the continuous time.  The
proposed discretization is possible if we assume that the switching
can occur at discrete instants of time.  We do not explain the
discretization procedure here, because it is beyond the scope of this
paper.  In our case study, we consider that the switching to occur at
any integer time instant.  Then, we modeled four different discretized
affine hybrid systems related to four different dwell times, i.e., $1s$,
$2s$, $4s$ and $20s$.  These discrete time systems were also modeled
in SpaceEx.

\emph{Implementation.}  We tried to smallest possible upper bounds on
$-e_1$, $-e_2$ and $-e_3$.  In our implementation, we choose the
primary template as the collection of the eigenvectors of the affine
maps in the two locations.  The secondary template is set to the zero
vector since there are no linear guards or staying conditions in this
example.  For the SpaceEx implementation, we considered three
different templates, i.e., octagon, fifty uniformly sampled support
vectors and hundred uniformly sampled support vectors.

\emph{Results.}  




