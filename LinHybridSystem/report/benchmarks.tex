We performed experiments on three benchmark examples from literature
and compared the results with that obtained by the tool SpaceEx. [Add
  configuration here]. [Add floating point error here]
\begin{table}
\center{UB: $>$1000, ~~NT: Not terminating in more than 180s, \newline
  n/a: Not applicable/not available, ~~ACZ: Augmented complex
  zonotope.\vspace{1em} }
\begin{minipage}{0.48\textwidth}
\centering
\begin{tabular}{|l|c|c|c|}
\hline
\multicolumn{2}{|c|}{\multirow{2}{*}{Method}} &
\multirow{2}{*}{$\lt|\psi\rt|\leq$} & Comp.\\
\multicolumn{2}{|c|}{} & & time (s)\\
\hline
\multirow{4}{*}{SpaceEx} & octagon & \multirow{2}{*}{UB} & \multirow{2}{*}{NT}\\
& template & & \\
\cline{2-4}
& 400 support & \multirow{2}{*}{UB} & \multirow{2}{*}{NT}\\
& vectors & &\\
\hline
\multicolumn{2}{|c|}{\multirow{2}{*}{Suggested in~\cite{TODO}}} &
\multirow{2}{*}{$1.39$} & \multirow{2}{*}{n/a}\\
\multicolumn{2}{|c|}{} & &\\
\hline
\multicolumn{2}{|c|}{\multirow{2}{*}{ACZ invariant}} & \multirow{2}{*}{$1.29$} &
\multirow{2}{*}{$4$}\\
\multicolumn{2}{|c|}{} & & \\
\hline
\end{tabular}
\caption{Unsaturated robot model results}
~\label{tab:robot-unsaturated}
\end{minipage}
\hspace{0em}
\begin{minipage}{0.48\textwidth}
\centering
\begin{tabular}{|l|c|c|c|}
\hline
\multicolumn{2}{|c|}{\multirow{2}{*}{Method}} &
\multirow{2}{*}{$\lt|\psi\rt|\leq$} & Comp.\\
\multicolumn{2}{|c|}{} & & time (s)\\
\hline
\multirow{4}{*}{SpaceEx} & octagon & \multirow{2}{*}{UB} &
\multirow{2}{*}{NT}\\
& template & & \\
\cline{2-4}
& 400 support & \multirow{2}{*}{UB} & \multirow{2}{*}{NT}\\
& vectors & & \\
\hline
\multicolumn{2}{|c|}{\multirow{2}{*}{Suggested in~\cite{TODO}}} &
$1.571-\epsilon:$ & \multirow{2}{*}{n/a}\\
\multicolumn{2}{|c|}{} & $\epsilon>0$ &\\
\hline
\multicolumn{2}{|c|}{\multirow{2}{*}{ACZ invariant}} & \multirow{2}{*}{$1.13$} &
\multirow{2}{*}{45}\\
\multicolumn{2}{|c|}{} & &\\
\hline
\end{tabular}
\caption{Saturated robot model results}
~\label{tab:robot-saturated}
\end{minipage}
\begin{minipage}{0.45\textwidth}
\begin{tabular}{|l|c|c|c|c|}
\hline
\multicolumn{2}{|c|}{\multirow{2}{*}{Method}} &
\multirow{2}{*}{$\lt|x_1\rt|\leq$} & \multirow{2}{*}{$\lt|x_2\rt|\leq$} & Comp.\\
\multicolumn{2}{|c|}{} & & & time (s) \\
\hline
\multirow{4}{*}{SpaceEx} & octagon & \multirow{2}{*}{0.38} &
\multirow{2}{*}{0.43} & \multirow{2}{*}{1.7}\\
& template & & &\\
\cline{2-5}
& 100 support & \multirow{2}{*}{0.38} & \multirow{2}{*}{0.43} & \multirow{2}{*}{23.6}\\
& vectors & & &\\
\hline
\multicolumn{2}{|c|}{\multirow{2}{*}{ACZ invariant}} &
\multirow{2}{*}{0.38} & \multirow{2}{*}{0.36} & 
\multirow{2}{*}{5.1}\\
\multicolumn{2}{|c|}{} & & &\\
\hline
\end{tabular}
\caption{Small invariant computation:\newline Perturbed double integrator}
\vspace{1em}
\end{minipage}
\hspace{6em}
\begin{minipage}{0.35\textwidth}
\begin{tabular}{|c|c|}
\hline
\multirow{2}{*}{Method} & Comp.\\
& time (s)\\
\hline
\multirow{2}{*}{MPT tool~\cite{TODO}} & \multirow{2}{*}{107}\\
& \\
\hline
\multirow{2}{*}{ACZ} & \multirow{2}{*}{12}\\
& \\
\hline
\end{tabular}
\caption{Large invariant computation: Perturbed double integrator}
\vspace{1em}
\end{minipage}
%
\begin{minipage}{0.45\textwidth}
\begin{tabular}{|c|c|c|c|c|}
\hline
Dwell & \multirow{2}{*}{$-e1<=$} & \multirow{2}{*}{$-e2\leq$} &
\multirow{2}{*}{$-e3\leq$} & Comp.\\
time (s) & & & & time (s)\\
\hline
\multirow{2}{*}{dt} & \multirow{2}{*}{result} &
\multirow{2}{*}{result} & \multirow{2}{*}{result} &
\multirow{2}{*}{result} \\
& & & & \\
\hline
\multirow{2}{*}{dt} & \multirow{2}{*}{result} &
\multirow{2}{*}{result} & \multirow{2}{*}{result} &
\multirow{2}{*}{result} \\
& & & & \\
\hline
\multirow{2}{*}{dt} & \multirow{2}{*}{result} &
\multirow{2}{*}{result} & \multirow{2}{*}{result} &
\multirow{2}{*}{result} \\
& & & & \\
\hline
\multirow{2}{*}{dt} & \multirow{2}{*}{result} &
\multirow{2}{*}{result} & \multirow{2}{*}{result} &
\multirow{2}{*}{result} \\
& & & & \\
\hline
\end{tabular}
\caption{Networked platoon model experiment in SpaceEx with the Octagon template}
\end{minipage}
\hspace{2.5em}
\begin{minipage}{0.45\textwidth}
\begin{tabular}{|c|c|c|c|c|}
\hline
Dwell & \multirow{2}{*}{$-e1<=$} & \multirow{2}{*}{$-e2\leq$} &
\multirow{2}{*}{$-e3\leq$} & Comp.\\
time (s) & & & & time (s)\\
\hline
\multirow{2}{*}{dt} & \multirow{2}{*}{result} &
\multirow{2}{*}{result} & \multirow{2}{*}{result} &
\multirow{2}{*}{result} \\
& & & & \\
\hline
\multirow{2}{*}{dt} & \multirow{2}{*}{result} &
\multirow{2}{*}{result} & \multirow{2}{*}{result} &
\multirow{2}{*}{result} \\
& & & & \\
\hline
\multirow{2}{*}{dt} & \multirow{2}{*}{result} &
\multirow{2}{*}{result} & \multirow{2}{*}{result} &
\multirow{2}{*}{result} \\
& & & & \\
\hline
\multirow{2}{*}{dt} & \multirow{2}{*}{result} &
\multirow{2}{*}{result} & \multirow{2}{*}{result} &
\multirow{2}{*}{result} \\
& & & & \\
\hline
\end{tabular}
\caption{Networked platoon model experiment in SpaceEx with 100 support
  vectors}
\vspace{0.5em}
\end{minipage}
$~$\\
$~$\\
\centering
\begin{tabular}{|c|c|c|c|c|}
\hline
Dwell & \multirow{2}{*}{$-e1<=$} & \multirow{2}{*}{$-e2\leq$} &
\multirow{2}{*}{$-e3\leq$} & Computation\\
time (s) & & & & time (s)\\
\hline
\multirow{2}{*}{dt} & \multirow{2}{*}{result} &
\multirow{2}{*}{result} & \multirow{2}{*}{result} &
\multirow{2}{*}{result} \\
& & & & \\
\hline
\multirow{2}{*}{dt} & \multirow{2}{*}{result} &
\multirow{2}{*}{result} & \multirow{2}{*}{result} &
\multirow{2}{*}{result} \\
& & & & \\
\hline
\multirow{2}{*}{dt} & \multirow{2}{*}{result} &
\multirow{2}{*}{result} & \multirow{2}{*}{result} &
\multirow{2}{*}{result} \\
& & & & \\
\hline
\multirow{2}{*}{dt} & \multirow{2}{*}{result} &
\multirow{2}{*}{result} & \multirow{2}{*}{result} &
\multirow{2}{*}{result} \\
& & & & \\
\hline
\end{tabular}
\caption{Networked platoon model experiment based on Augmented complex zonotopes}
\end{table}

\subsection{Robot with a saturated controller}   We consider the benchmark
model of a self-balancing two wheeled robot called NXTway-GS1 by
Yorihisa Yamamoto, presented in the ARCH workshop~\cite{TODO}.  The
model is a networked control system, i.e. a plant interacting with a
controller.  The controller has a hole, which is an unknown input to
the controller and is modeled as an additive disturbance input.  The
controller input received by the plant has a saturation limit.  Due to
the saturation, the composite system is modeled as a hybrid system.
Three different models of the controller are proposed in the
benchmark: continuous linear, sampled data (discrete time) linear and
non-linear.  In our experiment, we consider the sampled data linear
controller, with two kinds of interaction with the plant: saturated
(hybrid system) and unsaturated (linear system).  The sampling
time given in the benchmark is $4 ms$.  The safety requirement is that
the \emph{body pitch angle} of the robot, denoted $\psi$, should be
bounded within some value. In the benchmark, the following bounds on
the pitch angle were suggested as reasonable limits for the hyrbid
system model:
$\psi\in\lt[-\frac{\pi}{2}+\epsilon,\frac{\pi}{2}-\epsilon\rt]$ such
that $\epsilon>0$.  For the linear system model, the bounds were
suggested as $\psi\in\lt[\frac{-\pi}{2.26}~~\frac{\pi}{2.26}\rt]$.

In discrete time, the composite sampled data system of the plant and
controller could be modeled using thirteen continuous state variables
and four uncertain input variables.  The model, however, had unbounded
trajectories in some directions.  But we could decouple some bounded
directions from the unbounded directions by an appropriate linear
transformation of the co-ordinates, such that the body pitch angle and
the controller inputs belong to the bounded directions.  So, we
experimented with the transformed model.  The latter model has ten
continuous state variables and four uncertain input variables. The
controller input received by the plant is two dimensional, which we
denote by $u_1$ and $u_2$, respectively.  The saturation limit on
$u_i$ is $v_i=\delta d_p$, where $\delta=100$ and $d_p=0.0807$.  Then,
the saturated input is computed as $sat(u_i) =
max\lt(-v_i,min\lt(u_i,v_i\rt)\rt)$.  So, the two dimensional
controller input can be divided into nine regions such that the
saturation function is affine in each of these regions.

\tbf{Modeling}.  We model the saturated system as a ten dimensional
hybrid system using one location and nine self edges with appropriate
guards, such that all possible transitions occur only along the edges.
For the unsaturated model, we have one location and no edges, where
the system transition is given by the intralocation affine map.  The
initial set is the origin.

\tbf{Complexity of representation}: 10 dimensional, 1 location and 9 edges.

\tbf{Implementation.}  For the hybrid system, we choose the secondary
template as the pseudoinverse of the guarding hyperplane normals.  The
primary template for the hybrid system is the collection of the
(complex) eigenvectors of linear matrices of all affine maps for the
edge transitions, the orthonormal vectors to the guarding hyperplane
normals and the projections of the eigenvectors on the subspace
spanned by the orthonormal vectors.  For the linear system, we only
have a primary template, which is constituted by the eigenvectors of
the linear map, the input set template and its multiplication by the
linear matrix (related to affine map) and square of the linear matrix.
For the SpaceEx implementation, we tested with the octagon template
and a template with 400 uniformly sampled support vectors.

For the hybrid system, we tried to compute a single augmented complex
zonotope invariant for both the upper and lower safety bounds.  But
for the linear system, we computed two different invariants, verifying
the upper and the lower bounds separately.

\tbf{Results.}  For both the hybrid and the linear systems, we could
verify smaller magnitudes for the bounds on the pitch angle than what
is proposed in the benchmark~\cite{TODO}.  But the SpaceEx tool could
not find a finite bound for either of the above systems.  The bounds
along with the computation times are reported in the Tables~\ref{TODO}
and~\ref{TODO}.

%
\subsection{Perturbed double integrator}
We consider the model of a perturbed double integrator, given
in~\cite{TODO}.  The closed loop system with a feedback control has
four different affine dynamics in four different regions of space,
respectively, i.e. piecewise affine. The dynamics has a bounded
additive disturbance input.  The system is two dimensional.

%% \begin{equation}~\label{eqn:pwa-regions}
%% \trj{x}{t+1}=\lt(A_i+B_iK_i\rt)\trj{x}{t}+w,~\text{where}~
%% i=\left\{\begin{array}{l}
%% 1,~\text{if}~x_1\geq 0~\text{and}~x_2\geq 0\\
%% 2,~\text{if}~x_1\leq 0~\text{and}~x_2\leq 0\\
%% 3,~\text{if}~x_1\leq 0~\text{and}~x_2\geq 0\\
%% 4,~\text{if}~x_1\geq 0~\text{and}~x_2\leq 0\\
%% \end{array} \rt.
%% \end{equation}
%% %
%% \begin{align*}
%% & A_1 =\lt[\begin{array}{ll}
%% 1 & 1\\
%% 0 & 1
%% \end{array}\rt],~B_1 = \lt[\Calign{1}{0.5}\rt],~K_1 = \lt[-0.5897~
%%   -0.9347\rt]\\
%% & A_2 = \lt[\begin{array}{ll}
%% 1 & 1\\
%% 1 & 0
%% \end{array}
%% \rt],~B_2 = \lt[\Calign{-1}{-0.5}\rt],~K_2 = \lt[0.5897~~0.9387\rt]\\
%% & A_3 = \lt[\begin{array}{ll}
%% 1 & -1\\
%% 0 & 1
%% \end{array}
%% \rt],~B_3 = \lt[\Calign{-1}{0.5}\rt],~K_3 = \lt[0.5897~-0.9387\rt]\\
%% & A_4 = \lt[\begin{array}{ll}
%% 1 & -1\\
%% 0 & 1
%% \end{array}
%% \rt],~B_4 = \lt[\Calign{1}{-0.5}\rt],~K_4 = \lt[-0.5897~~0.9387\rt].\\
%% \end{align*}
%% %  
%% The additive disturbance input $w$ is bounded as $\|w\|_{\infty}\leq
%% 0.2$.  

We perform two different experiments on this system.  In the first
experiment, we try to verify the smallest possible magnitude of bounds
on the two coordinates $x_1$ and $x_2$.  In the second experiment, our
objective is to quickly compute a large invariant for the system under
the given safety constraints in~\cite{TODO}.  The given safety
constraints are $\|x\|_{\infty}\leq 5$ and $\lt|K_i(x)\leq
1\rt|~\forall i\in\lt\{1,2\rt\}$.  In order to do this, we
maximize the sum of the scaling factors and differences of the
upper and lower interval bounds of the augmented complex zonotopic
invaraint.  The above maximization is a second order conic program.
Although we may not find the largest (maximal) invariant by this
method, still the optimizer would find a large invaraint.  For this
experiment, we draw comparison in terms of the computation time with
the MPT tool.

\tbf{Modeling.}  We model the system as a two dimensional hybrid
system with four locations and twelve edges constituting all possible
directed edges.  Appropriate staying conditions are specified in each
location, reflecting the division of the state space into different
regions where the dynamics is affine.  The initial set is the origin.
The same model is specified in SpaceEx.

\tbf{Complexity of representation}: 2 dimensional, 4 locations and 12 edges.

\tbf{Implementation}.  We choose the secondary template as
the pseudoinverse (in this case equal to) the hyperplane normals of
the staying conditions.  For the primary template, we collected the
(complex) eigenvectors of all linear matrices of the affine maps and their
binary products. For the SpaceEx tool, we experimented with two
templates, the octagon template and a template with 100 uniformly
sampled support vectors.

\tbf{Results.}  In the first experiment on the perturbed double
integrator model, the bounds verified for the second coordinate are
smaller than that of SpaceEx.  For the first coordinate, the bounds
verified are equal to that of SpaceEx.  In our second experiment on
the perturbed double integrator model, our computation time for
finding a large invariant is significantly smaller than that of the
reported result of the MPT tool.  In our implementation, we first
decompose the given safety constraints as an intersection of four
different safety constraints.  The actual invariant is then given as
the intersection of different invaraints satisfying each of the four
different safety constraints.  For each of the four safety
constraints, our method took less than $3$ seconds to compute a large
invariant.  So, the total computation time is less than $4\times 3$
seconds $=12$ seconds.  In comparison, the reported computation time
of the MPT tool is 107 seconds.  The results are summarized in the
Tables~\ref{TODO} and~\ref{TODO}.



\subsection{Networked platoon of vehicles}
We consider the example of a networked cooperative platoon, presented
as a benchmark in the ARCH workshop~\cite{TODO}.  The platoon consists
of three vehicles $M_1$, $M_2$ and $M_3$ along with a leader board
ahead.  Each vehicle has a reference distance to the vehicle ahead of
it.  The difference between the actual distance of a vehicle $M_i$ to
its successor and the reference distance is denoted as $e_i$.  Then, any
upper bound on $-e_i$ is a lower limit on the reference distance
above which the platoon is guaranteed not to collide.

The movement of the vehicles is dependent on the communication between
them.  In the benchmark proposal, the dynamics of the vechicles is
described as a hybrid system with two locations having different
dynamics.  In one location, there is communication between all the
vehicles, while in another location, there is complete communication
failure.  In the general model described in the benchmark, there can
be staying conditions for each location and time constraints on the
switching time.  The paper then considers a specific case where the
dwell time (minimum switching time) is greater than 20 seconds, (this
is also in the distributed SpaceEx implementation~\cite{TODO}).
However, for our case study, we consider four different dwell times,
i.e., 20s, 4s, 2s, and 1s.


\tbf{Modeling.}  Since our implementation is concerned with discrete
time hybrid systems, we need to find a discrete time system whose
reachable set overapproximates that of the continuous time.  The
proposed discretization is possible if we assume that the switching
can occur at discrete instants of time.  We do not explain the
discretization procedure here, because it is beyond the scope of this
paper.  In our case study, we consider that the switching to occur at
any integer time instant.  The same discrete time models are
specified in SpaceEx as well. 

\tbf{Complexity of representation}: 9 dimensional, 2 locations and 2 edges. 

\tbf{Implementation.}  We choose the primary template as the
collection of the (complex) eigenvectors of linear matrices of the
affine maps in the the two locations and their binary products, the
axis alligned box template and the templates used for
overapproximating the input sets.  The secondary template is set to
the zero vector since there are no linear guards or staying conditions
in this example.  For SpaceEx, we considered the
the octagon template and also a template with hundred uniformly
sampled support vectors.

\tbf{Results.}  For the larger dwell times, i.e., $20s$ and $4s$,
SpaceEx could verify slightly smaller bounds on $-e_1$, $-e_2$, and
$-e_3$, compared to our approach.  However, for the smaller dwell
times, i.e., $2s$ and $1s$, SpaceEx could not find even a finite set of
bounds, while our approach could verify a finite set of bounds.  These
results and the computation times are reported in the
Table~\ref{TODO}.


%
\begin{table}
\begin{minipage}{0.45\textwidth}
\begin{tabular}{|c|c|c|c|c|}
\hline
Dwell & \multirow{2}{*}{$-e1<=$} & \multirow{2}{*}{$-e2\leq$} &
\multirow{2}{*}{$-e3\leq$} & Computation\\
time (s) & & & & time (s)\\
\hline
\multirow{2}{*}{dt} & \multirow{2}{*}{result} &
\multirow{2}{*}{result} & \multirow{2}{*}{result} &
\multirow{2}{*}{result} \\
& & & & \\
\hline
\multirow{2}{*}{dt} & \multirow{2}{*}{result} &
\multirow{2}{*}{result} & \multirow{2}{*}{result} &
\multirow{2}{*}{result} \\
& & & & \\
\hline
\multirow{2}{*}{dt} & \multirow{2}{*}{result} &
\multirow{2}{*}{result} & \multirow{2}{*}{result} &
\multirow{2}{*}{result} \\
& & & & \\
\hline
\multirow{2}{*}{dt} & \multirow{2}{*}{result} &
\multirow{2}{*}{result} & \multirow{2}{*}{result} &
\multirow{2}{*}{result} \\
& & & & \\
\hline
\end{tabular}
\caption{Networked platoon model experiment in SpaceEx with the Octagon template}
\end{minipage}
\hspace{3em}
\begin{minipage}{0.45\textwidth}
\begin{tabular}{|c|c|c|c|c|}
\hline
Dwell & \multirow{2}{*}{$-e1<=$} & \multirow{2}{*}{$-e2\leq$} &
\multirow{2}{*}{$-e3\leq$} & Computation\\
time (s) & & & & time (s)\\
\hline
\multirow{2}{*}{dt} & \multirow{2}{*}{result} &
\multirow{2}{*}{result} & \multirow{2}{*}{result} &
\multirow{2}{*}{result} \\
& & & & \\
\hline
\multirow{2}{*}{dt} & \multirow{2}{*}{result} &
\multirow{2}{*}{result} & \multirow{2}{*}{result} &
\multirow{2}{*}{result} \\
& & & & \\
\hline
\multirow{2}{*}{dt} & \multirow{2}{*}{result} &
\multirow{2}{*}{result} & \multirow{2}{*}{result} &
\multirow{2}{*}{result} \\
& & & & \\
\hline
\multirow{2}{*}{dt} & \multirow{2}{*}{result} &
\multirow{2}{*}{result} & \multirow{2}{*}{result} &
\multirow{2}{*}{result} \\
& & & & \\
\hline
\end{tabular}
\caption{Networked platoon model experiment in SpaceEx with 100 support
  vectors}
\end{minipage}
$~$\\
$~$\\
\centering
\begin{tabular}{|c|c|c|c|c|}
\hline
Dwell & \multirow{2}{*}{$-e1<=$} & \multirow{2}{*}{$-e2\leq$} &
\multirow{2}{*}{$-e3\leq$} & Computation\\
time (s) & & & & time (s)\\
\hline
\multirow{2}{*}{dt} & \multirow{2}{*}{result} &
\multirow{2}{*}{result} & \multirow{2}{*}{result} &
\multirow{2}{*}{result} \\
& & & & \\
\hline
\multirow{2}{*}{dt} & \multirow{2}{*}{result} &
\multirow{2}{*}{result} & \multirow{2}{*}{result} &
\multirow{2}{*}{result} \\
& & & & \\
\hline
\multirow{2}{*}{dt} & \multirow{2}{*}{result} &
\multirow{2}{*}{result} & \multirow{2}{*}{result} &
\multirow{2}{*}{result} \\
& & & & \\
\hline
\multirow{2}{*}{dt} & \multirow{2}{*}{result} &
\multirow{2}{*}{result} & \multirow{2}{*}{result} &
\multirow{2}{*}{result} \\
& & & & \\
\hline
\end{tabular}
\caption{Networked platoon model experiment based on Augmented complex zonotopes}
\end{table}



