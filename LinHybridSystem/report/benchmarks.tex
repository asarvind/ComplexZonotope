We performed experiments on three benchmark examples from literature
and compared the results with that obtained by the tool SpaceEx. [Add
  configuration here]

\subsection{Robot controller model with saturation}   We consider the benchmark
model of a self-balancing two wheeled robot called NXTway-GS1 by
Yorihisa Yamamoto, presented in the ARCH workshop~\cite{TODO}.  The
model is a networked control system, i.e. a plant interacting with a
controller.  The controller has a hole, which is an unknown input to
the controller.  The controlled input received by the plant has a
saturation limit and this saturation can be modeled in a hybrid
system.  Different models of the controller are presented like
continuous linear, sampled data (discrete time) linear and non-linear.
The plant can receive unsaturated controlled inputs or have saturation
limits on the controlled input.  In the latter case, the system can be
described as a hybrid system.

In our experiment, we consider both the discrete time affine hybrid
system model, i.e., with the saturation limits, as well as the
unsaturated discrete time affine system model.  The safety requirement
is that the \emph{body pitch angle} of the robot, denoted $\psi$,
should be bounded within some value.  For the affine system model
without saturation, a pair of reasonable bounds were specified as
$\psi\in\lt[-\frac{\pi}{2.26},\frac{\pi}{2.26}\rt]$.  Whereas, for the
affine hybrid system model, i.e. with saturation, the bounds were
specified as
$\psi\in\lt[-\frac{\pi}{2}-\epsilon,\frac{\pi}{2}+\epsilon\rt]$ such
that $\epsilon>0$.

The composite system with the plant and controller together could be
modeled using thirteen continuous state variables and four uncertain input
variables.  This model had unbounded trajectories in some
directions.  But it was possible to decouple some bounded
directions from the unbounded directions by an appropriate linear
transformation of the co-ordinates, such that the body pitch angle and the
controller inputs belong to the bounded directions.  So, we
experimented with the transformed model.  The latter model has ten
continuous state variables and four uncertain input variables.  We consider
the uncertain input as an additive disturbance input.

The controller input received by the plant is two dimensional, which
we denote by $u_1$ and $u_2$.  The saturation limit is $v_i=\delta
d_p$, where $\delta=100$ and $d_p=0.0807$.  The saturated input is
computed as $sat(u_i) = max\lt(-v_i,min\lt(u_i,v_i\rt)\rt)$.  So, the
two dimensional controller input can be divided into nine regions such
that the saturation function is piecewise affine with respect to these
regions.  We model this hybrid system using one location and nine self
edges with appropriate guards, such that all possible transitions
occur only along the edges.  So, we can set a constant value for the
intralocation affine map as the origin.  For the unsaturated model, we
have one location and no edges, where the system transition is given
by the intralocation affine map.
\vspace{0.2em}

\emph{Implementation.}  Using our approach and also the SpaceEx tool,
we tried to find the smallest possible safety bounds for the discrete
time affine unsaturated and hybrid (saturated) models, respectively.
In our approach, we choose the secondary template as the pseudoinverse
of the normal for the hyperplane guards in the hybrid model.  The
primary template is choosen as the collection of eigenvectors of all
affine maps for the edge transitions, orthonormal vectors to the
normals of the hyperplane guards and projections of the eigenvectors
on the subspace orthogonal to the normal of hyperplane guards.  For
the SpaceEx implementation, we tested with the octagon template and
also a uniformly sampled template with 400 directions.

\emph{Results.}  We could prove smaller safety bounds that what is
proposed in the paper.  In comparison, the SpaceEx tool exceeded the
proposed limits by a large margin in fourty iterations and could not
find a fix point.  The results of this experiment are reported in
Table~\ref{TODO}.

