We performed experiments on three benchmark examples from literature
and compared the results with that obtained by the tool SpaceEx. [Add
  configuration here]. [Add floating point error here]


\subsection{Robot with a saturated controller}   Our first example is a benchmark
model of a self-balancing two wheeled robot called NXTway-GS1 by
Yorihisa Yamamoto, presented in the ARCH
workshop~\cite{heinz2014benchmark}.  The model is a networked control
system, i.e. a plant interacting with a controller.  The controller
has a hole, which is an unknown input to the controller and is modeled
as an additive disturbance input.  The controller input received by
the plant has a saturation limit.  Due to the saturation, the
composite system is modeled as a hybrid system.  Three different
models of the controller are proposed in the benchmark: continuous
linear, sampled data (discrete time) linear and non-linear.  In our
experiment, we consider the sampled data linear controller, with two
kinds of interaction with the plant: saturated, i.e., hybrid system
and unsaturated, i.e., linear system.  The sampling time given in the
benchmark is $4 ms$.  The safety requirement is that the \emph{body
  pitch angle} of the robot, denoted $\psi$, should be bounded within
some value. In the benchmark, it was suggested that
$\psi\in\lt[-\frac{\pi}{2}+\epsilon,\frac{\pi}{2}-\epsilon\rt]$ for
any $\epsilon>0$, is given as a safe set.  For the linear system
model, $\psi\in\lt[\frac{-\pi}{2.26},~\frac{\pi}{2.26}\rt]$ is given
as a safe set.


In discrete time, the composite sampled data system of the plant and
controller could be modeled using thirteen continuous state variables
and four uncertain input variables.  The model, however, had unbounded
trajectories in some directions.  Therefore, we decoupled some bounded
directions from the unbounded directions by an appropriate linear
transformation of the co-ordinates, such that the body pitch angle and
the controller inputs belong to the bounded directions.  The latter
model has ten continuous state variables and four uncertain input
variables. The controller input received by the plant is two
dimensional, which we denote by $u_1$ and $u_2$, respectively.  The
saturation limit on $u_i$ is $v_i=\delta d_p$, where $\delta=100$ and
$d_p=0.0807$.  Then, the saturated input is computed as $sat(u_i) =
max\lt(-v_i,min\lt(u_i,v_i\rt)\rt)$.  Thus, the two dimensional
controller input can be divided into nine regions such that in each
region, the saturation function is affine.

\tbf{Modeling}.  We model the saturated system (after transformation)
as a ten dimensional hybrid system using one location and nine self
edges with appropriate guards, such that all possible transitions
occur only along the edges.  For the unsaturated model, we have one
location and no edges, and the only system transition is specified by
the intralocation affine map.  The initial set is the origin.

\tbf{Size of model}: 10 dimensional, 1 location and 9 edges.

\tbf{Implementation.}  For the hybrid system, we choose the secondary
template as the pseudoinverse of the guarding hyperplane normals, in
conformity with Theroem~\ref{thm:main}.  The primary template for the
hybrid system is chosen as the collection of the (complex) eigenvectors of
linear matrices of all affine maps for the edge transitions, the
orthonormal vectors to the guarding hyperplane normals and the
projections of the eigenvectors on the subspace spanned by the
orthonormal vectors.  For the linear system, we only have a primary
template, which is constituted by the eigenvectors of the linear map,
the input set template and its multiplication by the linear matrix
(related to affine map) and square of the linear matrix.  For the
SpaceEx implementation, we tested with the octagon template and a
template with 400 uniformly sampled support vectors.

%% For the hybrid system, we computed a single augmented complex
%% zonotopic invariant satisfying both the upper and lower safety bounds.
%% But for the linear system, we computed two different invariants, one
%% each satisfying the upper and lower bounds, respectively.

\tbf{Results.}  For both the hybrid and the linear systems, we could
verify smaller magnitudes for the bounds on the pitch angle than what
is proposed in the benchmark~\cite{heinz2014benchmark}.  But the
SpaceEx tool could not find a finite bound for either of the above
systems.  The results are reported in the
Tables~\ref{tab:robot-unsaturated} and~\ref{tab:robot-saturated}.

\tbf{Remarks.}  Although the unsaturated model is linear,
SpaceEx could not find an invariant with as many as 400 support
vectors.  In theory, although a linear system has a polytopic
invariant, but the number of faces of the polytope can be arbitrarily
large for a fixed dimension if the eigenvalues are complex.  The
robot model has some complex eigenvalues whose absolute values were
close to one.  On the other hand, since augmented complex zonotopes
can incorporate the complex eigenvectors in the template, we are
guaranteed to find an invariant for the linear (unsaturated) system.
Futhermore, we could also a safe invariant for the hybrid model of the
saturated system.

\begin{table}
\center{UB: $>$1000, ~~NT: Not terminating in more than 180s, \newline
  n/a: Not applicable/not available, ~~ACZ: Augmented complex
  zonotope.\vspace{1em} }
\begin{minipage}{0.48\textwidth}
\centering
\begin{tabular}{|l|c|c|c|}
\hline
\multicolumn{2}{|c|}{\multirow{2}{*}{Method}} &
\multirow{2}{*}{$\lt|\psi\rt|\leq$} & Comp.\\
\multicolumn{2}{|c|}{} & & time (s)\\
\hline
\multirow{4}{*}{SpaceEx} & octagon & \multirow{2}{*}{UB} & \multirow{2}{*}{NT}\\
& template & & \\
\cline{2-4}
& 400 support & \multirow{2}{*}{UB} & \multirow{2}{*}{NT}\\
& vectors & &\\
\hline
\multicolumn{2}{|c|}{\multirow{2}{*}{Suggested in~\cite{heinz2014benchmark}}} &
\multirow{2}{*}{$1.39$} & \multirow{2}{*}{n/a}\\
\multicolumn{2}{|c|}{} & &\\
\hline
\multicolumn{2}{|c|}{\multirow{2}{*}{ACZ invariant}} & \multirow{2}{*}{$1.29$} &
\multirow{2}{*}{$4$}\\
\multicolumn{2}{|c|}{} & & \\
\hline
\end{tabular}
\caption{Unsaturated robot model: results}
~\label{tab:robot-unsaturated}
\end{minipage}
\hspace{0em}
\begin{minipage}{0.48\textwidth}
\centering
\begin{tabular}{|l|c|c|c|}
\hline
\multicolumn{2}{|c|}{\multirow{2}{*}{Method}} &
\multirow{2}{*}{$\lt|\psi\rt|\leq$} & Comp.\\
\multicolumn{2}{|c|}{} & & time (s)\\
\hline
\multirow{4}{*}{SpaceEx} & octagon & \multirow{2}{*}{UB} &
\multirow{2}{*}{NT}\\
& template & & \\
\cline{2-4}
& 400 support & \multirow{2}{*}{UB} & \multirow{2}{*}{NT}\\
& vectors & & \\
\hline
\multicolumn{2}{|c|}{\multirow{2}{*}{Suggested in~\cite{heinz2014benchmark}}} &
$1.571-\epsilon:$ & \multirow{2}{*}{n/a}\\
\multicolumn{2}{|c|}{} & $\epsilon>0$ &\\
\hline
\multicolumn{2}{|c|}{\multirow{2}{*}{ACZ invariant}} & \multirow{2}{*}{$1.13$} &
\multirow{2}{*}{45}\\
\multicolumn{2}{|c|}{} & &\\
\hline
\end{tabular}
\caption{Saturated robot model: results}
~\label{tab:robot-saturated}
\end{minipage}
\begin{minipage}{0.45\textwidth}
\begin{tabular}{|l|c|c|c|c|}
\hline
\multicolumn{2}{|c|}{\multirow{2}{*}{Method}} &
\multirow{2}{*}{$\lt|x_1\rt|\leq$} & \multirow{2}{*}{$\lt|x_2\rt|\leq$} & Comp.\\
\multicolumn{2}{|c|}{} & & & time (s) \\
\hline
\multirow{4}{*}{SpaceEx} & octagon & \multirow{2}{*}{0.38} &
\multirow{2}{*}{0.43} & \multirow{2}{*}{1.7}\\
& template & & &\\
\cline{2-5}
& 100 support & \multirow{2}{*}{0.38} & \multirow{2}{*}{0.43} & \multirow{2}{*}{23.6}\\
& vectors & & &\\
\hline
\multicolumn{2}{|c|}{\multirow{2}{*}{ACZ invariant}} &
\multirow{2}{*}{0.38} & \multirow{2}{*}{0.36} & 
\multirow{2}{*}{5.1}\\
\multicolumn{2}{|c|}{} & & &\\
\hline
\end{tabular}
\caption{Small invariant computation:\newline Perturbed double
  integrator}
~\label{tab:smallinv-pdi}
\vspace{1em}
\end{minipage}
\hspace{4em}
\begin{minipage}{0.4\textwidth}
\begin{tabular}{|c|c|}
\hline
\multirow{2}{*}{Method} & Comp.\\
& time (s)\\
\hline
\multirow{2}{*}{MPT tool~\cite{rakovic2004computation}} & \multirow{2}{*}{107}\\
& \\
\hline
\multirow{2}{*}{ACZ} & \multirow{2}{*}{12}\\
& \\
\hline
\end{tabular}
\caption{Large invariant computation: Perturbed double integrator}
~\label{tab:largeinv-pdi}
\vspace{1em}
\end{minipage}
%
\begin{tabular}{|l|c|c|c|c|c|}
\hline
\multicolumn{2}{|c|}{\multirow{2}{*}{Method}} &
\multirow{2}{*}{$-e_1\leq$} & \multirow{2}{*}{$-e_2\leq$} & \multirow{2}{*}{$-e_3\leq$} & Comp.\\
\multicolumn{2}{|c|}{} & & & & time (s)\\
\hline
\multirow{4}{*}{SpaceEx} & octagon & \multirow{2}{*}{28} &
\multirow{2}{*}{27} & \multirow{2}{*}{10} &
\multirow{2}{*}{NT}\\
& template & & & & \\
\cline{2-6}
& 100 support & \multirow{2}{*}{28} & \multirow{2}{*}{25} &
\multirow{2}{*}{13} & \multirow{2}{*}{1.3}\\
& vectors & & & & \\
\hline
\multicolumn{2}{|c|}{\multirow{2}{*}{Real zonotope~\cite{makhlouf2014networked}}} &
\multirow{2}{*}{25} & \multirow{2}{*}{25} & \multirow{2}{*}{10}
 & \multirow{2}{*}{n/a}\\
\multicolumn{2}{|c|}{} & & & & \\
\hline
\multicolumn{2}{|c|}{\multirow{2}{*}{ACZ invariant}} &
\multirow{2}{*}{28} & \multirow{2}{*}{26} &
\multirow{2}{*}{12} & \multirow{2}{*}{12}\\
\multicolumn{2}{|c|}{} & & & &\\
\hline
\end{tabular}
\caption{Large minimum dwell time (20s) model of networked
  vehicle platoon: results}
~\label{tab:largedwell-platoon}
 $~$\\
\begin{tabular}{|l|c|c|c|c|c|}
\hline
\multicolumn{2}{|c|}{\multirow{2}{*}{Method}} &
\multirow{2}{*}{$-e_1\leq$} & \multirow{2}{*}{$-e_2\leq$} & \multirow{2}{*}{$-e_3\leq$} & Comp.\\
\multicolumn{2}{|c|}{} & & & & time (s)\\
\hline
\multirow{4}{*}{SpaceEx} & octagon & \multirow{2}{*}{UB} &
\multirow{2}{*}{UB} & \multirow{2}{*}{UB} &
\multirow{2}{*}{NT}\\
& template & & & & \\
\cline{2-6}
& 100 support & \multirow{2}{*}{UB} & \multirow{2}{*}{UB} &
\multirow{2}{*}{UB} & \multirow{2}{*}{NT}\\
& vectors & & & & \\
\hline
\multicolumn{2}{|c|}{\multirow{2}{*}{ACZ invariant}} &
\multirow{2}{*}{46} & \multirow{2}{*}{54} &
\multirow{2}{*}{57} & \multirow{2}{*}{12.6}\\
\multicolumn{2}{|c|}{} & & & &\\
\hline
\end{tabular}
\caption{Small minimum dwell time (1s) model of networked vehicle
  platoon: results}
~\label{tab:smalldwell-platoon}
\end{table}
%
\subsection{Perturbed double integrator}
Our second example is a perturbed double integrator system that is
described in~\cite{rakovic2004computation}.  The closed loop system
with a feedback control is piecewise affine, having four different
affine dynamics in four different regions of space.  The system is two
dimensional and has a bounded additive disturbance input.  We perform
two different experiments on this system.  In the first experiment, we
try to verify the smallest possible magnitude of bounds on the two
coordinates, denoted $x_1$ and $x_2$. We compare these bounds with
that found by the SpaceEx tool.

%% \begin{equation}~\label{eqn:pwa-regions}
%% \trj{x}{t+1}=\lt(A_i+B_iK_i\rt)\trj{x}{t}+w,~\text{where}~
%% i=\left\{\begin{array}{l}
%% 1,~\text{if}~x_1\geq 0~\text{and}~x_2\geq 0\\
%% 2,~\text{if}~x_1\leq 0~\text{and}~x_2\leq 0\\
%% 3,~\text{if}~x_1\leq 0~\text{and}~x_2\geq 0\\
%% 4,~\text{if}~x_1\geq 0~\text{and}~x_2\leq 0\\
%% \end{array} \rt.
%% \end{equation}
%% %
%% \begin{align*}
%% & A_1 =\lt[\begin{array}{ll}
%% 1 & 1\\
%% 0 & 1
%% \end{array}\rt],~B_1 = \lt[\Calign{1}{0.5}\rt],~K_1 = \lt[-0.5897~
%%   -0.9347\rt]\\
%% & A_2 = \lt[\begin{array}{ll}
%% 1 & 1\\
%% 1 & 0
%% \end{array}
%% \rt],~B_2 = \lt[\Calign{-1}{-0.5}\rt],~K_2 = \lt[0.5897~~0.9387\rt]\\
%% & A_3 = \lt[\begin{array}{ll}
%% 1 & -1\\
%% 0 & 1
%% \end{array}
%% \rt],~B_3 = \lt[\Calign{-1}{0.5}\rt],~K_3 = \lt[0.5897~-0.9387\rt]\\
%% & A_4 = \lt[\begin{array}{ll}
%% 1 & -1\\
%% 0 & 1
%% \end{array}
%% \rt],~B_4 = \lt[\Calign{1}{-0.5}\rt],~K_4 = \lt[-0.5897~~0.9387\rt].\\
%% \end{align*}
%% %  
%% The additive disturbance input $w$ is bounded as $\|w\|_{\infty}\leq
%% 0.2$.  

In the second experiment, we try to quickly compute a large invariant
for the system under the safety constraints given
in~\cite{rakovic2004computation}.  %% The
%% given safety
%% constraints are $\|x\|_{\infty}\leq 5$ and $\lt|K_i(x)\leq
%% 1\rt|~\forall i\in\lt\{1,2\rt\}$.
In the latter case, we maximize the sum of the scaling factors and
differences of the upper and lower interval bounds of the augmented
complex zonotopic invaraint.  Furthermore, we decompose the given
safety constraints as the intersection of four different sets of
safety constraints.  For each set of safety constraints, we compute a
large augmented complex zonotopic invariant.  Then the desired
invariant is the intersection of four augmented complex zonotopic
invaraints.  Although we may not find the largest possible (maximal)
invariant by this approach, still the optimizer would find a large
invaraint.  We draw comparison in terms of the computation time with
the reported result for the MPT tool~\cite{rakovic2004computation}.

\tbf{Modeling.}  In our formalism, we model the system as two
dimensional with four locations and twelve edges.  Appropriate staying
conditions are specified in each location, reflecting the division of
the state space into different regions where the dynamics is affine.
The edges constitute all possible interlocation transitions.  The
initial set is the origin.  The same model is specified in SpaceEx.

\tbf{Size of model}: 2 dimensional, 4 locations and 12 edges.

\tbf{Implementation}.  We choose the secondary template in each
location as the pseudoinverse (in this case equal to) the hyperplane
normals of the staying conditions in that location, so that
Theorem~\ref{thm:main} is applicable.  For the primary template, we
collected the (complex) eigenvectors of all linear matrices of the
affine maps and their binary products. For the SpaceEx tool, we
experimented with two different templates, the octagon template and a
template with 100 uniformly sampled support vectors.

\tbf{Results.}  In the first experiment on this example, the bounds
verified bounds for the first coordinate by our method is equal to
that of SpaceEx. But for the second coordinate, we verfied smaller
bounds than that of SpaceEx.  In our second experiment on this
example, the computation time for finding a large invariant by our
method is significantly smaller than that of the reported result for
the MPT tool.  The results are summarized in the Tables~\ref{tab:smallinv-pdi}
and~\ref{tab:largeinv-pdi}.

\tbf{Remark.}  The overapproximation quality of SpaceEx reduces when
there are large number of possible transitions, because then a union
of many different polytopes has to be overapproximated using support
vectors.  In contrast, our approach uses optimization to learn an
appropriate invariant, avoiding the union operations.  Possibly
because of this reason, we could verify smaller bounds than SpaceEx,
for the second coordinate.


\subsection{Networked platoon of vehicles}
Our third example is a model of a networked cooperative platoon of
vehicles, which is presented as a benchmark in the ARCH
workshop~\cite{makhlouf2014networked}.  The platoon consists of three
vehicles $M_1$, $M_2$ and $M_3$ along with a leader board ahead.  Each
vehicle has a reference distance to the vehicle ahead of it.  The
difference between the actual distance of a vehicle $M_i$ to its
successor and the reference distance is denoted as $e_i$.  Any upper
bound on $-e_i$ is a safe lower limit on the reference distance above
which the platoon is guaranteed not to collide.

The movement of the vehicles is dependent on the communication between
them.  In the benchmark proposal, the dynamics of the vechicles is
described as a hybrid system with two locations having different
dynamics.  In one location, there is communication between all the
vehicles, while in another location, there is complete communication
failure.  In the general model described in the benchmark, there can
be staying conditions for each location and time constraints on the
switching time.  The benchmark then considers a specific case where
the minimum dwell time is 20 seconds (also specified in the
distributed SpaceEx
implementation\footnote{http://cps-vo.org/node/15096}).  In this
paper, we consider two cases, minimum dwell times of 20 seconds and 1
second, respectively.


\tbf{Modeling.}  Since our method is applicable to discrete time
hybrid systems, we need to find a discrete time system whose reachable
set overapproximates that of the given continuous time system.  For
the large minimum dwell time of $20s$, it is possible efficiently
discretize the system, satisfying the aforementioned requirement.  We
do not, however, explain the discretization procedure here, because it
is beyond the scope of this paper.  But for the small minimum dwell
time of $1s$, a similar discretization of the system would lead to a
very high complexity model.  So, for our experiment in the latter
case, we alternatively considered a model in which the switching can
only happen at integer instants of time.  With this assumption,
efficient discretization was possible for the $1s$ minimum dwell
model.  Henceforth, we model the above two systems in the formalism
described in this paper. The same models are also used for the
discrete time SpaceEx implementation.  The size of both the models is
given below.

\tbf{Size of large dwell time model}: 9 dimensional, 2
locations and 4 edges.

\tbf{Size of small dwell time model}: 9 dimensional, 2
locations, 2 edges.

\tbf{Implementation.}  In our approach, we choose the primary template
as the collection of the (complex) eigenvectors of linear matrices of
the affine maps in the the two locations and their binary products,
the axis alligned box template and the templates used for
overapproximating the input sets.  The secondary template is set to
the zero vector since there are no linear guards or staying conditions
in this example.  For the SpaceEx tool, we experimented with two
templates, octagon and hundred uniformly sampled support vectors.

\tbf{Results.}  For the large minimum dwell time of $20s$, the
discrete time SpaceEx implementation and also a method based on using
real zonotopes~\cite{makhlouf2014networked} could verify slightly
smaller bounds on $-e_1$, $-e_2$ and $-e_3$, compared to our approach.
But for the small minimum dwell time ($1s$) model, SpaceEx could not
even find a finite set of bounds, whereas our approach could verify a
finite set of bounds.  These results are reported in the
Tables~\ref{tab:largedwell-platoon} and~\ref{tab:smalldwell-platoon}.

\tbf{Remarks.}  The dynamics in both locations being stable, the
lyapunov exponent (measure of stability) tends to be higher for larger
dwell time.  Threfore, SpaceEx and the zonotope based approach could
perform somewhat better than our approach when the minimum dwell time
is 20s.  But when the minimum dwell time is 1s, SpaceEx failed
possibly because the lyapunov exponent is smaller in this case.
However, our approach found an invariant for the smaller minimum dwell
time case as well.  Besides, we note that the zonotope based approach
can be difficult to use when the dwell time is smaller because then a
union of a large number of zonotopes has to be stored or
overapproximated in each step.  In contrast, our approach avoids the
union operations and instead formulates an optimization program.


