We performed experiments on three benchmark examples from literature
and compared the results with that obtained by the tool SpaceEx. [Add
  configuration here]. [Add floating point error here]

\subsection{Robot controller model with saturation}   We consider the benchmark
model of a self-balancing two wheeled robot called NXTway-GS1 by
Yorihisa Yamamoto, presented in the ARCH workshop~\cite{TODO}.  The
model is a networked control system, i.e. a plant interacting with a
controller.  The controller has a hole, which is an unknown input to
the controller.  The controlled input received by the plant has a
saturation limit and this saturation can be modeled in a hybrid
system.  Different models of the controller are presented like
continuous linear, sampled data (discrete time) linear and non-linear.
The plant can receive unsaturated controlled inputs or have saturation
limits on the controlled input.  In the latter case, the system can be
described as a hybrid system.

In our experiment, we consider both the discrete time affine hybrid
system model, i.e., with the saturation limits, as well as the
unsaturated discrete time affine system model.  The safety requirement
is that the \emph{body pitch angle} of the robot, denoted $\psi$,
should be bounded within some value.  For the affine system model
without saturation, a pair of reasonable bounds were specified as
$\psi\in\lt[-\frac{\pi}{2.26},\frac{\pi}{2.26}\rt]$.  Whereas, for the
affine hybrid system model, i.e. with saturation, the bounds were
specified as
$\psi\in\lt[-\frac{\pi}{2}-\epsilon,\frac{\pi}{2}+\epsilon\rt]$ such
that $\epsilon>0$.

The composite system with the plant and controller together could be
modeled using thirteen continuous state variables and four uncertain
input variables.  This model had unbounded trajectories in some
directions.  But it was possible to decouple some bounded directions
from the unbounded directions by an appropriate linear transformation
of the co-ordinates, such that the body pitch angle and the controller
inputs belong to the bounded directions.  So, we experimented with the
transformed model.  The latter model has ten continuous state
variables and four uncertain input variables.  We consider the
uncertain input as an additive disturbance input.  The controller
input received by the plant is two dimensional, which we denote by
$u_1$ and $u_2$, respectively.  The saturation limit on $u_i$ is
$v_i=\delta d_p$, where $\delta=100$ and $d_p=0.0807$.  The saturated
input is computed as $sat(u_i) = max\lt(-v_i,min\lt(u_i,v_i\rt)\rt)$.
So, the two dimensional controller input can be divided into nine
regions such that the saturation function is piecewise affine with
respect to these regions.

\emph{Modeling}.  We model the hybrid system using one location and
nine self edges with appropriate guards, such that all possible
transitions occur only along the edges.  For the unsaturated model, we
have one location and no edges, where the system transition is given
by the intralocation affine map.  The initial set is the origin.

\emph{Implementation.}  We tried to find the smallest possible
magnitude of the safety bounds based on our approach and the SpaceEx
tool.  In our approach, we choose the secondary template as the
pseudoinverse of the normal for the hyperplane guards in the hybrid
model.  The primary template is choosen as the collection of (complex)
eigenvectors of all affine maps for the edge transitions, orthonormal
vectors to the normals of the hyperplane guards and projections of the
eigenvectors on the subspace orthogonal to the normal of hyperplane
guards.  For the SpaceEx implementation, we tested with the octagon
template and also a uniformly sampled template with 400 directions.

\emph{Results.}  We could prove smaller magnitude of safety bounds
than what is proposed in the paper.  In comparison, the SpaceEx tool
exceeded the proposed limits by a large margin in fourty iterations
and could not find a fix point.  These results are reported in
Table~\ref{TODO}.  The computation times are reported in
Table~\ref{TODO}.

\subsection{Perturbed double integrator}
We consider the model of a perturbed double integrator, given
in~\cite{TODO}.  The closed loop system with feedback control is piecewise affine described as
\[\trj{x}{t+1}=\lt(A_i+B_iK_i\rt)\trj{x}{t}+w,~\text{where}~
i=\left\{\begin{array}{l}
1,~\text{if}~x_1\geq 0~\text{and}~x_2\geq 0\\
2,~\text{if}~x_1\leq 0~\text{and}~x_2\leq 0\\
3,~\text{if}~x_1\leq 0~\text{and}~x_2\geq 0\\
4,~\text{if}~x_1\geq 0~\text{and}~x_2\leq 0\\
\end{array} \rt.\]
%
\begin{align*}
& A_1 =\lt[\begin{array}{ll}
1 & 1\\
0 & 1
\end{array}\rt],~B_1 = \lt[\Calign{1}{0.5}\rt],~K_1 = \lt[-0.5897~
  -0.9347\rt]\\
& A_2 = \lt[\begin{array}{ll}
1 & 1\\
1 & 0
\end{array}
\rt],~B_2 = \lt[\Calign{-1}{-0.5}\rt],~K_2 = \lt[0.5897~~0.9387\rt]\\
& A_3 = \lt[\begin{array}{ll}
1 & -1\\
0 & 1
\end{array}
\rt],~B_3 = \lt[\Calign{-1}{0.5}\rt],~K_3 = \lt[0.5897~-0.9387\rt]\\
& A_4 = \lt[\begin{array}{ll}
1 & -1\\
0 & 1
\end{array}
\rt],~B_4 = \lt[\Calign{1}{-0.5}\rt],~K_4 = \lt[-0.5897~~0.9387\rt].\\
\end{align*}
%  
The additive disturbance input $w$ is bounded as $\|w\|_{\infty}\leq
0.2$.  

\emph{Modeling.}  In our formalism, the system is modeled by four
locations with the staying conditions corresponding to the divisions
of the state space described above.  Every location is connected to
every other location by an edge with a corresponding affine map.  In
this model, the linear constraints on transitions are comprised of the
staying conditions in the locations.  The same is modeled in SpaceEx.
The initial set is the origin.

\emph{Implementation}.  We choose the secondary template as
$\lt[\begin{array}{ll}1 & 0\\0 & 1\end{array}\rt]$, whose columns are
the pseudoinverse (in this case equal to) the normals of the
hyperplane guards.  For the primary template, we collected the
eigenvectors of the affine maps as well as eigenvectors of the binary
products of affine maps in different locations.  For the SpaceEx tool,
we choose the octagon template.

\emph{Results.}  Firstly, using our approach and also SpaceEx, we
tried to find the smallest magnitude of bounds on the two continuous
state variables.  Our method found smaller magnitude of bounds than
that of SpaceEx for the second coordinate, while these bounds were
equal to that of SpaceEx for the first coordinate.  The results are
reported in Table~\ref{TODO}.  The computation time for our method is
less than TODO seconds on this example.

Next, we draw comparison with the reported results of the MPT
tool~\cite{TODO}.  The method of~\cite{TODO} computed the maximal
invariant for the system under the safety constraints
$\|x\|_{\inf}\leq 5$ and $\lt|K_i(x)\leq 1\rt|~\forall i\in\tup{2}$.
Our method may not be able to find the maximal invariant.  Still, we
can find a large invariant by maximizing an objective function defined
as the sum of scaling factors and differences between the upper and
lower interval bounds of the augmented complex zonotope.  The
maximization of this objective function is done by second order conic
optimization in the CVX tool.  In our implementation, we first
decompose the above safety constraints as an intersection of four
different safety constraints.  The actual invariant is then computed
as an intersection of invaraints corresponding to the four different
safety constraints.  For each individual safety constraint, our method
takes less than 3 seconds.  So, the total computation time is less
than $4\times 3$ seconds $=12$ seconds.  In comparison, the MPT tool
took 107 seconds.  This is reported in Table~\ref{TODO}.





