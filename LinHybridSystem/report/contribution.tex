

For discrete time affine hybrid systems, the eigenvectors of the
products of linear matrices related to the affine dynamics of
different subsystems can possibly capture some of the stable
directions for the overall hybrid dynamics.  As such, for invariant
computation, template complex zonotopes have the advantage that they
can include the possibly complex eigenvectors among the generators,
while usual (real) zonotopes can not.  In an earlier
work~\cite{tcz2017}, numerically efficiently solvable conditions for
computing a template complex zonotopic invariant subject to linear
safety constraints were obtained for a limited class of hybrid
systems, i.e., having uncontrolled switching.  However, a formidable
hurdle in extending the approach for more general affine hybrid
systems, where switching is controlled by linear constraints, is that
we have to handle the intersection of template complex zonotopes with
the linear constraints.  In this regard, template complex zonotopes
share the drawback of usual zonotopes that these classes of sets are
not closed under intersection with linear constraints.

In this paper, we circumvent this problem as follows.  We observe that
it is possible to compute or reasonably overapproximate the
intersection of a template complex zonotope with a class of linear
constraints, called subparallelotpic, by appropriately choosing the
template of the complex zonotope.  We use a slightly more general set
representation, called augmented complex zonotope, with which the
intersection operation can be succinctly presented.  Geometrically
speaking, augmented complex zonotopes and template complex zonotopes
describe the same classes of sets in terms of their real valued
projections.  However, it is easier and more succinct, using the
representation of an augmented complex zonotope instead of a template
complex zonotope, to represent the resultant intersection with linear
constraints.  Then, we derive a numerically efficiently solvable
sufficient condition for computing an augmented complex zonotopic
invaraint satisfying linear safety constraints, for a discrete time
affine hybrid system with subparallelotopic switching constraints and
bounded additive disturbance input.  The sufficient condition is
expressed as a set of second order conic constraints.  We also note
that the class of sub-parallelotopic constraints that we consider are
quite general and can be used in the specification of many examples
affine hybrid systems.  We implemented our approach on three benchmark
examples from literature and compared the results with that of the
SpaceEx tool on the same discrete time models, and also the reported
benchmark results from literature.  In some experiments, we could
verify finite safety bounds when SpaceEx could not even find an
invariant. In other experiments, we could verify competitive safety
bounds.  Also, our computation time is quite reasonable in all the
experiments, depending on the size of the specification.

The rest of the paper is organized as follows.  Firstly, we explain
some of the mathematical notation used in this paper.  Then in
Section~\ref{sec:system}, we describe the model of a discrete time
affine hybrid system, controlled by sub-parallelotopic switching
conditions and having a bounded additive disturbance input.  In
Section~\ref{sec:review}, we review some existing set representations
before presenting augmented complex zonotopes.  In
Section~\ref{sec:acz}, we present the set representaion of augmented
complex zonotopes and discuss some important operations and relations
like intersection with sub-parallelotopic constraints, projection in
any direction, linear transformation, Minkowski sum and inclusion.  In
Section~\ref{sec:invcomp}, we derive a set of second order conic
constraints to compute an augmented complex zonotopic invariant,
satisfying linear safety constraints and containing an initial set.
Furthermore, we explain how to choose the template.  In
Section~\ref{sec:exp}, we discuss the experimental results.  The
conclusion and future work are given in
Section~\ref{sec:conclusion}.  We annex the proofs of the lemmas
presented in the paper as an Appendix.

{\bf Notation.} Some of the notations used in this paper, for which we
consider explanation may be required, is described below.  If $S$ is a
set of complex numbers, then $real(S)$ represents the real projection
of $S$.  If $z$ is a complex number, then $|z|$ denotes the absolute
value of $z$.  On the other hand, if $X$ is a complex matrix, then $\lt|X\rt|$ denotes
the matrix containing the absolute values of the elements of $X$.  The
diagonal square matrix containing the entries of a complex vector $z$
along the diagonal is denoted by $\dg(z)$.  Let
$K\in\mat{k}{n}{\realset}$ such that $k\leq n$ and $KK^T$ is
non-singular.  Then, we denote $\pinv{K}=K^T\lt(KK^T\rt)^{-1}$,
which is the pseudo-inverse of $K$.
