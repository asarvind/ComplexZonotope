

For discrete time affine hybrid systems, the eigenvectors of the
products of linear matrices related to the affine dynamics of
different subsystems can possibly capture some of the stable
directions for the overall hybrid dynamics.  As such, for invariant
computation, template complex zonotopes have the advantage that they
can include the possibly complex eigenvectors among the generators,
while usual (real) zonotopes can not.  In an earlier
work~\cite{tcz2017}, numerically efficiently solvable conditions for
computing a template complex zonotopic invariant subject to linear
safety constraints were obtained for a limited class of hybrid
systems, i.e., having uncontrolled switching.  However, a formidable
hurdle in extending the approach for more general affine hybrid
systems, where switching is controlled by linear constraints, is that
we have to handle the intersection of template complex zonotopes with
the linear constraints.  In this regard, template complex zonotopes
share the drawback of usual zonotopes that these classes of sets are
not closed under intersection with linear constraints.

In this paper, we circumvent this problem as follows.  We observe that
it is possible to compute or reasonably overapproximate the
intersection of a template complex zonotope with a class of linear
constraints, called subparallelotpic, by appropriately choosing the
template of the complex zonotope.  We use a slightly more general set
representation, called augmented complex zonotope, with which the
intersection operation can be succinctly presented.  %% Geometrically
%% speaking, augmented complex zonotopes and template complex zonotopes
%% describe the same classes of sets in terms of their real valued
%% projections.  However, it is easier and more succinct, using the
%% representation of an augmented complex zonotope instead of a template
%% complex zonotope, to represent the resultant intersection with linear
%% constraints. 
Then, we derive a numerically efficiently solvable
sufficient condition for computing an augmented complex zonotopic
invaraint satisfying linear safety constraints, for a discrete time
affine hybrid system with subparallelotopic switching constraints and
bounded additive disturbance input.  The sufficient condition is
expressed as a set of second order conic constraints.  We also note
that the class of sub-parallelotopic constraints that we consider are
quite general and can be used in the specification of many examples of 
affine hybrid systems.  To corroborate our approach by presenting the
experimental results for three benchmark examples from literature.
%% We implemented our approach on three benchmark examples from
%% literature and compared the results with that of the SpaceEx tool
%% on the same discrete time models, and also the reported benchmark
%% results from literature.  In some experiments, we could verify
%% finite safety bounds when SpaceEx could not even find an
%% invariant. In other experiments, we could verify competitive safety
%% bounds.  Also, our computation time is quite reasonable in all the
%% experiments, depending on the size of the specification.

