Zonotopes have the advantage that they accurately capture matrix
multiplication and linear transformation operations, but they are used
mainly for bounded time reachability computation.  For approximating
unbounded time reachable sets of arbitrarily switched affine hybrid
systems based on positive invariants, template complex zonotopes were
introduced in~\cite{tcz2017}, which have the following useful
property.  Any template complex zonotope generated by the eigenvectors
of a Schur stable linear transformation is positively invariant with
respect to the transformation.  Therefore, template complex zonotopes
can exploit the possibly complex eigenstructure of the system dynamics
for computing invariants, while real zonotopes can not.  However, a
formidable hurdle using them for invarariant computation of more
general affine hybrid systems, where switching is state-dependent and
controlled by linear constraints, is that we have to handle the
intersection of template complex zonotopes with the guard sets that
trigger switching. In this regard, template complex zonotopes share
the drawback of usual zonotopes that these classes of sets are not
closed under intersection with linear constraints. In this paper, we
address this problem as follows. We use a slightly more general set
representation, called \emph{augmented complex zonotope}, based on
which we propose an algebraic overapproximation of the intersection
with a class of linear constraints, called sub-parallelotopic.
Henceforth, we derive a numerically efficiently solvable sufficient
condition for computing an augmented complex zonotopic invariant
satisfying linear safety constraints, for a discrete-time affine
hybrid system with subparallelotopic switching constraints and bounded
additive disturbance input.  The sufficient condition is expressed as
a set of second order conic constraints.  We also note that the class
of subparallelotopic constraints that we consider are quite general
and can be used in the specification of many practical affine hybrid
systems. We corroborate our approach by presenting the experimental
results for three benchmark examples from the literature.


