

For discrete time affine hybrid systems, the eigenvectors of the
products of linear matrices related to the affine dynamics of
different subsystems, can possibly capture some of the stable
directions for the overall hybrid dynamics.  As such, for invariant
computation, template complex zonotopes~\cite{tcz2017} have the
advantage that they can include the possibly complex eigenvectors
among the generators, while usual (real-valued) zonotopes can not. In
an earlier work~\cite{tcz2017}, to compute a template complex
zonotopic invariant subject to linear safety constraints, numerically
efficiently solvable conditions were obtained for a class of affine
hybrid systems having uncontrolled switching. However, a formidable
hurdle in extending the approach for more general affine hybrid
systems, where switching is state-dependent and controlled by linear
constraints, is that we have to handle the intersection of template
complex zonotopes with the guard sets that trigger switching. In this
regard, template complex zonotopes share the drawback of usual
zonotopes that these classes of sets are not closed under
such intersection. In this paper, we circumvent this problem as follows. We
use a slightly more general set representation, called \emph{augmented
complex zonotope}, based on which we propose an affine
overapproximation of the intersection with a class of linear
constraints, called sub-parallelotopic.  Henceforth, we derive a
numerically efficiently solvable sufficient condition for computing an
augmented complex zonotopic invariant satisfying linear safety
constraints, for a discrete-time affine hybrid system with
subparallelotopic switching constraints and bounded additive
disturbance input.  The sufficient condition is expressed as a set of
second order conic constraints.  We also note that the class of
subparallelotopic constraints that we consider are quite general and
can be used in the specification of many practical affine hybrid
systems. We corroborate our approach by presenting the experimental
results for three benchmark examples from the literature.\\


