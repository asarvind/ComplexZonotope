In the previous section, we derived a sufficient condition for an
augmented complex zonotope to be a positive invariant with some
additional requirements.  In this section, we will discuss how to
compute the positive invaraint based on the sufficient condition.  Our
computational procedure involves two steps.  In the first step, we
choose suitable primary and secondary templates for the augmented
complex zonotope.  In the next step, we solve for the other variables
that satisfy the sufficient condition derived in
Theorem~\ref{thm:main}.

\paragraph{Solving for the variables.}  The sufficient condition we
derived for the required positive invariant is equivalent to the
satisfaction of a set of convex constraints.  By a convex constraint,
we mean a constraint of the form $f(x)\leq 0$ where $f$ is a convex
function.  We formalize our observation as follows.
%
\begin{proposition}~\label{prop:convex-inclusion}
Both the below statements are true.
\begin{enumerate}
\item For fixed templates $V\in\mat{n}{m}{\mb{C}}$ and
$V^\pr\in\mat{n}{m^\pr}{\mb{C}}$, the partial order
$\cz{V}{c}{s}\order\cz{V^\pr}{c^\pr}{s^\pr}$ is equivalent to a set of
convex constraints on $c,s,c^\pr,s^\pr$ and some auxillary variables.
\item For fixed templates $V\in\mat{n}{m}{\mb{C}}$,
$W\in\mat{n}{k}{\realset}$, 
$V^\pr\in\mat{n}{m^\pr}{\mb{C}}$ and
$W^\pr\in\mat{n}{k^\pr}{\realset}$, the partial order
$\gcz{V}{c}{s}{W}{l}{u}\order\gcz{V^\pr}{c^\pr}{s^\pr}{W^\pr}{l^\pr}{u^\pr}$
is equivalent to a set of convex constraints on
$c,s,l,u,c^\pr,s^\pr,l^\pr,u^\pr$ and some auxillary variables.
\end{enumerate}
\end{proposition}
%
If a variable is assigned to an affine function of another variable,
then a convex constraint on the former variable is equivalent to
another convex constraint on the latter variable.  We have stated in
Lemma~\ref{lem:min-max-approx} that the min and max approximation
functions are affine functions in the first argument, for a fixed
second argument.  These affine approximations are used in
Equations~(\ref{eqn:locinv}) and~(\ref{eqn:edgeinv}).  Therefore, we
deduce the following observation about the former equations.
%
\begin{proposition}
Both the below statements are true.
\begin{enumerate}
\item Consider the Equation~(\ref{eqn:locinv}).  For any location
$\loc\in\locationset$, given fixed templates $V(\loc)$ and $W(\loc)$,
the Equation~(\ref{eqn:locinv}) is equivalent to a set of convex
constraints on the variables $c(\loc),s(\loc),l(\loc),u(\loc)$ and
some auxillary variables.
\item Consider the Equation~(\ref{eqn:edgeinv}).  For any edge
$\edge\in\edgeset$, given fixed templates
$V(\preloc{\edge})$, $W(\preloc{\edge}),V(\postloc{\edge})$ and
$W(\postloc{\edge})$, the Equation~(\ref{eqn:edgeinv}) is equivalent
to a set of convex constraint on the variables
$c(\preloc{\edge}),s(\preloc{\edge}),l(\preloc{\edge}),u(\preloc{\edge}),
c(\postloc{\edge}),s(\postloc{\edge}),l(\postloc{\edge}),u(\postloc{\edge})$
and some auxillary variables.
\end{enumerate}
\end{proposition}
%
