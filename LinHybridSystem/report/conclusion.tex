We introduced augmented complex zonotopes as a more general set
representation than template complex zonotopes, based on which we
derived efficiently solvable conditions for computing invariants,
subject to polytopic safety requirements, discrete time affine hybrid
systems with linear guards and additive disturbance input.  Like
template complex zonotopes, augmented complex zonotopes have the
advantage that we can meaningfully choose the templates based on the
eigenstructure and and other relevant aspects of the dynamics, for
efficient fixpoint computation.  Furthermore, we overcame a drawback
of template complex zonotopes in that we derived a reasonable
overapproximation of the intersection with a class of linear
constraints, which is expressed as a simple algebraic expression.  We
use this expression in the derivation of a set of second order conic
constraints that can be efficiently solved to compute an invariant.
In contrast to the step-by-step reachability computation approaches
that iteratively accumulate overapproximation error, we
instead compute an invariant in a single convex optimization step such
that the optimizer inherently minimizes the overapproximation error in
the single computation step.  We demonstrated the efficiency of our
approach on some benchmark examples.

As future work, we can investigate ways to minimize the
overapproximation error in the intersection operation, such that the
overapproximation can still be algebrically computed.  In particular,
the relation between the choice of the template and the
over-approximation error in intersection has to be analyzed.  Also, we
would like to the extend this computational framework to continuous
time hybrid systems.
