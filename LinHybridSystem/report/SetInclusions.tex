The real projection of an augmented complex zonotope can be
equivalently tranformed as the real projection of a template complex
zonotope, as follows.  However, we note that an augmented complex
zonotope, as a complex valued set, is more general and can not be represented as a
template complex zonotope.
%
\begin{lemma}~\label{lem:conversion}
$\real\lt(\gcz{V}{c}{s}{W}{l}{u}\rt) = \real\lt(\cz{\lt[V~W\rt]}{c+W\lt(\frac{u+l}{2}\rt)}{\ColumnJoin{\hspace{0.5em}s}{\frac{u-l}{2}}}\rt)$.
\end{lemma}
%
Because of the above relationship, checking the inclusion between real
projections of augmented complex zonotopes amounts to checking
inclusion between real projections of template complex zonotopes.
Therefore, we first discuss the inclusion relationship between
template complex zonotopes.

In general, checking the exact inclusion between two template complex
zonotopes amounts to solving a non-convex optimization problem, which
could be inefficient.  Therefore, a convex condition was proposed
in~[CITE], which is sufficient to guarantee inclusion between template
complex zonotopes.  Here, we present this condition as a
partial order between template complex zonotopes.

%
\begin{definition}
We define a relation ``$\order$'' between template complex zonotopes
as\\ $\cz{V^\pr_{n\times m^\pr}}{c^\pr}{s^\pr}\order \cz{V_{n\times
    m}}{c}{s}$ if all of the following is true.
\begin{align}
\begin{split}
& \exists X\in\mat{m}{m^\pr}{\mb{C}}~\text{and}~y\in\mb{C}^{m}~\text{s.t.}\\
& \transfer{V}{V^\pr}{s^\pr}{X},~~~\centertransfer{V}{c}{c^\pr}{y}\\
& \scalebound{X}{y}{s}{m}{m^\pr}\leq 0\\
\end{split}
\end{align}
\end{definition}
%
The following lemma states that the relation defined above is a partial
order and is sufficient to
guarantee inclusion between template complex zonotopes.

\begin{lemma}[Ordering: template complex
  zonotopes]~\label{lem:zon-zon} Both the following statements are
  true.
\begin{enumerate}
\item The relation ``$\order$'' between template complex zonotopes is
  a partial order.
\item The inclusion $\cz{V^\pr}{c^\pr}{s^\pr}\subseteq
  \cz{V}{c}{s}$ holds if the
  relation\\ $\cz{{V^\pr}}{c^\pr}{s^\pr}\order
  \cz{V}{c}{s}$ holds.
\end{enumerate}
\end{lemma}

Based on the convertibility of a real projection of augmented zonotope
to the real projection of a template complex zonotope, we define the
following partial order.  This partial order is a sufficient condition
for checking inclusion between real projections of augmented complex
zonotopes.

\begin{definition}
We say that $\gcz{V^\pr}{c^\pr}{s^\pr}{W^\pr}{l^\pr}{u^\pr}\order
\gcz{V}{c}{s}{W}{l}{u}$ if\\ $\cz{\lt[V^\pr~W^\pr\rt]}{c^\pr+W^\pr\lt(\frac{u^\pr+l^\pr}{2}\rt)}{\ColumnJoin{\hspace{0.5em}s^\pr}{\frac{u-l}{2}}}
\order
\cz{\lt[V~W\rt]}{c+W\lt(\frac{u+l}{2}\rt)}{\ColumnJoin{\hspace{0.5em}s}{\frac{u-l}{2}}}.$
\end{definition}

\begin{lemma}[Ordering: augmented complex
    zonotopes]~\label{lem:gcz-gcz} Both the following statements are
  true.
\begin{enumerate}
\item The relation ``$\order$'' between augmented complex zonotopes is
  a partial order.
\item The real inclusion
$\real\lt(\gcz{V^\pr}{c^\pr}{s^\pr}{W^\pr}{l^\pr}{u^\pr}\rt)\subseteq
\real\lt(\gcz{V}{c}{s}{W_{n\times k}}{l}{u}\rt)$ holds if $\gcz{V^\pr}{c^\pr}{s^\pr}{W^\pr}{l^\pr}{u^\pr}\order \gcz{V}{c}{s}{W_{n\times k}}{l}{u}$.
\end{enumerate}
\end{lemma}

For certain kinds of templates, the intersection of a
sub-parallelotope with an augmented complex zonotope can be exactly
specified as another augmented complex zonotope with the same
template.  This is stated in the following lemma 
\begin{lemma}[Intersection with
  sub-parallelotope]~\label{lem:intersection} If a sub-parallelotope
  $\sptope{K}{\wh{l}}{\wh{u}}$ and an augmented complex zonotope
  $\gcz{V}{c}{s}{\pseudoinverse{K}}{l}{u}$ are such that $KV=0$, then
  
\begin{equation}~\label{eqn:intersection}
\gcz{V}{c}{s}{\pseudoinverse{K}}{l}{u}\bigcap \sptope{K}{\wh{l}}{\wh{u}} =
  \gcz{V}{c}{s}{\pseudoinverse{K}}{\max\lt(l,\wh{l}\rt)}{\min\lt(u,\wh{u}\rt)}.
\end{equation}
\end{lemma}

The above intersection operation employs the minimum and maximum
function.  These functions are not affine functions, hence, when
composed with convex constraints, the resulting constraints need not
be convex.  However, the minimum function can be written as the
minimum of a finite number of affine functions, while the maximum
function can be written as the maximum of a finite number of affine
functions.  We will call such functions as \emph{policies}, which are
defined below.  Latter, we use these functions to derive a policy
iteration framework for finding positive invariants, where for each
policy in the iteration, we  get convex constraints.

\begin{definition}[Min-approximation policy] A function
  $\minaffinefunc:\realset^k\times\comprealset^k$ is called a
  min-approximation policy on $n$ variables, if there exists a boolean
  vector $v\in\{0,1\}^k$ such that for all $a\in\realset^k$,
  $b\in\comprealset^k$ and $i\in\tup{n}$,
\begin{align}
\lt(\minaffine{a}{b}\rt)_i= & \left\{
\begin{array}{l}
v_ia_i+(1-v_i)b_i~~\text{if}~b_i<\inf\\
a_i~~\text{if}~b_i=\inf
\end{array}
\right.
\end{align}
\end{definition}
%


\begin{definition}[Max-approximation policy] A function
  $\maxaffinefunc:\realset^k\times\comprealset^k$ is called a
  max-approximation policy on $n$ variables, if there exists a boolean
  vector $v\in\{0,1\}^k$ such that for all $a\in\realset^k$,
  $b\in\comprealset^k$ and $i\in\tup{n}$,
\begin{align}
\lt(\maxaffine{a}{b}\rt)_i= &\left\{
\begin{array}{l}
v_ia_i+(1-v_i)b_i~~\text{if}~b_i>-\inf\\
a_i~~\text{if}~b_i=-\inf
\end{array}
\right.
\end{align}
\end{definition}
%
Observe that in the above definitions, each min-approximation or
max-approximation policy on $k$ variables is defined by a boolean
vector of size $k$.  So, the number of min-approximation policies or
max-approximation policies on $k$ variables is $2^k$.  We denote the
set of min-approximation policies on $k$ variables as
$\minaffineset(k)$ and max-approximation policies as
$\maxaffineset(k)$.


\begin{lemma}[Intersection using policies]~\label{lem:affineapproximation} Let a sub-parallelotope
  $\sptope{K}{\wh{l}}{\wh{u}}$ and an augmented complex zonotope
  $\gcz{V}{c}{s}{\pseudoinverse{K}}{l}{u}$ be such that $VK=0$.  Then,
  all
  the following statements are true.
\begin{enumerate}
\item Let $\minaffinefunc\in\minaffineset(k)$ and
  $\maxaffinefunc\in\maxaffineset(k)$.  % For a fixed ,
  % the functions $\minaffine{.}{b}:\realset^n=k\ra\realset^k$ and
  % $\maxaffine{.}{b}:\realset^k\ra\realset^n$ are affine functions
  % (i.e., in the first argument).
  Then, for all $a
  \in\realset^k$ and $b\in\comprealset^k$, we have the inequalities $\minaffine{a}{b}\geq \min(a,b)$ and
  $\maxaffine{a}{b}\leq \max(a,b)$.
  Therefore, \[\gcz{V}{c}{s}{\pseudoinverse{K}}{l}{u}\bigcap \sptope{K}{\wh{l}}{\wh{u}} \subseteq
  \gcz{V}{c}{s}{W}{\maxaffine{l}{\wh{l}}}{\minaffine{u}{\wh{u}}}.\]
\item For every $a\in\realset^k$ and $b\in\comprealset^k$, there exist
  $\minaffinefunc\in\minaffineset(k)$ and
  $\maxaffinefunc\in\maxaffineset(k)$ such that
  $\min(a,b)=\minaffine{a}{b}$ and
  $\max(a,b)=\maxaffine{a}{b}$. Consequently, there exist policies
  $\minaffinefunc\in\minaffineset(k)$ and
  $\maxaffinefunc\in\maxaffineset(k)$ such that
\[\gcz{V}{c}{s}{\pseudoinverse{K}}{l}{u}\bigcap \sptope{K}{\wh{l}}{\wh{u}} =
  \gcz{V}{c}{s}{W}{\maxaffine{l}{\wh{l}}}{\minaffine{u}{\wh{u}}}.\]
\end{enumerate}
\end{lemma}





