Since a complex zonotope can have a non-polyhedral real projection,
its intersection with a single linear constraint may not have a
template complex zonotopic real projection.  Moreover, to our
knowledge, there is no efficient way to overapproximate the
intersection with reasonable accuracy as a template complex zonotope.
But in the case of real zonotopes, we discussed the possibility of
efficiently computing the intersection as a zonotope for a specific
case in~Lemma~\ref{lem:motivation}.  However, for invariant
computation, template complex zonotopes have the advantage that they
can incorporate complex valued eigenvectors in the template, while
real zonotopes can not.  Therefore, we define a new set
representation, called augmented complex zonotopes, which combines the
template complex zonotope and the real zonotope by their Minkowski
sum.  Then, the real zonotopic part of the augmented complex zonotope
can be used to compute intersections, while the template complex
zonotopic part can incorporate complex eigenvectors.
%
\begin{definition}[Augmented complex zonotope]
Let $V\in\mat{n}{m}{C}$ called primary template, $W\in\mat{n}{k}{R}$
called secondary template, $c\in\mb{R}^n$ called primary offset,
$s\in\mb{R}^m$ called scaling factors, $u,l\in\mb{R}^k$ called lower
and upper interval bounds, respectively, such that $l\leq u$.  The
following is an augmented complex
zonotope.
\begin{equation*}
\gcz{V}{c}{s}{W}{l}{u} = \cz{V}{c}{s}\oplus\zon{W}{l}{u}.
\end{equation*}
\end{definition}
%
The limits of the projection of an augmented complex zonotope along
any direction are stated in the following lemma.
%
\begin{lemma}
Let $V\in\mat{n}{m}{\mb{C}}$ and $v\in\realset^n$.  Then,
\[
\max_{x\in\gcz{V}{c}{s}{W}{l}{u}}v^Tx = v^T\lt(c+W\frac{l+u}{2}\rt)+\lt|v^T[V~~W]\rt|\lt(\ColumnJoin{~~s}{\frac{l-u}{2}}\rt)
\]
\end{lemma}
%
The reason for considering subparallelotopic constraints for
describing the staying conditions and guards of our affine hybrid
system description is related to the following lemma.  It computes the
overapproximation of the intersection between an augmented complex
zonotope and a suitably aligned subparallelotope as another augmented
complex zonotope.  Furthermore, under an orthogonality condition
stated below, we compute the exact intersection.
%
\begin{lemma}~\label{lem:acz-int}
Let $\ptemplate$ be a sub-paralleotopic template and $c\in\mb{C}^n$
such that $\ptemplate c=0$.  Then,
\begin{enumerate}
\item
  $\gcz{V}{c}{s}{\pinv{\ptemplate}}{l}{u}\bigcap\sptope{\ptemplate}{\wh{l}}{\wh{u}}\subseteq
  \gcz{V}{c}{s}{\pinv{\ptemplate}}{l\bigvee\wh{l}}{u\bigwedge\wh{u}}$.
\item If $KV=0$, then $\gcz{V}{c}{s}{\pinv{\ptemplate}}{l}{u}\bigcap\sptope{\ptemplate}{\wh{l}}{\wh{u}}=
  \gcz{V}{c}{s}{\pinv{\ptemplate}}{l\bigvee\wh{l}}{u\bigwedge\wh{u}}$.
\end{enumerate}
\end{lemma}
%
Linear transformations and Minkowski sums preserve the class of
augmented complex zonotopes, and these can be efficiently computed as
follows.
%
\begin{lemma}[Linear transformation and Minkowski sum]
\begin{enumerate}
\item $A\gcz{V}{c}{s}{W}{l}{u} = \gcz{AV}{Ac}{s}{AW}{l}{u}$.
\item $\gcz{V}{c}{s}{W}{l}{u}\oplus
  \gcz{V^\pr}{c^\pr}{s^\pr}{W^\pr}{l^\pr}{u^\pr}$\\
$= \gcz{\lt[V~~V^\pr\rt]}{c+c^\pr}{\ColumnJoin{s}{s^\pr}}{\lt[W~~W^\pr\rt]}{\ColumnJoin{l}{l^\pr}}{\ColumnJoin{u}{u^\pr}}$
\end{enumerate}
\end{lemma}
%
The real projection of an augmented complex zonotope can be
equivalently tranformed as the real projection of a template complex
zonotope, as follows.  However, we note that as a complex valued set,
an augmented complex zonotope is more general and can not be
represented as a template complex zonotope.  Furthermore,
representation as an augmented complex zonotope is more succinct and
hence more convenient for deriving conditions for invariant
computation.
%
\begin{lemma}~\label{lem:conversion}
$\real\lt(\gcz{V}{c}{s}{W}{l}{u}\rt) = \real\lt(\cz{\lt[V~W\rt]}{c+W\lt(\frac{u+l}{2}\rt)}{\ColumnJoin{\hspace{0.5em}s}{\frac{u-l}{2}}}\rt)$.
\end{lemma}
%
Because of the above relationship, checking the inclusion between the
real projections of two augmented complex zonotopes amounts to
checking the inclusion between real projections of two template
complex zonotopes.  Recall the partial order between two template
complex zonotopes defined in the previous section. We extend this
partial order to augmented complex zonotopes as follows.
%
\begin{definition}
We say that $\gcz{V^\pr}{c^\pr}{s^\pr}{W^\pr}{l^\pr}{u^\pr}\order
\gcz{V}{c}{s}{W}{l}{u}$ if\\ $\cz{\lt[V^\pr~W^\pr\rt]}{c^\pr+W^\pr\lt(\frac{u^\pr+l^\pr}{2}\rt)}{\ColumnJoin{\hspace{0.5em}s^\pr}{\frac{u-l}{2}}}
\order
\cz{\lt[V~W\rt]}{c+W\lt(\frac{u+l}{2}\rt)}{\ColumnJoin{\hspace{0.5em}s}{\frac{u-l}{2}}}.$
\end{definition}
%
\begin{lemma}[Ordering: augmented complex
    zonotopes]~\label{lem:gcz-gcz} The relation ``$\order$'' among
augmented complex zonotopes is a partial order.  Furthermore, the real
inclusion\\
$\real\lt(\gcz{V^\pr}{c^\pr}{s^\pr}{W^\pr}{l^\pr}{u^\pr}\rt)\subseteq \real\lt(\gcz{V}{c}{s}{W_{n\times
k}}{l}{u}\rt)$ holds if the order relation
$\gcz{V^\pr}{c^\pr}{s^\pr}{W^\pr}{l^\pr}{u^\pr}\order \gcz{V}{c}{s}{W_{n\times
k}}{l}{u}$ holds.
\end{lemma}

The intersection of an augmented complex zonotope with a
subparallelotope involves the meet and join operations, as stated in
Lemma~\ref{lem:acz-int}.  As functions, the meet and join operations
piecewise affine but not affine, hence composition with convex
functions can give non-convex functions.  But in the case of affine
hybrid systems, the guards and staying conditions define the limits on
the interval bounds of a sub-parallelotope.  So, we propose the
following affine upper and lower bound functions for the meet and join
operations, respectively.

Let us consider a binary operation,
$\minaffinefunc:\realset^k\times\comprealset^k$, which we
call \emph{min-approximation} function, defined as follows.  For
$u\in\realset^k$ and $\wh{u}\in\comprealset^k$,
$\lt(\minaffine{u}{\wh{u}}\rt)_i = \left\{
\begin{array}{l}
\wh{u}_i~~\text{if}~\wh{u}_i<\inf\\
u_i~~\text{if}~\wh{u}_i=\inf
\end{array}
\right.$
Similarly, let us consider a binaray operation
$\maxaffinefunc:\realset^k\times\comprealset^k$,
called \emph{max-approximation} function, defined as follows.  For
$l\in\realset^k$ and $\wh{l}\in\comprealset^k$,
$\lt(\maxaffine{l}{\wh{l}}\rt)_i = \left\{
\begin{array}{l}
\wh{l}_i~~\text{if}~\wh{l}_i>\inf\\
l_i~~\text{if}~\wh{l}_i=-\inf
\end{array}
\right.$
%
The following lemma states that the min-approximation and
max-approximation functions are affine upper and lower bound functions
for the meet and join operations, respectively.
%
\begin{lemma}~\label{lem:min-max-approx}
All of the below statements are true.
\begin{enumerate}
\item Let $l,u\in\realset^k$ and $\wh{l},\wh{u}\in\comprealset^k$.
  Then, $\maxaffine{l}{\wh{l}}\leq l\bigcup\wh{l}$ and
  $u\bigcap\wh{u}\leq\minaffine{u}{\wh{u}}$.
\item For a fixed $\wh{l},\wh{u}\in\comprealset^k$, the functions
  $\maxaffine{.}{\wh{l}}:\realset^k\ra\realset^k$ and
  $\minaffine{.}{\wh{u}}:\realset^k\ra\realset^k$ are affine functions.
\end{enumerate}
\end{lemma}





