An augmented complex zonotope is the Minkowski sum of a template
complex zonotope and a real zonotope. In terms of expressivity, an
augmented complex zonotope is slightly more general than template
complex zonotopes.  But geometrically, the sets that can be described
as real projections of augmented complex zonotopes can also be
described as real projections of template complex zonotopes.  However,
with augmented complex zonotopes, the intersection with
subparallelotopic constraints can be succinctly specified, as we will
see latter. Consequently, this representation is more convenient to
derive conditions for computing invariants for the affine hybrid
system.
%
\begin{definition}[Augmented complex zonotope]
Let $V\in\mat{n}{m}{C}$ called primary template, $W\in\mat{n}{k}{R}$
called secondary template, $c\in\mb{R}^n$ called primary offset,
$s\in\mb{R}^m$ called scaling factors, $u,l\in\mb{R}^k$ called lower
and upper interval bounds, respectively, such that $l\leq u$.  The
following is an augmented complex
zonotope.
\begin{equation*}
\gcz{V}{c}{s}{W}{l}{u} = \cz{V}{c}{s}\oplus\zon{W}{l}{u}.
\end{equation*}
\end{definition}
%
The following lemma gives an overapproximation of the intersection
between an augmented complex zonotope and a suitably aligned
subparallelotope as another augmented complex zonotope.  Furthermore,
under an orthogonality condition stated below, we compute the exact
intersection.
%
\begin{lemma}~\label{lem:acz-int}
Let $\ptemplate$ be a subparalleotopic template and $c\in\mb{C}^n$
such that $\ptemplate c=0$.  Then,
\begin{enumerate}
\item
  $\gcz{V}{c}{s}{\pinv{\ptemplate}}{l}{u}\bigcap\sptope{\ptemplate}{\wh{l}}{\wh{u}}\subseteq
  \gcz{V}{c}{s}{\pinv{\ptemplate}}{l\bigvee\wh{l}}{u\bigwedge\wh{u}}$.
\item If $\ptemplate V=0$, then $\gcz{V}{c}{s}{\pinv{\ptemplate}}{l}{u}\bigcap\sptope{\ptemplate}{\wh{l}}{\wh{u}}=
  \gcz{V}{c}{s}{\pinv{\ptemplate}}{l\bigvee\wh{l}}{u\bigwedge\wh{u}}$.
\end{enumerate}
\end{lemma}
%
%% To illustrate, the intersection of $\gcz{\lt[\begin{array}{cc}1+2i &
%% 2+i\\1-2i & 2-i\\0 &
%% 0\end{array}\rt]}{\lt[\begin{array}{c}1\\1\\0\end{array}\rt]}{\lt[\begin{array}{c}1\\1\\1\end{array}\rt]}
%% {\lt[\begin{array}{c}0\\0\\1\end{array}\rt]}{-2}{2}$. with a
%% constraint on the third coordinate $-1\leq x_3\leq 1$ is exactly\\
%% $\gcz{\lt[\begin{array}{cc}1+2i & 2+i\\1-2i & 2-i\\0 &
%% 0\end{array}\rt]}{\lt[\begin{array}{c}1\\1\\0\end{array}\rt]}{\lt[\begin{array}{c}1\\1\\1\end{array}\rt]}
%% {\lt[\begin{array}{c}0\\0\\1\end{array}\rt]}{-1}{1}$.  We observe that
%% the center $\lt(\begin{array}{c}1\\1\\0\end{array}\rt)$ is
%% perpendicular to the third axis, i.e., the vector
%% $\lt[\begin{array}{ccc}0 & 0 & 1\end{array}\rt]$, which is a required
%% condition in the above lemma.  Furthermore, since the primary template
%% is orthogonal to the subparallelotopic template in this example, i.e.
%% $\lt[\begin{array}{ccc}0 & 0 &
%% 1\end{array}\rt]\lt[\begin{array}{cc}1+2i & 2+i\\1-2i & 2-i\\0 &
%% 0\end{array}\rt]=0$, the resultant intersection is exactly an
%% augmented complex zonotope.  On the other hand, if the primary
%% template is not orthogonal with the secondary template, like in the
%% case of $\gcz{\lt[\begin{array}{cc}1+2i & 2+i\\1-2i & 2-i\\1 &
%% 1\end{array}\rt]}{\lt[\begin{array}{c}1\\1\\0\end{array}\rt]}{\lt[\begin{array}{c}1\\1\\1\end{array}\rt]}
%% {\lt[\begin{array}{c}0\\0\\1\end{array}\rt]}{-2}{2}$, then the
%% intersection with $-1\leq x_3\leq 1$ is overapproximated by
%% $\gcz{\lt[\begin{array}{cc}1+2i & 2+i\\1-2i & 2-i\\1 &
%% 1\end{array}\rt]}{\lt[\begin{array}{c}1\\1\\0\end{array}\rt]}{\lt[\begin{array}{c}1\\1\\1\end{array}\rt]}
%% {\lt[\begin{array}{c}0\\0\\1\end{array}\rt]}{-1}{1}$, but this is not
%% the exact intersected set.


Similar to usual zonotopes, augmented
complex zonotopes are closed under Minkowski sums and linear
transformations, and their computations are also similar. The computation of some important operations are summarized as follows.

\begin{enumerate}
\item $A\gcz{V}{c}{s}{W}{l}{u} = \gcz{AV}{Ac}{s}{AW}{l}{u}$.
\item $\gcz{V}{c}{s}{W}{l}{u}\oplus
  \gcz{V^\pr}{c^\pr}{s^\pr}{W^\pr}{l^\pr}{u^\pr}$\\
$= \gcz{\lt[V~~V^\pr\rt]}{c+c^\pr}{\ColumnJoin{s}{s^\pr}}{\lt[W~~W^\pr\rt]}{\ColumnJoin{l}{l^\pr}}{\ColumnJoin{u}{u^\pr}}$

%
\item The limits of the projection of an augmented complex zonotope along
any direction can be computed as follows. For $v\in\realset^n$,
\begin{equation}\label{lem:polylimits-acz}
\max_{x\in\gcz{V}{c}{s}{W}{l}{u}}v^Tx = v^T\lt(c+W\frac{l+u}{2}\rt)+\lt|v^T[V~~W]\rt|\lt(\ColumnJoin{s}{\frac{u-l}{2}}\rt)
\end{equation}
\end{enumerate}
%
%\begin{lemma}[Linear transformation and Minkowski sum]
%\begin{enumerate}
%\item $A\gcz{V}{c}{s}{W}{l}{u} = \gcz{AV}{Ac}{s}{AW}{l}{u}$.
%\item $\gcz{V}{c}{s}{W}{l}{u}\oplus
%  \gcz{V^\pr}{c^\pr}{s^\pr}{W^\pr}{l^\pr}{u^\pr}$\\
%$= \gcz{\lt[V~~V^\pr\rt]}{c+c^\pr}{\ColumnJoin{s}{s^\pr}}{\lt[W~~W^\pr\rt]}{\ColumnJoin{l}{l^\pr}}{\ColumnJoin{u}{u^\pr}}$
%\end{enumerate}
%\end{lemma}
%%
%The limits of the projection of an augmented complex zonotope along
%any direction are stated in the following lemma.
%%
%\begin{lemma}~\label{lem:polylimits-acz}
%Let $V\in\mat{n}{m}{\mb{C}}$ and $v\in\realset^n$.  Then,
%\[
%\max_{x\in\gcz{V}{c}{s}{W}{l}{u}}v^Tx = v^T\lt(c+W\frac{l+u}{2}\rt)+\lt|v^T[V~~W]\rt|\lt(\ColumnJoin{s}{\frac{u-l}{2}}\rt)
%\]
%\end{lemma}
%%%%%%%%%%%


The real projection of an augmented complex zonotope can be
equivalently transformed as the real projection of a template complex
zonotope, as follows.
%
\begin{lemma}~\label{lem:conversion}
$\real\lt(\gcz{V}{c}{s}{W}{l}{u}\rt) = \real\lt(\cz{\lt[V~W\rt]}{c+W\lt(\frac{u+l}{2}\rt)}{\ColumnJoin{s}{\frac{u-l}{2}}}\rt)$.
\end{lemma}
%
Because of the above relationship, checking the inclusion between the
real projections of two augmented complex zonotopes amounts to
checking the inclusion between real projections of two template
complex zonotopes.  Recall the relation between template complex
zonotopes that was a sufficient condition for inclusion.  We extend
the relation to augmented complex zonotopes as follows.
%
\begin{definition}~\label{defn:gcz-order}
We say that $\gcz{V^\pr}{c^\pr}{s^\pr}{W^\pr}{l^\pr}{u^\pr}\order
\gcz{V}{c}{s}{W}{l}{u}$ if\\ $\cz{\lt[V^\pr~W^\pr\rt]}{c^\pr+W^\pr\lt(\frac{u^\pr+l^\pr}{2}\rt)}{\ColumnJoin{s^\pr}{\frac{u-l}{2}}}
\order
\cz{\lt[V~W\rt]}{c+W\lt(\frac{u+l}{2}\rt)}{\ColumnJoin{s}{\frac{u-l}{2}}}.$
\end{definition}
%
\begin{lemma}[Inclusion: augmented complex
    zonotopes]~\label{lem:gcz-gcz} The real inclusion\\
$\real\lt(\gcz{V^\pr}{c^\pr}{s^\pr}{W^\pr}{l^\pr}{u^\pr}\rt)\subseteq \real\lt(\gcz{V}{c}{s}{W_{n\times
k}}{l}{u}\rt)$ holds if the relation\\
$\gcz{V^\pr}{c^\pr}{s^\pr}{W^\pr}{l^\pr}{u^\pr}\order \gcz{V}{c}{s}{W_{n\times
k}}{l}{u}$ is true.
\end{lemma}

The intersection of an augmented complex zonotope with a
subparallelotope involves the meet and join operations, as stated in
Lemma~\ref{lem:acz-int}.  These operations are piecewise affine
functions of their arguments, but not affine.  Hence, their
composition with a convex function may be non-convex.  But since we
are interested in deriving convex conditions for finding an invariant,
in this regard, we define the following upper and lower bound
functions for the join and meet operations, respectively.

Let us define a binary function
$\minaffinefunc:\realset^k\times\comprealset^k$,
called \emph{min-approximation} function, as follows.  For
$u\in\realset^k$ and $\wh{u}\in\comprealset^k$,
$\lt(\minaffine{u}{\wh{u}}\rt)_i = \left\{
\begin{array}{l}
\wh{u}_i~~\text{if}~\wh{u}_i<\inf\\
u_i~~\text{if}~\wh{u}_i=\inf
\end{array}
\right..$
Similarly, let us define another binary function       
$\maxaffinefunc:\realset^k\times\comprealset^k$,
called \emph{max-approximation} function, as follows.  For
$l\in\realset^k$ and $\wh{l}\in\comprealset^k$,
$\lt(\maxaffine{l}{\wh{l}}\rt)_i = \left\{
\begin{array}{l}
\wh{l}_i~~\text{if}~\wh{l}_i>\inf\\
l_i~~\text{if}~\wh{l}_i=-\inf
\end{array}
\right..$
%
\begin{lemma}~\label{lem:min-max-approx}
Both the following statements are true.
\begin{enumerate}
\item Let $l,u\in\realset^k$ and $\wh{l},\wh{u}\in\comprealset^k$.
  Then, $\maxaffine{l}{\wh{l}}\leq l\bigvee\wh{l}$ and
  $\minaffine{u}{\wh{u}}\geq u\bigwedge\wh{u}$.
\item For fixed $\wh{l},\wh{u}\in\comprealset^k$, the functions
  $\maxaffine{.}{\wh{l}}:\realset^k\ra\realset^k$ and
  $\minaffine{.}{\wh{u}}:\realset^k\ra\realset^k$ are both affine functions.
\end{enumerate}
\end{lemma}





