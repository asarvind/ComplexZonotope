As discussed in the previous section, a complex zonotope can have a
non-polyhedral real projection.  Hence, when a complex zonotope is
intersected with even a single linear constraint, we may not be able
to computing a reasonably accurate complex zonotopic overapproximation
of the intersection.  In contrast, Lemma~\ref{lem:motivation}
discussed the possibility of efficiently computing the intersection
between a real zonotope and a sub-parallelotope under some assumption.
This being a motivation, we combine the template complex zonotopic set
representation and the real zonotopic set representation by their
Minkowski sum.  We call the resulting representation as an
\emph{augmented complex zonotope}.
%
\begin{definition}[Augmented complex zonotope]
Let $V\in\mat{n}{m}{C}$ called primary template, $W\in\mat{n}{k}{R}$
called secondary template, $c\in\mb{R}^n$ called primary offset,
$s\in\mb{R}^m$ called scaling factors, $u,l\in\mb{R}^k$ called lower
and upper interval bounds, respectively, such that $l\leq u$.  The
following is an augmented complex
zonotope.
\begin{multline}
\gcz{V}{c}{s}{W}{l}{u} =
\lt\{
  c+V\epsilon+W\zeta:\epsilon\in\mb{C}^m,\zeta\in\mb{R}^k,\rt.\\ \lt.  \lt|\epsilon_i\rt|\leq
 s_i~\forall i\in\tup{m},~\zeta_j\in[l_j,u_j]~\forall j\in \tup{k}
\rt\}
\end{multline}
\end{definition}
%
Before we discuss the intersection between an augmented complex
zonotope and a sub-parallelotope, we introduce the following lemma.
It states the limits of the projection of an augmented complex
zonotope along any direction.  This lemma is latter used to reason
about the intersection.
%
\begin{lemma}
Let us consider an augmented complex zonotope $\gcz{V}{c}{s}{W}{l}{u}$
where $V\in\mat{n}{m}{\mb{C}}$.  If $v\in\realset^n$, then
\[
\max_{x\in\gcz{V}{c}{s}{W}{l}{u}}v^Tx = v^T\lt(c+W\frac{l+u}{2}\rt)+\lt|v^T[V~~W]\rt|\lt(\ColumnJoin{s}{\frac{l-u}{2}}\rt)
\]
\end{lemma}
%
The following lemma gives an overapproximation of the intersection
between an augmented complex zonotope and a sub-parallelotope.
Furthermore, it also states a condition when the intersection is
exactly computed.
%
\begin{lemma}
Let $\ptemplate$ be a sub-paralleotopic template and $c\in\mb{C}^n$
such that $\ptemplate c=0$.  Then both the following statements are
true.
\begin{enumerate}
\item
  $\gcz{V}{c}{s}{\pinv{\ptemplate}}{l}{u}\bigcap\sptope{\ptemplate}{\wh{l}}{\wh{u}}\subseteq
  \gcz{V}{c}{s}{\pinv{\ptemplate}}{l\bigvee\wh{l}}{u\bigwedge\wh{u}}$.
\item If $KV=0$, then $\gcz{V}{c}{s}{\pinv{\ptemplate}}{l}{u}\bigcap\sptope{\ptemplate}{\wh{l}}{\wh{u}}=
  \gcz{V}{c}{s}{\pinv{\ptemplate}}{l\bigvee\wh{l}}{u\bigwedge\wh{u}}$.
\end{enumerate}
\end{lemma}
%
The linear transformation and Minkowski sum operations on
augmented complex zonotopes are computed as follows.
%
\begin{lemma}[Linear transformation and Minkowski sum]
\begin{enumerate}
\item $A\gcz{V}{c}{s}{W}{l}{u} = \gcz{AV}{Ac}{s}{AW}{l}{u}$.
\item $\gcz{V}{c}{s}{W}{l}{u}\oplus
  \gcz{V^\pr}{c^\pr}{s^\pr}{W^\pr}{l^\pr}{u^\pr}$\\
$= \gcz{\lt[V~~V^\pr\rt]}{c+c^\pr}{\ColumnJoin{s}{s^\pr}}{\lt[W~~W^\pr\rt]}{\ColumnJoin{l}{l^\pr}}{\ColumnJoin{u}{u^\pr}}$
\end{enumerate}
\end{lemma}
%
The real projection of an augmented complex zonotope can be
equivalently tranformed as the real projection of a template complex
zonotope, as follows.  However, we note that as a complex valued set,
an augmented complex zonotope is more general and can not be
represented as a template complex zonotope.
%
\begin{lemma}~\label{lem:conversion}
$\real\lt(\gcz{V}{c}{s}{W}{l}{u}\rt) = \real\lt(\cz{\lt[V~W\rt]}{c+W\lt(\frac{u+l}{2}\rt)}{\ColumnJoin{\hspace{0.5em}s}{\frac{u-l}{2}}}\rt)$.
\end{lemma}
%
Because of the above relationship, checking the inclusion between the
real projections of two augmented complex zonotopes amounts to checking the
inclusion between real projections of two template complex zonotopes.
%
Recall the partial order between two template complex zonotopes
defined in the previous section. We extend this partial order to
augmented complex zonotopes as follows.

\begin{definition}
We say that $\gcz{V^\pr}{c^\pr}{s^\pr}{W^\pr}{l^\pr}{u^\pr}\order
\gcz{V}{c}{s}{W}{l}{u}$ if\\ $\cz{\lt[V^\pr~W^\pr\rt]}{c^\pr+W^\pr\lt(\frac{u^\pr+l^\pr}{2}\rt)}{\ColumnJoin{\hspace{0.5em}s^\pr}{\frac{u-l}{2}}}
\order
\cz{\lt[V~W\rt]}{c+W\lt(\frac{u+l}{2}\rt)}{\ColumnJoin{\hspace{0.5em}s}{\frac{u-l}{2}}}.$
\end{definition}

\begin{lemma}[Ordering: augmented complex
    zonotopes]~\label{lem:gcz-gcz} 
\begin{enumerate}
\item The relation ``$\order$'' between
  augmented complex zonotopes is a partial order.  
\item The real inclusion
  $\real\lt(\gcz{V^\pr}{c^\pr}{s^\pr}{W^\pr}{l^\pr}{u^\pr}\rt)\subseteq
  \real\lt(\gcz{V}{c}{s}{W_{n\times k}}{l}{u}\rt)$ holds if
  $\gcz{V^\pr}{c^\pr}{s^\pr}{W^\pr}{l^\pr}{u^\pr}\order
  \gcz{V}{c}{s}{W_{n\times k}}{l}{u}$.
\end{enumerate}
\end{lemma}

The intersection of an augmented complex zonotope with a
sub-parallelotope involves the meet and join operations, according to
Lemma~\ref[REFER].  The meet and join operations on the variables give
rise to non-affine functions, which when composed with convex
functions, may result in non-convex functions.  However, in case of
hybrid systems the guards and staying conditions define the limits on
the interval bounds of a sub-parallelotope.  Since we are interested
in computing large invariants, these bounds are usually attained by
the sufficiently large invariant augmented complex zonotope.  Based on
this observation, we propose the following affine upper and lower
bound functions for the meet and join operations, respectively.

Let us define a binary operation called \emph{min-approximation} function,
$\minaffinefunc:\realset^k\times\comprealset^k$, defined as follows.
For $u\in\realset^k$ and $\wh{u}\in\comprealset^k$, $\lt(\minaffine{u}{wh{u}}\rt)_i = \left\{
\begin{array}{l}
\wh{u}_i~~\text{if}~\wh{u}_i<\inf\\
u_i~~\text{if}~\wh{u}_i=\inf
\end{array}
\right.$

Similarly, define a binaray operation called \emph{max-approximation}
function, $\maxaffinefunc:\realset^k\times\comprealset^k$, defined as
follows.  For $l\in\realset^k$ and $\wh{l}\in\comprealset^k$,
$\lt(\maxaffine{l}{wh{l}}\rt)_i = \left\{
\begin{array}{l}
\wh{l}_i~~\text{if}~\wh{l}_i>\inf\\
l_i~~\text{if}~\wh{l}_i=-\inf
\end{array}
\right.$

[Give example here]

The following lemma states that the min-approximation and
max-approximation functions are affine upper and lower bound functions
for the meet and join operations, respectively.
%
\begin{lemma}
All of the below statements are true.
\begin{enumerate}
\item Let $l,u\in\realset^k$ and $\wh{l},\wh{u}\in\comprealset^k$.
  Then, $\maxaffine{l}{\wh{l}}\leq l\bigcup\wh{l}$ and
  $u\bigcap\wh{u}\leq\minaffine{u}{\wh{u}}$.
\item For a fixed $\wh{l},\wh{u}\in\comprealset^k$, the functions
  $\maxaffine{.}{\wh{l}}:\realset^k\ra\realset^k$ and
  $\minaffine{.}{\wh{u}}:\realset^k\ra\realset^k$ are affine functions
  functions.
\end{enumerate}
\end{lemma}





