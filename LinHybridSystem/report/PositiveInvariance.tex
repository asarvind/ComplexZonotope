In this section, we first derive a sufficient condition for positive
invariance of an augmented complex zonotope.  Also, we state
conditions for containment of an initial set and satisfaction of
polytopic safety constraints.  Latter, we explain how find the
augmented complex zontope based on these conditions.

Earlier, we had computed the linear transformations and Minkowki sums
of augmented complex zonotope and possible overapproximations of their
intersection with subparalleotopic constraints.  Accordingly, we can
compute the overapproximation of the reachable set of an augmented
complex zonotope as another augmented complex zonotope.  Then, we
utilize the partial order given in Definition~\ref{defn:gcz-order} to
  deduce a sufficient condition for positive invariance, as follows.
%
We consider a state set $\hybridset:\locationset\ra\realset^n$ whose
  projection in each location $\loc\in\locationset$ is
  $\real\lt(\gcz{V(\loc)}{c(\loc)}{s(\loc)}{\pinv{\ptemplate}(\loc)}{l(\loc)}{u(\loc)}\rt)$.
  Let us consider that the additive input for an intralocation
  transition in any location $\loc\in\locationset$ is overapproximated
  as
  $\inputset(\loc)\subseteq \gcz{V^{in}(\loc)}{c^{in}(\loc)}{s^{in}(\loc)}{W^{in}(\loc)}{l^{in}(\loc)}{u^{in}(\loc)}$.
  Similarly, let the additive input for an interlocation transition
  along any edge $\edge\in\edgeset$, be overapproximated as
  $\edgeinp(\edge)\subseteq \gcz{V^{in}(\edge)}{c^{in}(\edge)}{s^{in}(\edge)}{W^{in}(\edge)}{l^{in}(\edge)}{u^{in}(\edge)}$.
  Furthermore, for any $\loc\in\locationset$, the safe set in the
  location is $\safeset(\loc)=\polytope{T(\loc)}{d(\loc)}$ and the
  initial set is
  $\mc{I}(\loc)=\real\lt(\gcz{V^I(\loc)}{c^I(\loc)}{s^I(\loc)}{W^I(\loc)}{l^I(\loc)}{u^I(\loc)}\rt).$
  


\begin{lemma}[Positive invariance]
  The condition for positive invariance of the state set $\hybridset$
  is the following.
\begin{enumerate}
\item For any location $\loc\in\locationset$, the inclusion
  $\locationtransition{\loc}\lt(\hybridset(\loc)\rt)\subseteq
  \hybridset(\loc)$ holds if all of the below statements are
  collectively true.
\begin{align}~\label{eqn:locinv}
\begin{split}
& \pinv{\ptemplate}(\loc)c(\loc) = 0~\text{and there exists a complex vector}~
  c^\pr(\loc)~\text{and}\\
& \text{real
    vectors}~s^\pr(\loc),l^\pr(\loc),u^\pr(\loc),l^\dpr(\loc),u^\dpr(\loc)~\text{such
    that}\\
\end{split}
\end{align}
\vspace{-1.5em}
\begin{align}
& c^\pr(\loc) = \map(\loc)c(\loc)+c^{in}(\loc),~~s^\pr(\loc) =
  \ColumnJoin{s(\loc)}{s^{in}(\loc)}
\end{align}
\vspace{-1.5em}
\begin{align}
& l^\pr(\loc) =
  \ColumnJoin{\maxaffine{l(\loc)}{\stay^-(\loc)}}{~~~~~~~~~l^{in}(\loc)~~~},~~u^\pr
  =
  \ColumnJoin{\minaffine{u(\loc)}{\stay^+(\loc)}}{~~~~~~~~~u^{in}(\loc)~~~}
\end{align}
\vspace{-1.5em}
\begin{align}
& \lt\{\Calign{\gcz{\lt[\map(\loc)V(\loc)~~V^{in}(\loc)\rt]}{c^\pr(\loc)}{s^\pr(\loc)}
          {\lt[\map(\loc)\pinv{\ptemplate}(\loc)~~W^{in}(\loc)\rt]}{l^\pr(\loc)}{u^\pr(\loc)}}
 {~~\order
   \gcz{V(\loc)}{c(\loc)}{s(\loc)}{\pinv{\ptemplate}(\loc)}{l^\dpr(\loc)}{u^\dpr(\loc)}}\rt.
\end{align}
\vspace{-1.5em}
\begin{align}
& \maxaffine{l^\dpr(\loc)}{\stay^-(\loc)}\geq l(\loc)~\text{and}~~
\minaffine{u^\dpr(\loc)}{\stay^+(\loc)}\leq u(\loc).
 \end{align}
\item For any edge $\edge\in\edgeset$, the inclusion
  $\edgetransition{\edge}\lt(\hybridset\lt(\preloc{\edge}\rt)\rt)
  \subseteq \hybridset\lt(\postloc{\edge}\rt)$ holds if 
  all of the below statements are collectively true.
\begin{align}~\label{eqn:edgeinv}
\begin{split}
& \pinv{\ptemplate}\lt(\preloc{\edge}\rt)c\lt(\preloc{\edge}\rt) =
  0~\text{and there exists a complex
  vector}~c^\pr(\postloc{\edge})~\text{and}\\
& \text{real
    vectors}~s^\pr(\postloc{\edge}),l^\pr(\postloc{\edge}),u^\pr(\postloc{\edge}),l^\dpr(\postloc{\edge}),u^\dpr(\postloc{\edge})~~\text{such
  that}
\end{split}
\end{align}
\vspace{-1.5em}
\begin{align}
& c^\pr(\edge) = \edgemap(\edge)c(\preloc{\edge})+c^{in}(\edge),~~s^\pr(\edge) =
  \ColumnJoin{s(\preloc{\edge})}{s^{in}(\edge)}
\end{align}
\vspace{-1.5em}
\begin{align}
& l^\pr(\edge) =
  \ColumnJoin{\maxaffine{l(\preloc{\edge})}{\stay^-(\preloc{\edge})\bigvee\loweredgebound{\edge}}}{~~~~~~~~~l^{in}(\edge)~~~},~~u^\pr =
  \ColumnJoin{\minaffine{u(\preloc{\edge})}{\stay^+(\preloc{\edge})\bigwedge\upperedgebound{\edge}}}{~~~~~~~~~u^{in}(\edge)~~~}
\end{align}
\vspace{-1.5em}
\begin{align}
& \lt\{\Calign{\gcz{\lt[\map(\edge)V(\preloc{\edge})~~V^{in}(\edge)\rt]}{c^\pr(\edge)}{s^\pr(\edge)}
          {\lt[\map(\edge)\pinv{\ptemplate}(\preloc{\edge})~~W^{in}(\edge)\rt]}{l^\pr(\edge)}{u^\pr(\edge)}}
 {~~\order
   \gcz{V(\postloc{\edge})}{c(\postloc{\edge})}{s(\postloc{\edge})}{\pinv{\ptemplate}(\postloc{\edge})}{l^\dpr(\edge)}{u^\dpr(\edge)}}\rt.
\end{align}
\vspace{-1.5em}
\begin{align}
& \maxaffine{l^\dpr(\edge)}{\stay^-(\postloc{\edge})}\geq l(\postloc{\edge})~\text{and}~~
\minaffine{u^\pr(\edge)}{\stay^+(\postloc{\edge})}\leq u(\postloc{\edge}).
\end{align}
\vspace{-1.5em}
\end{enumerate}
\end{lemma}
%
Next, we state a sufficient condition for an augmented complex
zonotopic state set to contain an initial set overapproximated by an
augmented complex zonotope.  This is given by the inclusion relation
between augmented complex zonotopes from~Lemma~\ref{lem:gcz-gcz}).
\begin{lemma}
 $\forall\loc\in\locationset,~\mc{I}(\loc)\subseteq
  \hybridset(\loc)$ if $\forall\loc\in\locationset$,
\begin{align}~\label{eqn:initcont}
\gcz{V^I(\loc)}{c^I(\loc)}{s^I(\loc)}{W^I(\loc)}{l^I(\loc)}{u^I(\loc)}\order\gcz{V(\loc)}{c(\loc)}{s(\loc)}{\pinv{\ptemplate}(\loc)}{l(\loc)}{u(\loc)}.
\end{align}
\end{lemma}
%
For satisfaction of polytopic safety constraints by an augmented
complex zonotope, the following lemma gives a sufficient condition,
which is just the reformulation of Lemma~\ref{lem:polylimits-acz} in
the below context.
%
\begin{lemma}
For any location $\loc\in\locationset$,
  $\hybridset(\loc)\subseteq \safeset(\loc)$ if,
\begin{align}~\label{eqn:safecont}
T(\loc)\lt(c(\loc)+\pinv{\ptemplate}(\loc)\lt(\frac{u(\loc)+l(\loc)}{2}\rt)\rt)+\lt|T\lt[V(\loc),~\pinv{\ptemplate}(\loc)\rt]\rt|\ColumnJoin{\hspace{1.5em}s}{\frac{u(\loc)-l(\loc)}{2}}\leq d(\loc).
\end{align}
\end{lemma}
%
Consolidating the previous results in this section, we state following
sufficient condition for positive invariance, along with satisfaction
of a given set of polytopic safety constraints and containment of an initial set.
%
\begin{theorem}~\label{thm:main} If
  $\forall \loc\in\locationset$ and $\forall \edge\in\edgeset$, all of
  the Equations[\ref{eqn:locinv}-\ref{eqn:safecont}] are collectively
  true, then the state set $\hybridset$ is a positive invariant,
  satisfies the given safety constraints and contains the given
  initial set
\end{theorem}

\tbf{Solving the conditions.}  Firstly, we note that the secondary
template in a location is predefined as the pseudoinverse of the
subparallelotopic template in the location, in accordance with
Theorem~\ref{thm:main}.  Then, we observe that for a fixed primary
template in each location, the set of
Equations[\ref{eqn:locinv}-\ref{eqn:safecont}] are equivalent to
second order conic constraints on the primary offset, upper and lower
interval bounds in each location and some additional varaibles.  This
can be inferred from the Proposition~\ref{lem:zon-socc} and the fact
that the min-approximation and max approximation functions we defined
earlier are affine. So, we first fix the primary template in each
location and solve the aforementioned constraints as a convex program.
The choice of the primary template is explained below.

\tbf{Choosing the primary template.}  We may collect all or some of
the following vectors in the primary template.
%
\begin{enumerate}
\item Eigenvectors of the transformation matrices of the affine maps
  of the intralocation and interlocation transitions and also
  eigenvectors of products of the matrices.  This is because the
  eigenvectors can capture stable directions of the dynamics.
\item The primary and secondary templates of the augmented complex
  zonotopes which overapproimate the additive disturbance input sets.
  Also, we can incorporate the products of the linear matrices of the
  affine maps with these templates.  This is because the input set and
  its transformations are added in reachable set
  computation.
\item Orthogonal projections of the above vectors on the null
  space of the subparallelotopic template.  This is because the
  proposed intersection in Lemma~\ref{lem:acz-int} is exact when the
  primary template belongs to the null space of the subparallelotopic
  template.
\item Adding any set of arbitrary vectors will only increase the chance of computing a desired
  invariant, but at a computational expense.  This is because the
  scaling factors will be adjusted accordingly by the optimizer. 
\end{enumerate}
