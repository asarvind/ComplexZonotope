In this section, we first derive a sufficient condition for positive
invariance of an augmented complex zonotope.  Also, we state
conditions for containment of an initial set and satisfaction of
polytopic safety constraints.  Latter, we explain how compute the
augmented complex zontope based on these conditions.

Earlier, we had computed the linear transformations and Minkowki sums
of augmented complex zonotope and possible overapproximations of their
intersection with subparalleotopic constraints.  Accordingly, we can
compute the overapproximation of the reachable set of an augmented
complex zonotope as another augmented complex zonotope.  Then, we
utilize the partial order given in Definition~\ref{defn:gcz-order} to
  deduce a sufficient condition for positive invariance, as follows.
%
We consider a state set $\hybridset$ given as, for a location
  $\loc\in\locationset$,
  $\hybridset_\loc=\real\lt(\gcz{V_\loc}{c_\loc}{s_\loc}{\pinv{\ptemplate}_\loc}{l_\loc}{u_\loc}\rt)$.
  Let us consider that the additive input for an intralocation
  transition in any location $\loc\in\locationset$ is overapproximated
  as
  $\inputset_\loc\subseteq \gcz{V^{in}_\loc}{c^{in}_\loc}{s^{in}_\loc}{W^{in}_\loc}{l^{in}_\loc}{u^{in}_\loc}$.
  Similarly, for an edge $\edge\in\edgeset$, let the additive input
  set be
  overapproximated as
  $\edgeinp_\edge\subseteq \gcz{V^{in}_\edge}{c^{in}_\edge}{s^{in}_\edge}{W^{in}_\edge}{l^{in}_\edge}{u^{in}_\edge}$.
  Furthermore, for any $\loc\in\locationset$, the safe set in the
  location is $\safeset_\loc=\polytope{T_\loc}{d_\loc}$ and the
  initial set is
  $\mc{I}_\loc=\real\lt(\gcz{V^I_\loc}{c^I_\loc}{s^I_\loc}{W^I_\loc}{l^I_\loc}{u^I_\loc}\rt).$
  


\begin{lemma}[Positive invariance]
  The condition for positive invariance of the state set $\hybridset$
  is the following.
\begin{enumerate}
\item For any location $\loc\in\locationset$, the inclusion
  $\locationtransition{\loc}\lt(\hybridset_\loc\rt)\subseteq
  \hybridset_\loc$ holds if all of the below statements are
  collectively true.
\begin{align}~\label{eqn:locinv}
\begin{split}
& \pinv{\ptemplate}_\loc c_\loc = 0~~\text{i.e., primary offset is orthogonal to
  secondary template},\\
&  \text{there exist real
    vectors}~
  c^\pr_\loc,s^\pr_\loc,l^\pr_\loc,u^\pr_\loc,l^\dpr_\loc,u^\dpr_\loc~\text{such
    that}\\
\end{split}
\vspace{-1em}
\end{align}
/* intersection with staying conditions and one continuous step  */
\vspace{-1em}
\begin{align}
\begin{split}
& c^\pr_\loc = \map_\loc c_\loc+c^{in}_\loc,~~s^\pr_\loc =
  \ColumnJoin{s_\loc}{s^{in}_\loc}\\
& l^\pr_\loc =
  \ColumnJoin{\maxaffine{l_\loc}{\stay^-_\loc}}{~~~~~~~~~l^{in}_\loc~~~},~~u^\pr
  =
  \ColumnJoin{\minaffine{u_\loc}{\stay^+_\loc)}}{~~~~~~~~~u^{in}_\loc~~~}
\end{split}
\end{align}
/*  inclusion condition */
\begin{align}
\begin{split}
& \gcz{\lt[\begin{array}{cc}\map_\loc V_\loc & ~~V^{in}_\loc\end{array}\rt]}{c^\pr_\loc}{s^\pr_\loc}
          {\lt[\map_\loc\pinv{\ptemplate}_\loc~~W^{in}_\loc\rt]}{l^\pr_\loc}{u^\pr_\loc}
 \order
   \gcz{V_\loc}{c_\loc}{s_\loc}{\pinv{\ptemplate}_\loc}{l^\dpr_\loc}{u^\dpr_\loc} \\
%\end{align}
%\begin{align}
& \maxaffine{l^\dpr_\loc}{\stay^-_\loc}\geq l_\loc~\text{and}~~
\minaffine{u^\dpr_\loc}{\stay^+_\loc}\leq u_\loc.
\end{split}
 \end{align}
\item For any edge $\edge\in\edgeset$, the inclusion
  $\edgetransition{\edge}\lt(\hybridset_{\preloc{\edge}}\rt)
  \subseteq \hybridset_{\postloc{\edge}}$ holds if 
  all of the below statements are collectively true.
\begin{align}~\label{eqn:edgeinv}
\begin{split}
& \pinv{\ptemplate}_{\preloc{\edge}}c_{\preloc{\edge}} =
  0~~\text{i.e., primary offset is orthogonal to
  secondary template},\\
& \text{there exist real
    vectors}~c^\pr_{\postloc{\edge}},s^\pr_{\postloc{\edge}},l^\pr_{\postloc{\edge}},u^\pr_{\postloc{\edge}},l^\dpr_{\postloc{\edge}},u^\dpr_{\postloc{\edge}}~~\text{such
  that}
\end{split}
\end{align}
/* intersection with staying condition and guard of current location */
\vspace{-0.9em}
\begin{align}
& c^\pr_\edge = \edgemap_\edge c_{\preloc{\edge}}+c^{in}_\edge,~~s^\pr_\edge =
  \ColumnJoin{s_{\preloc{\edge}}}{s^{in}_\edge}
\end{align}
\vspace{-1.5em}
\begin{align}
& l^\pr_\edge =
  \ColumnJoin{\maxaffine{l_{\preloc{\edge}}}{\stay^-_{\preloc{\edge}}\bigvee\loweredgebound{\edge}}}{~~~~~~~~~l^{in}_{\edge}~~~},~~u^\pr =
  \ColumnJoin{\minaffine{u_{\preloc{\edge}}}{\stay^+_{\preloc{\edge}}\bigwedge\upperedgebound{\edge}}}{~~~~~~~~~u^{in}_\edge~~~}
\end{align}
/*  intersection with staying condition of target location and inclusion condition */
\begin{align}
& \gcz{\lt[\edgemap_\edge V_{\preloc{\edge}}~~V^{in}_{\edge}\rt]}{c^\pr_\edge}{s^\pr_\edge}
          {\lt[\edgemap_\edge\pinv{\ptemplate}_{\preloc{\edge}}~~W^{in}_\edge\rt]}{l^\pr_\edge}{u^\pr_\edge}
 \order
   \gcz{V_{\postloc{\edge}}}{c_{\postloc{\edge}}}{s_{\postloc{\edge}}}{\pinv{\ptemplate}_{\postloc{\edge}}}{l^\dpr_\edge}{u^\dpr_\edge} \nonumber \\
& \maxaffine{l^\dpr_\edge}{\stay^-_{\postloc{\edge}}}\geq l_{\postloc{\edge}}~\text{and}~~
\minaffine{u^\pr_\edge}{\stay^+_{\postloc{\edge}}}\leq u_{\postloc{\edge}}.
\end{align}
\vspace{-1.5em}
\end{enumerate}
\end{lemma}
%
Next, we state a sufficient condition for an augmented complex
zonotopic state set to contain an initial set overapproximated by an
augmented complex zonotope.  This is given by the inclusion relation
between augmented complex zonotopes from Lemma~\ref{lem:gcz-gcz}.
\begin{lemma}
 For a location $\loc \in\locationset$, $\mc{I}_\loc\subseteq
  \hybridset_\loc$ if,
\begin{align}~\label{eqn:initcont}
\gcz{V^I_\loc}{c^I_\loc}{s^I_\loc}{W^I_\loc}{l^I_\loc}{u^I_\loc}\order\gcz{V_\loc}{c_\loc}{s_\loc}{\pinv{\ptemplate}_\loc}{l_\loc}{u_\loc}.
\end{align}
\end{lemma}
%
For satisfaction of polytopic safety constraints by an augmented
complex zonotope, the following lemma gives a sufficient condition,
which is just the reformulation of~\ref{lem:polylimits-acz} in
the below context.
%
\begin{lemma}
For any location $\loc\in\locationset$,
  $\hybridset_\loc\subseteq \safeset_\loc$ if,
\begin{align}~\label{eqn:safecont}
T_\loc\lt(c_\loc+\pinv{\ptemplate}_\loc\lt(\frac{u_\loc+l_\loc}{2}\rt)\rt)+\lt|T\lt[V_\loc,~\pinv{\ptemplate}_\loc\rt]\rt|\ColumnJoin{\hspace{1.5em}s}{\frac{u_\loc-l_\loc}{2}}\leq d_\loc.
\end{align}
\end{lemma}
%
By simply collecting all the results of this section for computing a safe
positive invariant, we state the following theorem.
%
\begin{theorem}~\label{thm:main} If
  $\forall \loc\in\locationset$ and $\forall \edge\in\edgeset$, all of
  the Equations[\ref{eqn:locinv}-\ref{eqn:safecont}] are collectively
  true, then the state set $\hybridset$ is a positive invariant,
  satisfies the given safety constraints and contains the given
  initial set
\end{theorem}

\tbf{Solving the conditions.}  Firstly, we note that the secondary
template in a location is predefined as the pseudoinverse of the
subparallelotopic template in the location, in accordance with the
above results in this section.  Then, we observe that for a fixed
primary template in each location, the set of
Equations[\ref{eqn:locinv}-\ref{eqn:safecont}] are equivalent to
second order conic constraints on the primary offset, upper and lower
interval bounds in each location and some additional varaibles.  This
can be inferred from the Proposition~\ref{lem:zon-socc} and the fact
that the min-approximation and max approximation functions we defined
earlier are affine. So, we first fix the primary template in each
location and solve the aforementioned constraints as a convex program.
The choice of the primary template is explained below.

\tbf{Choosing the primary template.}  We may collect all or some of
the following vectors in the primary template.
%
\begin{enumerate}
\item Eigenvectors of the transformation matrices and their products, for the
   different transition maps.  This is because the eigenvectors can
  possibly capture some of the stable directions of the dynamics.
\item The primary and secondary templates of the augmented complex
  zonotopes which overapproimate the additive disturbance input sets.
  Also, we can incorporate the products of the linear matrices of the
  transition maps with these templates.  This is because the input set and
  its transformations are added in reachable set
  computation.
\item Orthogonal projections of the above vectors on the null
  space of the subparallelotopic template.  This is because the
  proposed intersection in Lemma~\ref{lem:acz-int} is exact when the
  primary template belongs to the null space of the subparallelotopic
  template.
\item Adding any set of arbitrary vectors will only increase the chance of computing a desired
  invariant, but at a computational expense.  This is because the
  scaling factors will be adjusted accordingly by the optimizer. 
\end{enumerate}
