
In a discrete-time affine hybrid system, we have a finite set of
discrete variables, called locations, and a finite set of continuous
variables whose valuation is in the real Euclidean space of dimension
$n\in\pint$.  In each location, there are a set of linear constraints,
called \emph{staying conditions}, within which the continuous state of
the system in that location is constrained.  Furthermore, there is an
affine transition map with (possibly) additive uncertain but bounded
disturbance input specifying the evolution of the continuous
variables. A set of labeled directed edges specify possible discrete
transitions between locations, accompanied by affine reset map on
continuous variables with a bounded additive disturbance input. Each
edge transition is controlled by a set of linear constraints on the
continuous variables, called guards.

In this paper, we consider a specific class of linear constraints
called, sub-parallelotopic, for defining guards and staying
conditions, such that their intersection with the reachable set
represented by augmented complex zonotopes (introduced later) can be
effectively computed. The sets corresponding to sub-parallelotopic
constraints can be seen as a generalization of parallelotopes to
possibly unbounded sets.  We discuss the aforementioned intersection
operation later after defining augmented complex zonotopes.
%
\begin{definition}[Sub-parallelotope]~\label{defn:sub-parallelotope} Let
  $K\in\mat{k}{n}{R}$ such that $k\leq n$ and $\lt(KK^T\rt)$ is
  non-singular.  We call such a matrix $K$ a
  \emph{sub-parallelotopic template}.  Let
  $\wh{u},\wh{l}\in\comprealset^n$ such that $\wh{u}\leq\wh{l}$.  Then
  the following is a sub-parallelotopic set.
\[
\sptope{K}{\wh{l}}{\wh{u}} = \lt\{x\in\realset^n: \wh{l}\leq Kx \leq \wh{u}\rt\}
\]
\end{definition}
%
For example, the set of linear constraints $-1\leq x+y-z\leq
1~\wedge~~ x-y+z\leq 3$ is equivalent to a sub-parallelotope
$$\sptope{\ColumnJoin{\lt[1~~~1~-1\rt]}{\lt[1~-1~1\rt]}}{\ColumnJoin{-1}{-\infty}}{\ColumnJoin{1}{3}},$$
because the rows of the sub-parallelotopic template are linearly
independent.  On the other hand, the set of constraints $-1\leq
x+y-z\leq 1~\wedge~~x+y+z\leq 2\wedge~~-1\leq x+y$ do not constitute a
sub-parallelotope, because the three row vectors $\lt[\begin{array}{c
c c}1 & 1 & -1\end{array}\rt]$, $\lt[\begin{array}{c c c}1 & 1 &
1\end{array}\rt]$, and $\lt[\begin{array}{c c c}1 & 1 &
0\end{array}\rt]$ together are linearly dependent. 

%Nevertheless, for many examples of affine hybrid systems, the guards and staying conditions can be specified by sub-parallelotopes.


\tbf{System model.}
We consider discrete-time affine hybrid systems defined by a tuple 
%
\[
\system =
\lt(\locationset,\ptemplate,\stay,\linearmapset,\inputset,\edgeset\rt).
\]
%
Here, $\locationset$ is a finite set of locations.  For each location
$\loc\in\locationset$, a sub-parallelotopic template
$\ptemplate_\loc\in\mat{k_q}{n}{\realset}$, i.e.,
$\ptemplate_\loc\lt(\ptemplate_\loc\rt)^T$ is non-singular, and $k(q)$
is the number of rows of the template, is used for defining the
staying conditions and the guards on edges emanating from the
location.  Then, a pair of upper and lower bounds
$\stay_\loc=\lt(\stay^-_\loc,\stay^+_\loc\rt)\in\mb{R}^{k_q}\times\mb{R}^{k_q}:
~~\stay^-_\loc\leq\stay^+_\loc$ together with the sub-parallelotopic
template define the sub-parallelotopic staying set as
$\sptope{\ptemplate_\loc}{\stay^-_\loc}{\stay^+_\loc}$.  The matrix
$A_\loc$ and a bounded set $\inp_\loc\subseteq\realset^n$ define the
linear transformation and the additive input set in the location.  The
set of edges is $\edgeset$, where $\edge\in\edgeset$ is a tuple $\edge
= \lt(\preloc{\edge},\postloc{\edge},\loweredgebound{\edge},\upperedgebound{\edge},\edgemap_\edge,\edgeinp_\edge\rt)$.
The pre and post locations of the edge are
$\preloc{\edge}\in\locationset$ and $\postloc{\edge}\in\locationset$,
respectively.  The pair of upper and lower bounds
$\lt(\edge^-,\edge^+\rt)\in\realset^{k_{\preloc{\edge}}}\times\realset^{k_{\preloc{\edge}}}:~~\edge^-\leq\edge^+$,
gives the sub-parallelotopic guard set
$\sptope{\ptemplate_{\preloc{\edge}}}{\edge^-}{\edge^+}$, which
is a precondition on the edge transition.  The matrix $\edgemap_\edge$
and a bounded set $\edgeinp_\edge\subseteq\realset^n$, respectively,
give the linear transfomation and the additive input set for all edge
(interlocation) transitions.

\tbf{Dynamics.}
The state of the hybrid system is a pair $(x,\loc)$, where
$x\in\realset^n$ is called the continuous state and
$\loc\in\locationset$ is called the discrete state.  The
evolution of the state of the system in time is called a
\emph{trajectory} of the system.  The trajectory is a function
$\systrj{x}{\loc}:\wholenums\ra\realset^n\times\locationset$, such
that for all $t\in\wholenums$, one of the following is true.

\begin{enumerate}
\item Intralocation dynamics.
\begin{align}~\label{eqn:intralocation}
\begin{split}
& \exists u\in\inputset_{\trj{\loc}{t}}~~\text{such that all of
    the following  are collectively true.}\\
& \trj{x}{t+1} = \map_{\trj{\loc}{t}}\trj{x}{t}+u,~~~\trj{\loc}{t+1} = \trj{\loc}{t}~~
\text{and}\\
& \trj{x}{t},~\trj{x}{t+1}\in\sptope{\ptemplate_{\trj{\loc}{t}}}{\stay^-_{\trj{\loc}{t}}}{\stay^+_{\trj{\loc}{t}}}.
\end{split}
\end{align}
\item Interlocation dynamics.
\begin{align} 
\begin{split}
& \exists \edge\in\edgeset~\text{and}~u\in\edgeinp_{\edge}~\text{such
that all of the following are collectively true.}\\
& \trj{\loc}{t}=\preloc{\edge},~~~\trj{x}{t}\in\sptope{\ptemplate_{\preloc{\edge}}}{\loweredgebound{\edge}\bigvee\stay^-_{\preloc{\edge}}}{\upperedgebound{\edge}\bigwedge\stay^+_{\preloc{\edge}}} \\
& \trj{x}{t+1} = \edgemap_{\trj{\loc}{t}}\trj{x}{t}+u,~~~\trj{\loc}{t+1}
= \postloc{\edge}\\
& \trj{x}{t+1}\in \sptope{\ptemplate_{\postloc{\edge}}}{\stay^-_{\postloc{\edge}}}{\stay^+_{\postloc{\edge}}}.
\end{split}
\end{align}
\end{enumerate}

Given a set of continuous states $S\in\realset^n$, we compute the set
of reachable continuous states in the next time step of intralocation transition
in a location $\loc\in\locationset$ or interlocation transition along
an edge $\edge\in\edgeset$, by the functions
$\locationtransition{\loc}:2^{\realset^n}\ra 2^{\realset^n}$ or
$\edgetransition{\edge}:2^{\realset^n}\ra 2^{\realset^n}$,
respectively, defined as
\begin{align*}
&\locationtransition{q}\lt(S\rt) = \lt\{\Calign{\lt(\map_{\loc}\lt(S\bigcap\staysptope{\loc}\rt)\oplus
\inputset_\loc                   \rt)}{~~\bigcap~~\staysptope{\loc}}\rt..\\
&\edgetransition{\edge}\lt(S\rt) =  \lt\{\Calign{\lt(\edgemap_\edge\lt(S\bigcap
\guardsptope{\edge}\rt)\oplus\edgeinp_\edge\rt)}{~~\bigcap~~\staysptope{\postloc{\edge}}}\rt..
\end{align*}

We shall identify a set of states by a mapping of the kind
$\hybridset:\locationset\ra 2^{\realset^n}$, called a \emph{state
set}, which corresponds to the set of states
$\lt\{\lt(x,\loc\rt):x\in\hybridset\lt(\loc\rt)\rt\}$.  For notational
convenience, we shall denote $\Gamma_\loc$ as the set of continuous
states of $\Gamma$ in a location $\loc$.  A \emph{positive invariant}
is a set of states of the system such that all trajectories beginning
at any state in the positive invariant remain within the positive
invariant.  Equivalently, a state set is a positive invariant if the
reachable set in one time step by both the intralocation and
interlocation dynamics is contained within the original state set.
\begin{definition}
A state set $\hybridset$ is a positive invariant if
the following is true.
\[
 \forall\loc\in\locationset,~~\locationtransition{q}\lt(\hybridset_\loc\rt) \subseteq \hybridset_\loc~\label{eqn:pi1}~\text{and}~~
 \forall\edge\in\edgeset,~~\edgetransition{\edge}\lt(\hybridset_{\preloc{\edge}}\rt) \subseteq
  \hybridset_{\postloc{\edge}}.
\]
\end{definition}
