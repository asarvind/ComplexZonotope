Before introducing augmented complex zonotope, we briefly discuss
related set representations.  Usual zonotopes are Minkowski sums of
line segemnts, represented as a linear combination of real vectors,
called \emph{generators}, such that the corresponding combining coefficients
are bounded in real valued intervals.
\begin{definition}[Real zonotope]
Let $W\in\mat{n}{k}{\mb{R}}$ and $l,u\in\mb{R}^m: l\leq u$.  Then the
following is a real zonotope.
\begin{equation*}
\zon{V}{l}{u} = \lt\{V\zeta: \zeta\in\mb{R}^k,~\zeta_j\in[l_j,u_j]~\forall j\in \tup{k}\rt\}
\end{equation*}
\end{definition}

To incorporate information about a possibly complex eigenstructure of
linear maps while finding invariants,  the template complex zonotope
set representation was introduced in~\cite{todo} that has complex
vectors as generators with complex combining coefficients bounded in
their absolute values.
\begin{definition}[Template complex zonotope]
Let $V\in\mat{n}{m}{\mb{C}}$ and $s\in\mb{R}^m_{\geq 0}$ and
$c\in\mb{C}^n$.  Then the following is a template complex zonotope.
\begin{equation*}
\cz{V}{c}{s} =
\lt\{V\epsilon:\epsilon\in\mb{C}^m,~\lt|\epsilon_i\rt|\leq s_i~\forall
i\in\tup{m}\rt\}
\end{equation*}
\end{definition}

In affine hybrid systems, the state transitions are controlled by some
linear constraints on the states.  If such linear constraints are
obtained by a linear transformation of product of constrained
intervals along each dimension, then we call them as paralleotopes.
To be more general, we may consider sets obtained by linear
transformation of product of intervals which can be either constrained
or unconstrianed, and call them as sub-parallelotopes.  This set
representation is defined as follows.
%
\begin{definition}[Sub-parallelotope]~\label{defn:sub-parallelotope} Let
  $K\in\mat{k}{n}{R}$ such that $k\leq n$ and $\lt(KK^T\rt)$ is
  non-singular, which we call \emph{sub-parallelotopic template}.  Let
  $\wh{u},\wh{l}\in\comprealset^n$ such that $u\leq l$.  Then the following
  is a sub-parallelotopic set.
\[
\sptope{K}{\wh{l}}{\wh{u}} = \lt\{x\in\realset^n: \wh{l}\leq Kx \leq \wh{u}\rt\}
\]
\end{definition}

One major drawback of template complex zonotopes and zonotopes is that
intersection with linear constraints does not necessarily give a
template complex zonotope or a usual zonotope.  Furthermore, there is
no efficient way of abstracting such an intersection opearation as a
zonotope or template complex zonotope.  In contrast for
sub-parallelotopes, if the linear constraints are alligned with the
sub-parallelotope, then the intersection can be exactly specified as a
sub-parallelotope.  However, a sub-parallelotope has
an equivalent representation as a zonotope, as follows.
\begin{proposition}
We have the equivalence $\sptope{K}{l}{u}=\zon{\pinv{K}}{l}{u}$.
\end{proposition}

Motivated by the above equivalence between sub-parallelotopes and
zonotopes, we introduce \emph{augmented complex zonotope}, which is
a Minkowski sum of template complex zonotope and usual (real)
zonotope.  Then, the real zonotopic part of the augmented complex
zonotope can be used to represent parallelotopes, in which case
intersection with linear constraints can be specified exactly under
some assumptions.  The assumptions under which such an intersection
can be exactly specified will be explained later.  Firstly, we define
augmented complex zonotopes, where a subset of the combining
coefficients and generators are complex and bounded in absolute
values, while the rest are real and bounded in real intervals.

\begin{definition}[Augmented complex zonotope]
Let $V\in\mat{n}{m}{C}$ called primary template, $W\in\mat{n}{k}{R}$
called secondary template, $c\in\mb{R}^n$ called primary offset,
$s\in\mb{R}^m$ called scaling factors, $u,l\in\mb{R}^k$ called lower
and upper interval bounds, respectively, such that $l\leq u$.  The
following is an augmented complex
zonotope.
\begin{multline}
\gcz{V}{c}{s}{W}{l}{u} =
\lt\{
  c+V\epsilon+W\zeta:\epsilon\in\mb{C}^m,\zeta\in\mb{R}^k,\rt.\\ \lt.  \lt|\epsilon_i\rt|\leq
 s_i~\forall i\in\tup{m},~\zeta_j\in[l_j,u_j]~\forall j\in \tup{k}
\rt\}
\end{multline}
\end{definition}
