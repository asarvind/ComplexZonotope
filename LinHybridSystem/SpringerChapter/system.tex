\section{Switched linear system with additive
  disturbances}\label{sec:switched}

Formally, a discrete-time switched linear system with additive disturbances
is specified by a tuple $\sys=\sw{L}$, where $L$ is a finite
set of modes, $\mapp: L\ra\mRnn$ is a mapping whose image is called the
set of switching matrices, $\inp$ is a bounded additive
disturbance set and $\init$ is the initial set.
A function $\tbf{x}: \mb{Z}_{\geq 0}\ra \mb{R}^n$
is called a trajectory of the system if there exists a switching signal
$\sigma:\mb{Z}_{\geq 0}\ra L$ and a disturbance signal $w:\mb{Z}_{\geq
  0}\ra \inp$ such that
\begin{equation}\label{eqn:sw-system}
\begin{split}
& \tbf{x}(t+1)=\mapp(\sigma(t))\tbf{x}(t)+w(t)~\text{and}~
 \tbf{x}(0)\in \init.
\end{split}
\end{equation}
Here $\sigma(t)\in L$ is the mode of the system at time $t$. In this
work, we consider arbitrary switching signals. 

From now on for simplicity, by a \emph{switched system} we refer to a
discrete-time switched linear system with additive disturbances,
unless otherwise explicitly stated.  For any $l\in L$, we use the
notation $\mapp(l)=A^l$ and call $A^l$ the switching matrix of mode
$l$.
Starting from a set $\Psi$, the set of all points that can be reached at
the next time instant is denoted by $R(\Psi)$.  We can generalize this to
complex sets as well, according to the following definition where
$\Psi\subseteq \mCn$,
%
%\begin{equation}\label{eqn:sw-reach}
$R(\Psi)=\bigcup_{l\in L}\lt(A^l\Psi\oplus\inp\rt)$.
%\end{equation}
%\subsubsection{Positive invariant}
Then a possibly complex set is called positive invariant of the
switched system if
the reachable set at the next time instant is contained within the original
set, and it also contains the initial set.  
%
\begin{defn}\label{defn:pi}[Positive invariant]
  A set $\Psi\subseteq \mCn$ is called a positive invariant of the
  switched system~(\ref{eqn:sw-system}) if $\init\subseteq \Psi$ and
  $\forall l\in L:\; \lt(A^l\Psi\oplus\inp\rt)\subseteq \Psi$.
\end{defn}
%
\paragraph*{Verification of linear invariance properties}
A linear invariant of a switched system is a collection of linear
constraints such that every trajectory starting in the initial set
always satisfies the linear constraints. We consider the problem
of verifying a linear invariance property around the origin.  This is
motivated by the fact that analysis of switched linear systems is
often done around the origin which is an equilibrium state.  The
results we present in the following can indeed be extended to the case
where the origin may not satisfy the linear constraints of the
property, however not presented in this paper.  Before we define a
linear invariance property, we introduce the following notation.

Since we consider linear
invariants containing the origin and the system dyanmics is symmetric
about the origin, a linear invariance property can be defined by
polyhedral norm constraints, as follows.
%
\begin{defn}[Linear invariant]
Let $k\in\pint$, $T\in\mat{k}{n}{R}$ and $b\in\nzrl$.  A tuple $\lt(T,b\rt)$ is called a linear invariance property if
$\forall t\in\pint,~x\in R^t(\Psi)$ and for every trajectory
$\tbf{x}:\mb{R}_{\geq 0}\ra\mb{R}^n$ such that $\trj{0}\in\init$, we have
$\|T\trj{t}\|_{\infty}\leq b.$
\end{defn}
%
From now on, we assume that $T\in\mat{k}{n}{R}$ and $b\in\nzrl$.  For
the set of linear constraints to be invariant, we require that every
trajectory of the system beginning in the initial set does not violate
the constraints at all times of during its evolution.  In other words,
the reachable set of the initial set has to be a positive invariant
satisfying the linear constraints.  On the other hand, if we find a
positive invariant satisfying the linear constraints, then the
trajectories of the system are bounded by the linear constraints since
they are bounded by the positive invariant.  trajectory of the system
always satisfies the constraints.  Thus, we have the following
proposition.
%
\begin{prop}[Condition for linear invariance]\label{prop:pi-li}
The set of constraints ${\lcons}$ is a linear invariant iff there
exists a positively invariant set $\Psi$ such that $\forall x\in real(\Psi)$,
we have
$\|Tx\|_{\infty}\leq b$.
\end{prop}
%
Therefore, to verify a linear invariance property, it
suffices to find a positively invariant set containing an initial
set.  This defines our problem.

%
\emph{Problem.} %[Linear invariance verification by finding positive invariants]
  For $T\in\mat{k}{n}{R}$ and $b\in\nzrl$, find a positively invariant
  set $\Psi$ such that $\forall x\in\Psi$, we have
  $\|Tx\|_{\infty}\leq b$.  Then $(T,b)$ is a linear
  invariant of~(\ref{eqn:sw-system}) by Proposition~\ref{prop:pi-li}.


Before describing our solution, we briefly review related work.




\paragraph*{Related work}  Firstly, we remark that the techniques for
computing positive invariants in the absence of additive disturbance
may not be able to handle the presence of additive disturbance.
Hence, we compare here with the previous work for computing positive
invariants for switched systems when additive disturbance is present,
which is the problem considered in this paper.  This can be divided
into two categories, one uses ellipsoids and the other
polyhedra. Given that additive disturbance is present, earlier
bi-convex constraints (non-convex conditions) were derived for
computing \emph{minimum norm ellipsoidal positive invariants} for
linear systems~\cite{polyak2006rejection}, whose number increases with
the number of switching matrices.  As such, positively invariant
ellipsoid based verification of linear invariants of a switched system
with additive disturbances requires handling biconvex constraints that
can be computationally expensive.  In contrast, using template complex
zonotopes, we derive second-order conic constraints that can be
efficiently solved.  %% A recent approach based on ellipsoidal join
%% operations~\cite{allamigeon2015scalable} demonstrates efficiency in
%% computation of ellipsoidal invariants for switched systems without
%% additive disturbance. But this work does not handle additive
%% disturbance and extending it to systems with additive disturbance
%% requires abstraction of Minkowski sums as ellipsoids. In comparison,
%% our approach can efficiently handle additive disturbances because
%% template complex zonotopes are closed under the Minkowski sums
%% operation, for which inclusion relationship with respect to the
%% original template complex zonotope can be expressed by efficiently
%% solvable SOCP constraints on scaling factors.
Concerning
\emph{polyhedral invariants}, \cite{kouramas2005minimal,DBLP:conf/aplas/DangG11} propose an
iterative outer-approximation of the reachable set until a fixpoint is
reached. But the procedure may take many iterations to converge to a fixpoint.  In
contrast, our approach involves a single step of second-order conic
optimization.




 


More concretely, we verify invariance of linear constraints by finding
positively invariant template complex zonotopes satisfying a linear
constraint. First, we extend the linear constraints to complex-valued
vectors, leading to conic constraints, and state the following
necessary and sufficient condition for their satisfaction by a
template complex zonotope.
%
\begin{lem}\label{lem:sat}
  $\forall x\in \czo{V}{s}$, the inequality
  $\|Tx\|_{\infty}\leq b$ holds if and only if
  $\|TV\dg(s)\|_{\infty}\leq b$.
\end{lem}
%
\paragraph*{Procedure for verifying a linear invariant property:} We
want to find a positively invariant template complex zonotope
$\czo{V}{s}$ satisfying a given set of linear constraints
$\lt<T,b\rt>$ so as to verify a linear invariance property.
Firstly, we fix the template $V$, which is generally the collection of
eigenvectors of the switching matrices, although additional vectors
can be added to the template to have more accuracy in verification.
Next, we want to synthesize scaling factors $s$ such that $\czo{V}{s}$
is a positively invariant template complex zonotope satisfying the
linear constraints.  For positive invariance, we formulate second-order conic constraints as sufficient
conditions for checking the required set inclusion relations in
Definition~\ref{defn:pi}.  Additionally, we formulate again second-order
conic constraints for satisfaction of the linear constriants by the
template complex zonotope, based on Lemma~\ref{lem:sat}.  We then
minimize the offset $b$ of the linear invariance property for which
the second-order conic constraints are satisfied using convex
optimization.  If the $b$ we derive is less that the given value for
the linear invariance property, then we have verified the property.
Otherwise, our result is inconclusive because we our conditions are
sufficient but not necessary.

%based on Theorem~\ref{toref} 


%
%% \begin{thm}[Sufficient condition for verifying linear invariant]\label{theorem:swiched}
%%   Given a switched system $\sys=\sw{L}$ where the disturbance and initial sets are respectively $\inp=\czo{G}{h}$ and
%%   $\init=\czo{W}{z}$ (with the template $G\in\mat{n}{r}{C}$, $W\in\mat{n}{p}{C}$ and the scaling factors
%%   $h\in\mb{R}^r_{\geq 0}$ and $z\in\mb{R}^p_{\geq 0}$), a tuple $\lcons$, where $T\in\mat{k}{n}{R}$ and
%%   $b\in\mRo$, is linear invariant of $\sys$ if all the following set
%%   of constraints are collectively satisfied, for a given $V \in\mat{n}{m}{C}$.
%% {\allowdisplaybreaks
%% %\begin{equation}\label{eqn:cons}
%% \begin{align*}
%% & \exists s\in\mb{R}^m_{\geq 0}~(\text{scaling factors}),~
%% \exists X^l\in\mat{n}{(m+r)}{C}~\text{for each}~l\in L~\text{and}\\
%% &\exists Y\in\mat{n}{p}{C}~~\text{such that}:~~
%% VX^l=\lt[A^lV~~G\rt]\dg\cjoin{s}{h},~~
%% VY=W\dg(z), \displaybreak[0] \\
%% & \sum_{j=1}^{m+r}|X^l_{ij}|\leq s_i~~~~~~~\forall
%%   i\in\tup{m}, %% \wedge\forall l\in L
%%   ~~~~~~~~~~~\text{\textsl{(inclusion after switching)}}\\
%% & \sum_{k=1}^p|Y_{ik}|\leq s_i,~~~~~~~\forall
%%   i\in\tup{m}~~~~~~~~~~~~~\text{\textsl{(containment of
%%       initial set)}}\\
%% & \|TV\dg(s)\|_{\infty}\leq b,~~~~~~~~~~~~~~~~~~~~~~~~~~~~~~~~~~\text{\textsl{(satisfaction of linear
%%     constraints)}}
%% \end{align*}
%% %\end{equation}
%% }
%% %
%% \end{thm}
%% %
%% In fact, for a fixed $T$, we can minimize $b$ for which the constraints in the 
%% theorem are satisfied using second-order conic
%% optimization.

%% \tbf{Choosing the template.} In the restricted case of linear systems
%% without switching, a positive invariant exists if and only if a
%% template complex zonotope having eigenvectors as the template is
%% positively invariant. Motivated by this, for a switched system,
%% we can choose the template as the collection of eigenvectors of
%% the switching matrices.  However if required, we can add new direction
%% vectors to the template to increase the chances of
%% successful verification.



