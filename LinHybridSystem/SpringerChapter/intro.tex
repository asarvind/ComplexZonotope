Embedded control systems combine computer software with the physical
world, and their global behavior can be modeled using hybrid systems.
To assert correctness of such systems, their global behavior under all
possible non-determinism arising in the interaction between continuous
and discrete dynamics should be accurately analyzed. Thus, one of the
key ingredients for safe design and verification of embedded control
systems is a set representation which, on one hand, is expressive
enough to describe the evolution of sets of hybrid trajectories and,
on the other hand, can be manipulated by time-efficient
algorithms. For most hybrid systems with non-trivial continuous
dynamics, exact computation of hybrid trajectories is impossible, so
the focus is put on approximate computation of reachable states. In
the area of abstract interpretation within program verification, there
is a similar need for data structures for set manipulation, called
abstract domains, which should be fine-grained enough to be accurate,
yet computationally tractable to deal with complex programs.
%\vspace{0.2em}
%The likelihood of finding a positive invariant and its
%computational efficiency depend on the abstract domain used for
%representing the positive invariant.  

 
Two classical abstract domains are intervals~\cite{CousotCousot76-1}
and convex polyhedra~\cite{DBLP:conf/popl/CousotH78}, and their
variants have been developed to achieve a good compomise between
computational speed and precision, such as
zones~\cite{DBLP:conf/pado/Mine01},
octagons~\cite{DBLP:journals/lisp/Mine06}, linear
templates~\cite{VMCAI05}, zonotopes~\cite{HSCC05}, and tropical
polyhedra~\cite{DBLP:conf/sas/AllamigeonGG08}. For hybrid model
checking, convex polyhedra and their special classes such as
parallelotopes and zonotopes are also among popular set
representations.  Beyond polyhedral set representations, ellipsoids
can be used for reachable set computations~\cite{KurzhanskiVaraiya00}.
In abstract interpretation, polynomial inequalities are used for
invariant computation via their reduction to linear inequalities
in~\cite{bagrodzafSAS05} and polynomial equalities via Gr\"{o}ner
basis methods~\cite{Rodriguez-Carbonell:2007}. Quadratic templates are
also proposed, where semi-definite relaxations are used for deriving
non-linear invariants (for instance quadratic invariants inspired by
Lyapunov functions)~\cite{Feron2010,AdjeGaubertGoubaultESOP2010}.

Recently, complex zonotopes were introduced in~\cite{arvind2016lis} as
extension of usual zonotopes to handle dynamics with complex
eigenvectors.  The efficiency of complex zonotopes is demonstrated for
for stability verification of nearly periodic linear impulsive
systems. However, a drawback of complex zonotopes is that adding new
generators can distort the positive invariance of a set.  Therefore,
refining these sets in a verification method is not easy.
 


%% Recently, complex zonotopes~\cite{arvind2016lis} extend usual zonotopes to the complex domain, which geometrically speaking are Minkowski sum of line segments and some ellipsoids.
%% Other extensions of zonotopes, such as quadratic~\cite{DBLP:conf/aplas/AdjeGW15} and more general
%% polynomial zonotopes~\cite{althoff2013} have been proposed. Complex zonotopes are however different from polynomial zonotopes because while a polynomial zonotope is a set-valued polynomial function of
%% \emph{intervals}, a complex zonotope is a set valued function of unit
%% \emph{circles} in the complex plane.
%% \vspace{0.2em}


%% In this paper, we extend complex zonotopes to a new set
%% representation called template complex zonotopes, which has more
%% efficient computational procedures and broader applicability.  The
%% interest of complex zonotopes can be explained as follows. An
%% important problem in abstract interpretation and hybrid system
%% verification is computing \emph{positive invariants}, which can be
%% used as over-approximations of reachable sets. For linear systems,
%% positive invariants are closely related to the their
%% eigenstructures. For instance, when the eigenvectors of a linear
%% system are real, positively invariant polytopes and zonotopes can
%% be defined by considering the eigenvectors as normal vectors of the
%% faces of the polytopes and generators of the zonotopes,
%% respectively. Then, for hybrid systems with aribitrary switching
%% among linear dynamics having real eigenvectors, the existence of
%% such eigenvector-based positively invariant polytopes and zonotopes
%% for the subsystems is a necessary condition for the existence of a
%% positive invariant of the overall system. However, this analysis
%% using real-valued polytopes and zonotopes can no longer be used if
%% the eigenvectors are complex (having both imaginary and real
%% parts). To handle dynamics with complex eigenvectors, complex
%% zonotopes are introduced in~\cite{arvind2016lis} and their
%% efficiency is demonstrated for stability verification of nearly
%% periodic linear impulsive systems.  \vspace{0.2em}


 %% does not necessarily increase the likelihood of finding positive
 %% invariants because this can make the sets unnecessarily large in some
 %% directions 
Overcoming the afore drawback, in this paper we introduce a
more general set representation called \emph{template complex
zonotopes}, which has more efficient verification procedures and
broader applicability.  In a {template complex zonotope}, the bounds
on the complex combining coefficients, called \emph{scaling factors},
are treated as variables, while the directions for the generators are
fixed a-priori by a \emph{template} of complex vectors. Consequently,
compared to complex zonotopes, template complex zonotopes provide
better approximation quality by allowing adjusting the scaling factors
for the generators. Thus using template complex zonotopes leads to
significantly more efficient approach to verification of stability of
nearly periodic linear impulsive systems.  The second contribution of
this paper is an application of template complex zonotopes to verify
linear invariance properties of switched systems with additive
disturbance, which indicates their potential application to a broader
class of hybrid systems.  %% Both the classes of hybrid systems
%% considered in this paper are widely used models for design and
%% implementation of embedded control systems (see for example
%% \cite{blanchini2015switching,naghshtabrizi2007delay,2008-naghshtabrizi-exponential}
%% and references there in). The effectiveness of the proposed methods is supported by
%% experiments on benchmark examples.
% \vspace{0.2em}

%% The remainder of the paper is organized as follows. We first describe
%% template complex zonotopes and the computation of some important
%% operations required by the verification procedures. Then, we present a
%% method which is improved compared to the work~\cite{arvind2016lis},
%% for stability verification of linear impulsive systems using template
%% complex zonotopes. In the next section, we show a procedure for
%% verification of linear invariance properties of switched systems with
%% additive disturbance. Finally, to demonstrate the efficiency of
%% template complex zonotopes, we report experimental results which are
%% either better or competitive, compared to the existing
%% results. Related work on switched systems and linear impulsive systems
%% and a comparison with our approach will be discussed in the respective
%% sections. Before continuing, we summarize the basic notation used in
%% the paper.
%% \vspace{0.1em}

%% {\bf Basic notation.} We represent integers by $\mb{Z}$, real numbers
%% by $\mb{R}$ and complex numbers by $\mb{C}$.  For integers $p$ and
%% $q$, the set of $p\times q$ matrices with entries drawn from a set
%% $\mb{Q}$ is denoted $\mat{p}{q}{Q}$.  If $z$ is a vector of complex
%% numbers, $real(z)$ represents the real part of $z$ and the imaginary
%% part is $img(z)$.  For a positive integer $i$, the $i^{th}$ component
%% of $z$ is denoted by $z_i$.  The diagonal square matrix containing
%% entries of $z$ along the diagonal is denoted by $\dg(z)$.  For a matrix
%% $X$ and positive integers $i, j$, $X_{ij}$ is the $i^{th}$ row and
%% $j^{th}$ column entry of $X$.  If $c$ is a scalar complex number, its
%% absolute value is $|c|=\lt(|real(c)|^2+|img(c)|^2\rt)^{1/2}$.  The
%% infinity norm of a possibly complex $n$-dimensional vector $z$ is
%% $\|z\|_{\infty}=\max_{i=1}^n{|z_i|}$. The infinity norm of a possibly
%% complex $n\times m$ matrix $X$ is $\max_{i=1}^{n}\sum{|X_{ij}|}$.



