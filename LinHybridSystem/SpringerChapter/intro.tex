Embedded control systems combine computer software with the physical
world, and their global behavior can be modeled using hybrid systems.
To assert correctness of such systems, their global behavior under all
possible non-determinism resulting from the interaction between
continuous and discrete dynamics should be accurately analyzed. Thus,
one of the key ingredients for safe design and verification of
embedded control systems is a set representation which, on one hand,
is expressive enough to describe the evolution of sets of hybrid
trajectories and, on the other hand, can be manipulated by
time-efficient algorithms. For most hybrid systems with non-trivial
continuous dynamics, exact computation of hybrid trajectories is
impossible, so the focus is put on approximate computation of
reachable states. In the area of abstract interpretation within
program verification, there is a similar need for data structures for
set manipulation, called abstract domains, which should be
fine-grained enough to be accurate, yet computationally tractable to
deal with complex programs.
 
Two classical abstract domains are intervals~\cite{CousotCousot76-1}
and convex polyhedra~\cite{DBLP:conf/popl/CousotH78}, and their
variants have been developed to achieve a good compomise between
computational speed and precision, such as
zones~\cite{DBLP:conf/pado/Mine01},
octagons~\cite{DBLP:journals/lisp/Mine06}, linear
templates~\cite{VMCAI05}, zonotopes~\cite{HSCC05}, and tropical
polyhedra~\cite{DBLP:conf/sas/AllamigeonGG08}. For hybrid model
checking, convex polyhedra and their special classes such as
parallelotopes and zonotopes are also among popular set
representations.  Beyond polyhedral set representations, ellipsoids
can be used for reachable set computations~\cite{KurzhanskiVaraiya00}.
In abstract interpretation, polynomial inequalities are used for
invariant computation via their reduction to linear inequalities
in~\cite{bagrodzafSAS05} and polynomial equalities via Gr\"{o}ner
basis methods~\cite{Rodriguez-Carbonell:2007}. Quadratic templates are
also proposed, where semi-definite relaxations are used for deriving
non-linear invariants (for instance quadratic invariants inspired by
Lyapunov functions)~\cite{Feron2010,AdjeGaubertGoubaultESOP2010}.
Recently, complex zonotopes~\cite{arvind2016lis} extended usual
zonotopes to the complex domain, which geometrically speaking are
Minkowski sum of line segments and some ellipsoids.  Other extensions
of zonotopes, such as quadratic~\cite{DBLP:conf/aplas/AdjeGW15} and
more general polynomial zonotopes~\cite{althoff2013} have been
proposed. Complex zonotopes are however different from polynomial
zonotopes because while a polynomial zonotope is a set-valued
polynomial function of
\emph{intervals}, a complex zonotope is a set valued function of unit
\emph{circles} in the complex plane.


Complex zonotopes can utilize the possibly complex eigenstructure
of the dynamics of linear impulsive systems to define
contractive sets for stability verification.  This is an advantage
over polytopes or usual zonotopes that can only utilize the real
eigenstructure but not the complex eigenstructure.  For stabililty
verification of nearly periodic linear impulsive systems, complex
zonotopes demonstrated good accuracy on some benchmark examples.
However, a drawback of complex zonotopes is that adding more
generators to it can violate the property of contraction with respect
to the dynamics.  Therefore, it is not easy to refine complex
zonotopes for verifying stability for larger intervals of sampling
times.  Moreover, we had to heuristically guess a suitable complex
zonotope for stability verification instead of systematically
synthesizing it.

Overcoming the aforementioned drawback, in this paper we introduce a
more general set representation called \emph{template complex
zonotopes}.  In a {template complex zonotope}, the bounds on the
complex combining coefficients, called \emph{scaling factors}, are
treated as variables, while the directions for the generators are
fixed a-priori by a \emph{template} of complex vectors.  This allows
us to systematically synthesize a suitable template complex zonotope
for stability verification, instead of guessing it like in the case of
complex zonotopes.  Furthermore, template complex zonotopes can be
refined easily by adding any arbitrary set of vectors to the existing
template, because the scaling factors can be adjusted accordingly.  We
present experiments on some benchmark examples where template complex
zonotopes contract faster than complex zonotopes, resulting in faster
verification.

The rest of the paper is organized as follows.  We introduce template
complex zonotopes in Section~\ref{sec:tcz} and dicuss important
operations on them used in the stability verification procedure, like
linear transformation inclusion checking.  In
Section~\ref{sec:lis-ex}, we first define a nearly-periodic linear
impulsive system and the problem of verifying stability

* ADD NOTATION 



%% {\bf Basic notation.} We represent integers by $\mb{Z}$, real numbers
%% by $\mb{R}$ and complex numbers by $\mb{C}$.  For integers $p$ and
%% $q$, the set of $p\times q$ matrices with entries drawn from a set
%% $\mb{Q}$ is denoted $\mat{p}{q}{Q}$.  If $z$ is a vector of complex
%% numbers, $real(z)$ represents the real part of $z$ and the imaginary
%% part is $img(z)$.  For a positive integer $i$, the $i^{th}$ component
%% of $z$ is denoted by $z_i$.  The diagonal square matrix containing
%% entries of $z$ along the diagonal is denoted by $\dg(z)$.  For a matrix
%% $X$ and positive integers $i, j$, $X_{ij}$ is the $i^{th}$ row and
%% $j^{th}$ column entry of $X$.  If $c$ is a scalar complex number, its
%% absolute value is $|c|=\lt(|real(c)|^2+|img(c)|^2\rt)^{1/2}$.  The
%% infinity norm of a possibly complex $n$-dimensional vector $z$ is
%% $\|z\|_{\infty}=\max_{i=1}^n{|z_i|}$. The infinity norm of a possibly
%% complex $n\times m$ matrix $X$ is $\max_{i=1}^{n}\sum{|X_{ij}|}$.



