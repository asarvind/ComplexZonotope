




We evaluated our algorithm on two benchkmark examples of linear
impulsive systems below and compared it with other state-of-the-art
approaches.  For convex optimization, we use CVX version 2.1 with
Matlab 8.5.0.197613 (R2015a).  The reported experimental results were
obtained on Intel(R) Core(TM) i5-3470 CPU @ 3.20GHz.

\emph{Example 1.} We consider a networked control system with
uncertain but bounded transmission period.  A networked control system
is composed of a plant and a controller that interact with each other
by transmission of feedback input from controller to the plant.  If
the system dynamics is linear with linear feedback, then for uncertain
but bounded transmission period, we can equivalently represent it
as a linear impulsive system where $A_c=\lt(\begin{array}{ccc} A_p & 0 & B_p\\ 0 & 0 &
  0\\ 0 & 0 & 0
\end{array}\rt),~A_r=\lt(\begin{array}{ccc}
\mb{I} & 0 & 0\\
B_oC_p & A_o & 0\\
D_oC_p & C_o & 0
\end{array}\rt)$ for some parameter matrices $A_p$, $B_p$, $B_o$,
$C_p$, $A_o$, $C_o$, and $D_o$.  The sampling interval $\Delta$ of the
linear impulsive system specifies bounds on the transmission interval.
%
Our example of a networked control system is taken from Bj\"{o}rn et
al.~\cite{wittenmark2002computer}.  The system is originally described
by discrete time transfer functions, which has an equivalent state
space representation with parameter matrices
$A_p=\lt(\begin{array}{cc}-1 & 0\\ 1 & 0\end{array}\rt)$,
  $B_p=\lt(\begin{array}{c}1\\ 0\end{array}\rt)$,
    $C_p=\lt(\begin{array}{cc}0 & 1\end{array}\rt)$, $A_o=0.4286$,
      $B_o=-0.8163$, $C_o=-1$ and $D_o=-3.4286$.  Given the lower
      bound on the transmission period as $t_{min}=0.8$, we want to
      find as high a value of $t_{max}$ as possible for which the
      system is GES.

\begin{table}
\begin{minipage}{0.49\textwidth}
\caption{Example 1}
\begin{tabular}{|l|c|r|}
  \hline
  Reference & $t_{min}$ & $t_{max}$ \\
\hline
  %& &\\
Value recommended in~\cite{wittenmark2002computer} & 0.08 & 0.22\\
  \hline
  %& &\\
  NCS toolbox~\cite{BauLoo_NECSYS12a} & 0.08 & 0.4 \\
  \hline
  %& &\\
Complex zonotope~\cite{arvind2016lis} & 0.08 & 0.5 \\
\hline
  %& &\\
  Template complex zonotope & 0.08 & 0.58 \\
  \hline
\end{tabular}
\label{tab:com1}
\end{minipage}
\begin{minipage}{0.49\textwidth}
\caption{Example 2}
\label{tab:com2}
\begin{tabular}{|l|c|r|}
\hline
    Reference & $t_{min}$ & $t_{max}$ \\
  \hline
  %& &\\
  Lyapunov, parametric LMI~\cite{2013hetel} & 0.1 & 0.3 \\
  \hline
  %& &\\
  Polytopic set contractiveness~\cite{2014-fiacchini-set} & 0.1 & 0.475 \\
  \hline
  %& &\\
Khatib et al.~\cite{AlKhatib2015} & 0.1 & 0.514\\
\hline
  %& &\\
Complex zonotope~\cite{arvind2016lis} & 0.1 & 0.49 \\
\hline
 % & &\\
  Template complex zonotope & 0.1 & 0.496 \\
  \hline
\end{tabular}
%\label{tab:com2}
\end{minipage}
\vspace{1em}
\caption{Template complex zonotopes (TCZ) vs
    Complex zonotopes (CZ)~\cite{arvind2016lis}}
\center
 \begin{tabular}{|l|c|r|}
  \hline
  $~$ & CZ & TCZ $~~~~~~~~$\\
  \hline
  Finding suitable zonotope & Requires guessing & Systematically synthesized\\
\hline
  No. of impulses for contraction & 2 (both examples) & 1 (both examples)\\
  \hline
    Computation time (Example 1) & 27.41 s & 14.9443 s \\
\hline
Computation time (Example 2) & 74.04 s & 10.6097 s\\
\hline
\end{tabular}
\label{tab:tcz-cz}
\end{table}

\emph{Example 2.} We consider the following linear impulsive system
from Hetel et. al.~\cite{2013hetel}, that describes an LMI based
approach to verify stability. The specification is given by
$A_c=\left(\begin{array}{ccc} 0 & -3 & 1\\ 1.4 & -2.6 & 0.6\\ 8.4 &
  -18.6 & 4.6
\end{array}\rt)
$ and $A_r=\lt(\begin{array}{ccc} 1 & 0 & 0\\ 0 & 1 & 0\\ 0 & 0 & 0
\end{array}\rt).$

\emph{Setting and Results.}  While implementing the algorithm for
stability verification, we used order Taylor expansion, a tolerance of
$tol=0.01$ for Example 1 and $tol=0.006$ for Example 2.  We required
$k=3$ number of reachability operators for both examples, for
synthesizing a suitable template complex zonotope used in checking
contraction.  We could verify exponential stability in a sampling
interval $[0.08, 0.58]$ for Example 1 and $[0.1,0.496]$ for Example 2.
The comparison of our approach with the state-of-the-art NCS
toolbox~\cite{BauLoo_NECSYS12a} and also other approaches is presented in
Tables~\ref{tab:com1} and~\ref{tab:com2}.  For the first example, our
method outperforms other approaches, while it is competitive with
other approaches on the second example.  Furthermore,
Table~\ref{tab:tcz-cz} shows that the template complex zonotope based
approach is also faster than the complex zonotope approach
of~\cite{arvind2016lis}.








