



\subsection{Nearly periodic linear impulsive systems}\label{sec:lis-ex}
%Experiments for linear impulsive systems

We evaluated the implementation of our algorithm on a number of
benchmark examples taken from~\cite{arvind2016lis} and compared it
with the state-of-the-art methods and tools and our algorithm produced
either better or competitive results.  We describe two of the examples
for linear impulsive systems below.

\begin{table*}
\caption{Template complex zonotopes (TCZ) vs
    Complex zonotopes (CZ)~\cite{arvind2016lis}}
\center
 \begin{tabular}{|l|c|r|}
  \hline
  $~$ & CZ & TCZ $~~~~~~~~$\\
  \hline
  Finding suitable zonotope & Requires guessing & Systematically synthesized\\
\hline
  No. of impulses for contraction & 2 (both examples) & 1 (both examples)\\
  \hline
    Computation time (Example 1) & 27.41 s & 4.6 s \\
\hline
Computation time (Example 2) & 74.04 s & 8.5 s\\
\hline
\end{tabular}
\label{tab:tcz-cz}
\end{table*}


\emph{Example 1.} We consider a networked control system with
uncertain but bounded transmission period.  A networked control system
is composed of a plant and a controller that interact with each other
by transmission of feedback input from controller to the plant.  If
the system dynamics is linear with linear feedback, then for uncertain
but bounded transmission period, we can equivalently represent it
as a linear impulsive system where $A_c=\lt(\begin{array}{ccc} A_p & 0 & B_p\\ 0 & 0 &
  0\\ 0 & 0 & 0
\end{array}\rt),~A_r=\lt(\begin{array}{ccc}
\mb{I} & 0 & 0\\
B_oC_p & A_o & 0\\
D_oC_p & C_o & 0
\end{array}\rt)$ for some parameter matrices $A_p$, $B_p$, $B_o$,
$C_p$, $A_o$, $C_o$, and $D_o$.  The sampling interval $\Delta$ of the
linear impulsive system specifies bounds on the transmission interval.
%
Our example of a networked control system is taken from Bj\"{o}rn et
al.~\cite{wittenmark2002computer}.  The system is originally described
by discrete time transfer functions, which has an equivalent state
space representation with parameter matrices
$A_p=\lt(\begin{array}{cc}-1 & 0\\ 1 & 0\end{array}\rt)$,
  $B_p=\lt(\begin{array}{c}1\\ 0\end{array}\rt)$,
    $C_p=\lt(\begin{array}{cc}0 & 1\end{array}\rt)$, $A_o=0.4286$,
      $B_o=-0.8163$, $C_o=-1$ and $D_o=-3.4286$.  Given the lower
      bound on the transmission period as $t_{min}=0.8$, we want to
      find as high a value of $t_{max}$ as possible for which the
      system is GES.

\emph{Example 2.} We consider the following linear impulsive system
from Hetel et. al.~\cite{2013hetel}, that describes an LMI based
approach to verify stability. The specification is given by
$A_c=\left(\begin{array}{ccc} 0 & -3 & 1\\ 1.4 & -2.6 & 0.6\\ 8.4 &
  -18.6 & 4.6
\end{array}\rt)
$ and $A_r=\lt(\begin{array}{ccc} 1 & 0 & 0\\ 0 & 1 & 0\\ 0 & 0 & 0
\end{array}\rt).$




While implementing the algorithm for stability
verification, we choose first order Taylor expansion, $tol=0.05$ for
Example 1 and $tol=0.006$ for Example 2.  We required $k=3$ as the
number of reachability operators used for synthesis of a suitable
template complex zonotope for stability verification in both examples.
Then we verified stability in a sampling interval $[0.08, 0.5]$ for
Example 1 and $[0.1,0495]$ for Example 2.  Our approach is compared
with the state-of-the-art NCS toolbox~\cite{BauLoo_NECSYS12a} and also
other approaches as listed in Table~\ref{tab:com1} and
Table~\ref{tab:com2}.  For the first example, our method outperforms
other approaches, while it is competitive with other approaches on the
second example.  Moreover, Table~\ref{tab:tcz-cz} shows that the
template complex zonotopic approach is computationally more efficient
than complex zonotopes~\cite{arvind2016lis}.

\begin{table}
{\caption{Example 1}
\begin{tabular}{|l|c|r|}
  \hline
  Reference & $t_{min}$ & $t_{max}$ \\
\hline
  %& &\\
Value recommended in~\cite{wittenmark2002computer} & 0.08 & 0.22\\
  \hline
  %& &\\
  NCS toolbox~\cite{BauLoo_NECSYS12a} & 0.08 & 0.4 \\
  \hline
  %& &\\
Complex zonotope~\cite{arvind2016lis} & 0.08 & 0.5 \\
\hline
  %& &\\
  Template complex zonotope & 0.08 & 0.53 \\
  \hline
\end{tabular}
\label{tab:com1}
}
\end{table}
\begin{table}
{\caption{Example 2}
\label{tab:com2}
\begin{tabular}{|l|c|r|}
\hline
    Reference & $t_{min}$ & $t_{max}$ \\
  \hline
  %& &\\
  Lyapunov, parametric LMI~\cite{2013hetel} & 0.1 & 0.3 \\
  \hline
  %& &\\
  Polytopic set contractiveness~\cite{2014-fiacchini-set} & 0.1 & 0.475 \\
  \hline
  %& &\\
Khatib et al.~\cite{AlKhatib2015} & 0.1 & 0.514\\
\hline
  %& &\\
Complex zonotope~\cite{arvind2016lis} & 0.1 & 0.49 \\
\hline
 % & &\\
  Template complex zonotope & 0.1 & 0.496 \\
  \hline
\end{tabular}
%\label{tab:com2}
}
\end{table}

%\end{figure}


%\begin{figure}[htbp]


%

%% For $k=3$, we chose the template as the collection of eigenvectors of
%% $k$-sampled reachability operators and found scaling factors
%% satisfying~(\ref{eqn:syn}) by SOCP.  Using this template complex
%% zonotope, we could verify contraction in a sampling interval $[0.1,
%%   0.495]$ according to the Algorithm~\ref{alg:1step} with first order
%% Taylor expansion, i.e., $r=1$.  This required $6.9076$ seconds.  But
%% the LMI based approach~\cite{2013hetel} could verify stability only
%% until 0.3.  The comparison with other approaches is given in
%% Table~\ref{tab:com2}.  The comparison with complex zonotopes is shown
%% in Table~\ref{tab:tcz-cz}.




