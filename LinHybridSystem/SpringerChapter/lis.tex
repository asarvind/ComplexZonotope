\section{Nearly periodic linear impulsive system}\label{sec:lis}
A nearly periodic linear impulsive system is a hybrid system
whose state evolves continuously by a linear differential equation
for some bounded time period, after which there is an instantaneous
linear impulse.  Formally, a linear impulsive system is specified by
a tuple $\mc{L}=\left<A_c,A_r,\Delta\right>$ where $A_c$ and $A_r$ are
$n\times n$ real matrices called the linear field matrix and impulse
matrix, respectively.  The positive integer $n$ is the dimension of
the state space and $\Delta=[t_{min},t_{max}]$ is an interval of
non-negative reals, called the sampling period interval.  The
dynamics of the nearly periodic linear impulsive system is described
as follows.  A function
$\tbf{x}:\mathbb{R}_{\geq 0}\ra \mathbb{R}^n$ is called a trajectory
of the system if there exists a sequence of sampling times
$(t_i)_{i=1}^\infty$ satisfying all the following:
%
	\begin{equation}\label{eqn:system} \begin{split}
&(t_{i+1}-t_i)\in\Delta~~\forall i\in\pint~~~~(\text{uncertainty in sampling period})\\
&\dot{\tbf{x}}(t)=A_c\tbf{x}(t)~~\forall
            t\in\nzrl:~t\neq t_i ~\forall i\in\pint~~(\text{continuous
              })\\
&\tbf{x}(t_i^+)=A_r\tbf{x}(t_i^-)~~\forall
            i\in\pint~~~~(\text{linear impulse})           
\end{split} \end{equation}
%
%% 	Here $\tbf{x}(t)\in\mathbb{R}^n$ is the state of the system at a time instant
%% $t$.  We denote the set of all possible trajectories of the system by $\Gamma$.
%% \vspace{0.2em}

We say that the linear impulsive system is globally exponentially
stable (GES) if all the trajectories of the system beginning at any
point in the state space eventually reach arbitrarily close to the
origin at an exponential rate, as %% Formally, GES is defined as
follows.
%
\begin{defn}[Global exponential stability]
  The system~(\ref{eqn:system}) is
  globally exponentially stable (GES) if there exist positive scalars
  $c>0$ and $\lambda\in[0,1)$ such that for all $t\in\mRo$,
  $\trj{t}\leq c\lambda^t\|\trj{0}\|$.
\end{defn}
We state the stability verification problem as follows.
%
\begin{problem}[Stability verification problem]
Given $A_c$, $A_r$, $t_{min}$, find the largest upper bound $t_{max}$ on the sampling time to guarantee exponential stability.
\end{problem}
%

\paragraph*{Related work on stability verification of nearly periodic linear impulsive systems} 

%A general approach to verifying exponential
%stability of nearly-periodic linear impulsive systems is to find
%decreasing Lyapunov functions
%~\cite{2013-briat-convex,Mikheev1988,Teel1998,2010-liu-stability,Mazenc2013,2010-fridman-refined}. If
%the Lyapunov functions are quadratic, they can be reduced to Linear
%Matrix Inequality (LMI) conditions~\cite{HetelDaafouz2006}.  Another
%approach is to find contractive
%sets~\cite{2014-fiacchini-set,Lazar2013,AlKhatib2015}, which is the
%inspiration to our work.  


A common approach to this problem is extending Lyapunov techniques, which results in Lyapunov Krakovskii functionals 
\cite{Mikheev1988,Teel1998} (using the framework of time-delay systems), and discrete-time Lyapunov functions \cite{2012-seuret-novel}. Stability with respect to time-varying input delay can also be handled by input/output approach 
\cite{DBLP:conf/amcc/KaoW14}. Stability verification problem for time-varying impulsive systems can also be formulated in a hybrid systems framework \cite{nevsic2004framework,Goebel2009,BauLoo_NECSYS12a}, for which various Lyapunov-based 
methods including discontinuous time-independent \cite{2008-naghshtabrizi-exponential} 
or time-dependent Lyapunov functions \cite{2010-fridman-refined} were developed. Another approach involves using convex embedding \cite{HetelDaafouz2006,Fujioka2009,2013hetel}. In this approach, stability conditions can be checked using parametric Linear Matrix Inequalities (LMIs) \cite{HetelDaafouz2006}, 
or as set contractiveness (such as, polytopic set contractiveness) \cite{2014-fiacchini-set,2013-briat-convex,AlKhatib2015}. Inspired by these results on 
set contractiveness conditions~\cite{arvind2016lis} provides a stability condition, expressed in terms of complex zonotopes, which is more conservative but can be efficiently verified. The novelty of this work is in the extension of complex zonotopes to template complex zonotopes which allows a systematic way to synthesize contractive zonotopic sets to verify stability. 

%Computationally speaking, our approach is close in spirit to abstract interpretation
%and hybrid systems analysis. Indeed the way we find contractive sets using such zonotopes is similar 
%to the way invariant sets are computed using some zonotopic
%\cite{Girard05reachabilityof,Althoff2011,DBLP:conf/sas/GoubaultPV12} and
%template-polyhedral abstract domains \cite{S riram2008,jeannet2009apron}.

%\cite{DBLP:conf/amcc/KaoW14,Omran2013}


%% To show exponential stability, we use reachability analysis as
%% follows.  Beginning at a state $x$ and by continuous evolution for a
%% time period $t$, the system reaches the state $e^{A_ct}x$, which we
%% obtain by integrating the continuous dynamics of the system.  On the
%% other hand, after a linear impulse at a state $x$, the system reaches
%% the state $A_rx$.  Overall, 

The state reached after a linear impulse followed by a continuous
evolution for time $t$ is $e^{A_ct}A_rx$.  So, let us denote
$H_t=e^{A_ct}A_r$ for any any positive real $t$, which we call as the
reachability operator if $t$ lies in the sampling period interval
$\Delta$.  Given a set $\Psi\subset\mathbb{R}^n$, the set of all
reachable points of $\Psi$, when acted upon by an impulse followed by
continuous evolution for sampling time period $t\in\Delta$ until
before the next impulse, is $\bigcup_{t\in\Delta}H_t\Psi$.  It was
shown previously in~\cite{2014-fiacchini-set,AlKhatib2015} that a
necessary and sufficient condition for exponential stability is the
existence of a convex, compact and closed set containing the origin in
its interior, called a $C$-set, that contracts between subsequent
impulses.  In other words, we can establish exponential stability by
finding a $C$-set $\Psi$ such that $H_t\Psi\subseteq\lambda\Psi$ for
some $\lambda\in[0,1)$ and for all $t\in\Delta$.  In this paper, we
  want to find contractive $C$-sets represented as template complex
  zonotopes.  We define \emph{contraction} of a
template complex due to a linear operator as follows, which when less
than one, implies that the zonotope is contractive.
%
\begin{defn}
For a template complex zonotope $\CZO$, the amount of contraction by a
square matrix $J\in\mat{n}{n}{R}$, denoted by $\chi(V,s,J)$
is \[\chi(V,s,J)=\min\lt\{\lambda\in\mRo:J\czo{V}{s}\subseteq
\lambda\czo{V}{s}\rt\}.\]
\end{defn}
%
Our motivation for considering
template complex zonotopes can be
inferred from the following proposition.  
%
\begin{prop}~\label{prop:eig-cont}
  Let $V$ contain only the eigenvectors of $H_t$ as its column vectors
  and $\mu$ be the vector of eigenvalues corresponding the columns of
  $V$.  Then $H_t\CZO=\czo{V}{\dg(|\mu|)s}$.
\end{prop}
%
For a fixed sampling time period, i.e., when $t_{min}=t_{max}=t$, we
can infer from the above proposition that the contraction of the
template complex zonotope formed by the eigenvector template is
bounded by the largest absolute value of the eigenvalues.  
Therefore, when the sampling period is fixed, we can find contractive
template complex zonotopes for exponentially stable systems by
choosing the template as the collection of eigenvectors.  However, we
are interested in the case where the sampling time period varies in
the interval $\Delta$, i.e. there are uncountably many reachability
operators parametrized over the time interval $\Delta$. Motivated by
the above analysis, when the sampling period is uncertain, we choose
the template as the collection of eigenvectors of a few reachability
operators and try to synthesize suitable scaling factors for which the
template complex zonotope contracts with respect to the chosen finite
set of reachability operators.  However, later we also verify that the
synthesized template complex zonotope actually contracts with respect
to all the (uncountably many) reachability operators.  First we
describe the procedure to synthesize the template complex zonotope.

 \emph{Synthesizing a candidate template complex
  zonotope.} This step of systematically synthesizing a suitable
template complex zonotope that is likely to be contractive constitutes
the main improvement over the procedure proposed
in~\cite{arvind2016lis}. Our criterion for synthesizing the template
complex zonotope is that it has to be contractive with respect to a
few reachability operators (called reference operators), which is a
necessary condition for contraction with respect to the overall system
dynamics.  Since the eigenstructure of the reachability operators are
related to the stability of the system, we include the eigenvectors of
a few reachability operators in the template.  For a fixed number
$k\in\pint$ of reference operators, they can be chosen incrementally
as follows.  Define $k$-sampled time points as
$\Lambda_k=\lt\{t_{min}+i\frac{(t_{max}-t_{min})}{k}:i\in\{0,\ldots,k-1\}\rt\}.$
Then we define the set of $k$-sampled reachability operators as
$\Gamma_k=\lt\{H_{t}:t\in\Lambda_k\rt\}$.  Let us denote the template
of collection of eigenvectors of all operators in $\Gamma_k$ as $E_k$.
For this template, we synthesize suitable scaling factors based on the
following theorem. The derivation of this theorem uses the inclusion
checking condition from Theorem~\ref{thm:inc}.
%
\begin{thm}
If $E_k$ have rank $n$.  For a
vector of scaling factors $s\in\mb{R}^{m}_{\geq 0}$, the template
complex zonotope $\mc{Z}=\czo{E_k}{s}$ would represent a $C$-set and also $H_t\mc{Z}\subseteq
\lambda\mc{Z}$ for $\lambda\in(0,1)$ if
following are all satisfied.
%
\begin{equation}\label{eqn:syn}
\begin{split}
& s\in\mb{R}^n_{\geq 1}~~\text{(sufficient condition for representing $C$-set)}\\
& \exists X_t\in\mat{m}{m}{C}~~\forall t\in\Lambda_k~~\text{s.t.}\\
& E_kX_t=H_tE_k\dg(s)~~~~(\text{transfer matrix condition
    })\\%~(\text{a finite set of time points})\\
& \sum_{j=1}^m\lt|\lt(X_t\rt)_{ij}\rt|\leq \lambda s_i~~\forall i\in\tup{m}~~\text{(bounding contraction)}\\
\end{split}
\end{equation}
\end{thm}

Therefore, we can synthesize a template complex zonotope that is
contractive with respect to a finite chosen number $k>0$ of reference
operators by solving for the scaling factors satisfying the
second-order conic constraints in~(\ref{eqn:syn}).

\emph{Verifying contraction.}  To verify that this synthesized
template complex zonotope actually contracts with respect to all the
reachability operators $H_t$, where $t$ is parametrized over the whole
sampling time interval $\Delta$, we divide the sampling interval into
small enough sub-intervals and verify contraction in each interval.
To bound the amount of contraction in small intervals, we use some
useful properties of contraction, which were earlier derived
in~\cite{arvind2016lis}.

Let $H_{t+\rho}$ be an operator where $\rho$ lies in the interval
$(0,\epsilon)$.  For an order of taylor expansion $r\in\pint$ and some
$\delta\in[0,\epsilon]$, define 
\[P^t_r(\rho)=\sum_{i=0}^r\frac{A_c^i\rho^i}{i!}H_t~~\text{and}~~
E^t_r(\delta)=\frac{A_c^{r+1}\delta^{r+1}}{(r+1)!}H_t.\]  Then based
on Taylor expansion, we get that \[H_{t+\rho} = P_r^t(\rho) +
E_r^t(\delta).\]
%
Furthermore, a bound on contraction as sum of contractions depending on $\epsilon$ is derived 
in~\cite{arvind2016lis} as
%
\begin{equation}\label{eqn:expbound}
\begin{split}
&~~~~~~~~~~~\chi\lt(V,s,H_{t+\rho}\rt)\leq
\lt(\max_{i=0}^r\chi\lt(V,s,P^t_r(\epsilon)\rt)\rt)+ \frac{\epsilon^{r+1}}{(r+1)!}\chi\lt(V,s,A_c^{r+1}H_t\rt)
\end{split}
\end{equation}
%
The right hand side of~(\ref{eqn:expbound}) can be bounded if we know
a bound on the contraction under a linear operation, which is
derived as follows.
%
\begin{lem}\label{lem:contLin}
Let $J\in\mRnn$.  Define 
\[\beta(V,s,J)=\min\{\|X\|_\infty:X\in\mat{m}{m}{C}~\wedge~
V\dg(s)X=JV\dg(s)\}\] Then we have $\chi(V,s,J)\leq \beta(V,s,J)$.
%
\end{lem}
\begin{proof}
If $JV\dg(s)=V\dg(s)X$, we want to prove that $J\czo{V}{s}=\czo{JV}{s}\subseteq \|X\|_\infty\CZO$.  We deduce
$\czo{JV}{s}=\{ JV\zeta \;:\; \forall
  i\in\tup{m}|\zeta_i|\leq s_i \} = \{JV\dg(s)\zeta^\pr:\|\zeta^\pr\|_{\infty} \leq 1\} = \{V\dg(s)X\zeta^\pr:\|\zeta^\pr\|_\infty\leq 1\}\subseteq
  \|X\|_\infty\{V\dg(s)\zeta^\pr:\|\zeta^\pr\|_{\infty}\leq
  1\} = \|X\|_\infty\{V\zeta \;:\; \forall i\in\tup{m}  |\zeta_i|\leq s_i \} = \|X\|_{\infty}\CZO$.
\end{proof}
%
Then the contraction for sampling time interval $(t,t+\delta)$ can be
bounded %% using Equations~(\ref{eqn:expbound}) and Lemma~(\ref{lem:contLin})
as follows.
%
\begin{thm}\label{thm:expbound}
  Let $\rho\in (t,t+\epsilon)$ and for $r\in\mb{Z}_{>0}$, $
  P^t_r(\rho)=\sum_{i=0}^r\frac{A_c^i\rho^i}{i!}H_t$.  Define
  $\eta_r(V,s,t,\epsilon)=
  \lt(\max_{i=0}^r\beta\lt(V,s,P^t_r(\epsilon)\rt)\rt)+
  \frac{\epsilon^{r+1}}{(r+1)!}\beta\lt(V,s,A_c^{r+1}H_t\rt)$, where
  the bound $\beta(.)$ is defined in Lemma~\ref{lem:contLin}.  Then
  $\chi\lt(V,s,H_{t+\rho}\rt)\leq \eta_r(V,s,t,\epsilon) $
\end{thm}
%
\emph{Verification algorithm.} We begin with $k=3$ reference operators
that correspond to the two end points of the sampling interval and the
middle point.  The algorithm first finds suitable scaling factors $s$
that gives a template complex zonotope which contracts with respect to
these reference operators.  Next, it checks whether the contraction of
the template complex zonotope $\czo{E_k}{s}$ with respect to all the
reachability operators is less than one. For checking contraction, we
use the algorithm earlier proposed in ~\cite{arvind2016lis}.  If
successful, then exponential stability of the system is verified.
Otherwise, $k$ is increased, the algorithm synthesizes the template
complex zonotope for the increased $k$ and then checks contraction. If
unsuccessful after a maximum value of $k$, the algorithm stops and the
result is inconclusive.  This is described in
Algorithm~\ref{alg:1step}. This algorithm has been implemented and
we defer experimental results to Section~\ref{sec:exp}.
%
 \begin{algorithm}
\caption{Exponential stability verification of system~(\ref{eqn:system})}
\label{alg:1step}
\begin{algorithmic}[1]
\State Initialize $k=3$.
\State Choose $M\in\pint$ as the largest value of $k$ and $tol$ as
discretization parameter.
\While{$k \leq M$}
\State Initialize $t=t_{min}$, $h=tol$, $r$ as order of Taylor
expansion (typically $\leq 2$).
\While{$h\geq \; tol$ and $t<t_{max}$}
\If{$\eta_r(E_k,s,t,h)<1$}
\State $t\gets t+h; h\gets h+tol$
\Else
\State $h\gets h-tol$
\EndIf
\EndWhile
\If{$t\geq t_{max}$}
\State {\bf|BreakLoop|}
\Else
\State $k\gets k+1$.  
\EndIf
\EndWhile
\If {$k \leq M$}
\State \emph{System is exponentially stable}
\Else
\State \emph{Inconclusive}
\EndIf
\end{algorithmic}
\end{algorithm}



