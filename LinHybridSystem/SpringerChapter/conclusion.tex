\vspace{-2em}
\section{Conclusion}
We extended complex zonotopes to template complex
zonotopes in order to improve the efficiency of the computation of
contractive sets and positive invariants.  Template complex zonotopes
retain a useful feature of complex zonotopes, which is the scope to
incorporate the eigenvectors of linear dynamics among the generators
because the eigenstructure is related to existence of positive
invariants.  In addition, compared to complex zonotopes, the advantage
template complex zonotopes have is the ability to regulate the
contribution of each generator to the set by using the scaling
factors.  Accordingly, we proposed a systematic and more efficient
procedure for verification of stability of nearly periodic impulsive
systems.  The advantage of this new set representation is attested by
the experimental results that are better or competitive, compared to
the state-of-the-art methods and tools on benchmark examples. This
work also contributes a method for exploiting the eigenstructure of
linear dynamics to algorithmically determine template directions,
required by most verification approaches using template sub-polyhedral
sets. A number of directions for future research can be
identified. First, we intend to extend these techniques to analysis to
switched systems under constrained switching laws. Also
computationally speaking, our approach is close in spirit to abstract
interpretation. Indeed the operations used to find contractive sets
can be extended to invariant computation for general hybrid
systems with state-dependent discrete transitions.

