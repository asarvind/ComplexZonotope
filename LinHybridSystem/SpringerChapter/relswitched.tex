



\paragraph*{Related work}  Firstly, we remark that the techniques for
computing positive invariants in the absence of additive disturbance
may not be able to handle the presence of additive disturbance.
Hence, we compare here with the previous work for computing positive
invariants for switched systems when additive disturbance is present,
which is the problem considered in this paper.  This can be divided
into two categories, one uses ellipsoids and the other
polyhedra. Given that additive disturbance is present, earlier
bi-convex constraints (non-convex conditions) were derived for
computing \emph{minimum norm ellipsoidal positive invariants} for
linear systems~\cite{polyak2006rejection}, whose number increases with
the number of switching matrices.  As such, positively invariant
ellipsoid based verification of linear invariants of a switched system
with additive disturbances requires handling biconvex constraints that
can be computationally expensive.  In contrast, using template complex
zonotopes, we derive second-order conic constraints that can be
efficiently solved.  %% A recent approach based on ellipsoidal join
%% operations~\cite{allamigeon2015scalable} demonstrates efficiency in
%% computation of ellipsoidal invariants for switched systems without
%% additive disturbance. But this work does not handle additive
%% disturbance and extending it to systems with additive disturbance
%% requires abstraction of Minkowski sums as ellipsoids. In comparison,
%% our approach can efficiently handle additive disturbances because
%% template complex zonotopes are closed under the Minkowski sums
%% operation, for which inclusion relationship with respect to the
%% original template complex zonotope can be expressed by efficiently
%% solvable SOCP constraints on scaling factors.
Concerning
\emph{polyhedral invariants}, \cite{kouramas2005minimal,DBLP:conf/aplas/DangG11} propose an
iterative outer-approximation of the reachable set until a fixpoint is
reached. But the procedure may take many iterations to converge to a fixpoint.  In
contrast, our approach involves a single step of second-order conic
optimization.




