We consider discrete time hyrbid systems with affine dynamics in each
location, and the guards and staying conditions are specified by
linear inequalities.  Additionally, for any location of the hybrid
system, we consider that the collection of staying conditions and
guards on the edges emanating from the location define a
sub-parallelotope.  Our hybrid system is specified by a tuple
\begin{equation}~\label{eqn:system}
\system =
\lt(\locationset,\linearmapset,\inputset,\ptemplate,\stay,\edgeset\rt)
\end{equation}
where $\locationset$ is a finite set of locations and
$\linearmapset:\locationset\ra\mat{n}{n}{R}$ and
$\inputset:\locationset\ra 2^{\mb{R}^n}$ define the affine dynamics in any
location $\loc\in\locationset$ as 
\begin{equation}~\label{eqn:next}
\trj{x}{t+1} = \map(\loc)\trj{x}{t}+u:~~~u\in\inp(\loc).
\end{equation}
For each location $\loc\in\locationset$,
$\ptemplate(\loc)\in\mat{k(q)}{n}{\realset}$ is the sub-parallelotopic
template used for defining the linear inequalities in the staying
conditions and the guards emanating from the location.  The staying
condition in a location $\loc\in\locationset$ is specified by a pair
of upper and lower bounds,
$\stay(\loc)=\lt(\stay^-(\loc),\stay^+(\loc)\rt)\in\mb{R}^n\times\mb{R}^n:
~~\stay^-(\loc)\leq\stay^+(\loc)$, on the sub-parallelotope
$\sptope{\ptemplate\lt(\loc\rt)}{\stay^-(\loc)}{\stay^+(\loc)}$.  The
set of edges is denoted by $\edgeset$, where an edge
$\edge\in\edgeset$ is denoted as by tuple $\edge =
\lt(\preloc{\edge},\postloc{\edge},\loweredgebound{\edge},\upperedgebound{\edge},\reset{\edge}\rt)$, described as
follows.  The pre and post locations of the edge are
$\preloc{\edge}\in\locationset$ and $\postloc{\edge}\in\locationset$,
respectively.  The pair of upper and lower bounds
$\lt(\edge^-,\edge^+\rt)\in\realset^{k\lt(\preloc{\edge}\rt)}\times\realset^{k\lt(\preloc{\edge}\rt)}:~~\edge^-\leq\edge^+$,
relate to the sub-parallelotopic guard set
$\sptope{\ptemplate\lt(\preloc{\edge}\rt)}{\edge^-}{\edge^+}$.  The
reset map is given by a transformation matrix
$\reset{\edge}\in\mat{n}{n}{R}$.

A \emph{state} of the hybrid system is a pair $(x,\loc)$, where
$x\in\realset^n$ is called the \emph{continuous state} and
$\loc\in\locationset$ is called the \emph{discrete state}.  The
evolution of the state of the system in time is called a
\emph{trajectory}.  Therefore, it is a function
$\systrj{x}{\loc}:\wholenums\ra\realset^n\times\locationset$, such
that for all $t\in\wholenums$, one of the following is true.

\begin{enumerate}
\item Intralocation dynamics.
\begin{align}~\label{eqn:intralocation}
\begin{split}
& \exists u\in\inputset\lt(\trj{\loc}{t}\rt)~~\text{such that}\\
& \trj{x}{t+1} = \map(\trj{\loc}{t})+u\\ 
& \trj{\loc}{t+1} = \trj{\loc}{t}~~
\text{and}\\
& \trj{x}{t},~\trj{x}{t+1}\in\sptope{\ptemplate\lt(\trj{\loc}{t}\rt)}{\stay^-\lt(\trj{\loc}{t}\rt)}{\stay^+\lt(\trj{\loc}{t}\rt)}.
\end{split}
\end{align}
\item Interlocation dynamics.
\begin{align} 
\begin{split}
& \exists \edge\in\edgeset~~\text{such that}\\
& \trj{\loc}{t}=\preloc{\edge},~~\trj{\loc}{t+1}=\postloc{\edge}\\
& \trj{x}{t}\in\sptope{\ptemplate\lt(\preloc{\edge}\rt)}{\max\lt(\loweredgebound{\edge},\stay^-\lt(\preloc{\edge}\rt)\rt)}{\min\lt(\upperedgebound{\edge},\stay^+\lt(\preloc{\edge}\rt)\rt)} \\
& \trj{x}{t+1}\in \sptope{\ptemplate\lt(\postloc{\edge}\rt)}{\stay^-\lt(\postloc{\edge}\rt)}{\stay^+\lt(\postloc{\edge}\rt)}\\
& \trj{x}{t+1} = \reset{\edge}\trj{x}{t}.
\end{split}
\end{align}
\end{enumerate}

The set of reachable continuous states in one time step of
intralocation and interlocation dynamics from a given set of
continuous states is given by the following two functions,
respectively.
\begin{enumerate}
\item For any location $\loc\in\locationset$, define $\locationtransition{\loc}:2^{\realset^n}\ra 2^{\realset^n}$ as
\begin{multline*}
\locationtransition{q}\lt(S\rt) = \lt(\map\lt(\loc\rt)\lt(S\bigcap\staysptope{\loc}\rt)\oplus \inputset\lt(\loc\rt)\rt)\\  \bigcap \staysptope{\loc}.
\end{multline*}
\item For any edge $\edge\in\edgeset$, define
  $\edgetransition{\edge}:2^{\realset^n}\ra 2^{\realset^n}$ as
\begin{multline*}
\edgetransition{\edge}\lt(S\rt) =  \reset{\edge}\lt(S\bigcap \guardsptope{\edge}\rt) \\ \bigcap\staysptope{\postloc{\edge}}.
\end{multline*}
\end{enumerate}

We always identify a set of states by a mapping of the kind
$\hybridset:\locationset\ra 2^{\realset^n}$, called a \emph{state
  set}, which corresponds to the set of states
$\lt\{\lt(x,\loc\rt):x\in\hybridset\lt(\loc\rt)\rt\}$.  A positive
invariant is a set of states of the system such that all trajectories
beginning at any state in the positive invariant remain withing the
positive invariant.  Equivalently, a state set is a positive invariant
if the reachable set in one time step by both the intralocation and
interlocation dynamics is contained within the original state set.
\begin{definition}
A state set $\hybridset$ is a positive invariant if
both the following are true.
\begin{align}~\label{eqn:PosInv}
& \forall\loc\in\locationset,~~\locationtransition{q}\lt(\hybridset(\loc)\rt) \subseteq \hybridset\lt(\loc\rt)\\
& \forall\edge\in\edgeset,~~\edgetransition{\edge}\lt(\hybridset\lt(\preloc{\edge}\rt)\rt) \subseteq
  \hybridset\lt(\postloc{\edge}\rt).
\end{align}
\end{definition}
