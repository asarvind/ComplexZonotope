We consider discrete time hyrbid systems with affine dynamics in each
location, and the guards and staying conditions are specified by
linear inequalities.  Additionally, we also consider that for any
location of the hybrid system, the collection of staying conditions
and guards on the edges emanating from the location define a (possibly
unbounded) parallelotope on a vector sub-space of the state space.  We
call such sets as \emph{sub-parallelotopes}, defined as follows.
%
\begin{definition}~\label{defn:sub-parallelotope} Let
  $K\in\mat{k}{n}{R}$ called \emph{sub-parallelotopic template}
  s.t. $k<n$ and $\lt(KK^T\rt)$ is non-singular.  Let
  $\wh{u},\wh{l}\in\comprealset^n$ s.t. $u\leq l$.  Then the following
  is a sub-parallelotopic set.
\[
\sptope{K}{\wh{l}}{\wh{u}} = \lt\{x\in\realset^n: \wh{l}\leq Kx \leq \wh{u}\rt\}
\]
\end{definition}

Our hybrid system is specified by a tuple 
\begin{equation}~\label{eqn:system}
\system =
\lt(\locationset,\initialset,\linearmapset,\inputset,\ptemplate,\stay,\edgeset\rt)
\end{equation}
where $\locationset$ is a finite set of locations,
$\initialset:\locationset\ra 2^{\mb{R}^n}$ gives the set of initial points
of the system in any location,
$\linearmapset:\locationset\ra\mat{n}{n}{R}$ and
$\inputset:\locationset\ra 2^{\mb{R}^n}$ define the affine dynamics in any
location $\loc\in\locationset$ as 
\begin{equation}~\label{eqn:next}
\trj{x}{t+1} = \map(\loc)\trj{x}{t}+u:~~~u\in\inp(\loc).
\end{equation}
For each location $\loc\in\locationset$, $\ptemplate(\loc)$ is the
sub-parallelotopic template used for defining the linear inequalities
in the staying conditions and the guards emanating from the location.
The staying condition in a location $\loc\in\locationset$ is specified
by
$\stay(\loc)=\lt(\stay^-(\loc),\stay^+(\loc)\rt)\in\mb{R}^n\times\mb{R}^n$
s.t $\stay^-(\loc)\leq\stay^+(\loc)$, where $\stay^-(\loc)$ is the
lower bound and $\stay^+(\loc)$ is the upper bound on the
sub-parallelotopic staying set.  The set of edges is denoted by
$\edgeset$, where an edge $\edge\in\edgeset$ is denoted as by tuple
$\edge =
\lt(\preloc{\edge},\postloc{\edge},\loweredgebound{\edge},\upperedgebound{\edge},\reset{\edge}\rt)$, described as
follows.  The pre and post locations of the edge are
$\edge^1_L\in\locationset$ and $\edge^2_L\in\locationset$,
respectively.  The lower and upper bounds of the sub-parallelotopic
constraints (guard) specified on the edge are denoted
$\edge^-,\edge^+\in\realset^n$, respectively, such that
$\edge^-\leq\edge^+$.  The reset map is given by a transformation
matrix $\edge_r\in\mat{n}{n}{R}$.

A trajectory of the system~$\system$ is a function
$\systrj{x}{\loc}:\wholenums\ra\realset^n\times\locationset$, such
that $\trj{x}{0}\in\initialset\lt(\trj{\loc}{0}\rt)$ and for all
$t\in\wholenums$, one of the following is true.
\begin{enumerate}
\item Intralocation dynamics.
\begin{align}~\label{eqn:intralocation}
\begin{split}
& \exists u\in\inputset\lt(\trj{\loc}{t}\rt)~~\text{such that}\\
& \trj{x}{t+1} = \map(\trj{\loc}{t})+u\\ 
& \trj{\loc}{t+1} = \trj{\loc}{t}~~
\text{and}\\
& \trj{x}{t},~\trj{x}{t+1}\in\sptope{\ptemplate\lt(\trj{\loc}{t}\rt)}{\stay^-\lt(\trj{\loc}{t}\rt)}{\stay^+\lt(\trj{\loc}{t}\rt)}.
\end{split}
\end{align}
\item Interlocation dynamics.
\begin{align} 
\begin{split}
& \exists \edge\in\edgeset~~\text{such that}\\
& \trj{\loc}{t}=\preloc{\edge},~~\trj{\loc}{t+1}=\postloc{\edge}\\
& \trj{x}{t}\in\sptope{\ptemplate\lt(\trj{\loc}{t}\rt)}{\max\lt(\loweredgebound{\edge},\stay^-\lt(\preloc{\edge}\rt)\rt)}{\min\lt(\upperedgebound{\edge},\stay^+\lt(\preloc{\edge}\rt)\rt)}\\
& \trj{x}{t+1}\in \sptope{\ptemplate\lt(\trj{x}{t+1}\rt)}{\stay^-\lt(\postloc{\edge}\rt)}{\stay^+\lt(\postloc{\edge}\rt)}.
\end{split}
\end{align}
\end{enumerate}
