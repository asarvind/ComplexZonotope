We consider discrete time hyrbid systems with affine dynamics in each
location, and the guards and staying conditions are specified by
linear inequalities.  Additionally, for any location of the hybrid
system, we consider that the collection of staying conditions and
guards on the edges emanating from the location define a parallelotope
(possibly unbounded) on a vector sub-space strictly contained in the
state space.  We call such sets as \emph{sub-parallelotopes}, defined
as follows.
%
\begin{definition}~\label{defn:sub-parallelotope} Let
  $K\in\mat{k}{n}{R}$ such that $k<n$ and $\lt(KK^T\rt)$ is
  non-singular, which we call \emph{sub-parallelotopic template}.  Let
  $\wh{u},\wh{l}\in\comprealset^n$ such that $u\leq l$.  Then the following
  is a sub-parallelotopic set.
\[
\sptope{K}{\wh{l}}{\wh{u}} = \lt\{x\in\realset^n: \wh{l}\leq Kx \leq \wh{u}\rt\}
\]
\end{definition}

Our hybrid system is specified by a tuple 
\begin{equation}~\label{eqn:system}
\system =
\lt(\locationset,\linearmapset,\inputset,\ptemplate,\stay,\edgeset\rt)
\end{equation}
where $\locationset$ is a finite set of locations,
$\initialset:\locationset\ra 2^{\mb{R}^n}$ gives the set of initial points
of the system in any location,
$\linearmapset:\locationset\ra\mat{n}{n}{R}$ and
$\inputset:\locationset\ra 2^{\mb{R}^n}$ define the affine dynamics in any
location $\loc\in\locationset$ as 
\begin{equation}~\label{eqn:next}
\trj{x}{t+1} = \map(\loc)\trj{x}{t}+u:~~~u\in\inp(\loc).
\end{equation}
For each location $\loc\in\locationset$,
$\ptemplate(\loc)\in\mat{k(q)}{n}{\realset}$ is the sub-parallelotopic
template used for defining the linear inequalities in the staying
conditions and the guards emanating from the location.  The staying
condition in a location $\loc\in\locationset$ is specified by a pair
of upper and lower bounds,
$\stay(\loc)=\lt(\stay^-(\loc),\stay^+(\loc)\rt)\in\mb{R}^n\times\mb{R}^n:
~~\stay^-(\loc)\leq\stay^+(\loc)$, on the sub-parallelotope
$\sptope{\ptemplate\lt(\loc\rt)}{\stay^-(\loc)}{\stay^+(\loc)}$.  The
set of edges is denoted by $\edgeset$, where an edge
$\edge\in\edgeset$ is denoted as by tuple $\edge =
\lt(\preloc{\edge},\postloc{\edge},\loweredgebound{\edge},\upperedgebound{\edge},\reset{\edge}\rt)$, described as
follows.  The pre and post locations of the edge are
$\preloc{\edge}\in\locationset$ and $\postloc{\edge}\in\locationset$,
respectively.  The pair of upper and lower bounds
$\lt(\edge^-,\edge^+\rt)\in\realset^n\times\realset^n:~~\edge^-\leq\edge^+$
specified on the edge define the sub-parallelotopic guard set
$\sptope{\ptemplate\lt(\preloc{\edge}\rt)}{\edge^-}{\edge^+}$.  The
reset map is given by a transformation matrix
$\edge_r\in\mat{n}{n}{R}$.

A trajectory of the system~$\system$ is a function
$\systrj{x}{\loc}:\wholenums\ra\realset^n\times\locationset$, such
that for all
$t\in\wholenums$, one of the following is true.
\begin{enumerate}
\item Intralocation dynamics.
\begin{align}~\label{eqn:intralocation}
\begin{split}
& \exists u\in\inputset\lt(\trj{\loc}{t}\rt)~~\text{such that}\\
& \trj{x}{t+1} = \map(\trj{\loc}{t})+u\\ 
& \trj{\loc}{t+1} = \trj{\loc}{t}~~
\text{and}\\
& \trj{x}{t},~\trj{x}{t+1}\in\sptope{\ptemplate\lt(\trj{\loc}{t}\rt)}{\stay^-\lt(\trj{\loc}{t}\rt)}{\stay^+\lt(\trj{\loc}{t}\rt)}.
\end{split}
\end{align}
\item Interlocation dynamics.
\begin{align} 
\begin{split}
& \exists \edge\in\edgeset~~\text{such that}\\
& \trj{\loc}{t}=\preloc{\edge},~~\trj{\loc}{t+1}=\postloc{\edge}\\
& \trj{x}{t}\in\sptope{\ptemplate\lt(\preloc{\edge}\rt)}{\max\lt(\loweredgebound{\edge},\stay^-\lt(\preloc{\edge}\rt)\rt)}{\min\lt(\upperedgebound{\edge},\stay^+\lt(\preloc{\edge}\rt)\rt)} \\
& \trj{x}{t+1}\in \sptope{\ptemplate\lt(\postloc{\edge}\rt)}{\stay^-\lt(\postloc{\edge}\rt)}{\stay^+\lt(\postloc{\edge}\rt)}\\
& \trj{x}{t+1} = \reset{\edge}\trj{x}{t}.
\end{split}
\end{align}
\end{enumerate}

A state of the hybrid system is a pair $(x,\loc)$, where
$x\in\realset^n$ and $\loc\in\locationset$.  Then, we call a mapping
$\hybridset:\locationset\ra 2^{\realset^n}$ as a \emph{state set}
because for such a mapping, we can identify a set of states
$\lt\{\lt(x,\loc\rt):x\in\hybridset\lt(\loc\rt)\rt\}$.  We call
$\hybridset\lt(\loc\rt)$ as the projection of $\hybridset$ in the
location $\loc$.

