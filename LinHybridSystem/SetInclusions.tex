


The real projection of an augmented complex zonotope can be
equivalently represented as the real projection of a template complex
zonotope, as follows.  
\begin{lemma}~\label{lem:conversion}
$\real\lt(\gcz{V}{c}{s}{W}{l}{u}\rt) = \real\lt(\cz{\lt[V~W\rt]}{c+W\lt(\frac{u+l}{2}\rt)}{\ColumnJoin{\hspace{0.5em}s}{\frac{u-l}{2}}}\rt)$.
\end{lemma}
Although the real projection of the augmented complex zonotope can
be expressed as a real projection of a template complex zonotope, a
augmented complex zonotope itself can not be represented as as a
template complex zonotope.  Therefore, augmented complex zonotopes
constitute a larger class of sets than template complex zonotopes.

Before stating a sufficient condition for inclusion between augmented complex
zonotopes, we review an inclusion relation between template complex
zonotopes, which was earlier derived in~\cite{todo}.
\begin{lemma}[Inclusion: template complex
  zonotopes]~\label{lem:zon-zon} The inclusion between template complex zonotopes,
  $\cz{V^\pr_{n\times m^\pr}}{c^\pr}{s^\pr}\subseteq \cz{V_{n\times m}}{c}{s}$ holds if,
\begin{align}
\begin{split}
& \exists X\in\mat{m}{m^\pr}{\mb{C}}~\text{and}~y\in\mb{C}^{m}~\text{s.t.}\\
& \transfer{V}{V^\pr}{s^\pr}{X},~~~\centertransfer{V}{c}{c^\pr}{y}\\
& \scalebound{X}{y}{s}{m}{m^\pr}\leq 0\\
\end{split}
\end{align}
\end{lemma}

Based on Lemmas~\ref{lem:conversion} and~\ref{lem:zon-zon}, we obtain
the following sufficient condition for inclusion between the real
projections of augmented complex zonotopes.
\begin{lemma}[Real Inclusion: augmented complex zonotopes]~\label{lem:gcz-gcz}
The inclusion relation
$\real\lt(\gcz{V^\pr_{n\times m^\pr}}{c^\pr}{s^\pr}{W^\pr_{n\times k^\pr}}{l^\pr}{u^\pr}\rt)\subseteq
\real\lt(\gcz{V_{n\times m}}{c}{s}{W_{n\times k}}{l}{u}\rt)$ holds if,
\begin{align}
\begin{split}
& \exists X\in\mat{(m+k)}{(m^\pr+k^\pr)}{\mb{C}}~\text{and}~y\in\mb{C}^{m+k}~s.t.\\
& \lt[V~W\rt]X = \lt[V^\pr~W^\pr\rt]\dg\ColumnJoin{\hspace{0.5em}s^\pr}{\frac{u^\pr-l^\pr}{2}}\\
& \lt[V~W\rt]y = c^\pr+W\frac{u^\pr+l^\pr}{2}-c-W\frac{u+l}{2}\\
& \forall i\in\tup{m},~|y_i|+\sum_{j=1}^{m^\pr+k^\pr}\lt|X_{ij}\rt|\leq s_i\\
& \forall i\in\{m+1,...,m+k\},~|y_i|+\sum_{j=1}^{m^\pr+k^\pr}\lt|X_{ij}\rt|\leq \frac{u^\pr_i-l^\pr_i}{2}.
\end{split}
\end{align}
\end{lemma}

For certain kinds of templates, the intersection of a
sub-parallelotope with an augmented complex zonotope can be exactly
specified as another augmented complex zonotope with the same
template.  This is stated in the following lemma 
\begin{lemma}~\label{lem:intersection} Let a sub-parallelotope
  $\sptope{K}{\wh{l}}{\wh{u}}$ and an augmented complex zonotope
  $\gcz{V}{c}{s}{\pseudoinverse{K}}{l}{u}$ and be such that $KV=0$.  Then,
\begin{equation}~\label{eqn:intersection}
\gcz{V}{c}{s}{\pseudoinverse{K}}{l}{u}\bigcap \sptope{K}{\wh{l}}{\wh{u}} =
  \gcz{V}{c}{s}{\pseudoinverse{K}}{\max\lt(l,\wh{l}\rt)}{\min\lt(u,\wh{u}\rt)}.
\end{equation}
\end{lemma}

The minimum and maximum functions are not affine functions, hence,
when composed with convex constraints, the resulting constraints need
not be convex.  However, the minimum function can be written as the
minimum of a finite number of affine functions, while the
maximum function can be written as the maximum of a finite number of
affine functions.  We will call such functions as \emph{policies}, which are
defined below.  Latter, we use these functions to derive a policy
iteration framework for finding positive invariants, where for each
policy in the iteration, we solve SOCP constraints.

\begin{definition}[Min-approximation policy] A function (affine)
  $\minaffine:\realset^k\times\comprealset^k$ is called a
  min-approximation policy on $n$ variables, if there exists a boolean
  vector $v\in\{0,1\}^k$ such that for all $a\in\realset^k$,
  $b\in\comprealset^k$ and $i\in\tup{n}$,
\begin{align}
\lt(\minaffine{a}{b}\rt)_i= & \left\{
\begin{array}{l}
v_ia_i+(1-v_i)b_i~~\text{if}~b_i<\inf\\
a_i~~\text{if}~b_i=\inf
\end{array}
\right.
\end{align}
\end{definition}
%


\begin{definition}[Max-approximation policy] A function (affine)
  $\maxaffine:\realset^k\times\comprealset^k$ is called a
  max-approximation policy on $n$ variables, if there exists a boolean
  vector $v\in\{0,1\}^k$ such that for all $a\in\realset^k$,
  $b\in\comprealset^k$ and $i\in\tup{n}$,
\begin{align}
\lt(\maxaffine{a}{b}\rt)_i= &\left\{
\begin{array}{l}
v_ia_i+(1-v_i)b_i~~\text{if}~b_i>-\inf\\
a_i~~\text{if}~b_i=-\inf
\end{array}
\right.
\end{align}
\end{definition}
%
Observe that in the above definitions, each min-approximation or
max-approximation policy on $k$ variables is defined by a boolean
vector of size $k$.  So, the number of min-approximation policies or
max-approximation policies on $k$ variables is $2^k$.  We denote the
set of min-approximation policies on $k$ variables as
$\minaffineset(k)$ and max-approximation policies as
$\maxaffineset(k)$.


\begin{lemma}~\label{lem:affineapproximation} Let a sub-parallelotope
  $\sptope{K}{\wh{l}}{\wh{u}}$ and an augmented complex zonotope
  $\gcz{V}{c}{s}{\pseudoinverse{K}}{l}{u}$ be such that $VK=0$.  Then,
  all
  the following statements are true.
\begin{enumerate}
\item Let $\minaffinefunc\in\minaffineset(k)$ and
  $\maxaffinefunc\in\maxaffineset(k)$.  % For a fixed ,
  % the functions $\minaffine{.}{b}:\realset^n=k\ra\realset^k$ and
  % $\maxaffine{.}{b}:\realset^k\ra\realset^n$ are affine functions
  % (i.e., in the first argument).
  Then, for all $a
  \in\realset^k$ and $b\in\comprealset^k$, we have the inequalities $\minaffine{a}{b}\geq \min(a,b)$ and
  $\maxaffine{a}{b}\leq \max(a,b)$.
  Therefore, \[\gcz{V}{c}{s}{\pseudoinverse{K}}{l}{u}\bigcap \sptope{K}{\wh{l}}{\wh{u}} \subseteq
  \gcz{V}{c}{s}{W}{\maxaffine{l}{\wh{l}}}{\minaffine{u}{\wh{u}}}.\]
\item For every $a\in\realset^k$ and $b\in\comprealset^k$, there exist
  $\minaffinefunc\in\minaffineset(k)$ and
  $\maxaffinefunc\in\maxaffineset(k)$ such that 
  $\min(a,b)=\minaffine{a}{b}$ and $\max(a,b)=\maxaffine{a}{b}$.  
\end{enumerate}
\end{lemma}





