The switching conditions in an affine hybrid system can be controlled
by linear constraints on the state variables.  In such a case,
over-approximating the reachable set by a set representation can
require computing an over-approximation of the intersection between
sets in the representation with sub-level sets of linear inequalities
controlling the transitions.  So, the accuracy of approximation of
reachable sets depends on the accuracy of approximation of the
intersection.  But like simple zonotopes, complex zonotopes also have
the drawback that they are not closed under intersection with
half-spaces.  However, we shall derive an over-approximation of the
intersection between a complex zonotope and a class of sub-level sets
of linear inequalities, called \emph{sub-parallelotopes}.  The
over-approximation shall be such that the error can be regulated by
modifying the scaling factors.

A sub-parallelotope is a set representation that encodes possibly
unbounded parallelotopes and is defined as follows.
%
\begin{definition}[Sub-parallelotope]
Let us consider a matrix $\qtemp\in\mat{k}{n}{\reals}$ where
$\qtemp\transpose{\qtemp}$ is invertible,
i.e. $\determinant{\qtemp\transpose{\qtemp}}\neq 0$,
$\plb\in\lt(\reals\bigcup\set{-\infty}\rt)^k$ and
$\pub\in\lt(\reals\bigcap\set{\infty}\rt)^k$ such that $\plb\leq\pub$.
The following is the representation of a sub-parallelotope.
%
\[
\ptope{\qtemp}{\lb}{\ub} = \set{x\in\reals^n: \plb\leq\qtemp x\leq\pub}.
\]
%
\end{definition}
%
For example, the set of linear constraints
%
\[ -1\leq x+y-z\leq
1~\wedge~~ x-y+z\leq 3\]
%
is equivalent to a sub-parallelotope
$$\sptope{\mymatrix{1 & 1 & 1\\1 &-1 & 1}}{\mymatrix{-1\\-\infty}}{\mymatrix{1\\3}},$$
because the rows of the sub-parallelotopic template are linearly
independent.  On the other hand, the set of constraints
%
\[
-1\leq
x+y-z\leq 1~\wedge~~x+y+z\leq 2\wedge~~-1\leq x+y
\]
%
do not constitute a sub-parallelotope, because the three row vectors
$\lt[\begin{array}{c c c}1 & 1 & -1\end{array}\rt]$,
$\lt[\begin{array}{c c c}1 & 1 & 1\end{array}\rt]$, and
$\lt[\begin{array}{c c c}1 & 1 & 0\end{array}\rt]$ together are
linearly dependent.  The reason we shall consider the special case of
intersection between sub-parallelotopes and complex zonotopes is
because the former is algebraically related to a generator
representation.  We can express %
\[
\sptope{K_{k\times
    n}}{\wh{l}}{\wh{u}}=\lt\{z+\pinv{K}\zeta:\begin{array}{l}~z\in\realset^n,\zeta\in\realset^k,
  \\~Kz=0,~\wh{l}\leq
\zeta\leq \wh{u}\end{array}\rt\}.
\]
%
%% Therefore, it is possible to compute an over-approximationexpress
%% the intersection of sub-parallelotope with a suitably aligned
%% zonotope as a simple algebraic expression, as we will see latter.
We shall derive an over-approximation for the intersection
between a sub-parallelotope and a complex zonotope as another complex
zonotope using the following affine
functions.  A function
$\minapproxsymbol:\reals^k\times\lt(\reals\bigcup\set{\infty}\rt)^k$ is a
min-approximation function if for all $i\in\set{1,...,k}$
%
\[
\lt(\minapprox{\ub}{\pub}\rt)_i=
\lt\{
\begin{array}{l}
\ub_i~\text{if}~\pub_i=\infty\\
\pub_i~\text{if}~\pub_i<\infty
\end{array}
\rt.
\]
%
Similarly, a function
$\maxapproxsymbol:\reals^k\times\lt(\reals\bigcup\set{-\infty}\rt)^k$
is a max-approximation function if for all $i\in\set{1,...,k}$
%
\[
\lt(\maxapprox{\lb}{\plb}\rt)_i=
\lt\{
\begin{array}{l}
\lb_i~\text{if}~\plb_i=-\infty\\
\plb_i~\text{if}~\plb_i>-\infty
\end{array}
\rt.
\]
%
The above functions output finite values and are affine expressions in
their first argument, when the second argument is fixed.  Now we shall
introduce a few results using which we can derive the
over-approximation for the intersection.  
For any $i\in\set{1,\ldots,k}$, we consider $\alpha_i\in\reals^k$ such
that $\forall j\in\set{1,\ldots,k}$, $\rqb{\alpha_i}_j=0$ if $i\neq j$
and $\rqb{\alpha_i}_j=0$ otherwise.
%
\begin{lemma}~\label{lem:convexhull}
Let $\Psi\in\reals^k$ be a convex set such that
$\forall i\in\set{1,\ldots,k},~\diag{\alpha_i}\Psi\subseteq\Psi$.
Then
%
\[
\forall v\in\Psi, \prod_{i=1}^k\convexhull{\set{0,v_i}}\subseteq\Psi.
\]
\end{lemma}
%
\begin{proof}
We prove the above by induction.  We have ${\compid{1}v=0\times\prod_{i=2}^k\set{v_i}\subseteq\Psi}.$
As $\set{v}=\prod_{i=1}^k\set{v}_i\subseteq \Psi$ and $\Psi$ is a convex set, we get
%
\begin{align*}
\convexhull{\set{0,v_1}}\times\prod_{i=2}^k\set{v_i}\subseteq\Psi~\numberthis\label{proof-hull3}.
\end{align*}
%
\begin{align*}
& \text{If for
~$j<k$}~\prod_{i=1}^j\convexhull{\set{0,v_i}}\times\prod_{i=j+1}^k\set{v_i}\subseteq \Psi,~\text{then}\\
&\compid{j+1}\prod_{i=1}^j\convexhull{\set{0,v_i}}\times\prod_{i=j+1}^k\set{v_i}\\
& \subseteq\compid{j+1}\Psi\subseteq\Psi\\
& \equivalent \prod_{i=1}^j\convexhull{\set{0,v_i}}\times\set{0}\times\prod_{i=j+2}^k\set{v_i}\subseteq\Psi\\
& \%\%~\text{As $v\in\Psi$ and $\Psi$ is a convex set}\\
& \implies \prod_{i=1}^{j+1}\convexhull{\set{0,v_i}}\times\prod_{i=j+2}^k\set{v_i}\subseteq\Psi.~\numberthis\label{proof-hull4}
\end{align*}
%
From Equations~\ref{proof-hull3} and~\ref{proof-hull4}, the lemma
follows by induction.             
\end{proof}
%
From here, we use the
following notations.
%
\begin{align*}
& \lb,\ub\in\reals^k:k\leq
n,~\plb\in\rqb{\reals\bigcup\set{-\infty}}^k,~\pub\in\rqb{\reals\bigcup\set{\infty}}^k,\\
& \qtemp\in\mat{k}{n}{\reals}:\det\rqb{\qtemp\transpose{\qtemp}}\neq 0.
\end{align*}
%
A complex zonotope can be written as the Minkowski sum of the
following complex set and a real set, which we call as \emph{template complex zonotope}
and \emph{interval zonotope}, respectively.
%
\begin{align*}
& \acztope{\ptemp}{\cen}{\sfact}{\stemp}{\lb}{\ub}  =\tcz{\ptemp}{\cen}{\sfact}\oplus\iztope{\stemp}{\lb}{\ub}\\
& \tcztope{\ptemp}{\cen}{\sfact}:= \set{\cen+\ptemp\zeta:~\zeta\in\compnums^n,~\absolute{\zeta}\leq\sfact}\\
& \iztope{\stemp}{\lb}{\ub}:=\set{\stemp\epsilon:~\epsilon\in\reals^k,~\lb\leq\zeta\leq\ub}.
\end{align*}
%
The following lemma gives an over-approximation as well as
under-approximation of the intersection of the Minkowski sum of a
convex set and an interval zonotope with a sub-parallelotope.

For any $k<n$, let us consider $Y\in\mat{n}{(n-k)}{\reals}$ such that the
column vectors of $Y$ form
the basis of $\null{\qtemp}$.
Otherwise when $k=n$, we consider $Y=0$.
%
\begin{lemma}~\label{lem:minsum-intersection}
Let $S\in\reals^n$ be a convex set and
%
\begin{align*}
& \forall i\in\set{1,...,k}~\compid{i}\qtemp S\subseteq\qtemp S,~\numberthis\label{eqn:intersection3}\\
& \lb\leq\maxapprox{\lb}{\plb}\leq\minapprox{\ub}{\pub}\leq\ub~\numberthis\label{eqn:intersection4}.\\
& \text{Then}~~\mymatrix{0\\ Y^T}S \oplus\mymatrix{\qtemp\\
 Y^T}\iztope{\pinv{\qtemp}}{\maxapprox{\lb}{\plb}}{\minapprox{\ub}{\pub}}
 \\ & \subseteq\mymatrix{\qtemp\\
 Y^T}\lt(\lt(S\oplus \iztope{\pinv{\qtemp}}{\lb}{\ub}\rt)\bigcap\ptope{\qtemp}{\plb}{\pub}\rt)~\numberthis\label{eqn:intersection1}\\
 &\subseteq \mymatrix{\qtemp\\
 Y^T}\lt(S \oplus\iztope{\pinv{\qtemp}}{\maxapprox{\lb}{\plb}}{\minapprox{\ub}{\pub}}\rt).\numberthis~\label{eqn:intersection2}
\end{align*}
%
\end{lemma}
%
\begin{proof}

First we shall prove Equation~\ref{eqn:intersection1}.
By Equation~\ref{eqn:intersection3}, we get that for any $v\in S$, $0=\prod_{i=1}^n\alpha_iv\in\qtemp S$.  So,
%
\begin{align*}
\mymatrix{0\\ Y^T}S\subseteq\mymatrix{\qtemp\\ Y^T}S.~\numberthis\label{eqn:corr-proof-intersection1}
\end{align*}
%
By
Equation~\ref{eqn:intersection4} we get
%
\begin{align*}
\mymatrix{\qtemp\\Y^T}\iztope{\pinv{\qtemp}}{\minapprox{\lb}{\plb}}{\maxapprox{\ub}{\pub}}\subseteq\mymatrix{\qtemp\\Y^T}\iztope{\pinv{\qtemp}}{\lb}{\ub}.~\numberthis\label{eqn:corr-proof-intersection2}
\end{align*}
%
Then, Equation~\ref{eqn:intersection1} follows from Equations~\ref{eqn:corr-proof-intersection1} and~\ref{eqn:corr-proof-intersection2}.

Now we shall prove Equation~\ref{eqn:intersection2}.

Let
$x\in \lt(S\oplus \iztope{\pinv{\qtemp}}{\lb}{\ub}\rt)\bigcap\ptope{\qtemp}{\plb}{\pub}$
be a point complex vector that $x=v+\pinv{\qtemp}\zeta:~v\in S,~\lb\leq\zeta\leq\ub$.
%
\begin{align*}
& 
x\in\ptope{\qtemp}{\plb}{\pub} \implies
 \plb\leq \qtemp \lt(v+\pinv{\qtemp}\zeta\rt)\leq\pub\\
& \implies\plb\leq\qtemp v+\zeta\leq\pub.~\numberthis\label{proof-intersection1}
\end{align*}
%
Let us consider
\begin{align*}
& \epsilon\in\reals^k:~\epsilon_i=\lt\{\begin{array}{l}
\min\set{\ub_i,\pub_i}~\text{if}~\zeta_i>\min\set{\ub_i,\pub_i}\\
\max\set{\lb_i,\plb_i}~\text{if}~\zeta_i<\max\set{\lb_i,\plb_i}\\
\zeta_i~\text{otherwise}.
\end{array}
\rt.
\end{align*}
%
Then, it follows from the definitions of min-approximation and max-approximation functions that
\begin{align*}
&\maxapprox{\lb}{\plb}\leq \epsilon\leq\minapprox{\ub}{\pub}.~~
~\numberthis\label{proofnew-intersection1}
\end{align*}
%
Let
$w=x-\pinv{\qtemp}\epsilon=v+\pinv{\qtemp}\zeta-\pinv{\qtemp}\epsilon$.  We shall first show that $\qtemp w\in\prod_{i=1}^n\convexhull{\set{0,\qtemp_iv}}$.
For proving this,  we analyze the following cases for any $i\in\set{1,...,k}$.


{\it Case 1:} Let us consider $\zeta_i>\min\set{\ub_i,\pub_i}$.  Then
$\epsilon_i=\min\set{\ub_i,\pub_i}$.  Also by definition $\zeta_i\leq\ub_i$, which gives
${\min\set{\ub_i,\pub_i}=\pub_i}=\epsilon_i$.  So, $\qtemp_iw=\qtemp_ix-\epsilon_i$
%
\begin{align*}
& \leq\pub_i-\pub_i=0
\end{align*}
%
Also by using Equations~\ref{proof-intersection1} and~\ref{proofnew-intersection1}, we get
%
\begin{align*}
& \qtemp_iw=\qtemp_iv+\zeta_i-\epsilon_i\\
& \geq \qtemp_iv+\min\set{\ub_i,\pub_i}-\min\set{\ub_i,\pub_i}=\qtemp_i
v.
\end{align*}
%
\begin{align*}
& \text{Therefore}~~ \qtemp_iw\in\convexhull{\set{0,\qtemp_iv}}.~\numberthis\label{proofnew-intersection2}
\end{align*}
%
{\it Case 2:} Let us consider that $\zeta_i<\max\set{\lb_i,\plb_i}$.
Then $\epsilon_i=\max{\set{\lb_i,\plb_i}}$.  Also $\zeta_i\geq\lb_i$ by definition, which gives
$\max{\set{\lb_i,\plb_i}}=\plb_i=\epsilon_i$.  So, $\qtemp_iw=\qtemp_ix-\epsilon_i$ 
%
\begin{align*}
& \geq \plb_i-\plb_i=0~\text{\%\% as $x\in\ptope{\qtemp}{\plb}{\pub}$}
\end{align*}
%
Also by using Equations~\ref{proof-intersection1} and~\ref{proofnew-intersection1}, we derive
\begin{align*}
& \qtemp_iw=\qtemp_iv+\zeta_i-\epsilon_i\\
& \leq \qtemp_iv+\max\set{\lb_i,\plb_i}-\max\set{\lb_i,\plb_i}=\qtemp_i
v.
\end{align*}
%
\begin{align*}
& \text{Therefore}~~    \qtemp_iw\in\convexhull{\set{0,\qtemp_iv}}.~\numberthis\label{proofnew-intersection3}
\end{align*}
%
{\it Case 3:}  Let us consider that the above two cases are not true.
Then $\epsilon_i=\zeta_i$ by definition.  So,
%
\begin{align*}
& \qtemp_iw=\qtemp_iv+\zeta_i-\epsilon_i=\qtemp_iv+0=v_i.~\numberthis\label{proofnew-intersection4}
\end{align*}
%
From Equations~\ref{proofnew-intersection2}\textendash\ref{proofnew-intersection4}, we
get
%
\begin{align*}
& \qtemp w\in\prod_{i=1}^k\convexhull{\set{0,v_i}}.\numberthis\label{proofnew-intersection5}
\end{align*}
%
As $S$ is a convex set and $v\in S$, using Equations~\ref{eqn:intersection3} with Lemma~\ref{lem:convexhull} gives $\prod_{i=1}^k\convexhull{\set{0,v_i}}\subseteq \qtemp S$ 
%
\begin{align*}
& (\%\%~~\text{by Equation~\ref{proofnew-intersection5}}) \implies \qtemp w\in \qtemp S~\numberthis\label{eqn:intersectionnew1}
\end{align*}
%
As $Y^T$ is orthogonal to $\qtemp$, we also get
%
\begin{align*}
& Y^Tw=Yv+Y\pinv{\qtemp}(\zeta-\epsilon)=Y^Tv+0=Y^Tv\in Y^TS.\\
& \text{Combining with Equation~\ref{eqn:intersectionnew1}},~~\mymatrix{\qtemp\\ Y^T}w\in \mymatrix{\qtemp\\ Y^T}S~\numberthis\label{eqn:intersectionnew2}
\end{align*}
%
By
Equation~\ref{proofnew-intersection1}, we get
$\pinv{\qtemp}\epsilon\in\iztope{\pinv{\qtemp}}{\maxapprox{\lb}{\plb}}{\minapprox{\ub}{\pub}}$.
Combining the above with Equation~\ref{eqn:intersectionnew2}, we get
%
\begin{align*}
& \mymatrix{\qtemp\\ Y^T}x=\mymatrix{\qtemp\\ Y^T}\lt(w+\pinv{\qtemp}\epsilon\rt)\\
& \in\mymatrix{\qtemp\\ Y^T}\lt(S\oplus\iztope{\pinv{\qtemp}}{\maxapprox{\lb}{\plb}}{\minapprox{\ub}{\pub}}\rt).
\end{align*}
%
As the above is true for any ${x\in
S\oplus\iztope{\pinv{\qtemp}}{\maxapprox{\lb}{\plb}}{\minapprox{\ub}{\pub}}}$,
we have proved Equation~\ref{eqn:intersection2}.
\end{proof}
%

