The switching conditions in an affine hybrid system can be controlled
by linear constraints on the state variables.  In such a case,
over-approximating the reachable set by a set representation can
require computing an over-approximation of the intersection between
sets in the representation with sub-level sets of linear inequalities
controlling the transitions.  So, the accuracy of approximation of
reachable sets depends on the accuracy of approximation of the
intersection.  But like simple zonotopes, complex zonotopes also have
the drawback that they are not closed under intersection with
half-spaces.  However, we shall derive an over-approximation of the
intersection between a complex zonotope and a class of sub-level sets
of linear inequalities, called \emph{sub-parallelotopes}.  The
over-approximation shall be such that the error can be regulated by
modifying the scaling factors.

A sub-parallelotope is a set representation that encodes possibly
unbounded parallelotopes and is defined as follows.
%
\begin{definition}[Sub-parallelotope]
Let us consider a matrix $\qtemp\in\mat{k}{n}{\reals}$ where
$\qtemp\transpose{\qtemp}$ is invertible,
i.e. $\determinant{\qtemp\transpose{\qtemp}}\neq 0$,
$\plb\in\lt(\reals\bigcup\set{-\infty}\rt)^k$ and
$\pub\in\lt(\reals\bigcap\set{\infty}\rt)^k$ such that $\plb\leq\pub$.
The following is the representation of a sub-parallelotope.
%
\[
\ptope{\qtemp}{\lb}{\ub} = \set{x\in\reals^n: \plb\leq\qtemp x\leq\pub}.
\]
%
\end{definition}
%
For example, the set of linear constraints
%
\[ -1\leq x+y-z\leq
1~\wedge~~ x-y+z\leq 3\]
%
is equivalent to a sub-parallelotope
$$\sptope{\mymatrix{1 & 1 & 1\\1 &-1 & 1}}{\mymatrix{-1\\-\infty}}{\mymatrix{1\\3}},$$
because the rows of the sub-parallelotopic template are linearly
independent.  On the other hand, the set of constraints
%
\[
-1\leq
x+y-z\leq 1~\wedge~~x+y+z\leq 2\wedge~~-1\leq x+y
\]
%
do not constitute a sub-parallelotope, because the three row vectors
$\lt[\begin{array}{c c c}1 & 1 & -1\end{array}\rt]$,
$\lt[\begin{array}{c c c}1 & 1 & 1\end{array}\rt]$, and
$\lt[\begin{array}{c c c}1 & 1 & 0\end{array}\rt]$ together are
linearly dependent.  The reason we shall consider the special case of
intersection between sub-parallelotopes and complex zonotopes is
because the former is algebraically related to a generator
representation.  We can express %
\[
\sptope{K_{k\times
    n}}{\wh{l}}{\wh{u}}=\lt\{z+\pinv{K}\zeta:\begin{array}{l}~z\in\realset^n,\zeta\in\realset^k,
  \\~Kz=0,~\wh{l}\leq
\zeta\leq \wh{u}\end{array}\rt\}.
\]
%
%% Therefore, it is possible to compute an over-approximationexpress
%% the intersection of sub-parallelotope with a suitably aligned
%% zonotope as a simple algebraic expression, as we will see latter.
The over-approximation that we derive later for the intersection
between a sub-parallelotope and a complex zonotope as another complex
zonotope shall be expressed in terms of the following affine
functions, called \emph{min-approximation} and
\emph{max-approximation} functions, respectively.  A function
$\minapproxsymbol:\reals^k\times\lt(\reals\bigcup\set{\infty}\rt)^k$ is a
min-approximation function if for all $i\in\set{1,...,k}$
%
\[
\lt(\minapprox{\ub}{\pub}\rt)_i=
\lt\{
\begin{array}{l}
\ub_i~\text{if}~\pub_i=\infty\\
\pub_i~\text{if}~\pub_i<\infty
\end{array}
\rt.
\]
%
The function $\minapprox{\ub}{\pub}$ is an affine function of its
first argument $\ub$ and outputs a finite valued real vector.
Similarly, a function
$\maxapproxsymbol:\reals^k\times\lt(\reals\bigcup\set{-\infty}\rt)^k$
is a max-approximation function if for all $i\in\set{1,...,k}$
%
\[
\lt(\maxapprox{\lb}{\plb}\rt)_i=
\lt\{
\begin{array}{l}
\lb_i~\text{if}~\plb_i=-\infty\\
\plb_i~\text{if}~\plb_i>-\infty
\end{array}
\rt.
\]
%
The function $\maxapprox{\lb}{\plb}$ is an affine function of its
first argument $\lb$ and outputs a finite valued real vector.

For any real vector with possibly unbounded componenets, i.e.,
$v\in\lt(\reals\bigcup\set{\infty,-\infty}\rt)^k$, we define the index
of co-ordinates along which the unbounded components are defined as
$\unbdind{v}$ as follows.
%
\[
\unbdind{v}=\set{i\in\set{1,\ldots,k}:\sup\set{-v_i,v_i}=\infty}.
\]
%
For any $i\in\set{1,...,k}$, let us consider a vector
$\alpha_i\in\reals^k$ defined as follows.  For any $j\in\set{1,\ldots,k}$,
%
\[
\alpha_{i_j}=\lt\{
\begin{array}{l}
1~\text{if}~i\neq j\\
0~\text{otherwise}
\end{array}.
\rt.
\]
%
The following lemma gives an over-approximation of a complex zonotope
with a suitably aligned sub-parallelotope when a certain convex
condition is satisified.  Furthermore, an upper bound is provided on
the Hausdorff distance between the over-approximation and the acutual
intersection.  We shall denote the Hausdorff distance between two sets
$S_1$ and $S_2$ as $\hausdorff{S_1}{S_2}$.
%
{\color{red} define hausdorff here}.
%
\begin{lemma}
Let us consider that $Y\in\mat{n}{(n-k)}{\reals}$ is a matrix whose column vectors form an
orthonormal basis of $\nullspace{\qtemp}$, the null space of
$\qtemp$.  If $\nullspace{\qtemp}=0$, the we consider $Y=0$.  Furthermore,
$S_1=\acztope{\ptemp}{\cen}{\sfact}{\pinv{\qtemp}}{\lb}{\ub}\bigcap\ptope{\qtemp}{\plb}{\pub}$ and
$S_2=\acztope{\ptemp}{\cen}{\sfact}{\pinv{\qtemp}}{\maxapprox{\lb}{\plb}}{\minapprox{\ub}{\pub}}$
such that all of the following conditions are true.
%
\begin{align*}
& \lb\leq\minapprox{\lb}{\plb}\leq\maxapprox{\ub}{\pub}\leq\ub,~\numberthis\label{eqn:intcond1}\\
& \forall
i\in\unbdind{\plb}\bigcup\unbdind{\pub}\\
& ~\acztope{\alpha_i\qtemp\ptemp}{\alpha_i\qtemp\cen}{\sfact}{0}{0}{0}\order\acztope{\qtemp\ptemp}{\qtemp\cen}{\sfact}{0}{0}{0}.~\numberthis\label{eqn:intcond2}
\end{align*}
%
Then $S_1\subseteq S_2$ and $\hausdorff{S_1}{S_2}\leq todo$
\end{lemma}
%
\begin{proof}
Let us consider $x\in\mymatrix{\qtemp\\\transpose{Y}}S_1$,
$y=\prod_{i\in\unbdind{\plb}\bigcup\unbdind{\pub}}\alpha_ix$ and $z=x-y$.
\end{proof}
%
