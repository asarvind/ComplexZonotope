The switching conditions in an affine hybrid system can be controlled
by linear constraints on the state variables.  In such a case,
over-approximating the reachable set by a set representation can
require computing an over-approximation of the intersection between
sets in the representation with sub-level sets of linear inequalities
controlling the transitions.  Then the accuracy of approximation of
reachable sets depends on the accuracy of approximation of the
intersection.  But like simple zonotopes, complex zonotopes also have
the drawback that they are not closed under intersection with
half-spaces.  However, we shall derive an over-approximation of the
intersection between a complex zonotope and a class of sub-level sets
of linear inequalities, called \emph{sub-parallelotopes}, which are
algebraically related to the generator representation.  The
over-approximation shall be such that the error can be regulated by
modifying the scaling factors.

A sub-parallelotope is a set representation that encodes possibly
unbounded parallelotopes and is defined as follows.
%
\begin{definition}[Sub-parallelotope]
Let us consider a matrix $\qtemp\in\mat{k}{n}{\reals}$ where
$\qtemp\transpose{\qtemp}$ is invertible,
i.e. $\determinant{\qtemp\transpose{\qtemp}}\neq 0$,
$\plb\in\lt(\reals\bigcup\set{-\infty}\rt)^k$ and
$\pub\in\lt(\reals\bigcap\set{\infty}\rt)^k$ such that $\plb\leq\pub$.
The following is the representation of a sub-parallelotope.
%
\[
\ptope{\qtemp}{\lb}{\ub} = \set{x\in\reals^n: \plb\leq\qtemp x\leq\pub}.
\]
%
\end{definition}
%
For example, the set of linear constraints
%
\[ -1\leq x+y-z\leq
1~\wedge~~ x-y+z\leq 3\]
%
is equivalent to a sub-parallelotope
$$\sptope{\mymatrix{1 & 1 & 1\\1 &-1 & 1}}{\mymatrix{-1\\-\infty}}{\mymatrix{1\\3}},$$
because the rows of the sub-parallelotopic template are linearly
independent.  On the other hand, the set of constraints
%
\[
-1\leq
x+y-z\leq 1~\wedge~~x+y+z\leq 2\wedge~~-1\leq x+y
\]
%
do not constitute a sub-parallelotope, because the three row vectors
$\lt[\begin{array}{c c c}1 & 1 & -1\end{array}\rt]$,
$\lt[\begin{array}{c c c}1 & 1 & 1\end{array}\rt]$, and
$\lt[\begin{array}{c c c}1 & 1 & 0\end{array}\rt]$ together are
linearly dependent.  The reason we shall consider the special case of
intersection between sub-parallelotopes and complex zonotopes is
because the former is algebraically related to a generator
representation.  We can express %
\[
\sptope{K_{k\times
    n}}{\wh{l}}{\wh{u}}=\lt\{z+\pinv{K}\zeta:\begin{array}{l}~z\in\realset^n,\zeta\in\realset^k,
  \\~Kz=0,~\wh{l}\leq
\zeta\leq \wh{u}\end{array}\rt\}.
\]
%
%% Therefore, it is possible to compute an over-approximationexpress
%% the intersection of sub-parallelotope with a suitably aligned
%% zonotope as a simple algebraic expression, as we will see latter.
The over-approximation of the intersection between a sub-parallelotope
and a complex zonotope that we derive later shall be expressed in
terms of the following affine function, called \emph{min-approximation} and
\emph{max-approximation} functions, respectively.  A function
$\minapproxsymbol:\reals^k\times\lt(\reals\bigcup\set{\infty}\rt)^k$ is a
min-approximation function if for all $i\in\set{1,...,k}$
%
\[
\lt(\minapprox{\ub}{\pub}\rt)_i=
\lt\{
\begin{array}{l}
\ub_i~\text{if}~\pub_i=\infty\\
\pub_i~\text{if}~\pub_i<\infty
\end{array}
\rt.
\]
%
The function $\minapprox{\ub}{\pub}$ is an affine function of its
first argument $\ub$ and outputs a finite valued real vector.
Similarly, a function
$\maxapproxsymbol:\reals^k\times\lt(\reals\bigcup\set{-\infty}\rt)^k$
is a max-approximation function if for all $i\in\set{1,...,k}$
%
\[
\lt(\maxapprox{\lb}{\plb}\rt)_i=
\lt\{
\begin{array}{l}
\lb_i~\text{if}~\plb_i=-\infty\\
\plb_i~\text{if}~\plb_i>-\infty
\end{array}
\rt.
\]
%
The function $\maxapprox{\lb}{\plb}$ is an affine function of its
first argument $\lb$ and outputs a finite valued real vector.
%
