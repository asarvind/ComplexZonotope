Checking inclusion between two complex zonotopes amounts to
solving a non-convex optimization problem which can be computationally
intractable.  However, we shall provide a sufficient condition for
the inclusion that can be checked by convex optimization.  In
this regard we define the following relation between two complex
zonotopes, which we later show is sufficient for the inclusion.
%
\begin{definition}[Relation for checking inclusion]~\label{defn:inclusion}
Let us consider $\ptemp^\pr\in\mat{n}{r}{\compnums}$ and
$\stemp^\pr\in\mat{n}{h}{\reals}$.  We say that
$\acztope{\ptemp^\pr}{\cen^\pr}{\sfact^\pr}{\stemp^\pr}{\lb^\pr}{\ub^\pr}
\order \acztope{\ptemp}{\cen}{\sfact}{\stemp}{\lb}{\ub}$
if all of the following is collectively true.
%
\begin{align*}
  & \exists X\in\mat{(m+k)}{(r+h)}{\compnums},~y\in\compnums^{m+k}:\\
   & \mymatrix{\ptemp & \stemp}{y} = \cen^\pr+\stemp^\pr\frac{\ub^\pr+\lb^\pr}{2}-\cen-\stemp\frac{\ub+\lb}{2}, \\
  & \mymatrix{\ptemp & \stemp} X = \mymatrix{\ptemp^\pr &
    \stemp^\pr}\diagonal{\mymatrix{\sfact
      \\ \frac{\ub^\pr-\lb^\pr}{2}}},\\
  & \forall i\in\set{1,\ldots,m}~\absolute{y_i}+\sum_{j = 1}^{r+h}\absolute{X_{i_j}}
  \leq \sfact_i,\\
  & \forall i\in\set{1,\ldots,k}~\absolute{y_i}+\sum_{j =
    1}^{r+h}\absolute{X_{(m+i)_j}}\leq \lt(\frac{\ub-\lb}{2}\rt)_i.
\end{align*}
%
\end{definition}
%
The following lemma states that the above relation between complex
zonotopes is sufficient to guarantee the inclusion between them.
%
\begin{lemma}[Sufficient condition for inclusion]~\label{lem:inclusion}
Let us consider that
$\acztope{\ptemp^\pr}{\cen^\pr}{\sfact^\pr}{\stemp^\pr}{\lb^\pr}{\ub^\pr}
\order \acztope{\ptemp}{\cen}{\sfact}{\stemp}{\lb}{\ub}$.  Then
$\acztope{\ptemp^\pr}{\cen^\pr}{\sfact^\pr}{\stemp^\pr}{\lb^\pr}{\ub^\pr}
\subseteq \acztope{\ptemp}{\cen}{\sfact}{\stemp}{\lb}{\ub}$.
\end{lemma}
%
\begin{proof}
Let us consider a point
$x\in\acztope{\ptemp^\pr}{\cen^\pr}{\sfact^\pr}{\stemp^\pr}{\lb^\pr}{\ub^\pr}$.
Then there exists $\zeta^\pr\in\compnums^r$ and
$\epsilon^\pr\in\reals^k$ such that
%
\begin{align*}
x=\ptemp^\pr\zeta^\pr+\stemp^\pr\epsilon^\pr,~\absolute{\zeta^\pr}\leq\sfact^\pr,~\text{and}~\lb\leq\epsilon^\pr\leq\ub.%~\numberthis\label{eqn:spec}
\end{align*}
%
Let us consider $\alpha\in\compnums^r$ and $\beta\in\reals^h$ defined as follows.
%
\begin{align*}
&  \forall i\in\set{1,\ldots,r}~\alpha_i=\lt\{\begin{array}{l}
  \frac{\zeta_i}{\sfact^\pr_i}~\text{if}~\zeta_i\neq 0\\
  0~\text{otherwise}.
  \end{array}
  \rt.\\
  & \forall i\in\set{1,\ldots,h}~\beta_i=\lt\{
  \begin{array}{l}
  \frac{2\zeta^\pr-\ub^\pr+\lb^\pr}{\ub^\pr-\lb^\pr}~\text{if}~\zeta^\pr\neq\frac{\ub^\pr-\lb^\pr}{2}\\
  0~\text{otherwise}.
    \end{array}
    \rt.\\ \text{Then}~
%% \end{align*}
%% %
%% Then we get
%% %
%% \begin{align*}
&  \zeta^\pr=\diagonal{\sfact^\pr}\alpha,~\epsilon^\pr=\diagonal{\frac{\ub^\pr-\lb^\pr}{2}}\beta+\frac{\ub^\pr+\lb^\pr}{2},\\
&  \infnorm{\alpha}\leq 1,~\infnorm{\beta}\leq 1.%~\numberthis\label{eqn:conversion}
\end{align*}
%
Let us consider $\mymatrix{\zeta\\\epsilon}=y+X\mymatrix{\alpha\\\beta}+\mymatrix{0\\\frac{\ub-\lb}{2}}$.
We derive
%
\begin{align*}
  &  x=\cen^\pr+\ptemp^\pr\zeta^\pr+\stemp^\pr\epsilon^\pr\\
&   = \cen^\pr + \mymatrix{\ptemp^\pr & \stemp^\pr}\mymatrix{\diagonal{\sfact^\pr}\alpha\\
  \diagonal{\frac{\ub^\pr-\lb^\pr}{2}}\beta +
  \frac{\ub^\pr+\lb^\pr}{2}}\\
  & = \cen^\pr+\stemp^\pr\frac{\ub^\pr+\lb^\pr}{2}+\mymatrix{\ptemp &
    \stemp}X\mymatrix{\alpha\\\beta}\\
  & = \cen +\stemp\frac{\ub-\lb}{2}+\mymatrix{\ptemp &
    \stemp}{\lt(y+X\mymatrix{\alpha\\\beta}\rt)}\\
  & = \cen+\mymatrix{\ptemp & \stemp}\mymatrix{\zeta\\ \epsilon}.~\numberthis\label{eqn:conversion1}
\end{align*}
%
We derive the following bounds on $\zeta$ and $\epsilon$.
%
\begin{align*}
& \forall
i\in\set{1,\ldots,m}~\absolute{\zeta_i}=\absolute{y_i+X_i\alpha}\\
& \leq \absolute{y_i}+\sum_{i=1}^{r+h}\absolute{X_{i_j}}\leq \sfact_i~\lt(\because
\infnorm{\alpha}\leq 1\rt)~\numberthis\label{eqn:bound1}\\
& \forall i\in\set{1,\ldots,k}~\absolute{y_i+X_i\beta}\\
& \leq \absolute{y_i}+\sum_{i=1}^{r+h}\absolute{X_{i_j}}\leq\frac{\ub-\lb}{2}~\lt(\because
\infnorm{\beta}\leq 1\rt)~\numberthis\label{eqn:bound2}\\
& \%\%~\because \epsilon_i =
y_i+X_i\beta_i+\frac{\ub+\lb}{2},~\text{by
  Equation~\ref{eqn:bound2}}:\\
& \lb\leq\epsilon\leq \ub.~\numberthis\label{eqn:bound3}
\end{align*}
%
By Equations~\ref{eqn:conversion1},~\ref{eqn:bound1}
and~\ref{eqn:bound3}, we get that
$x\in\acztope{\ptemp}{\cen}{\sfact}{\stemp}{\lb}{\ub}$.  As this is
true for all
$x\in\acztope{\ptemp^\pr}{\cen^\pr}{\sfact^\pr}{\stemp^\pr}{\lb^\pr}{\ub^\pr}$,
we get
$\acztope{\ptemp^\pr}{\cen^\pr}{\sfact^\pr}{\stemp^\pr}{\lb^\pr}{\ub^\pr}\subseteq\acztope{\ptemp}{\cen}{\sfact}{\stemp}{\lb}{\ub}$.
\end{proof}
%
We note that for fixed templates, the relation in
Definition~\ref{defn:inclusion} is a set of convex constraints on the
scaling factors, primary offset and interval bounds of the two complex
zonotopes.  In fact, these constraints can be equivalently
expressed as a second order conic program of polynomial size in the
specification of the two complex zonnotopes.
