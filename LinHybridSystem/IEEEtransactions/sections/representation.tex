Before introducing a complex zonotope, we shall briefly describe the
previously known simple/real valued zonotope.  A simple zonotope
consists of points generated by a linear combination of real valued
vectors translated by a center such that the combining coefficients
are bounded inside symmetric intervals.  Geometrically speaking, a
simple zonotope is a Minkowski sum of line segments.  It can be
represented by matrix of generators and a center, as follows.
%
\begin{definition}
Let us consider $\stemp\in\mat{n}{k}{\reals}$ and $\cen\in\reals^n$.
A real zonotope is represented as follows.
%
\[
\rztope{\stemp}{\cen} = \set{\cen+\stemp\zeta:~\zeta\in[-1,1]^k}.
\]
%
\end{definition}
%
A main advantage of the above representation of a simple zonotope is
that besides being closed under linear transformation and Minkowski
sum, these operations can be computed efficiently (see~\cite{todo}).
However, for a stable linear transformation having complex
eigenvalues, it is not known whether there exists a positively
invariant simple zonotope with non-empty interior.  %% In contrast, some
%% set representations like ellipsoids can efficiently encode positive
%% invariants of stable linear transformations~\cite{todo}.  However,
%% ellipsoids are not closed under Minkowski sum.
Henceforth, we want to
develop a set representation that can efficiently encode positive
invariants of linear transformation with complex eigenstructure, while
the Minkowski sum and linear transformation opeartions are closed and
efficiently computable.  In this regard, we extend simple
zonotopes to the complex valued domain to yeild a set
representation called complex zonotope that has the above
desirable property.  A complex zonotope consists of points represented
by a linear combination of complex valued vectors with translation,
where some of the combining coefficients are complex valued and
bounded in their absolute values, while others are real valued and
bounded inside real intervals.  We shall later show in Lemma~\ref{lem:lintransform}
how a complex zonotope can efficiently encode a positive invariant of
a stable linear transformation using the complex eigenstructure.
%
\begin{definition}
Let us consider $\ptemp\in\mat{n}{m}{\compnums}$, called a
\emph{primary template}, $\sfact\in\reals_{\geq 0}$ called scaling
factors, $\cen\in\reals^n$ called \emph{primary offset},
$\stemp\in\mat{n}{k}{\reals}$ called secondary template,
$\lb,\ub\in\reals^k:~\lb\leq\ub$, called \emph{lower and upper
  interval bounds}, respectively.  An complex zonotope is
represented as follows.
%
\[
\acztope{\ptemp}{\cen}{\sfact}{\stemp}{\lb}{\ub} =
\set{
  \cen+\ptemp\zeta+\stemp\epsilon:
  \begin{array}{l}
    \zeta\in\compnums^m,~\epsilon\in\reals^k,\\
    \absolute{\zeta}\leq
    \sfact,~\lb\leq\epsilon\leq\ub
   \end{array}
}.
\]
%
\end{definition}
%
Geometrically speaking, the real projection of a complex zonotope can
be a Minkowski sum of both ellipses and line segments embedded inside
possibly higher dimensional spaces.  For example, Figure~\ref{} shows
the real projection of a complex zonotope which is a Minkowski sum of
two ellipses and one line segment.  Therefore, complex zonotopes,
which can describe some non-polyhedral sets in addition to polytopic
simple zonotopes, are geometrically more expressive than the latter.

The linear transformation of a complex zonotope can be computed
efficiently by an algebraic expression which is a linear function of
the scaling factors, primary offset and the interval bounds.
Furthermore, a complex zonotope can efficiently encode a positive
invariant of an invertible stable linear transformation by capturing
contraction along the complex eigenvectors.  This is described in the
following lemma.

\emph{Notation}:  From now, we use the following notation unless
otherwise specified:
%
$
{\ptemp\in\mat{n}{m}{\compnums}},~{\cen\in\reals^n},{\sfact\in\reals_{\geq 0}^m},~\stemp\in\mat{n}{k}{\reals},~{\lb,\ub\in\reals^k}.
$
%
\begin{lemma}[Linear transformation and eigenstructure based contraction]~\label{lem:lintransform}
Let us consider $\Xi\in\mat{n}{n}{\compnums}$ contains the complex
eigenvectors of $A\in\mat{n}{n}{\reals}$ as its columns vectors such
that $A\Xi=\Xi\diagonal{\mu}$ for a vector of eigenvalues
$\mu\in\compnums^n$.  All of the following statements are true.
%
\begin{enumerate}
\item
  $A\acztope{\ptemp}{\cen}{\sfact}{\stemp}{\lb}{\ub}=\acztope{A\ptemp}{A\cen}{\sfact}{A\stemp}{\lb}{\ub}$.
\item
  $A\acztope{\Xi}{0}{\sfact}{0}{0}{0}=\acztope{\Xi}{0}{\diagonal{\absolute{\mu}}\sfact}{0}{0}{0}$.
\item If $\infnorm{\mu}\leq 1$, then
  \[A\acztope{\Xi}{0}{\sfact}{0}{0}{0}\subseteq\acztope{\Xi}{0}{\sfact}{0}{0}{0}.\]
\end{enumerate}
%
\end{lemma}
%
\begin{proof}
The proof of first statement is derived as follows.
%
\begin{align*}
& A\acztope{\ptemp}{\cen}{\sfact}{\stemp}{\lb}{\ub}\\
& = A\set{
  \cen+\ptemp\zeta+\stemp\epsilon:
  \begin{array}{l}
    \zeta\in\compnums^m,~\epsilon\in\reals^k,\\
    \absolute{\zeta}\leq
    \sfact,~\lb\leq\epsilon\leq\ub
   \end{array}
  }\\
& = \set{
  A\cen+A\ptemp\zeta+A\stemp\epsilon:
  \begin{array}{l}
    \zeta\in\compnums^m,~\epsilon\in\reals^k,\\
    \absolute{\zeta}\leq
    \sfact,~\lb\leq\epsilon\leq\ub
   \end{array}
  }\\
& = \acztope{A\ptemp}{A\cen}{\sfact}{A\stemp}{\lb}{\ub}.
\end{align*}
%
By the above derivation we get
$A\acztope{\Xi}{0}{\sfact}{0}{0}{0}=\acztope{A\Xi}{0}{\sfact}{0}{0}{0}=\acztope{\Xi\diagonal{\mu}}{0}{\sfact}{0}{0}{0}$.
We have
%
\begin{align*}
& \set{\diagonal{\mu}\zeta:\zeta\in\compnums^m,\absolute{\zeta}\leq
    \sfact}\\
& =\set{\zeta^\pr:\zeta^\pr\in\compnums^m,\absolute{\zeta^\pr}\leq\diagonal{\absolute{\mu}}\sfact}\\
& \therefore
  A\acztope{\Xi}{0}{\sfact}{0}{0}{0}=\acztope{\Xi}{0}{\diagonal{\absolute{\mu}}\sfact}{0}{0}{0}.
\end{align*}
%
If $\infnorm{\mu}\leq 1$, then $\diagonal{\absolute{\mu}}{\sfact}\leq
\sfact$.  Then we get
$A\acztope{\Xi}{0}{\sfact}{0}{0}{0}=\acztope{\Xi}{0}{\diagonal{\absolute{\mu}}\sfact}{0}{0}{0}\subseteq\acztope{\Xi}{0}{\sfact}{0}{0}{0}$.
\end{proof}
%
If an $n\times n$ matrix is invertible, then it has $n$-linearly
independent eigenvectors.  So, a complex zonotope whose template
contains the $n$-linearly independent eigenvectors and the scaling
factors are non-zero has a non-empty interior.  Additionally, if the
invertible matrix is also Schur stable, then we get a complex zonotope
with non-empty interior, which after transformation by the matrix is
contained inside itself according to the above lemma.  In other words,
we can specify a positive invariant complex zonotope with non-empty
interior for a stable and invertible linear transformation using the
eigenstructure.

Complex zonotopes are also closed under Minkowski sum, which can be
computed by an affine expression as
follows.
%
\begin{lemma}
Let us consider $\ptemp^\pr\in\mat{n}{r}{\compnums}$ and
$\stemp^\pr\in\mat{n}{h}{\reals}$.  Then the following is true.
%
\begin{align*}
&
  \acztope{\ptemp}{\cen}{\sfact}{\stemp}{\lb}{\ub}\oplus\acztope{\ptemp^\pr}{\cen^\pr}{
    \sfact^\pr}{\stemp^\pr}{\lb^\pr}{\ub^\pr}\\
& = \acztope{
    \mymatrix{\ptemp & \ptemp^\pr}
  }
          {
            \mymatrix{\cen+\cen^\pr}
          }
          {\mymatrix{\sfact\\\sfact^\pr}
          }
          {\mymatrix{\stemp & \stemp^\pr}
          }
          {\mymatrix{\lb\\ \lb^\pr}
          }
          {\mymatrix{\ub\\ \ub^\pr}
          }.
\end{align*}
%
\end{lemma}
%
\begin{proof}
$\acztope{\ptemp}{\cen}{\sfact}{\stemp}{\lb}{\ub}\oplus\acztope{\ptemp^\pr}{\cen^\pr}{
    \sfact^\pr}{\stemp^\pr}{\lb^\pr}{\ub^\pr}=$
%
\begin{align*}
&  \lt\{\cen+\cen^\pr+\ptemp\zeta+\stemp\epsilon+\ptemp^\pr\zeta^\pr+\stemp^\pr\epsilon^\pr\rt.:
  \lt. \zeta\in\compnums^m,\zeta^\pr\in\compnums^r,\rt.\\
  & \lt. \epsilon\in\reals^k,\epsilon^\pr\in\reals^h,
    \absolute{\zeta}\leq\sfact,\lb\leq\epsilon\leq
    \ub,\absolute{\zeta^\pr}\leq\sfact^\pr,\lb^\pr\leq\epsilon^\pr\leq
    \ub^\pr\rt\}=\\
& \lt\{\lt(\cen+\cen^\pr\rt)+\mymatrix{\ptemp
      &\ptemp^\pr}\mymatrix{\zeta\\\zeta^\pr}+\mymatrix{\stemp
      &\stemp^\pr}\mymatrix{\epsilon\\ \epsilon^\pr}: \mymatrix{\zeta\\\zeta^\pr}\in\compnums^{m+r}\rt.\\
    & \lt. \mymatrix{\epsilon\\\epsilon^\pr}\in\reals^{k+h},\rt.
    \lt.\absolute{\begin{matrix}\zeta\\\zeta^\pr\end{matrix}}\leq
        \mymatrix{\sfact\\\sfact^\pr},
        \mymatrix{\lb\\\lb^\pr}\leq\mymatrix{\epsilon\\\epsilon^\pr}\leq\mymatrix{\ub\\\ub^\pr}
        \rt\}\\
& = \acztope{
    \mymatrix{\ptemp & \ptemp^\pr}
  }
          {
            \mymatrix{\cen+\cen^\pr}
          }
          {\mymatrix{\sfact\\\sfact^\pr}
          }
          {\mymatrix{\stemp & \stemp^\pr}
          }
          {\mymatrix{\lb\\ \lb^\pr}
          }
          {\mymatrix{\ub\\ \ub^\pr}
          }.\hspace{1em}\qedhere
\end{align*}
%
\end{proof}
%
Checking inclusion between two complex zonotopes amounts to
solving a non-convex optimization problem which can be computationally
intractable.  However, we shall provide a sufficient condition for
the inclusion that can be checked by convex optimization.  In
this regard we define the following relation between two complex
zonotopes, which we later show is sufficient for the inclusion.
%
\begin{definition}[Relation for checking inclusion]~\label{defn:inclusion}
Let us consider $\ptemp^\pr\in\mat{n}{r}{\compnums}$ and
$\stemp^\pr\in\mat{n}{h}{\reals}$.  We say that
$\acztope{\ptemp^\pr}{\cen^\pr}{\sfact^\pr}{\stemp^\pr}{\lb^\pr}{\ub^\pr}
\order \acztope{\ptemp}{\cen}{\sfact}{\stemp}{\lb}{\ub}$
if all of the following is collectively true.
%
\begin{align*}
  & \exists X\in\mat{(m+k)}{(r+h)}{\compnums},~y\in\compnums^{m+k}:\\
   & \mymatrix{\ptemp & \stemp}{y} = \cen^\pr+\stemp^\pr\frac{\ub^\pr+\lb^\pr}{2}-\cen-\stemp\frac{\ub+\lb}{2}, \\
  & \mymatrix{\ptemp & \stemp} X = \mymatrix{\ptemp^\pr &
    \stemp^\pr}\diagonal{\mymatrix{\sfact
      \\ \frac{\ub^\pr-\lb^\pr}{2}}},\\
  & \forall i\in\set{1,\ldots,m}~\absolute{y_i}+\sum_{j = 1}^{r+h}\absolute{X_{i_j}}
  \leq \sfact_i,\\
  & \forall i\in\set{1,\ldots,k}~\absolute{y_i}+\sum_{j =
    1}^{r+h}\absolute{X_{(m+i)_j}}\leq \lt(\frac{\ub-\lb}{2}\rt)_i.
\end{align*}
%
\end{definition}
%
The following lemma states that the above relation between complex
zonotopes is sufficient to guarantee the inclusion between them.
%
\begin{lemma}[Sufficient condition for inclusion]~\label{lem:inclusion}
Let us consider that
$\acztope{\ptemp^\pr}{\cen^\pr}{\sfact^\pr}{\stemp^\pr}{\lb^\pr}{\ub^\pr}
\order \acztope{\ptemp}{\cen}{\sfact}{\stemp}{\lb}{\ub}$.  Then
$\acztope{\ptemp^\pr}{\cen^\pr}{\sfact^\pr}{\stemp^\pr}{\lb^\pr}{\ub^\pr}
\subseteq \acztope{\ptemp}{\cen}{\sfact}{\stemp}{\lb}{\ub}$.
\end{lemma}
%
\begin{proof}
Let us consider a point
$x\in\acztope{\ptemp^\pr}{\cen^\pr}{\sfact^\pr}{\stemp^\pr}{\lb^\pr}{\ub^\pr}$.
Then there exists $\zeta^\pr\in\compnums^r$ and
$\epsilon^\pr\in\reals^k$ such that
%
\begin{align*}
x=\ptemp^\pr\zeta^\pr+\stemp^\pr\epsilon^\pr,~\absolute{\zeta^\pr}\leq\sfact^\pr,~\text{and}~\lb\leq\epsilon^\pr\leq\ub.%~\numberthis\label{eqn:spec}
\end{align*}
%
Let us consider $\alpha\in\compnums^r$ and $\beta\in\reals^h$ defined as follows.
%
\begin{align*}
&  \forall i\in\set{1,\ldots,r}~\alpha_i=\lt\{\begin{array}{l}
  \frac{\zeta_i}{\sfact^\pr_i}~\text{if}~\zeta_i\neq 0\\
  0~\text{otherwise}.
  \end{array}
  \rt.\\
  & \forall i\in\set{1,\ldots,h}~\beta_i=\lt\{
  \begin{array}{l}
  \frac{2\zeta^\pr-\ub^\pr+\lb^\pr}{\ub^\pr-\lb^\pr}~\text{if}~\zeta^\pr\neq\frac{\ub^\pr-\lb^\pr}{2}\\
  0~\text{otherwise}.
    \end{array}
    \rt.\\ \text{Then}~
%% \end{align*}
%% %
%% Then we get
%% %
%% \begin{align*}
&  \zeta^\pr=\diagonal{\sfact^\pr}\alpha,~\epsilon^\pr=\diagonal{\frac{\ub^\pr-\lb^\pr}{2}}\beta+\frac{\ub^\pr+\lb^\pr}{2},\\
&  \infnorm{\alpha}\leq 1,~\infnorm{\beta}\leq 1.%~\numberthis\label{eqn:conversion}
\end{align*}
%
Let us consider $\mymatrix{\zeta\\\epsilon}=y+X\mymatrix{\alpha\\\beta}+\mymatrix{0\\\frac{\ub-\lb}{2}}$.
We derive
%
\begin{align*}
  &  x=\cen^\pr+\ptemp^\pr\zeta^\pr+\stemp^\pr\epsilon^\pr\\
&   = \cen^\pr + \mymatrix{\ptemp^\pr & \stemp^\pr}\mymatrix{\diagonal{\sfact^\pr}\alpha\\
  \diagonal{\frac{\ub^\pr-\lb^\pr}{2}}\beta +
  \frac{\ub^\pr+\lb^\pr}{2}}\\
  & = \cen^\pr+\stemp^\pr\frac{\ub^\pr+\lb^\pr}{2}+\mymatrix{\ptemp &
    \stemp}X\mymatrix{\alpha\\\beta}\\
  & = \cen +\stemp\frac{\ub-\lb}{2}+\mymatrix{\ptemp &
    \stemp}{\lt(y+X\mymatrix{\alpha\\\beta}\rt)}\\
  & = \cen+\mymatrix{\ptemp & \stemp}\mymatrix{\zeta\\ \epsilon}.~\numberthis\label{eqn:conversion1}
\end{align*}
%
We derive the following bounds on $\zeta$ and $\epsilon$.
%
\begin{align*}
& \forall
i\in\set{1,\ldots,m}~\absolute{\zeta_i}=\absolute{y_i+X_i\alpha}\\
& \leq \absolute{y_i}+\sum_{i=1}^{r+h}\absolute{X_{i_j}}\leq \sfact_i~\lt(\because
\infnorm{\alpha}\leq 1\rt)~\numberthis\label{eqn:bound1}\\
& \forall i\in\set{1,\ldots,k}~\absolute{y_i+X_i\beta}\\
& \leq \absolute{y_i}+\sum_{i=1}^{r+h}\absolute{X_{i_j}}\leq\frac{\ub-\lb}{2}~\lt(\because
\infnorm{\beta}\leq 1\rt)~\numberthis\label{eqn:bound2}\\
& \%\%~\because \epsilon_i =
y_i+X_i\beta_i+\frac{\ub+\lb}{2},~\text{by
  Equation~\ref{eqn:bound2}}:\\
& \lb\leq\epsilon\leq \ub.~\numberthis\label{eqn:bound3}
\end{align*}
%
By Equations~\ref{eqn:conversion1},~\ref{eqn:bound1}
and~\ref{eqn:bound3}, we get that
$x\in\acztope{\ptemp}{\cen}{\sfact}{\stemp}{\lb}{\ub}$.  As this is
true for all
$x\in\acztope{\ptemp^\pr}{\cen^\pr}{\sfact^\pr}{\stemp^\pr}{\lb^\pr}{\ub^\pr}$,
we get
$\acztope{\ptemp^\pr}{\cen^\pr}{\sfact^\pr}{\stemp^\pr}{\lb^\pr}{\ub^\pr}\subseteq\acztope{\ptemp}{\cen}{\sfact}{\stemp}{\lb}{\ub}$.
\end{proof}
%
We note that for fixed templates, the relation in
Definition~\ref{defn:inclusion} is a set of convex constraints on the
scaling factors, primary offset and interval bounds of the two complex
zonotopes.  In fact, these constraints can be equivalently
expressed as a second order conic program of polynomial size in the
specification of the two complex zonnotopes.

