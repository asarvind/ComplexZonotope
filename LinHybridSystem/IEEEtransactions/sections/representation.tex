Before introducing a complex zonotope, we shall briefly describe the
previously known simple/real valued zonotope.  A simple zonotope
consists of points generated by a linear combination of real valued
vectors translated by a center such that the combining coefficients
are bounded inside symmetric intervals.  Geometrically speaking, a
simple zonotope is a Minkowski sum of line segments.  It can be
represented by matrix of generators and a center, as follows.
%
\begin{definition}
Let us consider $\stemp\in\mat{n}{k}{\reals}$ and $\cen\in\reals^n$.
A real zonotope is represented as follows.
%
\[
\rztope{\stemp}{\cen} = \set{\cen+\stemp\zeta:~\zeta\in[-1,1]^k}.
\]
%
\end{definition}
%
A main advantage of the above representation of a simple zonotope is
that besides being closed under linear transformation and Minkowski
sum, these operations can be computed efficiently (see~\cite{todo}).
However, for a stable linear transformation having complex
eigenvalues, it is not known whether there exists a positively
invariant simple zonotope with non-empty interior.  %% In contrast, some
%% set representations like ellipsoids can efficiently encode positive
%% invariants of stable linear transformations~\cite{todo}.  However,
%% ellipsoids are not closed under Minkowski sum.
Henceforth, we want to
develop a set representation that can efficiently encode positive
invariants of linear transformation with complex eigenstructure, while
the Minkowski sum and linear transformation opeartions are closed and
efficiently computable.  In this regard, we extend simple
zonotopes to the complex valued domain to yeild a set
representation called complex zonotope that has the above
desirable property.  A complex zonotope consists of points represented
by a linear combination of complex valued vectors with translation,
where some of the combining coefficients are complex valued and
bounded in their absolute values, while others are real valued and
bounded inside real intervals.  We shall later show in Lemma~\ref{lem:lintransform}
how a complex zonotope can efficiently encode a positive invariant of
a stable linear transformation using the complex eigenstructure.
%
\begin{definition}
Let us consider $\ptemp\in\mat{n}{m}{\compnums}$, called a
\emph{primary template}, $\sfact\in\reals_{\geq 0}$ called scaling
factors, $\cen\in\reals^n$ called \emph{primary offset},
$\stemp\in\mat{n}{k}{\reals}$ called secondary template,
$\lb,\ub\in\reals^k:~\lb\leq\ub$, called \emph{lower and upper
  interval bounds}, respectively.  An complex zonotope is
represented as follows.
%
\[
\acztope{\ptemp}{\cen}{\sfact}{\stemp}{\lb}{\ub} =
\set{
  \cen+\ptemp\zeta+\stemp\epsilon:
  \begin{array}{l}
    \zeta\in\compnums^m,~\epsilon\in\reals^k,\\
    \absolute{\zeta}\leq
    \sfact,~\lb\leq\epsilon\leq\ub
   \end{array}
}.
\]
%
\end{definition}
%
Geometrically speaking, the real projection of a complex zonotope can
be a Minkowski sum of both ellipses and line segments embedded inside
possibly higher dimensional spaces.  For example, Figure~\ref{} shows
a complex zonotope which is a Minkowski sum of two ellipses and one
line segment.  Therefore, complex zonotopes, which can describe some
non-polyhedral sets in addition to polytopic simple zonotopes, are
geometrically more expressive than the latter.

The linear transformation of a complex zonotope can be computed
efficiently by an algebraic expression which is a linear function of
the scaling factors, primary offset and the interval bounds.
Furthermore, a complex zonotope can efficiently encode a positive
invariant of an invertible stable linear transformation by capturing
contraction along the complex eigenvectors.  This is described in the
following lemma.

\emph{Notation}:  From now, we use the following notation unless
otherwise specified:
%
$
{\ptemp\in\mat{n}{m}{\compnums}},~{\cen\in\reals^n},{\sfact\in\reals_{\geq 0}^m},~\stemp\in\mat{n}{k}{\reals},~{\lb,\ub\in\reals^k}.
$
%
\begin{lemma}[Linear transformation and eigenstructure based contraction]~\label{lem:lintransform}
Let us consider $\Xi\in\mat{n}{n}{\compnums}$ contains the complex
eigenvectors of $A\in\mat{n}{n}{\reals}$ as its columns vectors such
that $A\Xi=\Xi\diagonal{\mu}$ for a vector of eigenvalues
$\mu\in\compnums^n$.  All of the following statements are true.
%
\begin{enumerate}
\item
  $A\acztope{\ptemp}{\cen}{\sfact}{\stemp}{\lb}{\ub}=\acztope{A\ptemp}{A\cen}{\sfact}{A\stemp}{\lb}{\ub}$.
\item
  $A\acztope{\Xi}{0}{\sfact}{0}{0}{0}=\acztope{\Xi}{0}{\diagonal{\absolute{\mu}}\sfact}{0}{0}{0}$.
\item If $\infnorm{\mu}\leq 1$, then
  \[A\acztope{\Xi}{0}{\sfact}{0}{0}{0}\subseteq\acztope{\Xi}{0}{\sfact}{0}{0}{0}.\]
\end{enumerate}
%
\end{lemma}
%
\begin{proof}
The proof of first statement is derived as follows.
%
\begin{align*}
& A\acztope{\ptemp}{\cen}{\sfact}{\stemp}{\lb}{\ub}\\
& = A\set{
  \cen+\ptemp\zeta+\stemp\epsilon:
  \begin{array}{l}
    \zeta\in\compnums^m,~\epsilon\in\reals^k,\\
    \absolute{\zeta}\leq
    \sfact,~\lb\leq\epsilon\leq\ub
   \end{array}
  }\\
& = \set{
  A\cen+A\ptemp\zeta+A\stemp\epsilon:
  \begin{array}{l}
    \zeta\in\compnums^m,~\epsilon\in\reals^k,\\
    \absolute{\zeta}\leq
    \sfact,~\lb\leq\epsilon\leq\ub
   \end{array}
  }\\
& = \acztope{A\ptemp}{A\cen}{\sfact}{A\stemp}{\lb}{\ub}.
\end{align*}
%
By the above derivation we get
$A\acztope{\Xi}{0}{\sfact}{0}{0}{0}=\acztope{A\Xi}{0}{\sfact}{0}{0}{0}=\acztope{\Xi\diagonal{\mu}}{0}{\sfact}{0}{0}{0}$.
We have
%
\begin{align*}
& \set{\diagonal{\mu}\zeta:\zeta\in\compnums^m,\absolute{\zeta}\leq
    \sfact}\\
& =\set{\zeta^\pr:\zeta^\pr\in\compnums^m,\absolute{\zeta^\pr}\leq\diagonal{\absolute{\mu}}\sfact}\\
& \therefore
  A\acztope{\Xi}{0}{\sfact}{0}{0}{0}=\acztope{\Xi}{0}{\diagonal{\absolute{\mu}}\sfact}{0}{0}{0}.
\end{align*}
%
If $\infnorm{\mu}\leq 1$, then $\diagonal{\absolute{\mu}}{\sfact}\leq
\sfact$.  Then we get
$A\acztope{\Xi}{0}{\sfact}{0}{0}{0}=\acztope{\Xi}{0}{\diagonal{\absolute{\mu}}\sfact}{0}{0}{0}\subseteq\acztope{\Xi}{0}{\sfact}{0}{0}{0}$.
\end{proof}
%
If an $n\times n$ matrix is invertible, then it has $n$-linearly
independent eigenvectors.  So, a complex zonotope whose template
contains the $n$-linearly independent eigenvectors and the scaling
factors are non-zero has a non-empty interior.  Additionally, if the
invertible matrix is also Schur stable, then we get a complex zonotope
with non-empty interior, which after transformation by the matrix is
contained inside itself according to the above lemma.  In other words,
we can specify a positive invariant complex zonotope with non-empty
interior for a stable and invertible linear transformation using the
eigenstructure.

Complex zonotopes are also closed under Minkowski sum, which can be
computed by an affine expression as
follows.
%
\begin{lemma}
Let us consider $\ptemp^\pr\in\mat{n}{r}{\compnums}$ and
$\stemp^\pr\in\mat{n}{h}{\reals}$.  Then the following is true.
%
\begin{align*}
&
  \acztope{\ptemp}{\cen}{\sfact}{\stemp}{\lb}{\ub}\oplus\acztope{\ptemp^\pr}{\cen^\pr}{
    \sfact^\pr}{\stemp^\pr}{\lb^\pr}{\ub^\pr}\\
& = \acztope{
    \mymatrix{\ptemp & \ptemp^\pr}
  }
          {
            \mymatrix{\cen+\cen^\pr}
          }
          {\mymatrix{\sfact\\\sfact^\pr}
          }
          {\mymatrix{\stemp & \stemp^\pr}
          }
          {\mymatrix{\lb\\ \lb^\pr}
          }
          {\mymatrix{\ub\\ \ub^\pr}
          }.
\end{align*}
%
\end{lemma}
%
\begin{proof}
$\acztope{\ptemp}{\cen}{\sfact}{\stemp}{\lb}{\ub}\oplus\acztope{\ptemp^\pr}{\cen^\pr}{
    \sfact^\pr}{\stemp^\pr}{\lb^\pr}{\ub^\pr}=$
%
\begin{align*}
&  \lt\{\cen+\cen^\pr+\ptemp\zeta+\stemp\epsilon+\ptemp^\pr\zeta^\pr+\stemp^\pr\epsilon^\pr\rt.:
  \lt. \zeta\in\compnums^m,\zeta^\pr\in\compnums^r,\rt.\\
  & \lt. \epsilon\in\reals^k,\epsilon^\pr\in\reals^h,
    \absolute{\zeta}\leq\sfact,\lb\leq\epsilon\leq
    \ub,\absolute{\zeta^\pr}\leq\sfact^\pr,\lb^\pr\leq\epsilon^\pr\leq
    \ub^\pr\rt\}=\\
& \lt\{\lt(\cen+\cen^\pr\rt)+\mymatrix{\ptemp
      &\ptemp^\pr}\mymatrix{\zeta\\\zeta^\pr}+\mymatrix{\stemp
      &\stemp^\pr}\mymatrix{\epsilon\\ \epsilon^\pr}: \mymatrix{\zeta\\\zeta^\pr}\in\compnums^{m+r}\rt.\\
    & \lt. \mymatrix{\epsilon\\\epsilon^\pr}\in\reals^{k+h},\rt.
    \lt.\absolute{\begin{matrix}\zeta\\\zeta^\pr\end{matrix}}\leq
        \mymatrix{\sfact\\\sfact^\pr},
        \mymatrix{\lb\\\lb^\pr}\leq\mymatrix{\epsilon\\\epsilon^\pr}\leq\mymatrix{\ub\\\ub^\pr}
        \rt\}\\
& = \acztope{
    \mymatrix{\ptemp & \ptemp^\pr}
  }
          {
            \mymatrix{\cen+\cen^\pr}
          }
          {\mymatrix{\sfact\\\sfact^\pr}
          }
          {\mymatrix{\stemp & \stemp^\pr}
          }
          {\mymatrix{\lb\\ \lb^\pr}
          }
          {\mymatrix{\ub\\ \ub^\pr}
          }.\hspace{1em}\qedhere
\end{align*}
%
\end{proof}
%
While computing positive invariants using a set representation,
ascertaining the positive invariance of a set requires deciding the
inclusion of the next reachable set inside the given set.  In the case
of complex zonotopes, we shall show that checking the exact inclusion
amounts to solving a is a non-convex optimization problem.  Therefore,
we later find a sufficient condition expressed by convex constraints
for checking the inclusion.  The convex constraints we derive later
are specifically second order conic constraints, which are described
below.
%
\begin{definition}[Second order conic constraint]
A second order conic constraint on a variable $x$ taking values in
$\reals^n$ is one of the following expressions.
\begin{enumerate}
\item $\sqnorm{Ax+b}\leq c^Tx+d$ where $A\in\mat{r}{n}{\reals}$,
  $b\in\reals^r$, $c\in\reals^n$ and $d\in\reals$.
\item $p^Tx=q$ where $p\in\reals^n$ and $q\in\reals$.
\end{enumerate}
\end{definition}
%
\begin{example}
%
An inequality like $x^2+4y^2+25z^2-3x-4y+z+3\leq 0$ is a second order
  conic constraint because it can be written as
%
\[
\norm{\mymatrix{
    1 & 0 & 0\\
    0 & 2 & 0\\
    0 & 0 & 5
}\mymatrix{x\\y\\z}}\leq \mymatrix{3 & 4 & -1}\mymatrix{x\\y\\z}+2.
\]
A linear equality like $3x+2y-4z=5$ is also a second order conic
constraint.
\end{example}
%
In the case of complex zonotope, we shall later derive a set of second
order conic constraints, which have to be collectively satisfied to
guarantee inclusion.  Given a set of second order conic constraints on
a variable $x\in\reals^n$, solving the constraints refers to finding a
value $x^*\in\reals^n$ that satisfies the constraints.  A value
$x^\pr\in\reals^n$ is called an approximate solution within a
precision $\epsilon\in\reals_{\geq 0}$ if there exists a solution
$x^*\in\reals^n$ such that $\sqnorm{x^\pr-x^*}\leq \epsilon$.  There
are tools based on interior point methods (see~\cite{grant2008cvx})
that can efficiently find approximate solutions with very high
precision to second order conic constraints (SOCC).

Checking inclusion of a single point inside a template complex
zonotope is equivalent to solving SOCC, as described below.
%
\begin{lemma}[Inclusion of a point]
Let us consider a point $x\in\compnums^n$.  Then
$x\in\tcztope{\ptemp}{\cen}{\sfact}\subset\compnums^n$ if and only if
all of the following is collectively true.
%
\begin{align}
& \exists\zeta\in\compnums^m:\nonumber\\
& \ptemp\zeta = x-c~\label{eqn:lem-point-inclusion-1}\\
& \absolute{\zeta}\leq \sfact.~\label{eqn:lem-point-inclusion-2}
\end{align}
%
\end{lemma}
%
\begin{proof}
The above result follows from the fact that any point 
$x\in\tcztope{\ptemp}{\cen}{\sfact}$ is of the form
$x=\cen+\ptemp\zeta$ for some $\zeta\in\compnums^m$ such that
$\absolute{\zeta}\leq \sfact$.
\end{proof}
%
\begin{example}
Let us consider the template complex zonotope
$\tcztope{\ptemp}{\cen}{\sfact}\subset\compnums^2$ and a point $x\in\compnums^2$,
where
%
\[
\ptemp=\mymatrix{1+\iota & 1 & 0\\1 & 0 & 1},~~\cen = \mymatrix{\iota\\ 1},~~\sfact=\mymatrix{1\\1\\1}~\text{and}~x=\mymatrix{2\iota-2\\\iota+2}.
\]
%
To prove that $x\in\tcztope{\ptemp}{\cen}{\sfact}$, let us consider
$\zeta=\mymatrix{\iota & -1 & 1}^T$.  Then we get
%
\[
\ptemp\zeta = \mymatrix{\iota-2\\\iota+1}= \mymatrix{2\iota-2\\\iota+2}-\mymatrix{\iota\\1}=x-c.
\]
%
Therefore, Equation~\ref{eqn:lem-point-inclusion-1} is satisfied.  Furthermore,
$\absolute{\zeta}=\mymatrix{1 & 1 & 1}^T$.  So, Equation~\ref{eqn:lem-point-inclusion-2} is
also satisfied.  Henceforth, $x\in\tcztope{\ptemp}{\cen}{\sfact}$.
\end{example}
%
Equation~\ref{eqn:lem-point-inclusion-1} is an equality constraint on
$\zeta$, which is therefore an SOCC.  We know that the absolute value
of a complex number is the square norm of a two dimensional vector.
So, Equation~\ref{eqn:lem-point-inclusion-2} is equivalent to a set of
square norm constraints on the real and imaginary components of
$\zeta$, which are therefore SOCC constraints.  Hence, the inclusion
of a point inside a template complex zonotope can be checked by
solving second order conic constraints.

Now we state the necessary and sufficient condition for checking
inclusion between two template complex zonotopes.
%
\begin{lemma}[Exact inclusion between template complex zonotopes]~\label{lem:exact-inclusion}
Let us consider $\ptemp\in\mat{n}{m}{\compnums}$ and
$\ptemp^\pr\in\mat{n}{r}{\compnums}$.  The inclusion
$\tcztope{\ptemp^\pr}{\cen^\pr}{\sfact^\pr}\subseteq\tcztope{\ptemp}{\cen}{\sfact}$
holds if and only if
\begin{equation}\label{eqn:exact-inclusion}
\max_{\set{\zeta^\pr\in\compnums^{r}:\absolute{\zeta^\pr}\leq \sfact^\pr}}\min_{\set{\zeta\in\compnums^m:\ptemp\zeta=\ptemp^\pr\zeta^\pr+\cen^\pr-\cen}}\max_{i=1}^m\lt(\absolute{\zeta_i}-s_i\rt)\leq 0
\end{equation}
\end{lemma}
%
\begin{proof}
  We have
  %
  \begin{align*}
    &\tcztope{\ptemp}{\cen}{\sfact}=\set{\cen+\ptemp\zeta:~\zeta\in\compnums^m,~\absolute{\zeta}\leq\sfact},\\
    &\tcztope{\ptemp^\pr}{\cen^\pr}{\sfact^\pr}=\set{\cen^\pr+\ptemp^\pr\zeta^\pr:~\zeta^\pr\in\compnums^r,~\absolute{\zeta^\pr}\leq\sfact^\pr}.
  \end{align*}
  %
Therefore, we get
$\tcztope{\ptemp^\pr}{\cen^\pr}{\sfact^\pr}\subseteq\tcztope{\ptemp}{\cen}{\sfact}$
if and only if
for every $\zeta^\pr\in\compnums^r:\absolute{\zeta^\pr}\leq \sfact^\pr$,
there exists
$\zeta\in\compnums^m:\ptemp\zeta+\cen=\ptemp^\pr\zeta^\pr+\cen^\pr~\wedge~\absolute{\zeta}\leq
\sfact$.  This is equivalently expressed as the constraint in Equation~\ref{eqn:exact-inclusion}.
\end{proof}
%
The reason solving Equation~\ref{eqn:exact-inclusion}
requires non-convex optimization is explained as follows.  Let us consider
that $\ptemp$ has a pseudo-inverse $\pinv{\ptemp}$.  Then by the
rank-nullity theorem
%
\[
\set{\zeta:~\ptemp\zeta=\ptemp\zeta^\pr+\cen^\pr-\cen}=\set{\pinv{\ptemp}\lt(\zeta^\pr-c\rt)+v:~v\in\nullspace{\ptemp}}
\]
%
So,
%
\begin{align*}
& \min_{\set{\zeta\in\compnums:\ptemp\zeta=\ptemp^\pr\zeta^\pr+\cen^\pr-\cen}}\max_{i=1}^m\lt(\absolute{\zeta_i}-s_i\rt)\\
&
=\min_{\set{v\in\nullspace{\ptemp}}}\max_{i=1}^m\lt(\absolute{\pinv{\ptemp}\lt(\zeta^\pr-c\rt)+v}-s_i\rt)
\end{align*}
%
The absolute value of a complex variable is a convex quadratic
function of the real and imaginary components of the variable.  So,
the above function is a point-wise minimum (for points $v$ in the null
space $\null{\ptemp}$) of a set of convex quadratic functions over
$\zeta^\pr$, which is therefore a non-concave function of $\zeta^\pr$.
So the maximization
%
\[
\max_{\set{\zeta^\pr\in\compnums^{r}:\absolute{\zeta^\pr}\leq \sfact^\pr}}\min_{\set{\zeta\in\compnums^m:\ptemp\zeta=\ptemp^\pr\zeta^\pr+\cen^\pr-\cen}}\max_{i=1}^m\lt(\absolute{\zeta_i}-s_i\rt)
\]
%
is equivalent to maximizing a non-concave function of $\zeta^\pr$.
Maximizing a non-concave function is a non-convex optimization problem.

Alternatively, we shall now derive a sufficient condition, equivalent
to a set of second order conic constraints, for checking inclusion
between two template complex zonotopes.  The following result is used
to later derive the sufficient condition.
%
\begin{lemma}~\label{lem:transfer-matrix}
  Let us consider ${\sfact\in\reals^m_{\geq 0}}$,
  ${\sfact^\pr\in\reals^r_{\geq 0}}$, ${\zeta^\pr\in\compnums^r}$,
  ${\cen,\cen^\pr\in\compnums^n}$,${\ptemp\in\mat{n}{m}{\compnums}}$, ${\ptemp^\pr\in\mat{n}{r}{\compnums}}$ 
  ${\absolute{\zeta^\pr}\leq\sfact^\pr}$,
  ${\tmat\in\mat{m}{r}{\compnums}}$  and ${y\in\compnums^m}$ such that
  %
  \begin{align*}
&
    \ptemp\tmat=\ptemp^\pr\diagonal{\sfact^\pr},\hspace{1em}\ptemp\tvect=\lt(c^\pr-c\rt).~\numberthis\label{eqn:inclusion1}
    \\
&\text{Then}\hspace{2em}\min_{\set{\zeta\in\compnums:\ptemp\zeta=\ptemp^\pr\zeta^\pr+\cen^\pr-\cen}}\max_{i=1}^m\lt(\absolute{\zeta_i}-\sfact_i\rt)\leq \max_{i=1}^m\lt(\absolute{\tvect_i}+\sum_{j=1}^r\absolute{\tmat_{ij}}-\sfact_i\rt).~\numberthis\label{eqn:transfer-matrix}
\end{align*}
%
\end{lemma}
%
\begin{proof}
  Let us consider $\epsilon\in\compnums^{r}$, such that for any $i\in\set{1,...,r}$,
%
\[\left\{
\begin{array}{l}
\epsilon_i=\frac{\zeta^\pr}{s^\pr_i}~\text{if}~ s^\pr_i\neq 0\\
\epsilon_i=0~\text{otherwise}
\end{array}
\right..\]
%
From the above definition and the fact that $\absolute{\zeta^\pr}\leq
s^\pr$, we get $\zeta^\pr=\diagonal{s^\pr}\epsilon$ and
$\max_{j=1}^r\absolute{\epsilon_j}\leq 1$.  Then we derive
%
\begin{align*}
&\ptemp^\pr\zeta^\pr+c-c^\pr
=\ptemp^\pr\diagonal{\sfact^\pr}\epsilon+c-c^\pr
=\ptemp\tmat\epsilon+\ptemp\tvect
=\ptemp\lt(\tmat\epsilon+\tvect\rt)
\end{align*}
%
According the above equation,
%
\begin{align*}
& \tmat\epsilon+\tvect\in\set{\zeta\in\compnums:\ptemp\zeta=\ptemp^\pr\zeta^\pr+\cen^\pr-\cen}.\\
& \implies
  \min_{\set{\zeta\in\compnums:\ptemp\zeta=\ptemp^\pr\zeta^\pr+\cen^\pr-\cen}}\max_{i=1}^m\lt(\absolute{\zeta_i}-\sfact_i\rt)\leq
  \max_{i=1}^m\lt(\absolute{\lt(X\epsilon+y\rt)_i}-\sfact_i\rt)\\
&   ~~\%\%~\text{Using triangular inequality}\\
&\leq \max_{i=1}^m\lt(\absolute{\tvect_i}+\sum_{j=1}^r\absolute{\tmat_{ij}}\absolute{\epsilon_j}-\sfact_i\rt)\\
& ~~\%\%~\text{Since}~\max_{j=1}^r\absolute{\epsilon_j}\leq 1\\
& \leq  \max_{i=1}^m\lt(\absolute{\tvect_i}+\sum_{j=1}^r\absolute{\tmat_{ij}}-\sfact_i\rt).\hspace{3em}\qedhere
\end{align*}
%
\end{proof}
%
We define the following relation between two template
complex zonotopes, which we shall prove is a sufficient condition for
the inclusion between them.
%
\begin{definition}[Relation for inclusion-checking]~\label{defn:inclusion-tcz}
Let us consider $\ptemp\in\mat{n}{m}{\compnums}$ and
$\ptemp^\pr\in\mat{n}{r}{\compnums}$.  We say
$\tcztope{\ptemp}{\cen}{\sfact}\order\tcztope{\ptemp^\pr}{\cen^\pr}{\sfact^\pr}$
iff all of the following is collectively true.
%
\begin{align*}
& \exists \tmat\in\mat{m}{r}{\compnums},\tvect\in\compnums^m~~\text{such
that}\\
& \ptemp\tmat=\ptemp^\pr\diagonal{\sfact^\pr},~~\ptemp\tvect=\cen^\pr-\cen~\numberthis\label{eqn:inclusion-tcz1}\\
& \max_{i=1}^m\lt(\absolute{\tvect_i}+\sum_{j=1}^r\absolute{\tmat_{ij}}-\sfact_i\rt)\leq
0~\numberthis\label{eqn:inclusion-tcz2}.
\end{align*}
%
\end{definition}
%
\begin{theorem}[Inclusion checking]~\label{thm:suff-inclusion}
 If 
 ${\tcztope{\ptemp}{\cen}{\sfact}\order\tcztope{\ptemp^\pr}{\cen^\pr}{\sfact^\pr}}$
 then\\
${\tcztope{\ptemp}{\cen}{\sfact}\subseteq\tcztope{\ptemp^\pr}{\cen^\pr}{\sfact^\pr}}$.
%
\end{theorem}
%
\begin{proof}
The theorem follows from Lemmas~\ref{lem:exact-inclusion}
and~\ref{lem:transfer-matrix}.  By
Lemma~\ref{lem:exact-inclusion}, the inclusion
$\tcztope{\ptemp}{\cen}{\sfact}\subseteq\tcztope{\ptemp^\pr}{\cen^\pr}{\sfact^\pr}$
holds iff the L.H.S of Equation~\ref{eqn:exact-inclusion} is bounded
above by zero.  According to Lemma~\ref{lem:transfer-matrix}, if there
exists a $\tmat$ and $\tvect$ satisfying
Equation~\ref{eqn:inclusion1}, then the R.H.S of
Equation~\ref{eqn:transfer-matrix} is an upper bound on the L.H.S of
Equation~\ref{eqn:exact-inclusion}.  So, if there exist $\tmat$ and
$\tvect$ satisfying Equation~\ref{eqn:inclusion1} such that the R.H.S
of Equation~\ref{eqn:transfer-matrix} is bounded above by zero, then
the inclusion holds.  The relation
$\tcztope{\ptemp}{\cen}{\sfact}\order\tcztope{\ptemp^\pr}{\cen^\pr}{\sfact^\pr}$
implies that there exists $\tmat$ and $\tvect$ satisfying
Equations~\ref{eqn:inclusion1} such that the R.H.S of
Equation~\ref{eqn:transfer-matrix} is bounded above by zero.
\end{proof}
%
\begin{remark}~\label{rem:socc}
  The following constraints are equivalent to Equation~\ref{eqn:inclusion-tcz2} as
  %
  \begin{align*}
\forall
& i\in\set{1,...,m}, \exists ~a_i,b_i\\
& ~\norm{\mymatrix{\real\lt(X^T_i\rt)\\\img\lt(X^T_i\rt)}}\leq a_i, \hspace{2em}
\norm{\mymatrix{\real\lt(y_i\rt)\\\img\lt(y_i\rt)}}\leq b_i, \hspace{2em}
 0\leq \sfact_i-a_i-b_i.
\end{align*}
%
  The above is equivalent to $3m$ second order conic constraints on a
  \emph{real vector} of size at most $2mr$, which comprises the
  scaling factors and the additional variables.  Next, for a fixed
  template, Equation~\ref{eqn:inclusion-tcz1} is a set of $n(n+1)$
  linear constraints on a \emph{complex vector} of size at most
  $mr+2m+r$, which comprises the center, scaling factors and the
  additional variables.  On the other hand, the representation size of
  both the template complex zonotopes together is $(m+r)n$.
  Therefore, the size of the second order conic program for checking
  the inclusion between two template complex zonotopes can be at most
  of the order cubic in the size of the template complex zonotopes.
\end{remark}
%
%% Furthermore, the above relation is a partial order.
%
%% The relation ``$\order$'' is a partial order on the set of template
%% complex zonotopes, as stated in the following theorem.
%% %
%% \begin{theorem}[Partial ordering]
%% For any three template complex zonotopes\\
%% ${\tcz{\ptemp}{\cen}{\sfact},\tcz{\ptemp^\pr}{\cen^\pr}{\sfact^\pr}\text{
%%     and }\tcz{\ptemp^\dpr}{\cen^\dpr}{\sfact^\dpr}}$,
%% all of the following conditions are true.
%% %
%% \begin{enumerate}
%% \item Reflexivity:
%% $\tcz{\ptemp}{\cen}{\sfact}\order\tcz{\ptemp}{\cen}{\sfact}$.
%% \item Anti-symmetry: If~
%% $\tcz{\ptemp}{\cen}{\sfact}\order\tcz{\ptemp^\pr}{\cen^\pr}{\sfact^\pr}$
%% and
%% $\tcz{\ptemp^\pr}{\cen^\pr}{\sfact^\pr}\order\tcz{\ptemp}{\cen}{\sfact}$,
%% then
%% $\tcz{\ptemp}{\cen}{\sfact}=\tcz{\ptemp^\pr}{\cen^\pr}{\sfact^\pr}$.
%% \item Transitivity: If~
%% $\tcz{\ptemp^\pr}{\cen^\pr}{\sfact^\pr}\order\tcz{\ptemp}{\cen}{\sfact}$
%% and
%% $\tcz{\ptemp^\dpr}{\cen^\dpr}{\sfact^\dpr}\order\tcz{\ptemp^\pr}{\cen^\pr}{\sfact^\pr}$,
%% then $\tcz{\ptemp^\dpr}{\cen^\dpr}{\sfact^\dpr}\order\tcz{\ptemp}{\cen}{\sfact}$.
%% \end{enumerate}
%% %
%% \end{theorem}

%% \begin{proof}
%% We prove reflexivity, antisymmetry and transitivity separately as follows.
%% \begin{enumerate}
%% \item {\it Reflexivity}:  Let us consider
%%   %
%%   \begin{align*}
%% &  \tmat=\diagonal{\sfact}\text{ and }
%%   \tvect=\repmat{m}{1}{0}.\\
%% &\text{Then we get }~\ptemp\tmat=\ptemp\diagonal{\sfact},~~
%% \ptemp\tvect=0=c-c\text{ and }\\
%% & \max_{i=1}^m\lt(\absolute{y_i}+\sum_{j=1}^m\absolute{X_{ij}}-\sfact_i\rt)
%% =\max_{i=1}^m\lt(0+\sfact_i-\sfact_i\rt)=0\\
%% & \text{So}, \tcz{\ptemp}{\cen}{\sfact}\order\tcz{\ptemp}{\cen}{\sfact}.
%% \end{align*}
%% %
%% \item {\it Anti-symmetry}: Let us consider that
%% %
%% \begin{align*}
%% &
%% \tcz{\ptemp}{\cen}{\sfact}\order\tcz{\ptemp^\pr}{\cen^\pr}{\sfact^\pr}
%% \text{ and }
%% \tcz{\ptemp^\pr}{\cen^\pr}{\sfact^\pr}\order\tcz{\ptemp}{\cen}{\sfact}.\\
%% & \text{Then by Theorem~\ref{thm:suff-inclusion}, we get}\\ 
%% & \tcz{\ptemp}{\cen}{\sfact}\subseteq\tcz{\ptemp^\pr}{\cen^\pr}{\sfact^\pr}
%% \text{ and }
%% \tcz{\ptemp^\pr}{\cen^\pr}{\sfact^\pr}\subseteq\tcz{\ptemp}{\cen}{\sfact}\\
%% &\implies
%% \tcz{\ptemp}{\cen}{\sfact}=\tcz{\ptemp^\pr}{\cen^\pr}{\sfact^\pr}.
%% \end{align*}
%% %
%% \item {\it Transitivity:}  
%% \end{enumerate}
%% \end{proof}


When the template of the complex zonotope inside which containment is
checked is invertible, and the centers of the template complex
zonotopes are same, then the above sufficient condition for inclusion
checking is also a necessary condition.  This is described in the
following theorem.
%
\begin{theorem}
Let us consider
$\ptemp\in\mat{n}{n}{\compnums}$ and
$\ptemp^\pr\in\mat{n}{m}{\compnums}$ such that $\ptemp$ is a non-singular matrix.
Then
$\tcztope{\ptemp^\pr}{\cen^\pr}{\sfact^\pr}\subseteq\tcztope{\ptemp}{\cen}{\sfact}$
if and only if
$\tcztope{\ptemp^\pr}{\cen}{\sfact^\pr}\order\tcztope{\ptemp}{\cen}{\sfact}$.
\end{theorem}
%
\begin{proof}
By Theorem~\ref{thm:suff-inclusion}, we know that if
$\tcztope{\ptemp^\pr}{\cen}{\sfact^\pr}\order\tcztope{\ptemp}{\cen}{\sfact}$
is true,
then we get
$\tcztope{\ptemp^\pr}{\cen}{\sfact^\pr}\subseteq\tcztope{\ptemp}{\cen}{\sfact}$.
So, we have to prove the converse that if
$\tcztope{\ptemp^\pr}{\cen}{\sfact^\pr}\subseteq\tcztope{\ptemp}{\cen}{\sfact}$,
then
$\tcztope{\ptemp^\pr}{\cen}{\sfact^\pr}\order\tcztope{\ptemp}{\cen}{\sfact}$.

Let us consider
$\tcztope{\ptemp^\pr}{\cen}{\sfact^\pr}\subseteq\tcztope{\ptemp}{\cen}{\sfact}$.
Using
Lemma~\ref{lem:normalization}, we get
%
\begin{align*}
&
  \tcztope{\ptemp^\pr}{\cen}{\sfact^\pr}=\cztope{\ptemp^\pr\diagonal{\sfact^\pr}}{\cen}\subseteq\tcztope{\ptemp}{\cen}{\sfact}\\ &\equivalent~\set{\cen+\ptemp^\pr\diagonal{\sfact^\pr}\zeta^\pr:~\zeta^\pr\in\compnums^m,~\infnorm{\zeta^\pr}\leq
    1}\subseteq\set{\cen+\ptemp\zeta:~\zeta\in\compnums^n,~\absolute{\zeta}\leq
    \sfact}\\ & \%\%~\text{since}~\ptemp~\text{is non-singular}\\ 
  \equivalent&
  \set{\inv{\ptemp}\lt(\cen-\cen\rt)+\inv{\ptemp}\ptemp^\pr\diagonal{\sfact^\pr}\zeta^\pr:~\zeta^\pr\in\compnums^m,~\infnorm{\zeta^\pr}\leq
    1}\\& \subseteq
  \set{\zeta:\zeta\in\compnums^n,~\absolute{\zeta}\leq\sfact}\\
 & \equivalent~\set{\inv{\ptemp}\ptemp^\pr\diagonal{\sfact^\pr}\zeta^\pr:~\zeta^\pr\in\compnums^m,~\infnorm{\zeta^\pr}\leq
    1}\subseteq
  \set{\zeta:\zeta\in\compnums^n,~\absolute{\zeta}\leq\sfact} ~\numberthis\label{proof-necc-inc1}
\end{align*}
%
Let $\tmat=\inv{\ptemp}\ptemp^\pr\diagonal{\sfact^\pr}$ and
$\tvect=0$.  Then by
Equation~\ref{proof-necc-inc1}, we get for any $i\in\set{1,...,n}$
%
\begin{align*}
& \max_{\zeta^\pr\in\compnums^m:~\infnorm{\zeta}\leq 1}\absolute{\sum_{j=1}^m\tmat_{ij}\zeta_i}\leq\sfact_i.
\hspace{1em}\therefore \sum_{j=1}^m\absolute{\tmat_{ij}\frac{\absolute{\tmat_{ij}}}{\tmat_{ij}}}
    \leq\sfact_i\\
    & \%\%~~\text{since}~y_i=0\\
& \therefore \sum_{i=1}^n\absolute{\tmat_{ij}}  +\absolute{\tvect_i}\leq \sfact_i .~\numberthis\label{proof-necc-inc2}
\end{align*}
%
Furthermore, we have
%
\begin{align*}
&
  \ptemp\tmat=\ptemp\inv{\ptemp}\ptemp^\pr\diagonal{\sfact}=\ptemp^\pr\diagonal{\sfact^\pr}~\text{
    and }\\
& \ptemp\tvect=0=\cen-\cen.~\numberthis\label{proof-necc-inc3}      
\end{align*}
%
By Equations~\ref{proof-necc-inc2} and~\ref{proof-necc-inc3}, we get $\tcztope{\ptemp^\pr}{\cen}{\sfact^\pr}\order\tcztope{\ptemp}{\cen}{\sfact}$.
\end{proof}


