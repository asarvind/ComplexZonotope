Before introducing a complex zonotope, we shall briefly describe the
previously known simple/real valued zonotope.  A simple zonotope
consists of points generated by a linear combination of real valued
vectors translated by a center such that the combining coefficients
are bounded inside symmetric intervals.  Geometrically speaking, a
simple zonotope is a Minkowski sum of line segments.  It can be
represented by matrix of generators and a center, as follows.
%
\begin{definition}
Let us consider $\stemp\in\mat{n}{k}{\reals}$ and $\cen\in\reals^n$.
A real zonotope is represented as follows.
%
\[
\rztope{\stemp}{\cen} = \set{\cen+\stemp\zeta:~\zeta\in[-1,1]^k}.
\]
%
\end{definition}
%
A main advantage of the above representation of a simple zonotope is
that besides being closed under linear transformation and Minkowski
sum, these operations can be computed efficiently (see~\cite{todo}).
However, for a stable linear transformation having complex
eigenvalues, it is not known whether there exists a positively
invariant simple zonotope with non-empty interior.  %% In contrast, some
%% set representations like ellipsoids can efficiently encode positive
%% invariants of stable linear transformations~\cite{todo}.  However,
%% ellipsoids are not closed under Minkowski sum.
Henceforth, we want to
develop a set representation that can efficiently encode positive
invariants of linear transformation with complex eigenstructure, while
the Minkowski sum and linear transformation opeartions are closed and
efficiently computable.  In this regard, we extend simple
zonotopes to the complex valued domain to yeild a set
representation called complex zonotope that has the above
desirable property.  A complex zonotope consists of points represented
by a linear combination of complex valued vectors with translation,
where some of the combining coefficients are complex valued and
bounded in their absolute values, while others are real valued and
bounded inside real intervals.  We shall later show in Lemma~\ref{lem:lintransform}
how a complex zonotope can efficiently encode a positive invariant of
a stable linear transformation using the complex eigenstructure.
%
\begin{definition}
Let us consider $\ptemp\in\mat{n}{m}{\compnums}$, called a
\emph{primary template}, $\sfact\in\reals_{\geq 0}$ called scaling
factors, $\cen\in\reals^n$ called \emph{primary offset},
$\stemp\in\mat{n}{k}{\reals}$ called secondary template,
$\lb,\ub\in\reals^k:~\lb\leq\ub$, called \emph{lower and upper
  interval bounds}, respectively.  An complex zonotope is
represented as follows.
%
\[
\acztope{\ptemp}{\cen}{\sfact}{\stemp}{\lb}{\ub} =
\set{
  \cen+\ptemp\zeta+\stemp\epsilon:
  \begin{array}{l}
    \zeta\in\compnums^m,~\epsilon\in\reals^k,\\
    \absolute{\zeta}\leq
    \sfact,~\lb\leq\epsilon\leq\ub
   \end{array}
}.
\]
%
\end{definition}
%
Geometrically speaking, the real projection of a complex zonotope can
be a Minkowski sum of both ellipses and line segments embedded inside
possibly higher dimensional spaces.  For example, Figure~\ref{} shows
a complex zonotope which is a Minkowski sum of two ellipses and one
line segment.  Therefore, complex zonotopes, which can describe some
non-polyhedral sets in addition to polytopic simple zonotopes, are
geometrically more expressive than the latter.

The linear transformation of a complex zonotope can be computed
efficiently by an algebraic expression which is a linear function of
the scaling factors, primary offset and the interval bounds.
Furthermore, a complex zonotope can efficiently encode a positive
invariant of an invertible stable linear transformation by capturing
contraction along the complex eigenvectors.  This is described in the
following lemma.

\emph{Notation}:  From now, we use the following notation unless
otherwise specified:
%
$
{\ptemp\in\mat{n}{m}{\compnums}},~{\cen\in\reals^n},{\sfact\in\reals_{\geq 0}^m},~\stemp\in\mat{n}{k}{\reals},~{\lb,\ub\in\reals^k}.
$
%
\begin{lemma}[Linear transformation and eigenstructure based contraction]~\label{lem:lintransform}
Let us consider $\Xi\in\mat{n}{n}{\compnums}$ contains the complex
eigenvectors of $A\in\mat{n}{n}{\reals}$ as its columns vectors such
that $A\Xi=\Xi\diagonal{\mu}$ for a vector of eigenvalues
$\mu\in\compnums^n$.  All of the following statements are true.
%
\begin{enumerate}
\item
  $A\acztope{\ptemp}{\cen}{\sfact}{\stemp}{\lb}{\ub}=\acztope{A\ptemp}{A\cen}{\sfact}{A\stemp}{\lb}{\ub}$.
\item
  $A\acztope{\Xi}{0}{\sfact}{0}{0}{0}=\acztope{\Xi}{0}{\diagonal{\absolute{\mu}}\sfact}{0}{0}{0}$.
\item If $\infnorm{\mu}\leq 1$, then
  \[A\acztope{\Xi}{0}{\sfact}{0}{0}{0}\subseteq\acztope{\Xi}{0}{\sfact}{0}{0}{0}.\]
\end{enumerate}
%
\end{lemma}
%
\begin{proof}
The proof of first statement is derived as follows.
%
\begin{align*}
& A\acztope{\ptemp}{\cen}{\sfact}{\stemp}{\lb}{\ub}\\
& = A\set{
  \cen+\ptemp\zeta+\stemp\epsilon:
  \begin{array}{l}
    \zeta\in\compnums^m,~\epsilon\in\reals^k,\\
    \absolute{\zeta}\leq
    \sfact,~\lb\leq\epsilon\leq\ub
   \end{array}
  }\\
& = \set{
  A\cen+A\ptemp\zeta+A\stemp\epsilon:
  \begin{array}{l}
    \zeta\in\compnums^m,~\epsilon\in\reals^k,\\
    \absolute{\zeta}\leq
    \sfact,~\lb\leq\epsilon\leq\ub
   \end{array}
  }\\
& = \acztope{A\ptemp}{A\cen}{\sfact}{A\stemp}{\lb}{\ub}.
\end{align*}
%
By the above derivation we get
$A\acztope{\Xi}{0}{\sfact}{0}{0}{0}=\acztope{A\Xi}{0}{\sfact}{0}{0}{0}=\acztope{\Xi\diagonal{\mu}}{0}{\sfact}{0}{0}{0}$.
We have
%
\begin{align*}
& \set{\diagonal{\mu}\zeta:\zeta\in\compnums^m,\absolute{\zeta}\leq
    \sfact}\\
& =\set{\zeta^\pr:\zeta^\pr\in\compnums^m,\absolute{\zeta^\pr}\leq\diagonal{\absolute{\mu}}\sfact}\\
& \therefore
  A\acztope{\Xi}{0}{\sfact}{0}{0}{0}=\acztope{\Xi}{0}{\diagonal{\absolute{\mu}}\sfact}{0}{0}{0}.
\end{align*}
%
If $\infnorm{\mu}\leq 1$, then $\diagonal{\absolute{\mu}}{\sfact}\leq
\sfact$.  Then we get
$A\acztope{\Xi}{0}{\sfact}{0}{0}{0}=\acztope{\Xi}{0}{\diagonal{\absolute{\mu}}\sfact}{0}{0}{0}\subseteq\acztope{\Xi}{0}{\sfact}{0}{0}{0}$.
\end{proof}
%
If an $n\times n$ matrix is invertible, then it has $n$-linearly
independent eigenvectors.  So, a complex zonotope whose template
contains the $n$-linearly independent eigenvectors and the scaling
factors are non-zero has a non-empty interior.  Additionally, if the
invertible matrix is also Schur stable, then we get a complex zonotope
with non-empty interior, which after transformation by the matrix is
contained inside itself according to the above lemma.  In other words,
we can specify a positive invariant complex zonotope with non-empty
interior for a stable and invertible linear transformation using the
eigenstructure.

Complex zonotopes are also closed under Minkowski sum, which can be
computed by an affine expression as
follows.
%
\begin{lemma}
Let us consider $\ptemp^\pr\in\mat{n}{r}{\compnums}$ and
$\stemp^\pr\in\mat{n}{h}{\reals}$.  Then the following is true.
%
\begin{align*}
&
  \acztope{\ptemp}{\cen}{\sfact}{\stemp}{\lb}{\ub}\oplus\acztope{\ptemp^\pr}{\cen^\pr}{
    \sfact^\pr}{\stemp^\pr}{\lb^\pr}{\ub^\pr}\\
& = \acztope{
    \mymatrix{\ptemp & \ptemp^\pr}
  }
          {
            \mymatrix{\cen+\cen^\pr}
          }
          {\mymatrix{\sfact\\\sfact^\pr}
          }
          {\mymatrix{\stemp & \stemp^\pr}
          }
          {\mymatrix{\lb\\ \lb^\pr}
          }
          {\mymatrix{\ub\\ \ub^\pr}
          }.
\end{align*}
%
\end{lemma}
%
\begin{proof}
$\acztope{\ptemp}{\cen}{\sfact}{\stemp}{\lb}{\ub}\oplus\acztope{\ptemp^\pr}{\cen^\pr}{
    \sfact^\pr}{\stemp^\pr}{\lb^\pr}{\ub^\pr}=$
%
\begin{align*}
&  \lt\{\cen+\cen^\pr+\ptemp\zeta+\stemp\epsilon+\ptemp^\pr\zeta^\pr+\stemp^\pr\epsilon^\pr\rt.:
  \lt. \zeta\in\compnums^m,\zeta^\pr\in\compnums^r,\rt.\\
  & \lt. \epsilon\in\reals^k,\epsilon^\pr\in\reals^h,
    \absolute{\zeta}\leq\sfact,\lb\leq\epsilon\leq
    \ub,\absolute{\zeta^\pr}\leq\sfact^\pr,\lb^\pr\leq\epsilon^\pr\leq
    \ub^\pr\rt\}=\\
& \lt\{\lt(\cen+\cen^\pr\rt)+\mymatrix{\ptemp
      &\ptemp^\pr}\mymatrix{\zeta\\\zeta^\pr}+\mymatrix{\stemp
      &\stemp^\pr}\mymatrix{\epsilon\\ \epsilon^\pr}: \mymatrix{\zeta\\\zeta^\pr}\in\compnums^{m+r}\rt.\\
    & \lt. \mymatrix{\epsilon\\\epsilon^\pr}\in\reals^{k+h},\rt.
    \lt.\absolute{\begin{matrix}\zeta\\\zeta^\pr\end{matrix}}\leq
        \mymatrix{\sfact\\\sfact^\pr},
        \mymatrix{\lb\\\lb^\pr}\leq\mymatrix{\epsilon\\\epsilon^\pr}\leq\mymatrix{\ub\\\ub^\pr}
        \rt\}\\
& = \acztope{
    \mymatrix{\ptemp & \ptemp^\pr}
  }
          {
            \mymatrix{\cen+\cen^\pr}
          }
          {\mymatrix{\sfact\\\sfact^\pr}
          }
          {\mymatrix{\stemp & \stemp^\pr}
          }
          {\mymatrix{\lb\\ \lb^\pr}
          }
          {\mymatrix{\ub\\ \ub^\pr}
          }.\hspace{1em}\qedhere
\end{align*}
%
\end{proof}
%
