\documentclass{llncs}
\usepackage{times}
\usepackage{amsmath}
\usepackage{amssymb} 
\usepackage{algorithm}
\usepackage{algpseudocode}
\usepackage{multirow}
\usepackage{url}
\usepackage{tikz} 
\usetikzlibrary{arrows,backgrounds,decorations,decorations.pathmorphing,positioning,fit,automata,shapes,snakes,patterns,plotmarks,calc}
\usepackage{graphicx,color} 
\usepackage{enumitem}
\usepackage{float}
\usepackage{bm}
\usepackage{array}
\usepackage{caption}
\usepackage{cite}
\usepackage{relsize}
%\usepackage{natbib}

\setlength{\itemsep}{0ex}
\setlength{\parsep}{0ex}
\setlength{\parskip}{0ex}
\setlength{\partopsep}{0ex}
\setlength{\parindent}{0pt} 
%\setlength{\bibsep}{0.0pt}


% Input macros
% Symbols
\newcommand{\ra}{\rightarrow}
\newcommand{\ov}{\overline}
\newcommand{\ur}{\underline}
\newcommand{\pr}{\prime}
\newcommand{\tbf}{\textbf}
\newcommand{\omg}{\Omega}
\newcommand{\mc}{\mathcal}
\newcommand{\lt}{\left}
\newcommand{\rt}{\right}
\newcommand{\mb}{\mathbb}
\newcommand{\imp}{\implies}
\newcommand{\dimp}{\Leftrightarrow}
\newcommand{\wh}{\widehat}
\newcommand{\dg}{\mc{D}}


% Sets
\newcommand{\realset}{\mb{R}}
\newcommand{\comprealset}{\mb{\ov{R}}}
\newcommand{\pint}{\mb{Z}_{> 0}}
\newcommand{\wholenums}{\mb{Z}_\geq 0}
\newcommand{\nzrl}{\mb{R}_{\geq 0}}
\newcommand{\prl}{\mb{R}_{>0}}
\newcommand{\rl}{\mb{R}}
\newcommand{\fin}{\forall i\in\{1,...,n\}}
\newcommand{\tup}[1]{\{1,...,#1\}}
\newcommand{\seq}[2]{_{#1=1}^#2}
\newcommand{\mCnn}{\mb{M}_{n\times n}(\mb{C})}
\newcommand{\mCmm}{\mb{M}_{m\times m}(\mb{C})}
\newcommand{\mCpn}{\mb{M}_{p\times n}(\mb{C})}
\newcommand{\mCnm}{\mb{M}_{n\times m}(\mb{C})}
\newcommand{\mRnn}{\mb{M}_{n\times n}(\mb{R})}
\newcommand{\mRpn}{\mb{M}_{p\times n}(\mb{R})}
\newcommand{\mRnm}{\mb{M}_{n\times m}(\mb{R})}
\newcommand{\mRno}{{{\mb{R}}}^n_{\geq 0}}
\newcommand{\mRmo}{{{\mb{R}}}^m_{\geq 0}}
\newcommand{\mRo}{\mb{R}_{\geq 0}}
\newcommand{\mRn}{{\mb{R}}^n}
\newcommand{\mRm}{{\mb{R}}^m}
\newcommand{\mCn}{\mb{C}^n}
\newcommand{\mCm}{\mb{C}^m}
\newcommand{\inv}{\mb{GL}_n(\mb{R})}
\newcommand{\id}[1]{\mb{I}_{#1\times #1}}
\newcommand{\mat}[3]{\mb{M}_{#1\times #2}(\mb{#3})}


% matrix operations
\newcommand{\ColumnJoin}[2]{\left[\begin{array}{l}{#1}\\{#2} \end{array}\right]}
\newcommand{\cjoin}[3]{\begin{array}{l}#1\\#2\\#3\end{array}}
\newcommand{\pseudoinverse}[1]{#1^\dagger}
\newcommand{\pinv}{\pseudoinverse}

%% Local macros:

% Operators
\DeclareMathOperator{\real}{\operatorname{Re}}
\newcommand{\minaffine}[2]{\widehat{\Lambda}\lt(#1,#2\rt)}
\newcommand{\maxaffine}[2]{\overline{\Lambda}\lt(#1,#2\rt)}
\newcommand{\minaffinefunc}{\widehat{\Lambda}}
\newcommand{\maxaffinefunc}{\overline{\Lambda}}
\newcommand{\minaffineset}{\widehat{\Delta}}
\newcommand{\maxaffineset}{\overline{\Delta}}

% Set notations
\newcommand{\zon}[3]{\mathcal{Z}\lt(#1,#2,#3\rt)}
\newcommand{\CZ}{\lt(V,c,s\rt)}
\newcommand{\GCZ}{\gcz{V}{c}{s}{W}{l}{u}}
\newcommand{\cz}[3]{\mc{C}\lt(#1,#2,#3\rt)}
\newcommand{\CZO}{\lt(V,0,s\rt)}
\newcommand{\czo}[2]{\mc{Z}\lt(#1,0,#2\rt)}
\newcommand{\trj}[2]{{\bf #1}(#2)}
\newcommand{\IncTcz}[6]{\mc{T}\lt(#1,#2,#3,#4,#5,#6\rt)}
\newcommand{\IncGcz}[6]{\mc{G}\lt(#1,#2,#3,#4,#5,#6\rt)}
\newcommand{\Ptope}[3]{\mc{P}\left(#1,#2,#3\right)}
\newcommand{\gcz}[6]{\mathcal{G}\lt(#1,#2,#3,#4,#5,#6\rt)}
\newcommand{\sptope}[3]{\mathcal{P}\lt(#1,#2,#3\rt)}
\newcommand{\scalebound}[5]{\max_{i=1}^{#4}\lt(\lt(\sum_{j=1}^{#5}\lt|#1\rt|_j\rt)-#3_i+#2_i\rt)}
\newcommand{\contemp}[1]{#1^T\lt(#1#1^T\rt)^{-1}}
\newcommand{\sectemp}[1]{\contemp{\ptemplate\lt(#1\rt)}}

% System notations
\newcommand{\system}{\mb{H}}
\newcommand{\locationset}{Q}
\newcommand{\edgeset}{E}
\newcommand{\stay}{\gamma}
\newcommand{\linearmapset}{\mc{A}}
\newcommand{\inputset}{U}
\newcommand{\initialset}{\Omega}
\newcommand{\edge}{\sigma}
\newcommand{\loc}{q}
\newcommand{\map}{\linearmapset}
\newcommand{\inp}{\inputset}
\newcommand{\ptemplate}{\mc{K}}
\newcommand{\systrj}[2]{\lt({\bf #1},{\bf #2}\rt)}
\newcommand{\preloc}[1]{#1_{1}}
\newcommand{\postloc}[1]{#1_{2}}
\newcommand{\upperedgebound}[1]{#1^+}
\newcommand{\loweredgebound}[1]{#1^-}
\newcommand{\reset}[1]{#1_r}
\newcommand{\locationtransition}[1]{R_{#1}}
\newcommand{\edgetransition}[1]{R_{#1}}
\newcommand{\staysptope}[1]{\sptope{\ptemplate\lt(#1\rt)}{\stay^-\lt(#1\rt)}{\stay^+\lt(#1\rt)}}
\newcommand{\guardsptope}[1]{\sptope{\ptemplate\lt(\preloc{#1}\rt)}{\max\lt(\loweredgebound{#1},\stay^-\lt(\preloc{#1}\rt)\rt)}{\min\lt(\upperedgebound{#1},\stay^+\lt(\preloc{#1}\rt) \rt)}}
\newcommand{\hybridset}{\Gamma}

% Equations
\newcommand{\transfer}[4]{#1#4 = #2\dg\lt(#3\rt)}
\newcommand{\centertransfer}[4]{#1#4 = #3-#2}




% Headers
\title{Augmented complex zonotopes for computing invariants of affine hybrid systems\thanks{This research work is partially supported by ANR project MALTHY.}
} 

\author{Arvind\ Adimoolam and Thao\ Dang
} 
\institute{\ Verimag,~Grenoble, France\\ \url{{santosh.adimoolam,thao.dang}@univ-grenoble-alpes.fr}.
}


\begin{document}

\maketitle

\begin{abstract}
Zonotopes are a useful set representation for bounded time reach set
computation of affine hybrid systems because of their closure under
Minkowski sum and matrix multiplication operations.  For unbounded
time reach set approximation of arbitrarily switched affine hybrid
systems, template complex zonotopes and a corresponding invariant
computation procedure were introduced, which utilized the possibly
complex eigenstructure of the affine maps.  But a major hurdle in
extending the technique for computing invariants of more general
affine hybrid systems, where switching is state dependent and
controlled by linear constraints, is that the class of template
complex zonotopes is not closed under intersection with linear
constraints.  In this paper, we use a more expressive set
representation called augmented complex zonotopes, for which we
propose an algebraic over-approximation of the intersection with
linear constraints.  This over-approximation is then used to derive a
set of second order conic constraints for computing an augmented
complex zonotopic positive invariant for discrete time affine hybrid
systems with additive disturbance input and linear safety constraints.
We demonstrate the efficiency of this approach by experimenting on
some benchmark examples.
\end{abstract}

\section{Introduction}
In hybrid dynamical systems, a transition can be controlled by
constraints on the state variables which act as preconditions for the
transition.  In an affine hybrid system, the preconditions are linear
constraints on the state variables.  So, computing an
overapproximation of a reachable set in a set representation would
involve computing accurate overapproximation for the intersection with
linear constraints.  In case of complex zonotopes, similar to simple
zonotopes they are not closed under intersection with sub-level sets
of linear inequalities.  To address this problem, we shall introduce a
more general representation of a complex zonotopic set, called
augmented complex zonotope, by which we can over-approximate the
intersection with a particular class of linear constraints, called
\emph{sub-parallelotopic}.  Furthermore, we can control the error in
overapproximation by adjusting the scaling factors.  Before describing
an augmented complex zonotope, we briefly discuss related work on
zonotopes and compare it with the augmented complex zonotopes.  

{\bf Related set representations and problem with their extension to complex
zonotopes: } To accurately represent the intersection with linear
sub-level sets, real zonotopes have been extended to constrained
zonotopes~\cite{scott2016constrained} or more generally constrained
affine sets~\cite{Ghorbal2010}.  Constrained zonotopes are briefly
described in the introductory chapter.  In these representations, in
addition to the interval constraints on the combining coefficients,
there can be more general linear constraints.  This permits exact
representation of the intersection with hyperplanes, in the case of
constrained zonotopes, and half-spaces, in the case of constrained
affine sets.  In these extended representations, the support function
can still be computed efficiently by linear programming because there are
only linear constraints on the combining coefficients.  Similarly, in
our complex zonotope representation, although there are quadratic
constraints (absolute value bounds) on the combining coefficients, the
support function can be computed by a simple affine expression given
in Lemma~\ref{lem:support-tcz}.  However, if we introduce linear
constraints on the complex combining coefficients in addition to the
quadratic absolute value bounds, computing the support function
becomes intractable.  But accurate computation of the support function
is required in verifying bounds on the reachable set.  In other words,
extending the idea of constrained zonotope or constrained affine set
to a complex zonotope will make the computation of support function
intractable.

{\bf Our solution: } We observe that in certain cases, intersection of
a version of the real zonotope representation, called \emph{interval
zonotope}, with a particular class of linear sub-level sets,
called \emph{sub-parallelotopes}, can be computed efficiently.
Motivated by this, we introduce a more general representation of
complex zonotope, called \emph{augmented complex zonotope}, which
denotes the Minkowski sum of a complex zonotope and an interval
zonotope.  We show that the interval zonotope part of an augmented
complex zonotope can be used to over-approximate the intersection.  On
the other hand, we can still compute the support function efficiently
because an augmented complex zonotope is geometrically equivalent to a
template complex zonotope.  Furthermore, we show that the error in
overapproximation can be regulated by adjusting the scaling factors
and the center of the template complex zonotope part.

This chapter has three main sections.  In Section~\ref{sec:iztope}, we
introduce the interval zonotope and sub-parallelotope set
representations, and describe the intersection between them.
Furthermore, we discuss other operations on interval zonotopes like
linear transformation, Minkowski sum and computation of the support
function.  In Section~\ref{sec:acz-it}, we introduce the augmented
complex zonotope and described its intersection with a
sub-parallelotope.  We compute a bound on the over-approximation of
the intersection between them, which is dependent on the scaling
factors and the center.  In Section~\ref{sec:operations-acz}, we shall
discuss other operations on augmented complex zonotopes like linear
transformation, Minkowski sum and computation of the support function.


\section{Hybrid systems and positive invariants}~\label{sec:system}

In a discrete time affine hybrid system, we have a set of
discrete variables, called locations, and continuous
variables that take values in the real euclidean space of dimension
$n\in\pint$.  In each location, there are a set
of linear constraints, called \emph{staying conditions}, within which
the continuous state of the system in that location is contrained.
In each location, there is an affine transition map with (possibly)
additive uncertain but bounded disturbance input specifying the
transition of the continuous variables.  A set of labeled directed
edges specify possible transitions between different locations,
accompanied by affine transitions on continuous variables with a bounded
additive disturbance input.  Each edge transition is controlled by a
set of linear constraints on the continous variables, called guards.

In this paper, for specifying the staying conditions and guards, we consider a
specific class of linear constraints that represent possibly
unbounded parallelotopes.  We shall call them as
sub-parallelotopes, defined as follows.
%
\begin{definition}[Sub-parallelotope]~\label{defn:sub-parallelotope} Let
  $K\in\mat{k}{n}{R}$ such that $k\leq n$ and $\lt(KK^T\rt)$ is
  non-singular.  We call such a matrix $K$ as a
  ñ\emph{sub-parallelotopic template}.  Let
  $\wh{u},\wh{l}\in\comprealset^n$ such that $\wh{u}\leq\wh{l}$.  Then
  the following is a sub-parallelotopic set.
\[
\sptope{K}{\wh{l}}{\wh{u}} = \lt\{x\in\realset^n: \wh{l}\leq Kx \leq \wh{u}\rt\}
\]
\end{definition}
%
[GIVE EXAMPLE] For many examples of affine hybrid systems, the guards
and staying conditions can be specified as sub-parallelotopic sets. 


\paragraph{Specification.}
We specify the discrete time affine hybrid system by a tuple 
%
\begin{equation}~\label{eqn:system}
\system =
\lt(\locationset,\ptemplate,\stay,\edgeset,\linearmapset,\inputset\rt).
\end{equation}
%
Here, $\locationset$ is a finite set of locations.  For each location
$\loc\in\locationset$, a sub-parallelotopic template
$\ptemplate(\loc)\in\mat{k(q)}{n}{\realset}$ is used for defining the
linear constraints on the staying conditions and the guards on edges
emanating from the location.  Accordingly, a pair of upper and lower
bounds
$\stay(\loc)=\lt(\stay^-(\loc),\stay^+(\loc)\rt)\in\mb{R}^n\times\mb{R}^n:
~~\stay^-(\loc)\leq\stay^+(\loc)$ gives the sub-parallelotopic
staying set
$\sptope{\ptemplate\lt(\loc\rt)}{\stay^-(\loc)}{\stay^+(\loc)}$ in the
location.  The maps
$\linearmapset:\lt(\locationset\bigcup\edgeset\rt)\ra\mat{n}{n}{R}$
and $\inputset:\lt(\locationset\bigcup\edgeset\rt)\ra 2^{\mb{R}^n}$,
respectively, give the linear transfomation and additive input set for
all intralocation transitions and edge transitions.

The set of edges is $\edgeset$, where $\edge\in\edgeset$ is a tuple
$\edge =
\lt(\preloc{\edge},\postloc{\edge},\loweredgebound{\edge},\upperedgebound{\edge},\map(\edge),\inp(\edge)\rt)$,
described as follows.  The pre and post locations of the edge are
$\preloc{\edge}\in\locationset$ and $\postloc{\edge}\in\locationset$,
respectively.  The pair of upper and lower bounds
$\lt(\edge^-,\edge^+\rt)\in\realset^{k\lt(\preloc{\edge}\rt)}\times\realset^{k\lt(\preloc{\edge}\rt)}:~~\edge^-\leq\edge^+$,
relate to the sub-parallelotopic guard set
$\sptope{\ptemplate\lt(\preloc{\edge}\rt)}{\edge^-}{\edge^+}$, which
is a precondition on the edge transition.  The
affine transition along the edge is given by the linear map
$\map(\edge)$ and additive uncertain input set $\inp(\edge)$.

\paragraph{Dynamics.}
The state of the hybrid system is a pair $(x,\loc)$, where
$x\in\realset^n$ is called the continuous state and
$\loc\in\locationset$ is called the discrete state.  The
evolution of the state of the system in time is called a
\emph{trajectory}.  The trajectory is a function
$\systrj{x}{\loc}:\wholenums\ra\realset^n\times\locationset$, such
that for all $t\in\wholenums$, one of the following is true.

\begin{enumerate}
\item Intralocation dynamics.
\begin{align}~\label{eqn:intralocation}
\begin{split}
& \exists u\in\inputset\lt(\trj{\loc}{t}\rt)~~\text{such that all of
    the below conditions  are satisfied.}\\
& \trj{x}{t+1} = \map(\trj{\loc}{t})+u,~~~\trj{\loc}{t+1} = \trj{\loc}{t}~~
\text{and}\\
& \trj{x}{t},~\trj{x}{t+1}\in\sptope{\ptemplate\lt(\trj{\loc}{t}\rt)}{\stay^-\lt(\trj{\loc}{t}\rt)}{\stay^+\lt(\trj{\loc}{t}\rt)}.
\end{split}
\end{align}
\item Interlocation dynamics.
\begin{align} 
\begin{split}
& \exists \edge\in\edgeset~~\text{such that all of the below
    conditions are satisfied.}\\
&
  \trj{\loc}{t}=\preloc{\edge},~~~\trj{x}{t}\in\sptope{\ptemplate\lt(\preloc{\edge}\rt)}{\loweredgebound{\edge}\bigvee\stay^-\lt(\preloc{\edge}\rt)}{\upperedgebound{\edge}\bigwedge\stay^+\lt(\preloc{\edge}\rt)} \\
& \trj{x}{t+1} = \map(\trj{\loc}{t})+u,~~~\trj{\loc}{t+1} = \postloc{\edge}\\
& \trj{x}{t+1}\in \sptope{\ptemplate\lt(\postloc{\edge}\rt)}{\stay^-\lt(\postloc{\edge}\rt)}{\stay^+\lt(\postloc{\edge}\rt)}.
\end{split}
\end{align}
\end{enumerate}

The set of reachable continuous states in one time step of
intralocation and interlocation dynamics, respectively, from a given
set of continuous states is given by the following two functions.
\begin{enumerate}
\item For any location $\loc\in\locationset$, define $\locationtransition{\loc}:2^{\realset^n}\ra 2^{\realset^n}$ as
\begin{equation*}
\locationtransition{q}\lt(S\rt) = \lt\{\Calign{\lt(\map\lt(\loc\rt)\lt(S\bigcap\staysptope{\loc}\rt)\oplus
\inputset\lt(\loc\rt)\rt)}{~~\bigcap~~\staysptope{\loc}}\rt..
\end{equation*}
\item For any edge $\edge\in\edgeset$, define
  $\edgetransition{\edge}:2^{\realset^n}\ra 2^{\realset^n}$ as
\begin{multline*}
\edgetransition{\edge}\lt(S\rt) =  \lt\{\Calign{\lt(\map(\edge)\lt(S\bigcap
\guardsptope{\edge}\rt)\oplus\inp(\edge)\rt)}{~~\bigcap~~\staysptope{\postloc{\edge}}}\rt..
\end{multline*}
\end{enumerate}

We shall identify a set of states by a mapping of the kind
$\hybridset:\locationset\ra 2^{\realset^n}$, called a \emph{state
  set}, which corresponds to the set of states
$\lt\{\lt(x,\loc\rt):x\in\hybridset\lt(\loc\rt)\rt\}$.  A positive
invariant is a set of states of the system such that all trajectories
beginning at any state in the positive invariant remain withing the
positive invariant.  Equivalently, a state set is a positive invariant
if the reachable set in one time step by both the intralocation and
interlocation dynamics is contained within the original state set.
\begin{definition}
A state set $\hybridset$ is a positive invariant if
both the following conditions hold.
\begin{align}
& \forall\loc\in\locationset,~~\locationtransition{q}\lt(\hybridset(\loc)\rt) \subseteq \hybridset\lt(\loc\rt)~\label{eqn:pi1}\\
& \forall\edge\in\edgeset,~~\edgetransition{\edge}\lt(\hybridset\lt(\preloc{\edge}\rt)\rt) \subseteq~\label{eqn:pi2}
  \hybridset\lt(\postloc{\edge}\rt).
\end{align}
\end{definition}


%\section{Review of some set representations}~\label{sec:review}


\section{Augmented complex zonotopes}~\label{sec:acz}
In a generalized template complex zonotope,
a subset of the combining coefficients are real and bounded in real
intervals, while the rest are complex with the usual bound on their
absolute values.

\begin{definition}
Let $V\in\mat{n}{m}{C}$ called primary template, $W\in\mat{n}{k}{R}$
called secondary template, $c\in\mb{R}^n$ called primary offset,
$s\in\mb{R}^m$ called scaling factors, $u,l\in\mb{R}^k$ called lower
and upper rectangular bounds, respectively such that $l\leq u$.  The
following is a generalized template complex
zonotope.
\begin{multline}
\gcz{V}{c}{s}{W}{u}{l} =
\lt\{
  c+V\epsilon+W\zeta:\epsilon\in\mb{C}^m,\zeta\in\mb{R}^k,\rt.\\ \lt.  \lt|\epsilon_i\rt|\leq
 s_i~\forall i\in\tup{m},~\zeta_j\in[l_j,u_j]~\forall j\in \tup{k}
\rt\}
\end{multline}
\end{definition}
Since a complex zonotope can have a non-polyhedral real projection,
its intersection with a linear constraint may not have a template
complex zonotopic real projection.  Moreover, there is no unique way
to overapproximate such an intersection as a template complex
zonotope.  In contrast, we discussed in~Lemma~\ref{lem:motivation}
that the intersection of a real zonotope with a suitably aligned
subparallelotope can be a real zonotope.  However, for invariant
computation, template complex zonotopes have the advantage that they
can incorporate complex valued eigenvectors in the template, while
real zonotopes can not.  Therefore, we define a new set
representation, called augmented complex zonotopes, which combines the
template complex zonotope and the real zonotope by their Minkowski
sum.  Then, the real zonotopic part of the augmented complex zonotope
can be used to compute the intersections, while the template complex
zonotopic part can incorporate the complex eigenvectors.
%
\begin{definition}[Augmented complex zonotope]
Let $V\in\mat{n}{m}{C}$ called primary template, $W\in\mat{n}{k}{R}$
called secondary template, $c\in\mb{R}^n$ called primary offset,
$s\in\mb{R}^m$ called scaling factors, $u,l\in\mb{R}^k$ called lower
and upper interval bounds, respectively, such that $l\leq u$.  The
following is an augmented complex
zonotope.
\begin{equation*}
\gcz{V}{c}{s}{W}{l}{u} = \cz{V}{c}{s}\oplus\zon{W}{l}{u}.
\end{equation*}
\end{definition}
%
The following lemma gives an overapproximation of the intersection
between an augmented complex zonotope and a suitably aligned
subparallelotope as another augmented complex zonotope.  Furthermore,
under an orthogonality condition stated below, we compute the exact
intersection.
%
\begin{lemma}~\label{lem:acz-int}
Let $\ptemplate$ be a sub-paralleotopic template and $c\in\mb{C}^n$
such that $\ptemplate c=0$.  Then,
\begin{enumerate}
\item
  $\gcz{V}{c}{s}{\pinv{\ptemplate}}{l}{u}\bigcap\sptope{\ptemplate}{\wh{l}}{\wh{u}}\subseteq
  \gcz{V}{c}{s}{\pinv{\ptemplate}}{l\bigvee\wh{l}}{u\bigwedge\wh{u}}$.
\item If $\ptemplate V=0$, then $\gcz{V}{c}{s}{\pinv{\ptemplate}}{l}{u}\bigcap\sptope{\ptemplate}{\wh{l}}{\wh{u}}=
  \gcz{V}{c}{s}{\pinv{\ptemplate}}{l\bigvee\wh{l}}{u\bigwedge\wh{u}}$.
\end{enumerate}
\end{lemma}
%
To illustrate, the intersection of $\gcz{\lt[\begin{array}{cc}1+2i &
2+i\\1-2i & 2-i\\0 &
0\end{array}\rt]}{\lt[\begin{array}{c}1\\1\\0\end{array}\rt]}{\lt[\begin{array}{c}1\\1\\1\end{array}\rt]}
{\lt[\begin{array}{c}0\\0\\1\end{array}\rt]}{-2}{2}$. with a
constraint on the third coordinate $-1\leq x_3\leq 1$ is exactly\\
$\gcz{\lt[\begin{array}{cc}1+2i & 2+i\\1-2i & 2-i\\0 &
0\end{array}\rt]}{\lt[\begin{array}{c}1\\1\\0\end{array}\rt]}{\lt[\begin{array}{c}1\\1\\1\end{array}\rt]}
{\lt[\begin{array}{c}0\\0\\1\end{array}\rt]}{-1}{1}$.  We observe that
the center $\lt(\begin{array}{c}1\\1\\0\end{array}\rt)$ is
perpendicular to the third axis, i.e., the vector
$\lt[\begin{array}{ccc}0 & 0 & 1\end{array}\rt]$, which is a required
condition in the above lemma.  Furthermore, since the primary template
is orthogonal to the subparallelotopic template in this example, i.e.
$\lt[\begin{array}{ccc}0 & 0 &
1\end{array}\rt]\lt[\begin{array}{cc}1+2i & 2+i\\1-2i & 2-i\\0 &
0\end{array}\rt]=0$, the resultant intersection is exactly an
augmented complex zonotope.  On the other hand, if the primary
template is not orthogonal with the secondary template, like in the
case of $\gcz{\lt[\begin{array}{cc}1+2i & 2+i\\1-2i & 2-i\\1 &
1\end{array}\rt]}{\lt[\begin{array}{c}1\\1\\0\end{array}\rt]}{\lt[\begin{array}{c}1\\1\\1\end{array}\rt]}
{\lt[\begin{array}{c}0\\0\\1\end{array}\rt]}{-2}{2}$, then the
intersection with $-1\leq x_3\leq 1$ is overapproximated by
$\gcz{\lt[\begin{array}{cc}1+2i & 2+i\\1-2i & 2-i\\1 &
1\end{array}\rt]}{\lt[\begin{array}{c}1\\1\\0\end{array}\rt]}{\lt[\begin{array}{c}1\\1\\1\end{array}\rt]}
{\lt[\begin{array}{c}0\\0\\1\end{array}\rt]}{-1}{1}$, but this is not
the exact intersected set.


Linear transformations and Minkowski sums preserve the class of
augmented complex zonotopes, and these can be efficiently computed as
follows.
%
\begin{lemma}[Linear transformation and Minkowski sum]
\begin{enumerate}
\item $A\gcz{V}{c}{s}{W}{l}{u} = \gcz{AV}{Ac}{s}{AW}{l}{u}$.
\item $\gcz{V}{c}{s}{W}{l}{u}\oplus
  \gcz{V^\pr}{c^\pr}{s^\pr}{W^\pr}{l^\pr}{u^\pr}$\\
$= \gcz{\lt[V~~V^\pr\rt]}{c+c^\pr}{\ColumnJoin{s}{s^\pr}}{\lt[W~~W^\pr\rt]}{\ColumnJoin{l}{l^\pr}}{\ColumnJoin{u}{u^\pr}}$
\end{enumerate}
\end{lemma}
%
The limits of the projection of an augmented complex zonotope along
any direction are stated in the following lemma.
%
\begin{lemma}~\label{lem:polylimits-acz}
Let $V\in\mat{n}{m}{\mb{C}}$ and $v\in\realset^n$.  Then,
\[
\max_{x\in\gcz{V}{c}{s}{W}{l}{u}}v^Tx = v^T\lt(c+W\frac{l+u}{2}\rt)+\lt|v^T[V~~W]\rt|\lt(\ColumnJoin{s}{\frac{u-l}{2}}\rt)
\]
\end{lemma}
%
The real projection of an augmented complex zonotope can be
equivalently tranformed as the real projection of a template complex
zonotope, as follows.  However, we note that as a complex valued set,
an augmented complex zonotope is more general and can not be
represented as a template complex zonotope.  Furthermore, the
representation as an augmented complex zonotope is more succinct,
hence, more convenient for deriving the conditions for invariant
computation.
%
\begin{lemma}~\label{lem:conversion}
$\real\lt(\gcz{V}{c}{s}{W}{l}{u}\rt) = \real\lt(\cz{\lt[V~W\rt]}{c+W\lt(\frac{u+l}{2}\rt)}{\ColumnJoin{s}{\frac{u-l}{2}}}\rt)$.
\end{lemma}
%
Because of the above relationship, checking the inclusion between the
real projections of two augmented complex zonotopes amounts to
checking the inclusion between real projections of two template
complex zonotopes.  Recall the partial order between two template
complex zonotopes defined in the previous section. We extend this
partial order to augmented complex zonotopes as follows.
%
\begin{definition}~\label{defn:gcz-order}
We say that $\gcz{V^\pr}{c^\pr}{s^\pr}{W^\pr}{l^\pr}{u^\pr}\order
\gcz{V}{c}{s}{W}{l}{u}$ if\\ $\cz{\lt[V^\pr~W^\pr\rt]}{c^\pr+W^\pr\lt(\frac{u^\pr+l^\pr}{2}\rt)}{\ColumnJoin{s^\pr}{\frac{u-l}{2}}}
\order
\cz{\lt[V~W\rt]}{c+W\lt(\frac{u+l}{2}\rt)}{\ColumnJoin{s}{\frac{u-l}{2}}}.$
\end{definition}
%
\begin{lemma}[Ordering: augmented complex
    zonotopes]~\label{lem:gcz-gcz} The relation ``$\order$'' among
augmented complex zonotopes is a partial order.  Furthermore, the real
inclusion\\
$\real\lt(\gcz{V^\pr}{c^\pr}{s^\pr}{W^\pr}{l^\pr}{u^\pr}\rt)\subseteq \real\lt(\gcz{V}{c}{s}{W_{n\times
k}}{l}{u}\rt)$ holds if the order relation
$\gcz{V^\pr}{c^\pr}{s^\pr}{W^\pr}{l^\pr}{u^\pr}\order \gcz{V}{c}{s}{W_{n\times
k}}{l}{u}$ holds.
\end{lemma}

The intersection of an augmented complex zonotope with a
subparallelotope involves the meet and join operations, as stated in
Lemma~\ref{lem:acz-int}.  These operations are piecewise affine
functions of their arguments, but not affine.  Hence, their
composition with a convex function may be non-convex.  But in the case
of affine hybrid systems, the guards and staying conditions define the
limits on the interval bounds of a sub-parallelotope.  Therefore,
during invaraint computation, we may overapproximate the intersections
using the provided by the staying conditions and guards.  Formally, we
define the following upper and lower bound functions for the join and
meet operations, respectively.

Let us define a binary function
$\minaffinefunc:\realset^k\times\comprealset^k$,
called \emph{min-approximation} function, as follows.  For
$u\in\realset^k$ and $\wh{u}\in\comprealset^k$,
$\lt(\minaffine{u}{\wh{u}}\rt)_i = \left\{
\begin{array}{l}
\wh{u}_i~~\text{if}~\wh{u}_i<\inf\\
u_i~~\text{if}~\wh{u}_i=\inf
\end{array}
\right..$
Similarly, let us define another binaray function       
$\maxaffinefunc:\realset^k\times\comprealset^k$,
called \emph{max-approximation} function, as follows.  For
$l\in\realset^k$ and $\wh{l}\in\comprealset^k$,
$\lt(\maxaffine{l}{\wh{l}}\rt)_i = \left\{
\begin{array}{l}
\wh{l}_i~~\text{if}~\wh{l}_i>\inf\\
l_i~~\text{if}~\wh{l}_i=-\inf
\end{array}
\right..$
%
\begin{lemma}~\label{lem:min-max-approx}
Both the following statements are true.
\begin{enumerate}
\item Let $l,u\in\realset^k$ and $\wh{l},\wh{u}\in\comprealset^k$.
  Then, $\maxaffine{l}{\wh{l}}\leq l\bigvee\wh{l}$ and
  $\minaffine{u}{\wh{u}}\geq u\bigwedge\wh{u}$.
\item For fixed $\wh{l},\wh{u}\in\comprealset^k$, the functions
  $\maxaffine{.}{\wh{l}}:\realset^k\ra\realset^k$ and
  $\minaffine{.}{\wh{u}}:\realset^k\ra\realset^k$ are both affine functions.
\end{enumerate}
\end{lemma}








\section{Computation of positive invariants}~\label{sec:invcomp}

Our objective is to find a positive invariant of the following kind.
\begin{itemize}
\item The projection of the positive invariant in every location is an augmented complex zonotope.
\item The positive invariant is contained within a safe set specified by linear inequalities.
\item The positive invariant contains an initial set that is over-approximated by an augmented complex zonotope.
\end{itemize} 
We use a fixed template based approach for finding such a positive
invariant.  In this approach, we fix suitable primary and secondary
templates and solve for the primary offsets, scaling factors and
sub-parallelotopic bounds.  In this section, we derive constranints on
the aforementioned variables whose satisfaction yeilds a positive
invariant as above, assuming that the templates are given.  The way we
choose the primary and secondary templates will be discussed in a
latter section.


The following two lemmas state sufficient conditions for satisfaction of
the inclusion relations for positive invariance~(\ref{eqn:pi1},\ref{eqn:pi2}) by
an augmented complex zonotope.
\begin{lemma}
  Let us consider a state set $\hybridset$ whose projection in each
  location $\loc\in\locationset$ is $\hybridset(\loc) =
  \real\lt(\gcz{V_{n\times m(\loc)}(\loc)}{c(\loc)}{s(\loc)}{\pinv{\ptemplate}(\loc)}{l(\loc)}{u(\loc)}\rt)$
  such that $\ptemplate(\loc)V(\loc) = 0$.  Let us consider that the
  additive input in each location $\loc\in\locationset$ can be
  overapproximated as $\inputset(\loc) = \cz{B_{n\times
      m^\pr(q)}(\loc)}{b(\loc)}{r(\loc)}$.  Then we have
  $\locationtransition{\loc}\lt(\hybridset(\loc)\rt) \subseteq
  \hybridset\lt(\loc\rt)$ if,
\begin{align}
\begin{split}~\label{eqn:exists-intra}
& \exists
  X(\loc)\in\mat{\lt(m(\loc)+k(\loc)\rt)}{\lt(m(\loc)+k(\loc)+m^\pr(\loc)\rt)}{\mb{C}},~
  y(\loc)\in\mb{C}^{m(\loc)+k(\loc)}\\&~u^\pr(\loc),l^\pr(\loc)\in\realset^{k(\loc)}~\text{and}~s^\pr(\loc)\in\realset^{k(\loc)}~\text{and
    policies}~
\minaffinefunc_{\loc}^{post}\in\minaffineset\lt(k\lt(\loc\rt)\rt),\\
&\minaffinefunc_{\loc}^{pre}\in\minaffineset\lt(k\lt(\loc\rt)\rt),\maxaffinefunc_{\loc}^{pre}\in\maxaffineset\lt(k\lt(\loc\rt)\rt),
\maxaffinefunc_{\loc}^{post}\in\maxaffineset\lt(k\lt(\loc\rt)\rt),~\text{s.t.}
\end{split}
\end{align}
\begin{align}
\begin{split}
& \left[V(\loc),~\pinv{\ptemplate}(\loc)\rt]X(\loc) =
  [A(\loc)V(\loc),~A(\loc)\pinv{\ptemplate}(\loc),~B(\loc)]\dg\lt(\cjoin{s(\loc)}{s^\pr(\loc)}{r(\loc)}\rt)\\
\end{split}
\end{align}
\begin{align}
\begin{split}
& s^\pr(\loc) = \frac{\minaffinefunc_{\loc}^{pre}\lt(\stay^+(\loc),u(\loc)\rt)
      -\maxaffinefunc_{\loc}^{pre}\lt(\stay^-(\loc),l(\loc)\rt)}{2}
\end{split}
\end{align}
\begin{align}
\begin{split}
  & \left[V(\loc),~~\pinv{\ptemplate}(\loc)\rt]y(\loc) = 
    \map(\loc)\lt(c(\loc)+\pinv{\ptemplate}(\loc)\frac{\minaffinefunc_{\loc}^{pre}\lt(\stay^+(\loc),u(\loc)\rt)
     +\maxaffinefunc_{\loc}^{pre}\lt(\stay^-(\loc),l(\loc)\rt)}{2}\rt)\\ & \hspace{10em} +
  b(\loc)-c(\loc)-\pinv{\ptemplate}(\loc)\frac{u^\pr(\loc)+l^\pr(\loc)}{2}
\end{split}
\end{align}
\begin{align}
\begin{split}
& \forall
i\in\tup{m(\loc)},~\lt|y_i(\loc)\rt|+\sum_{i=1}^{m(\loc)+k(\loc)+m^\pr(\loc)}\lt|X_{ij}(\loc)\rt|\leq
s_i(\loc),
\end{split}
\end{align}
\begin{align}
\begin{split}
& \forall
i\in\lt\{m(\loc)+1,...,m(\loc)+k(\loc)\rt\},\\
&\lt|y_i(\loc)\rt|+\sum_{i=1}^{m(\loc)+k(\loc)+m^\pr(\loc)}\lt|X_{ij}(\loc)\rt|\leq
\frac{u^\pr(\loc)-l^\pr(\loc)}{2}
\end{split}
\end{align}
\begin{align}
& l^\pr(\loc)\leq u^\pr(\loc)\\
\begin{split}
& \maxaffinefunc_{\loc}^{post}\lt(l^\pr(\loc),\stay^-(\loc)\rt)\geq
l(\loc)~\text{and}~\minaffinefunc_{\loc}^{post}\lt(u^\pr(\loc),\stay^+(\loc)\rt)\leq u(\loc).
\end{split}
\end{align}
\end{lemma}
\begin{proof}
{\color{red} TODO}
\end{proof}

\begin{lemma}
  Let us consider a state set $\hybridset$ whose projection in each
  location $\loc\in\locationset$ is $\hybridset(\loc) =
  \real\lt(\gcz{V_{n\times
      m(\loc)}(\loc)}{c(\loc)}{s(\loc)}{\pinv{\ptemplate}(\loc)}{l(\loc)}{u(\loc)}\rt)$
  such that $\ptemplate(\loc)V(\loc) = 0$.  Then for an edge
  $\edge\in\edgeset$, the inclusion
  $\edgetransition{\edge}\lt(\hybridset\lt(\preloc{\edge}\rt)\rt)
  \subseteq \hybridset\lt(\postloc{\edge}\rt)$ holds if,
\begin{align}
\begin{split}
& \exists X(\edge)\in
\mat{\lt(m\lt(\postloc{\edge}\rt)+k\lt(\postloc{\edge}\rt)\rt)}{\lt(m\lt(\preloc{\edge}\rt)+k\lt(\preloc{\edge}\rt)\rt)}{\mb{C}}~\text{and}~y(\edge)\in
\mb{C}^{m\lt(\postloc{\edge}\rt)+k\lt(\postloc{\edge}\rt)},\\
& \exists u^\pr(\edge),l^\pr(\edge)\in \realset^{k\lt(\postloc{\edge}\rt)}, c^\pr(\sigma)\in\mb{R}^n,s^\pr(\edge)\in\realset^{k\lt(\preloc{\edge}\rt)}~\text{and policies}\\
&\minaffinefunc_{\edge}^{pre}\in\minaffineset\lt(k\lt(\preloc{\edge}\rt)\rt),\maxaffinefunc_{\edge}^{pre}\in\maxaffineset\lt(k\lt(\preloc{\edge}\rt)\rt),
\minaffinefunc_{\edge}^{post}\in\minaffineset\lt(k\lt(\postloc{\edge}\rt)\rt),\maxaffinefunc_{\edge}^{post}\in\maxaffineset\lt(k\lt(\postloc{\edge}\rt)\rt),~\text{s.t.}
\end{split}\\
\begin{split}
& \lt[V\lt(\postloc{\edge}\rt),~\pinv{\ptemplate}\lt(\postloc{\edge}\rt)\rt]X(\edge) = \reset{\edge}\lt[V\lt(\preloc{\edge}\rt),~\pinv{\ptemplate}\lt(\preloc{\edge}\rt)\rt]\dg
\ColumnJoin{s\lt(\preloc{\edge}\rt)}{s^\pr(\edge)}\\
\end{split}
\end{align}
\begin{align}
\begin{split}
s^\pr(\edge) = \frac{\minaffinefunc_{\edge}^{pre}\lt(u\lt(\preloc{\edge}\rt),\min\lt(\stay^+\lt(\preloc{\edge}\rt),\edge^+\rt)\rt)-
\maxaffinefunc_{\edge}^{pre}\lt(l\lt(\preloc{\edge}\rt),\max\lt(\stay^-\lt(\preloc{\edge}\rt),\edge^-\rt)\rt)}{2}\\
\end{split}
\end{align}
\begin{align}
\begin{split}
& \lt[V\lt(\postloc{\edge}\rt),~\pinv{\ptemplate}\lt(\postloc{\edge}\rt)\rt]y(\edge) = \reset{\edge}\lt(c\lt(\preloc{\edge}\rt)+c^\pr(\edge)\rt)- 
c\lt(\postloc{\edge}\rt) - \pinv{\ptemplate}\lt(\postloc{\edge}\rt)\frac{u^\pr\lt(\edge\rt)+l^\pr\lt(\edge\rt)}{2}\\
\end{split}
\end{align}
\begin{align}
\begin{split}
& c^\pr(\edge) = \pinv{\ptemplate}\lt(\preloc{\edge}\rt)\frac{\minaffinefunc_{\edge}^{pre}\lt(u\lt(\preloc{\edge}\rt),\min\lt(\stay^+\lt(\preloc{\edge}\rt),\edge^+\rt)\rt)+
\maxaffinefunc_{\edge}^{pre}\lt(l\lt(\preloc{\edge}\rt),\max\lt(\stay^-\lt(\preloc{\edge}\rt),\edge^-\rt)\rt)}{2}
\end{split}
\end{align}
\begin{align}
\begin{split}
& \forall i\in \tup{m\lt(\postloc{\edge}\rt)}:
 \lt|y_i(\edge)\rt|+\sum_{j=1}^{m\lt(\preloc{\edge}\rt)+k\lt(\preloc{\edge}\rt)}\lt|X_{ij}(\edge)\rt|\leq s_i\lt(\postloc{\edge}\rt)
\end{split}\\
\begin{split}
& \forall i\in
\lt\{m\lt(\postloc{\edge}\rt)+1,...,m\lt(\postloc{\edge}\rt)+k\lt(\postloc{\edge}\rt)\rt\}:\\
&\lt|y_i(\edge)\rt|+\sum_{j=1}^{m\lt(\preloc{\edge}\rt)+k\lt(\preloc{\edge}\rt)}\lt|X_{ij}(\edge)\rt|\leq \frac{u^\pr_i\lt(\edge\rt)-l^\pr_i\lt(\edge\rt)}{2}
\end{split}\\
\begin{split}
l^\pr(\edge)\leq u^\pr(\edge)
\end{split}\\
& \maxaffinefunc_{\edge}^{post}\lt(l^\pr(\edge),\stay^-\lt(\postloc{\edge}\rt)\rt)\geq
l\lt(\postloc{\edge}\rt)~\text{and}~\minaffinefunc_{\edge}^{post}\lt(u^\pr(\edge),\stay^+\lt(\postloc{\edge}\rt)\rt)\leq u\lt(\postloc{\edge}\rt).
\end{align}
\end{lemma}
\begin{proof}
{\color{red} TODO}
\end{proof}

Additionally, we may want the positive invariant to contain an initial
set.  The following lemma states the sufficient condition for
containing an initial set.
\begin{lemma}
  Let $\mc{I}$ and $\hybridset$ be two state sets whose projections in each location
  $\loc\in\locationset$ are respectively $\mc{I}(\loc)=\real\lt(\gcz{V^I_{n\times
      m^\pr }}{c^I(\loc)}{s^I(\loc)}{W_{n\times
      k^\pr}}{l^I(\loc)}{u^I(\loc)}\rt)$ and $\hybridset(\loc) =
  \gcz{V_{n\times m(\loc)}(\loc)}{c(\loc)}{s(\loc)}{\pinv{\ptemplate}(\loc)}{l(\loc)}{u(\loc)}$.
  Then
  $\forall\loc\in\locationset,~\mc{I}(\loc)\subseteq \hybridset(\loc)$
  if $\forall\loc\in\locationset$, 
\begin{align}
\begin{split}
& \exists X^I(\loc)\in\mat{(m(\loc)+k(\loc))}{(m^\pr+k^\pr)}{\mb{C}}~\text{and}~y^I\in\mb{C}^{m(\loc)+k(\loc)}~s.t.\\
& \lt[V,~\pinv{\ptemplate}(\loc)\rt]X^I = \lt[V^I~W\rt]\dg\ColumnJoin{\hspace{0.5em}s^I(\loc)}{\frac{u^I(\loc)-l^I(\loc)}{2}}\\
& \lt[V,~\pinv{\ptemplate}(\loc)\rt]y^I =
c^I(\loc)+W\frac{u^I(\loc)+l^I(\loc)}{2}-c(\loc)-\pinv{\ptemplate}(\loc)\frac{u(\loc)+l(\loc)}{2}\\
& \forall i\in\tup{m(\loc)},
\lt|y_i^I\rt|+\sum_{j=1}^{m^\pr+k^\pr}\lt|X_{ij}^I\rt|\leq s_i(\loc)\\
& \forall
i\in\lt\{m(\loc)+k(\loc)\rt\},\lt|y_i^I\rt|+\sum_{j=1}^{m^\pr+k^\pr}\lt|X_{ij}^I\rt|\leq \frac{u_i(\loc)-l_i(\loc)}{2}
\end{split}
\end{align}
\end{lemma}

One of the applications of positive invariants is to prove safety.  If
a positive invariant containing an initial set is also contained
inside a safe set, it implies that the system is safe.  In this
regard, the following lemma states a sufficient condition for an
augmented complex zonotope to be contained within a polytopic
set (safe set).

\begin{lemma}
  Let us consider a state set $\hybridset$ whose projection in each
  location $\loc \in\locationset$ is $\hybridset(\loc) =
  \real\lt(\gcz{V_{n\times m(\loc)}(\loc)}{c(\loc)}{s(\loc)}{\pinv{\ptemplate}(\loc)}{l(\loc)}{u(\loc)}\rt)$.
  Let us consider a polytopic state set
  $\mc{J}:\locationset\ra\realset^n$ described as $\forall
  \loc\in\locationset,~\mc{J}(\loc)=\lt\{x\in\realset^n:Tx\leq
  d(\loc)\rt\}$, where $T\in\mat{p}{n}{R}$ and
  $d(\loc)\in\comprealset^n$.  Then $\forall\loc\in\locationset$,
  $\hybridset(\loc)\subseteq \mc{J}(\loc)$ if $\forall
  q\in\locationset$,
\begin{align}~\label{eqn:safe-contain}
T\lt(c+W\lt(\frac{u(\loc)+l(\loc)}{2}\rt)\rt)+\lt|T\lt[V(\loc),~\pinv{\ptemplate}(\loc)\rt]\rt|\ColumnJoin{\hspace{1.5em}s}{\frac{u(\loc)-l(\loc)}{2}}\leq d(\loc).
\end{align}
\end{lemma}

Consolidating the previous results in this section gives the following
sufficient condition for positive invariance.  We also
include sufficient conditions so that the positive invariant is
contained within a polytopic safe set and contains an initial set.

\begin{theorem}~\label{thm:main} Let $\hybridset$ be a state set whose
  projection in each location $\loc\in\locationset$ is
  $\hybridset(\loc) = \real\lt(\gcz{V_{n\times
      m(\loc)}(\loc)}{c(\loc)}{s(\loc)}{\pinv{\ptemplate}(\loc)}{l(\loc)}{u(\loc)}\rt)$
  such that $\ptemplate(\loc)V(\loc) = 0$.  Consider
  $\mc{I}:~\forall\loc\in\locationset$,
  $\mc{I}(\loc)=\real\lt(\gcz{V^I_{n\times m^\pr
    }}{c^I(\loc)}{s^I(\loc)}{W_{n\times
      k^\pr}}{l^I(\loc)}{u^I(\loc)}\rt)$ as an initial state set.  Let
  us consider a safe set specified by a polytopic state set
  $\mc{J}:\locationset\ra\realset^n$ described as $\forall
  \loc\in\locationset,~\mc{J}(\loc)=\lt\{x\in\realset^n:Tx\leq
  d(\loc)\rt\}$, where $T\in\mat{p}{n}{R}$ and
  $\forall\loc\in\locationset,~d(\loc)\in\comprealset^n$.  If $\forall
  \loc\in\locationset$ and $\forall \edge\in\edgeset$,
  Equations[\ref{eqn:exists-intra}-\ref{eqn:safe-contain}] are all
  satisfied, then $\hybridset$ is a positive invariant and $\forall
  \loc\in\locationset,~\mc{I}(\loc)\subseteq\hybridset(\loc)\subseteq
  \mc{J}(\loc)$.
\end{theorem}


\vspace{-1em}
\section{Experiments}~\label{sec:exp}
We performed experiments on three benchmark examples from literature
and compared the results with that obtained by the tool SpaceEx. [Add
  configuration here]. [Add floating point error here]


\subsection{Robot with a saturated controller}   Our first example is a benchmark
model of a self-balancing two wheeled robot called NXTway-GS1 by
Yorihisa Yamamoto, presented in the ARCH
workshop~\cite{heinz2014benchmark}.  The model is a networked control
system, i.e. a plant interacting with a controller.  The controller
has a hole, which is an unknown input to the controller and is modeled
as an additive disturbance input.  The controller input received by
the plant has a saturation limit.  Due to the saturation, the
composite system is modeled as a hybrid system.  Three different
models of the controller are proposed in the benchmark: continuous
linear, sampled data (discrete time) linear and non-linear.  In our
experiment, we consider the sampled data linear controller, with two
kinds of interaction with the plant: saturated, i.e., hybrid system
and unsaturated, i.e., linear system.  The sampling time given in the
benchmark is $4 ms$.  The safety requirement is that the \emph{body
  pitch angle} of the robot, denoted $\psi$, should be bounded within
some value. In the benchmark, it was suggested that
$\psi\in\lt[-\frac{\pi}{2}+\epsilon,\frac{\pi}{2}-\epsilon\rt]$ for
any $\epsilon>0$, is given as a safe set.  For the linear system
model, $\psi\in\lt[\frac{-\pi}{2.26},~\frac{\pi}{2.26}\rt]$ is given
as a safe set.


In discrete time, the composite sampled data system of the plant and
controller could be modeled using thirteen continuous state variables
and four uncertain input variables.  The model, however, had unbounded
trajectories in some directions.  Therefore, we decoupled some bounded
directions from the unbounded directions by an appropriate linear
transformation of the co-ordinates, such that the body pitch angle and
the controller inputs belong to the bounded directions.  The latter
model has ten continuous state variables and four uncertain input
variables. The controller input received by the plant is two
dimensional, which we denote by $u_1$ and $u_2$, respectively.  The
saturation limit on $u_i$ is $v_i=\delta d_p$, where $\delta=100$ and
$d_p=0.0807$.  Then, the saturated input is computed as $sat(u_i) =
max\lt(-v_i,min\lt(u_i,v_i\rt)\rt)$.  Thus, the two dimensional
controller input can be divided into nine regions such that in each
region, the saturation function is affine.

\tbf{Modeling}.  We model the saturated system (after transformation)
as a ten dimensional hybrid system using one location and nine self
edges with appropriate guards, such that all possible transitions
occur only along the edges.  For the unsaturated model, we have one
location and no edges, and the only system transition is specified by
the intralocation affine map.  The initial set is the origin.

\tbf{Size of model}: 10 dimensional, 1 location and 9 edges.

\tbf{Implementation.}  For the hybrid system, we choose the secondary
template as the pseudoinverse of the guarding hyperplane normals, in
conformity with Theroem~\ref{thm:main}.  The primary template for the
hybrid system is chosen as the collection of the (complex) eigenvectors of
linear matrices of all affine maps for the edge transitions, the
orthonormal vectors to the guarding hyperplane normals and the
projections of the eigenvectors on the subspace spanned by the
orthonormal vectors.  For the linear system, we only have a primary
template, which is constituted by the eigenvectors of the linear map,
the input set template and its multiplication by the linear matrix
(related to affine map) and square of the linear matrix.  For the
SpaceEx implementation, we tested with the octagon template and a
template with 400 uniformly sampled support vectors.

%% For the hybrid system, we computed a single augmented complex
%% zonotopic invariant satisfying both the upper and lower safety bounds.
%% But for the linear system, we computed two different invariants, one
%% each satisfying the upper and lower bounds, respectively.

\tbf{Results.}  For both the hybrid and the linear systems, we could
verify smaller magnitudes for the bounds on the pitch angle than what
is proposed in the benchmark~\cite{heinz2014benchmark}.  But the
SpaceEx tool could not find a finite bound for either of the above
systems.  The results are reported in the
Tables~\ref{tab:robot-unsaturated} and~\ref{tab:robot-saturated}.

\tbf{Remarks.}  Although the unsaturated model is linear,
SpaceEx could not find an invariant with as many as 400 support
vectors.  In theory, although a linear system has a polytopic
invariant, but the number of faces of the polytope can be arbitrarily
large for a fixed dimension if the eigenvalues are complex.  The
robot model has some complex eigenvalues whose absolute values were
close to one.  On the other hand, since augmented complex zonotopes
can incorporate the complex eigenvectors in the template, we are
guaranteed to find an invariant for the linear (unsaturated) system.
Futhermore, we could also a safe invariant for the hybrid model of the
saturated system.

\begin{table}
\center{UB: $>$1000, ~~NT: Not terminating in more than 180s, \newline
  n/a: Not applicable/not available, ~~ACZ: Augmented complex
  zonotope.\vspace{1em} }
\begin{minipage}{0.48\textwidth}
\centering
\begin{tabular}{|l|c|c|c|}
\hline
\multicolumn{2}{|c|}{\multirow{2}{*}{Method}} &
\multirow{2}{*}{$\lt|\psi\rt|\leq$} & Comp.\\
\multicolumn{2}{|c|}{} & & time (s)\\
\hline
\multirow{4}{*}{SpaceEx} & octagon & \multirow{2}{*}{UB} & \multirow{2}{*}{NT}\\
& template & & \\
\cline{2-4}
& 400 support & \multirow{2}{*}{UB} & \multirow{2}{*}{NT}\\
& vectors & &\\
\hline
\multicolumn{2}{|c|}{\multirow{2}{*}{Suggested in~\cite{heinz2014benchmark}}} &
\multirow{2}{*}{$1.39$} & \multirow{2}{*}{n/a}\\
\multicolumn{2}{|c|}{} & &\\
\hline
\multicolumn{2}{|c|}{\multirow{2}{*}{ACZ invariant}} & \multirow{2}{*}{$1.29$} &
\multirow{2}{*}{$4$}\\
\multicolumn{2}{|c|}{} & & \\
\hline
\end{tabular}
\caption{Unsaturated robot model: results}
~\label{tab:robot-unsaturated}
\end{minipage}
\hspace{0em}
\begin{minipage}{0.48\textwidth}
\centering
\begin{tabular}{|l|c|c|c|}
\hline
\multicolumn{2}{|c|}{\multirow{2}{*}{Method}} &
\multirow{2}{*}{$\lt|\psi\rt|\leq$} & Comp.\\
\multicolumn{2}{|c|}{} & & time (s)\\
\hline
\multirow{4}{*}{SpaceEx} & octagon & \multirow{2}{*}{UB} &
\multirow{2}{*}{NT}\\
& template & & \\
\cline{2-4}
& 400 support & \multirow{2}{*}{UB} & \multirow{2}{*}{NT}\\
& vectors & & \\
\hline
\multicolumn{2}{|c|}{\multirow{2}{*}{Suggested in~\cite{heinz2014benchmark}}} &
$1.571-\epsilon:$ & \multirow{2}{*}{n/a}\\
\multicolumn{2}{|c|}{} & $\epsilon>0$ &\\
\hline
\multicolumn{2}{|c|}{\multirow{2}{*}{ACZ invariant}} & \multirow{2}{*}{$1.13$} &
\multirow{2}{*}{45}\\
\multicolumn{2}{|c|}{} & &\\
\hline
\end{tabular}
\caption{Saturated robot model: results}
~\label{tab:robot-saturated}
\end{minipage}
\begin{minipage}{0.45\textwidth}
\begin{tabular}{|l|c|c|c|c|}
\hline
\multicolumn{2}{|c|}{\multirow{2}{*}{Method}} &
\multirow{2}{*}{$\lt|x_1\rt|\leq$} & \multirow{2}{*}{$\lt|x_2\rt|\leq$} & Comp.\\
\multicolumn{2}{|c|}{} & & & time (s) \\
\hline
\multirow{4}{*}{SpaceEx} & octagon & \multirow{2}{*}{0.38} &
\multirow{2}{*}{0.43} & \multirow{2}{*}{1.7}\\
& template & & &\\
\cline{2-5}
& 100 support & \multirow{2}{*}{0.38} & \multirow{2}{*}{0.43} & \multirow{2}{*}{23.6}\\
& vectors & & &\\
\hline
\multicolumn{2}{|c|}{\multirow{2}{*}{ACZ invariant}} &
\multirow{2}{*}{0.38} & \multirow{2}{*}{0.36} & 
\multirow{2}{*}{5.1}\\
\multicolumn{2}{|c|}{} & & &\\
\hline
\end{tabular}
\caption{Small invariant computation:\newline Perturbed double
  integrator}
~\label{tab:smallinv-pdi}
\vspace{1em}
\end{minipage}
\hspace{4em}
\begin{minipage}{0.4\textwidth}
\begin{tabular}{|c|c|}
\hline
\multirow{2}{*}{Method} & Comp.\\
& time (s)\\
\hline
\multirow{2}{*}{MPT tool~\cite{rakovic2004computation}} & \multirow{2}{*}{107}\\
& \\
\hline
\multirow{2}{*}{ACZ} & \multirow{2}{*}{12}\\
& \\
\hline
\end{tabular}
\caption{Large invariant computation: Perturbed double integrator}
~\label{tab:largeinv-pdi}
\vspace{1em}
\end{minipage}
%
\begin{tabular}{|l|c|c|c|c|c|}
\hline
\multicolumn{2}{|c|}{\multirow{2}{*}{Method}} &
\multirow{2}{*}{$-e_1\leq$} & \multirow{2}{*}{$-e_2\leq$} & \multirow{2}{*}{$-e_3\leq$} & Comp.\\
\multicolumn{2}{|c|}{} & & & & time (s)\\
\hline
\multirow{4}{*}{SpaceEx} & octagon & \multirow{2}{*}{28} &
\multirow{2}{*}{27} & \multirow{2}{*}{10} &
\multirow{2}{*}{NT}\\
& template & & & & \\
\cline{2-6}
& 100 support & \multirow{2}{*}{28} & \multirow{2}{*}{25} &
\multirow{2}{*}{13} & \multirow{2}{*}{1.3}\\
& vectors & & & & \\
\hline
\multicolumn{2}{|c|}{\multirow{2}{*}{Real zonotope~\cite{makhlouf2014networked}}} &
\multirow{2}{*}{25} & \multirow{2}{*}{25} & \multirow{2}{*}{10}
 & \multirow{2}{*}{n/a}\\
\multicolumn{2}{|c|}{} & & & & \\
\hline
\multicolumn{2}{|c|}{\multirow{2}{*}{ACZ invariant}} &
\multirow{2}{*}{28} & \multirow{2}{*}{26} &
\multirow{2}{*}{12} & \multirow{2}{*}{12}\\
\multicolumn{2}{|c|}{} & & & &\\
\hline
\end{tabular}
\caption{Large minimum dwell time (20s) model of networked
  vehicle platoon: results}
~\label{tab:largedwell-platoon}
 $~$\\
\begin{tabular}{|l|c|c|c|c|c|}
\hline
\multicolumn{2}{|c|}{\multirow{2}{*}{Method}} &
\multirow{2}{*}{$-e_1\leq$} & \multirow{2}{*}{$-e_2\leq$} & \multirow{2}{*}{$-e_3\leq$} & Comp.\\
\multicolumn{2}{|c|}{} & & & & time (s)\\
\hline
\multirow{4}{*}{SpaceEx} & octagon & \multirow{2}{*}{UB} &
\multirow{2}{*}{UB} & \multirow{2}{*}{UB} &
\multirow{2}{*}{NT}\\
& template & & & & \\
\cline{2-6}
& 100 support & \multirow{2}{*}{UB} & \multirow{2}{*}{UB} &
\multirow{2}{*}{UB} & \multirow{2}{*}{NT}\\
& vectors & & & & \\
\hline
\multicolumn{2}{|c|}{\multirow{2}{*}{ACZ invariant}} &
\multirow{2}{*}{46} & \multirow{2}{*}{54} &
\multirow{2}{*}{57} & \multirow{2}{*}{12.6}\\
\multicolumn{2}{|c|}{} & & & &\\
\hline
\end{tabular}
\caption{Small minimum dwell time (1s) model of networked vehicle
  platoon: results}
~\label{tab:smalldwell-platoon}
\end{table}
%
\subsection{Perturbed double integrator}
Our second example is a perturbed double integrator system that is
described in~\cite{rakovic2004computation}.  The closed loop system
with a feedback control is piecewise affine, having four different
affine dynamics in four different regions of space.  The system is two
dimensional and has a bounded additive disturbance input.  We perform
two different experiments on this system.  In the first experiment, we
try to verify the smallest possible magnitude of bounds on the two
coordinates, denoted $x_1$ and $x_2$. We compare these bounds with
that found by the SpaceEx tool.

%% \begin{equation}~\label{eqn:pwa-regions}
%% \trj{x}{t+1}=\lt(A_i+B_iK_i\rt)\trj{x}{t}+w,~\text{where}~
%% i=\left\{\begin{array}{l}
%% 1,~\text{if}~x_1\geq 0~\text{and}~x_2\geq 0\\
%% 2,~\text{if}~x_1\leq 0~\text{and}~x_2\leq 0\\
%% 3,~\text{if}~x_1\leq 0~\text{and}~x_2\geq 0\\
%% 4,~\text{if}~x_1\geq 0~\text{and}~x_2\leq 0\\
%% \end{array} \rt.
%% \end{equation}
%% %
%% \begin{align*}
%% & A_1 =\lt[\begin{array}{ll}
%% 1 & 1\\
%% 0 & 1
%% \end{array}\rt],~B_1 = \lt[\Calign{1}{0.5}\rt],~K_1 = \lt[-0.5897~
%%   -0.9347\rt]\\
%% & A_2 = \lt[\begin{array}{ll}
%% 1 & 1\\
%% 1 & 0
%% \end{array}
%% \rt],~B_2 = \lt[\Calign{-1}{-0.5}\rt],~K_2 = \lt[0.5897~~0.9387\rt]\\
%% & A_3 = \lt[\begin{array}{ll}
%% 1 & -1\\
%% 0 & 1
%% \end{array}
%% \rt],~B_3 = \lt[\Calign{-1}{0.5}\rt],~K_3 = \lt[0.5897~-0.9387\rt]\\
%% & A_4 = \lt[\begin{array}{ll}
%% 1 & -1\\
%% 0 & 1
%% \end{array}
%% \rt],~B_4 = \lt[\Calign{1}{-0.5}\rt],~K_4 = \lt[-0.5897~~0.9387\rt].\\
%% \end{align*}
%% %  
%% The additive disturbance input $w$ is bounded as $\|w\|_{\infty}\leq
%% 0.2$.  

In the second experiment, we try to quickly compute a large invariant
for the system under the safety constraints given
in~\cite{rakovic2004computation}.  %% The
%% given safety
%% constraints are $\|x\|_{\infty}\leq 5$ and $\lt|K_i(x)\leq
%% 1\rt|~\forall i\in\lt\{1,2\rt\}$.
In the latter case, we maximize the sum of the scaling factors and
differences of the upper and lower interval bounds of the augmented
complex zonotopic invaraint.  Furthermore, we decompose the given
safety constraints as the intersection of four different sets of
safety constraints.  For each set of safety constraints, we compute a
large augmented complex zonotopic invariant.  Then the desired
invariant is the intersection of four augmented complex zonotopic
invaraints.  Although we may not find the largest possible (maximal)
invariant by this approach, still the optimizer would find a large
invaraint.  We draw comparison in terms of the computation time with
the reported result for the MPT tool~\cite{rakovic2004computation}.

\tbf{Modeling.}  In our formalism, we model the system as two
dimensional with four locations and twelve edges.  Appropriate staying
conditions are specified in each location, reflecting the division of
the state space into different regions where the dynamics is affine.
The edges constitute all possible interlocation transitions.  The
initial set is the origin.  The same model is specified in SpaceEx.

\tbf{Size of model}: 2 dimensional, 4 locations and 12 edges.

\tbf{Implementation}.  We choose the secondary template in each
location as the pseudoinverse (in this case equal to) the hyperplane
normals of the staying conditions in that location, so that
Theorem~\ref{thm:main} is applicable.  For the primary template, we
collected the (complex) eigenvectors of all linear matrices of the
affine maps and their binary products. For the SpaceEx tool, we
experimented with two different templates, the octagon template and a
template with 100 uniformly sampled support vectors.

\tbf{Results.}  In the first experiment on this example, the bounds
verified bounds for the first coordinate by our method is equal to
that of SpaceEx. But for the second coordinate, we verfied smaller
bounds than that of SpaceEx.  In our second experiment on this
example, the computation time for finding a large invariant by our
method is significantly smaller than that of the reported result for
the MPT tool.  The results are summarized in the Tables~\ref{tab:smallinv-pdi}
and~\ref{tab:largeinv-pdi}.

\tbf{Remark.}  The overapproximation quality of SpaceEx reduces when
there are large number of possible transitions, because then a union
of many different polytopes has to be overapproximated using support
vectors.  In contrast, our approach uses optimization to learn an
appropriate invariant, avoiding the union operations.  Possibly
because of this reason, we could verify smaller bounds than SpaceEx,
for the second coordinate.


\subsection{Networked platoon of vehicles}
Our third example is a model of a networked cooperative platoon of
vehicles, which is presented as a benchmark in the ARCH
workshop~\cite{makhlouf2014networked}.  The platoon consists of three
vehicles $M_1$, $M_2$ and $M_3$ along with a leader board ahead.  Each
vehicle has a reference distance to the vehicle ahead of it.  The
difference between the actual distance of a vehicle $M_i$ to its
successor and the reference distance is denoted as $e_i$.  Any upper
bound on $-e_i$ is a safe lower limit on the reference distance above
which the platoon is guaranteed not to collide.

The movement of the vehicles is dependent on the communication between
them.  In the benchmark proposal, the dynamics of the vechicles is
described as a hybrid system with two locations having different
dynamics.  In one location, there is communication between all the
vehicles, while in another location, there is complete communication
failure.  In the general model described in the benchmark, there can
be staying conditions for each location and time constraints on the
switching time.  The benchmark then considers a specific case where
the minimum dwell time is 20 seconds (also specified in the
distributed SpaceEx
implementation\footnote{http://cps-vo.org/node/15096}).  In this
paper, we consider two cases, minimum dwell times of 20 seconds and 1
second, respectively.


\tbf{Modeling.}  Since our method is applicable to discrete time
hybrid systems, we need to find a discrete time system whose reachable
set overapproximates that of the given continuous time system.  For
the large minimum dwell time of $20s$, it is possible efficiently
discretize the system, satisfying the aforementioned requirement.  We
do not, however, explain the discretization procedure here, because it
is beyond the scope of this paper.  But for the small minimum dwell
time of $1s$, a similar discretization of the system would lead to a
very high complexity model.  So, for our experiment in the latter
case, we alternatively considered a model in which the switching can
only happen at integer instants of time.  With this assumption,
efficient discretization was possible for the $1s$ minimum dwell
model.  Henceforth, we model the above two systems in the formalism
described in this paper. The same models are also used for the
discrete time SpaceEx implementation.  The size of both the models is
given below.

\tbf{Size of large dwell time model}: 9 dimensional, 2
locations and 4 edges.

\tbf{Size of small dwell time model}: 9 dimensional, 2
locations, 2 edges.

\tbf{Implementation.}  In our approach, we choose the primary template
as the collection of the (complex) eigenvectors of linear matrices of
the affine maps in the the two locations and their binary products,
the axis alligned box template and the templates used for
overapproximating the input sets.  The secondary template is set to
the zero vector since there are no linear guards or staying conditions
in this example.  For the SpaceEx tool, we experimented with two
templates, octagon and hundred uniformly sampled support vectors.

\tbf{Results.}  For the large minimum dwell time of $20s$, the
discrete time SpaceEx implementation and also a method based on using
real zonotopes~\cite{makhlouf2014networked} could verify slightly
smaller bounds on $-e_1$, $-e_2$ and $-e_3$, compared to our approach.
But for the small minimum dwell time ($1s$) model, SpaceEx could not
even find a finite set of bounds, whereas our approach could verify a
finite set of bounds.  These results are reported in the
Tables~\ref{tab:largedwell-platoon} and~\ref{tab:smalldwell-platoon}.

\tbf{Remarks.}  The dynamics in both locations being stable, the
lyapunov exponent (measure of stability) tends to be higher for larger
dwell time.  Threfore, SpaceEx and the zonotope based approach could
perform somewhat better than our approach when the minimum dwell time
is 20s.  But when the minimum dwell time is 1s, SpaceEx failed
possibly because the lyapunov exponent is smaller in this case.
However, our approach found an invariant for the smaller minimum dwell
time case as well.  Besides, we note that the zonotope based approach
can be difficult to use when the dwell time is smaller because then a
union of a large number of zonotopes has to be stored or
overapproximated in each step.  In contrast, our approach avoids the
union operations and instead formulates an optimization program.



\vspace{-0.5em}

\section{Conclusion}~\label{sec:conclusion}
We introduced augmented complex zonotopes as a more general set
representation than template complex zonotopes, based on which we
derived efficiently solvable conditions for computing invariants,
subject to linear safety constraints, for discrete time affine hybrid
systems with linear guards and additive disturbance input.  Like
template complex zonotopes, augmented complex zonotopes have the
advantage that we can meaningfully choose the templates for efficient
fixpoint computation, based on the eigenstructure and and other
relevant aspects of the dynamics.  But additionally, we overcame a
drawback of template complex zonotopes in that we derived a simple
algebraic expression for reasonable overapproximation of the
intersection with a class of linear constraints.  We use this
algebraic expression to obtain of a set of second order conic
constraints that can be efficiently solved to compute an invariant.
In contrast to the step-by-step reachability computation approaches
that iteratively accumulate overapproximation error, we instead
compute an invariant in a single convex optimization step such that
the optimizer inherently minimizes the overapproximation error.  We
demonstrated the efficiency of our approach on some benchmark
examples.

As future work, we can investigate ways to minimize the
overapproximation error in the intersection operation, such that the
overapproximation can still be algebrically computed.  In particular,
the relation between the choice of the template and the
over-approximation error in the intersection has to be analyzed.  Also, we
would like to extend this computational framework to continuous
time hybrid systems.


\newpage
%\bibliographystyle{plain}
\bibliographystyle{abbrv}
\bibliography{ref}

\appendix
%\section*{Proofs in section}
Appendix goes here

\section{Title} \label{ap:SA}
some citation \cite{Angluin87}

\end{document}
